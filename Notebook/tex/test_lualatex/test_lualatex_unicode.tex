Vitaly Ankh\today\documentclass[12pt,a4paper,oneside]{scrbook}
\tracinglostchars=2
\usepackage[match]{luatexja-fontspec}
\usepackage{polyglossia}
\setdefaultlanguage[variant=american]{english}
\usepackage{csquotes}
%\setotherlanguage{chinese}
\usepackage{xstring}

\defaultfontfeatures{ Ligatures=TeX,
                      Scale=MatchUppercase }

\setmainfont{Latin Modern Roman}[Scale=1.0]
\setmainjfont{AR PL UMing TW}[Renderer=HarfBuzz]
\(f(x)_{i} \)
tectonic -X compile /tmp/8d0429d611b937833d6e38e373a09194e1a943da.tex -Z shell-escape --outfmt xdv --outdir /tmp/

\begin{document}


The common transliteration of 易经 is {\enquote{I Ching,}}
following the now-obsolete Wade-Giles system: modern Romanization
of almost all modern Chinese texts follows the Hanyu Pinyin standard used here.
See [huang].

\end{document}
