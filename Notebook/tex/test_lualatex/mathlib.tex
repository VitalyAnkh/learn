\documentclass{article}
\usepackage{microtype}
\usepackage{hyperref}
\usepackage{fontspec}
\usepackage{xcolor}
\usepackage{newunicodechar}
\usepackage{bussproofs}
\usepackage{listings}
\usepackage{titlesec}
\usepackage{parskip}
\usepackage{enumerate}
\usepackage{contour}
\usepackage{ulem}
\author{leanprover-community https://leanprover-community.github.io}
\title{mathlib}
\AtBeginDocument{\newunicodechar{←} {\mathfont{←}}}
\AtBeginDocument{\newunicodechar{→} {\mathfont{→}}}
\AtBeginDocument{\newunicodechar{↓} {\mathfont{↓}}}
\AtBeginDocument{\newunicodechar{↔} {\mathfont{↔}}}
\AtBeginDocument{\newunicodechar{↦} {\mathfont{↦}}}
\AtBeginDocument{\newunicodechar{↪} {\mathfont{↪}}}
\AtBeginDocument{\newunicodechar{∀} {\mathfont{∀}}}
\AtBeginDocument{\newunicodechar{∃} {\mathfont{∃}}}
\AtBeginDocument{\newunicodechar{∅} {\mathfont{∅}}}
\AtBeginDocument{\newunicodechar{∈} {\mathfont{∈}}}
\AtBeginDocument{\newunicodechar{∉} {\mathfont{∉}}}
\AtBeginDocument{\newunicodechar{∘} {\mathfont{∘}}}
\AtBeginDocument{\newunicodechar{√} {\mathfont{√}}}
\AtBeginDocument{\newunicodechar{∞} {\mathfont{∞}}}
\AtBeginDocument{\newunicodechar{∣} {\mathfont{∣}}}
\AtBeginDocument{\newunicodechar{∥} {\mathfont{∥}}}
\AtBeginDocument{\newunicodechar{∧} {\mathfont{∧}}}
\AtBeginDocument{\newunicodechar{∨} {\mathfont{∨}}}
\AtBeginDocument{\newunicodechar{∩} {\mathfont{∩}}}
\AtBeginDocument{\newunicodechar{∪} {\mathfont{∪}}}
\AtBeginDocument{\newunicodechar{≃} {\mathfont{≃}}}
\AtBeginDocument{\newunicodechar{≅} {\mathfont{≅}}}
\AtBeginDocument{\newunicodechar{≠} {\mathfont{≠}}}
\AtBeginDocument{\newunicodechar{≡} {\mathfont{≡}}}
\AtBeginDocument{\newunicodechar{≤} {\mathfont{≤}}}
\AtBeginDocument{\newunicodechar{≥} {\mathfont{≥}}}
\AtBeginDocument{\newunicodechar{≫} {\mathfont{≫}}}
\AtBeginDocument{\newunicodechar{≺} {\mathfont{≺}}}
\AtBeginDocument{\newunicodechar{≼} {\mathfont{≼}}}
\AtBeginDocument{\newunicodechar{⊂} {\mathfont{⊂}}}
\AtBeginDocument{\newunicodechar{⊆} {\mathfont{⊆}}}
\AtBeginDocument{\newunicodechar{⊑} {\mathfont{⊑}}}
\AtBeginDocument{\newunicodechar{⊓} {\mathfont{⊓}}}
\AtBeginDocument{\newunicodechar{⊔} {\mathfont{⊔}}}
\AtBeginDocument{\newunicodechar{⊕} {\mathfont{⊕}}}
\AtBeginDocument{\newunicodechar{⊢} {\mathfont{⊢}}}
\AtBeginDocument{\newunicodechar{⊤} {\mathfont{⊤}}}
\AtBeginDocument{\newunicodechar{⊥} {\mathfont{⊥}}}
\AtBeginDocument{\newunicodechar{⋁} {\mathfont{⋁}}}
\AtBeginDocument{\newunicodechar{⋀} {\mathfont{⋀}}}
\AtBeginDocument{\newunicodechar{⋃} {\mathfont{⋃}}}
\AtBeginDocument{\newunicodechar{⋂} {\mathfont{⋂}}}
\AtBeginDocument{\newunicodechar{⋙} {\mathfont{⋙}}}
\AtBeginDocument{\newunicodechar{⋯} {\mathfont{⋯}}}
\AtBeginDocument{\newunicodechar{⌊} {\mathfont{⌊}}}
\AtBeginDocument{\newunicodechar{⌋} {\mathfont{⌋}}}
\AtBeginDocument{\newunicodechar{⟨} {\mathfont{⟨}}}
\AtBeginDocument{\newunicodechar{⟩} {\mathfont{⟩}}}
\AtBeginDocument{\newunicodechar{⟶} {\mathfont{⟶}}}
\AtBeginDocument{\newunicodechar{⥤} {\mathfont{⥤}}}
\AtBeginDocument{\newunicodechar{ᵢ} {\textsubscript{i}}}
\AtBeginDocument{\newunicodechar{ⱼ} {\textsubscript{j}}}
\AtBeginDocument{\newunicodechar{ᵣ} {\textsubscript{r}}}
\AtBeginDocument{\newunicodechar{ₘ} {\textsubscript{m}}}
\AtBeginDocument{\newunicodechar{ₙ} {\textsubscript{n}}}
\AtBeginDocument{\newunicodechar{ᵒ} {\textsubscript{o}}}
\AtBeginDocument{\newunicodechar{ₚ} {\textsubscript{p}}}
\AtBeginDocument{\newunicodechar{ₖ} {\textsubscript{k}}}
\AtBeginDocument{\newunicodechar{₀} {\textsubscript{0}}}
\AtBeginDocument{\newunicodechar{ᵖ} {\textsuperscript{p}}}
\newfontfamily{\mathfont}{STIX2Math.otf}
    \newfontfamily{\ttmathfont}{texgyrecursor-regular.otf}
    \hypersetup{
        colorlinks,
        linkcolor={red!50!black},
        citecolor={blue!50!black},
        urlcolor={blue!80!black}
    }
    \renewcommand{\ULdepth}{2.0pt}
    \contourlength{1pt}

    \newcommand{\fancyuline}[1]{%
        \uline{\phantom{#1}}%
      \llap{\contour{white}{#1}}%
    }

    \titleformat{\paragraph}[hang]{\normalfont\normalsize\bfseries}{\fancyuline{\theparagraph}}{1em}{\fancyuline}
    \titlespacing*{\paragraph}{0pt}{2.25ex plus 1ex minus .2ex}{0pt}
    \begin{document}
    \maketitle
    \clearpage
    \tableofcontents
    \clearpage
    \setmainfont[
         BoldFont={STIX2Text-Bold.otf},
         ItalicFont={STIX2Text-Italic.otf},
         BoldItalicFont={STIX2Text-BoldItalic.otf}
    ]
    {STIX2Text-Regular.otf}
    \section{analysis/filter.lean}\paragraph{cfilter}
\par
A 
\colorbox[RGB]{253,246,227}{{{{\color[RGB]{101, 123, 131} cfilter α σ }}}} is a realization of a filter (base) on 
\colorbox[RGB]{253,246,227}{{{{\color[RGB]{101, 123, 131} α }}}},
represented by a type 
\colorbox[RGB]{253,246,227}{{{{\color[RGB]{101, 123, 131} σ }}}} together with operations for the top element and
the binary inf operation.
\paragraph{cfilter.of\_equiv}
\par
Map a cfilter to an equivalent representation type.
\paragraph{cfilter.to\_filter}
\par
The filter represented by a 
\colorbox[RGB]{253,246,227}{{{{\color[RGB]{101, 123, 131} cfilter }}}} is the collection of supersets of
elements of the filter base.
\paragraph{filter.realizer}
\par
A realizer for filter 
\colorbox[RGB]{253,246,227}{{{{\color[RGB]{101, 123, 131} f }}}} is a cfilter which generates 
\colorbox[RGB]{253,246,227}{{{{\color[RGB]{101, 123, 131} f }}}}.
\paragraph{filter.realizer.of\_filter}
\par
A filter realizes itself.
\paragraph{filter.realizer.of\_equiv}
\par
Transfer a filter realizer to another realizer on a different base type.
\paragraph{filter.realizer.principal}
\par
\colorbox[RGB]{253,246,227}{{{{\color[RGB]{101, 123, 131} unit }}}} is a realizer for the principal filter
\paragraph{filter.realizer.top}
\par
\colorbox[RGB]{253,246,227}{{{{\color[RGB]{101, 123, 131} unit }}}} is a realizer for the top filter
\paragraph{filter.realizer.bot}
\par
\colorbox[RGB]{253,246,227}{{{{\color[RGB]{101, 123, 131} unit }}}} is a realizer for the bottom filter
\paragraph{filter.realizer.map}
\par
Construct a realizer for 
\colorbox[RGB]{253,246,227}{{{{\color[RGB]{101, 123, 131} map m f }}}} given a realizer for 
\colorbox[RGB]{253,246,227}{{{{\color[RGB]{101, 123, 131} f }}}}\paragraph{filter.realizer.comap}
\par
Construct a realizer for 
\colorbox[RGB]{253,246,227}{{{{\color[RGB]{101, 123, 131} comap m f }}}} given a realizer for 
\colorbox[RGB]{253,246,227}{{{{\color[RGB]{101, 123, 131} f }}}}\paragraph{filter.realizer.sup}
\par
Construct a realizer for the sup of two filters
\paragraph{filter.realizer.inf}
\par
Construct a realizer for the inf of two filters
\paragraph{filter.realizer.cofinite}
\par
Construct a realizer for the cofinite filter
\paragraph{filter.realizer.bind}
\par
Construct a realizer for filter bind
\paragraph{filter.realizer.Sup}
\par
Construct a realizer for indexed supremum
\paragraph{filter.realizer.prod}
\par
Construct a realizer for the product of filters
\section{analysis/topology.lean}\paragraph{ctop}
\par
A 
\colorbox[RGB]{253,246,227}{{{{\color[RGB]{101, 123, 131} ctop α σ }}}} is a realization of a topology (basis) on 
\colorbox[RGB]{253,246,227}{{{{\color[RGB]{101, 123, 131} α }}}},
represented by a type 
\colorbox[RGB]{253,246,227}{{{{\color[RGB]{101, 123, 131} σ }}}} together with operations for the top element and
the intersection operation.
\paragraph{ctop.of\_equiv}
\par
Map a ctop to an equivalent representation type.
\paragraph{ctop.realizer}
\par
A 
\colorbox[RGB]{253,246,227}{{{{\color[RGB]{101, 123, 131} ctop }}}} realizer for the topological space 
\colorbox[RGB]{253,246,227}{{{{\color[RGB]{101, 123, 131} T }}}} is a 
\colorbox[RGB]{253,246,227}{{{{\color[RGB]{101, 123, 131} ctop }}}}which generates 
\colorbox[RGB]{253,246,227}{{{{\color[RGB]{101, 123, 131} T }}}}.
\section{array/lemmas.lean}\section{bool.lean}\section{buffer/basic.lean}\section{char.lean}\section{complex/basic.lean}\section{complex/exponential.lean}\section{dfinsupp.lean}\paragraph{dfinsupp.map\_range}
\par
The composition of 
\colorbox[RGB]{253,246,227}{{{{\color[RGB]{101, 123, 131} f : β₁  }}}{{{\color[RGB]{133, 153, 0} → }}}{{{\color[RGB]{101, 123, 131}  β₂ }}}} and 
\colorbox[RGB]{253,246,227}{{{{\color[RGB]{101, 123, 131} g : Π₀ i, β₁ i }}}} is
\colorbox[RGB]{253,246,227}{{{{\color[RGB]{101, 123, 131} map\_range f hf g : Π₀ i, β₂ i }}}}, well defined when 
\colorbox[RGB]{253,246,227}{{{{\color[RGB]{101, 123, 131} f  }}}{{{\color[RGB]{108, 113, 196} 0 }}}{{{\color[RGB]{101, 123, 131}   }}}{{{\color[RGB]{181, 137, 0} = }}}{{{\color[RGB]{101, 123, 131}   }}}{{{\color[RGB]{108, 113, 196} 0 }}}}.
\paragraph{dfinsupp.filter}
\par
\colorbox[RGB]{253,246,227}{{{{\color[RGB]{101, 123, 131} filter p f }}}} is the function which is 
\colorbox[RGB]{253,246,227}{{{{\color[RGB]{101, 123, 131} f i }}}} if 
\colorbox[RGB]{253,246,227}{{{{\color[RGB]{101, 123, 131} p i }}}} is true and 0 otherwise.
\paragraph{dfinsupp.subtype\_domain}
\par
\colorbox[RGB]{253,246,227}{{{{\color[RGB]{101, 123, 131} subtype\_domain p f }}}} is the restriction of the finitely supported function
\colorbox[RGB]{253,246,227}{{{{\color[RGB]{101, 123, 131} f }}}} to the subtype 
\colorbox[RGB]{253,246,227}{{{{\color[RGB]{101, 123, 131} p }}}}.
\paragraph{dfinsupp.sum}
\par
\colorbox[RGB]{253,246,227}{{{{\color[RGB]{101, 123, 131} sum f g }}}} is the sum of 
\colorbox[RGB]{253,246,227}{{{{\color[RGB]{101, 123, 131} g i (f i) }}}} over the support of 
\colorbox[RGB]{253,246,227}{{{{\color[RGB]{101, 123, 131} f }}}}.
\paragraph{dfinsupp.prod}
\par
\colorbox[RGB]{253,246,227}{{{{\color[RGB]{101, 123, 131} prod f g }}}} is the product of 
\colorbox[RGB]{253,246,227}{{{{\color[RGB]{101, 123, 131} g i (f i) }}}} over the support of 
\colorbox[RGB]{253,246,227}{{{{\color[RGB]{101, 123, 131} f }}}}.
\section{dlist/basic.lean}\section{dlist/instances.lean}\section{equiv/algebra.lean}\section{equiv/basic.lean}\paragraph{equiv}
\par
\colorbox[RGB]{253,246,227}{{{{\color[RGB]{101, 123, 131} α ≃ β }}}} is the type of functions from 
\colorbox[RGB]{253,246,227}{{{{\color[RGB]{101, 123, 131} α  }}}{{{\color[RGB]{133, 153, 0} → }}}{{{\color[RGB]{101, 123, 131}  β }}}} with a two-sided inverse.
\paragraph{equiv.perm}
\par
\colorbox[RGB]{253,246,227}{{{{\color[RGB]{101, 123, 131} perm α }}}} is the type of bijections from 
\colorbox[RGB]{253,246,227}{{{{\color[RGB]{101, 123, 131} α }}}} to itself.
\paragraph{equiv.equiv\_sigma\_subtype}
\par
aka coimage
\paragraph{equiv.swap}
\par
\colorbox[RGB]{253,246,227}{{{{\color[RGB]{101, 123, 131} swap a b }}}} is the permutation that swaps 
\colorbox[RGB]{253,246,227}{{{{\color[RGB]{101, 123, 131} a }}}} and 
\colorbox[RGB]{253,246,227}{{{{\color[RGB]{101, 123, 131} b }}}} and
leaves other values as is.
\paragraph{equiv.set\_value}
\par
Augment an equivalence with a prescribed mapping 
\colorbox[RGB]{253,246,227}{{{{\color[RGB]{101, 123, 131} f a  }}}{{{\color[RGB]{181, 137, 0} = }}}{{{\color[RGB]{101, 123, 131}  b }}}}\paragraph{quot.congr\_right}
\par
Quotients are congruent on equivalences under equality of their relation.
An alternative is just to use rewriting with 
\colorbox[RGB]{253,246,227}{{{{\color[RGB]{101, 123, 131} eq }}}}, but then computational proofs get stuck.
\section{equiv/denumerable.lean}\paragraph{denumerable}
\par
A denumerable type is one which is (constructively) bijective with ℕ.
Although we already have a name for this property, namely 
\colorbox[RGB]{253,246,227}{{{{\color[RGB]{101, 123, 131} α ≃ ℕ }}}},
we are here interested in using it as a typeclass.
\section{equiv/encodable.lean}\paragraph{encodable}
\par
An encodable type is a "constructively countable" type. This is where
we have an explicit injection 
\colorbox[RGB]{253,246,227}{{{{\color[RGB]{101, 123, 131} encode : α  }}}{{{\color[RGB]{133, 153, 0} → }}}{{{\color[RGB]{101, 123, 131}  nat }}}} and a partial inverse
\colorbox[RGB]{253,246,227}{{{{\color[RGB]{101, 123, 131} decode : nat  }}}{{{\color[RGB]{133, 153, 0} → }}}{{{\color[RGB]{101, 123, 131}  option α }}}}. This makes the range of 
\colorbox[RGB]{253,246,227}{{{{\color[RGB]{101, 123, 131} encode }}}} decidable,
although it is not decidable if 
\colorbox[RGB]{253,246,227}{{{{\color[RGB]{101, 123, 131} α }}}} is finite or not.
\section{equiv/fin.lean}\section{equiv/functor.lean}\section{equiv/list.lean}\section{equiv/nat.lean}\section{erased.lean}\paragraph{erased}
\par
\colorbox[RGB]{253,246,227}{{{{\color[RGB]{101, 123, 131} erased α }}}} is the same as 
\colorbox[RGB]{253,246,227}{{{{\color[RGB]{101, 123, 131} α }}}}, except that the elements
of 
\colorbox[RGB]{253,246,227}{{{{\color[RGB]{101, 123, 131} erased α }}}} are erased in the VM in the same way as types
and proofs. This can be used to track data without storing it
literally.
\section{fin.lean}\paragraph{fin\_zero\_elim}
\par
\colorbox[RGB]{253,246,227}{{{{\color[RGB]{101, 123, 131} fin  }}}{{{\color[RGB]{108, 113, 196} 0 }}}} is empty
\paragraph{fin.last}
\par
The greatest value of 
\colorbox[RGB]{253,246,227}{{{{\color[RGB]{101, 123, 131} fin (n }}}{{{\color[RGB]{181, 137, 0} + }}}{{{\color[RGB]{108, 113, 196} 1 }}}{{{\color[RGB]{101, 123, 131} ) }}}}\paragraph{fin.cast\_lt}
\par
\colorbox[RGB]{253,246,227}{{{{\color[RGB]{101, 123, 131} cast\_lt i h }}}} embeds 
\colorbox[RGB]{253,246,227}{{{{\color[RGB]{101, 123, 131} i }}}} into a 
\colorbox[RGB]{253,246,227}{{{{\color[RGB]{101, 123, 131} fin }}}} where 
\colorbox[RGB]{253,246,227}{{{{\color[RGB]{101, 123, 131} h }}}} proves it belongs into.
\paragraph{fin.cast\_le}
\par
\colorbox[RGB]{253,246,227}{{{{\color[RGB]{101, 123, 131} cast\_le h i }}}} embeds 
\colorbox[RGB]{253,246,227}{{{{\color[RGB]{101, 123, 131} i }}}} into a larger 
\colorbox[RGB]{253,246,227}{{{{\color[RGB]{101, 123, 131} fin }}}} type.
\paragraph{fin.cast}
\par
\colorbox[RGB]{253,246,227}{{{{\color[RGB]{101, 123, 131} cast eq i }}}} embeds 
\colorbox[RGB]{253,246,227}{{{{\color[RGB]{101, 123, 131} i }}}} into a equal 
\colorbox[RGB]{253,246,227}{{{{\color[RGB]{101, 123, 131} fin }}}} type.
\paragraph{fin.cast\_add}
\par
\colorbox[RGB]{253,246,227}{{{{\color[RGB]{101, 123, 131} cast\_add m i }}}} embedds 
\colorbox[RGB]{253,246,227}{{{{\color[RGB]{101, 123, 131} i }}}} in 
\colorbox[RGB]{253,246,227}{{{{\color[RGB]{101, 123, 131} fin (n }}}{{{\color[RGB]{181, 137, 0} + }}}{{{\color[RGB]{101, 123, 131} m) }}}}.
\paragraph{fin.cast\_succ}
\par
\colorbox[RGB]{253,246,227}{{{{\color[RGB]{101, 123, 131} cast\_succ i }}}} embedds 
\colorbox[RGB]{253,246,227}{{{{\color[RGB]{101, 123, 131} i }}}} in 
\colorbox[RGB]{253,246,227}{{{{\color[RGB]{101, 123, 131} fin (n }}}{{{\color[RGB]{181, 137, 0} + }}}{{{\color[RGB]{108, 113, 196} 1 }}}{{{\color[RGB]{101, 123, 131} ) }}}}.
\paragraph{fin.succ\_above}
\par
\colorbox[RGB]{253,246,227}{{{{\color[RGB]{101, 123, 131} succ\_above p i }}}} embeds into 
\colorbox[RGB]{253,246,227}{{{{\color[RGB]{101, 123, 131} fin (n  }}}{{{\color[RGB]{181, 137, 0} + }}}{{{\color[RGB]{101, 123, 131}   }}}{{{\color[RGB]{108, 113, 196} 1 }}}{{{\color[RGB]{101, 123, 131} ) }}}} with a hole around 
\colorbox[RGB]{253,246,227}{{{{\color[RGB]{101, 123, 131} p }}}}.
\paragraph{fin.pred\_above}
\par
\colorbox[RGB]{253,246,227}{{{{\color[RGB]{101, 123, 131} pred\_above p i h }}}} embeds 
\colorbox[RGB]{253,246,227}{{{{\color[RGB]{101, 123, 131} i }}}} into 
\colorbox[RGB]{253,246,227}{{{{\color[RGB]{101, 123, 131} fin n }}}} by ignoring 
\colorbox[RGB]{253,246,227}{{{{\color[RGB]{101, 123, 131} p }}}}.
\paragraph{fin.sub\_nat}
\par
\colorbox[RGB]{253,246,227}{{{{\color[RGB]{101, 123, 131} sub\_nat i h }}}} subtracts 
\colorbox[RGB]{253,246,227}{{{{\color[RGB]{101, 123, 131} m }}}} from 
\colorbox[RGB]{253,246,227}{{{{\color[RGB]{101, 123, 131} i }}}}, generalizes 
\colorbox[RGB]{253,246,227}{{{{\color[RGB]{101, 123, 131} fin.pred }}}}.
\paragraph{fin.add\_nat}
\par
\colorbox[RGB]{253,246,227}{{{{\color[RGB]{101, 123, 131} add\_nat i h }}}} adds 
\colorbox[RGB]{253,246,227}{{{{\color[RGB]{101, 123, 131} m }}}} on 
\colorbox[RGB]{253,246,227}{{{{\color[RGB]{101, 123, 131} i }}}}, generalizes 
\colorbox[RGB]{253,246,227}{{{{\color[RGB]{101, 123, 131} fin.succ }}}}.
\paragraph{fin.nat\_add}
\par
\colorbox[RGB]{253,246,227}{{{{\color[RGB]{101, 123, 131} nat\_add i h }}}} adds 
\colorbox[RGB]{253,246,227}{{{{\color[RGB]{101, 123, 131} n }}}} on 
\colorbox[RGB]{253,246,227}{{{{\color[RGB]{101, 123, 131} i }}}}\section{finmap.lean}\paragraph{multiset.keys}
\par
Multiset of keys of an association multiset.
\paragraph{multiset.nodupkeys}
\par
\colorbox[RGB]{253,246,227}{{{{\color[RGB]{101, 123, 131} nodupkeys s }}}} means that 
\colorbox[RGB]{253,246,227}{{{{\color[RGB]{101, 123, 131} s }}}} has no duplicate keys.
\paragraph{finmap}
\par
\colorbox[RGB]{253,246,227}{{{{\color[RGB]{101, 123, 131} finmap β }}}} is the type of finite maps over a multiset. It is effectively
a quotient of 
\colorbox[RGB]{253,246,227}{{{{\color[RGB]{101, 123, 131} alist β }}}} by permutation of the underlying list.
\paragraph{alist.to\_finmap}
\par
The quotient map from 
\colorbox[RGB]{253,246,227}{{{{\color[RGB]{101, 123, 131} alist }}}} to 
\colorbox[RGB]{253,246,227}{{{{\color[RGB]{101, 123, 131} finmap }}}}.
\paragraph{finmap.lift\_on}
\par
Lift a permutation-respecting function on 
\colorbox[RGB]{253,246,227}{{{{\color[RGB]{101, 123, 131} alist }}}} to 
\colorbox[RGB]{253,246,227}{{{{\color[RGB]{101, 123, 131} finmap }}}}.
\paragraph{finmap.lift\_on₂}
\par
Lift a permutation-respecting function on 2 
\colorbox[RGB]{253,246,227}{{{{\color[RGB]{101, 123, 131} alist }}}}s to 2 
\colorbox[RGB]{253,246,227}{{{{\color[RGB]{101, 123, 131} finmap }}}}s.
\paragraph{finmap.has\_mem}
\par
The predicate 
\colorbox[RGB]{253,246,227}{{{{\color[RGB]{101, 123, 131} a ∈ s }}}} means that 
\colorbox[RGB]{253,246,227}{{{{\color[RGB]{101, 123, 131} s }}}} has a value associated to the key 
\colorbox[RGB]{253,246,227}{{{{\color[RGB]{101, 123, 131} a }}}}.
\paragraph{finmap.keys}
\par
The set of keys of a finite map.
\paragraph{finmap.has\_emptyc}
\par
The empty map.
\paragraph{finmap.singleton}
\par
The singleton map.
\paragraph{finmap.lookup}
\par
Look up the value associated to a key in a map.
\paragraph{finmap.replace}
\par
Replace a key with a given value in a finite map.
If the key is not present it does nothing.
\paragraph{finmap.foldl}
\par
Fold a commutative function over the key-value pairs in the map
\paragraph{finmap.erase}
\par
Erase a key from the map. If the key is not present it does nothing.
\paragraph{finmap.insert}
\par
Insert a key-value pair into a finite map, replacing any existing pair with
the same key.
\paragraph{finmap.extract}
\par
Erase a key from the map, and return the corresponding value, if found.
\paragraph{finmap.union}
\par
\colorbox[RGB]{253,246,227}{{{{\color[RGB]{101, 123, 131} s₁ ∪ s₂ }}}} is the key-based union of two finite maps. It is left-biased: if
there exists an 
\colorbox[RGB]{253,246,227}{{{{\color[RGB]{101, 123, 131} a ∈ s₁ }}}}, 
\colorbox[RGB]{253,246,227}{{{{\color[RGB]{101, 123, 131} lookup a (s₁ ∪ s₂)  }}}{{{\color[RGB]{181, 137, 0} = }}}{{{\color[RGB]{101, 123, 131}  lookup a s₁ }}}}.
\section{finset.lean}\paragraph{finset}
\par
\colorbox[RGB]{253,246,227}{{{{\color[RGB]{101, 123, 131} finset α }}}} is the type of finite sets of elements of 
\colorbox[RGB]{253,246,227}{{{{\color[RGB]{101, 123, 131} α }}}}. It is implemented
as a multiset (a list up to permutation) which has no duplicate elements.
\paragraph{finset.to\_set}
\par
Convert a finset to a set in the natural way.
\paragraph{finset.singleton}
\par
\colorbox[RGB]{253,246,227}{{{{\color[RGB]{101, 123, 131} singleton a }}}} is the set 
\colorbox[RGB]{253,246,227}{{{{\color[RGB]{101, 123, 131} \{a\} }}}} containing 
\colorbox[RGB]{253,246,227}{{{{\color[RGB]{101, 123, 131} a }}}} and nothing else.
\paragraph{finset.has\_insert}
\par
\colorbox[RGB]{253,246,227}{{{{\color[RGB]{101, 123, 131} insert a s }}}} is the set 
\colorbox[RGB]{253,246,227}{{{{\color[RGB]{101, 123, 131} \{a\} ∪ s }}}} containing 
\colorbox[RGB]{253,246,227}{{{{\color[RGB]{101, 123, 131} a }}}} and the elements of 
\colorbox[RGB]{253,246,227}{{{{\color[RGB]{101, 123, 131} s }}}}.
\paragraph{finset.has\_union}
\par
\colorbox[RGB]{253,246,227}{{{{\color[RGB]{101, 123, 131} s ∪ t }}}} is the set such that 
\colorbox[RGB]{253,246,227}{{{{\color[RGB]{101, 123, 131} a ∈ s ∪ t }}}} iff 
\colorbox[RGB]{253,246,227}{{{{\color[RGB]{101, 123, 131} a ∈ s }}}} or 
\colorbox[RGB]{253,246,227}{{{{\color[RGB]{101, 123, 131} a ∈ t }}}}.
\paragraph{finset.has\_inter}
\par
\colorbox[RGB]{253,246,227}{{{{\color[RGB]{101, 123, 131} s ∩ t }}}} is the set such that 
\colorbox[RGB]{253,246,227}{{{{\color[RGB]{101, 123, 131} a ∈ s ∩ t }}}} iff 
\colorbox[RGB]{253,246,227}{{{{\color[RGB]{101, 123, 131} a ∈ s }}}} and 
\colorbox[RGB]{253,246,227}{{{{\color[RGB]{101, 123, 131} a ∈ t }}}}.
\paragraph{finset.erase}
\par
\colorbox[RGB]{253,246,227}{{{{\color[RGB]{101, 123, 131} erase s a }}}} is the set 
\colorbox[RGB]{253,246,227}{{{{\color[RGB]{101, 123, 131} s  }}}{{{\color[RGB]{181, 137, 0} - }}}{{{\color[RGB]{101, 123, 131}  \{a\} }}}}, that is, the elements of 
\colorbox[RGB]{253,246,227}{{{{\color[RGB]{101, 123, 131} s }}}} which are
not equal to 
\colorbox[RGB]{253,246,227}{{{{\color[RGB]{101, 123, 131} a }}}}.
\paragraph{finset.has\_sdiff}
\par
\colorbox[RGB]{253,246,227}{{{{\color[RGB]{101, 123, 131} s \textbackslash{} t }}}} is the set consisting of the elements of 
\colorbox[RGB]{253,246,227}{{{{\color[RGB]{101, 123, 131} s }}}} that are not in 
\colorbox[RGB]{253,246,227}{{{{\color[RGB]{101, 123, 131} t }}}}.
\paragraph{finset.attach}
\par
\colorbox[RGB]{253,246,227}{{{{\color[RGB]{101, 123, 131} attach s }}}} takes the elements of 
\colorbox[RGB]{253,246,227}{{{{\color[RGB]{101, 123, 131} s }}}} and forms a new set of elements of the
subtype 
\colorbox[RGB]{253,246,227}{{{{\color[RGB]{101, 123, 131} \{x  }}}{{{\color[RGB]{181, 137, 0} / }}}{{{\color[RGB]{181, 137, 0} / }}}{{{\color[RGB]{101, 123, 131}  x ∈ s\} }}}}.
\paragraph{finset.decidable\_eq\_pi\_finset}
\par
decidable equality for functions whose domain is bounded by finsets
\paragraph{finset.filter}
\par
\colorbox[RGB]{253,246,227}{{{{\color[RGB]{101, 123, 131} filter p s }}}} is the set of elements of 
\colorbox[RGB]{253,246,227}{{{{\color[RGB]{101, 123, 131} s }}}} that satisfy 
\colorbox[RGB]{253,246,227}{{{{\color[RGB]{101, 123, 131} p }}}}.
\paragraph{finset.range}
\par
\colorbox[RGB]{253,246,227}{{{{\color[RGB]{101, 123, 131} range n }}}} is the set of natural numbers less than 
\colorbox[RGB]{253,246,227}{{{{\color[RGB]{101, 123, 131} n }}}}.
\paragraph{option.to\_finset}
\par
Construct an empty or singleton finset from an 
\colorbox[RGB]{253,246,227}{{{{\color[RGB]{101, 123, 131} option }}}}\paragraph{multiset.to\_finset}
\par
\colorbox[RGB]{253,246,227}{{{{\color[RGB]{101, 123, 131} to\_finset s }}}} removes duplicates from the multiset 
\colorbox[RGB]{253,246,227}{{{{\color[RGB]{101, 123, 131} s }}}} to produce a finset.
\paragraph{list.to\_finset}
\par
\colorbox[RGB]{253,246,227}{{{{\color[RGB]{101, 123, 131} to\_finset l }}}} removes duplicates from the list 
\colorbox[RGB]{253,246,227}{{{{\color[RGB]{101, 123, 131} l }}}} to produce a finset.
\paragraph{finset.image}
\par
\colorbox[RGB]{253,246,227}{{{{\color[RGB]{101, 123, 131} image f s }}}} is the forward image of 
\colorbox[RGB]{253,246,227}{{{{\color[RGB]{101, 123, 131} s }}}} under 
\colorbox[RGB]{253,246,227}{{{{\color[RGB]{101, 123, 131} f }}}}.
\paragraph{finset.card}
\par
\colorbox[RGB]{253,246,227}{{{{\color[RGB]{101, 123, 131} card s }}}} is the cardinality (number of elements) of 
\colorbox[RGB]{253,246,227}{{{{\color[RGB]{101, 123, 131} s }}}}.
\paragraph{finset.bind}
\par
\colorbox[RGB]{253,246,227}{{{{\color[RGB]{101, 123, 131} bind s t }}}} is the union of 
\colorbox[RGB]{253,246,227}{{{{\color[RGB]{101, 123, 131} t x }}}} over 
\colorbox[RGB]{253,246,227}{{{{\color[RGB]{101, 123, 131} x ∈ s }}}}\paragraph{finset.product}
\par
\colorbox[RGB]{253,246,227}{{{{\color[RGB]{101, 123, 131} product s t }}}} is the set of pairs 
\colorbox[RGB]{253,246,227}{{{{\color[RGB]{101, 123, 131} (a, b) }}}} such that 
\colorbox[RGB]{253,246,227}{{{{\color[RGB]{101, 123, 131} a ∈ s }}}} and 
\colorbox[RGB]{253,246,227}{{{{\color[RGB]{101, 123, 131} b ∈ t }}}}.
\paragraph{finset.sigma}
\par
\colorbox[RGB]{253,246,227}{{{{\color[RGB]{101, 123, 131} sigma s t }}}} is the set of dependent pairs 
\colorbox[RGB]{253,246,227}{{{{\color[RGB]{101, 123, 131} ⟨a, b⟩ }}}} such that 
\colorbox[RGB]{253,246,227}{{{{\color[RGB]{101, 123, 131} a ∈ s }}}} and 
\colorbox[RGB]{253,246,227}{{{{\color[RGB]{101, 123, 131} b ∈ t a }}}}.
\paragraph{finset.fold}
\par
\colorbox[RGB]{253,246,227}{{{{\color[RGB]{101, 123, 131} fold op b f s }}}} folds the commutative associative operation 
\colorbox[RGB]{253,246,227}{{{{\color[RGB]{101, 123, 131} op }}}} over the
\colorbox[RGB]{253,246,227}{{{{\color[RGB]{101, 123, 131} f }}}}-image of 
\colorbox[RGB]{253,246,227}{{{{\color[RGB]{101, 123, 131} s }}}}, i.e. 
\colorbox[RGB]{253,246,227}{{{{\color[RGB]{101, 123, 131} fold ( }}}{{{\color[RGB]{181, 137, 0} + }}}{{{\color[RGB]{101, 123, 131} ) b f \{ }}}{{{\color[RGB]{108, 113, 196} 1 }}}{{{\color[RGB]{101, 123, 131} , }}}{{{\color[RGB]{108, 113, 196} 2 }}}{{{\color[RGB]{101, 123, 131} , }}}{{{\color[RGB]{108, 113, 196} 3 }}}{{{\color[RGB]{101, 123, 131} \}  }}}{{{\color[RGB]{181, 137, 0} = }}}{{{\color[RGB]{101, 123, 131}   }}}}f 1 + f 2 + f 3 + b
`
.
\paragraph{finset.sup}
\par
Supremum of a finite set: 
\colorbox[RGB]{253,246,227}{{{{\color[RGB]{101, 123, 131} sup \{a, b, c\} f  }}}{{{\color[RGB]{181, 137, 0} = }}}{{{\color[RGB]{101, 123, 131}  f a ⊔ f b ⊔ f c }}}}\paragraph{finset.inf}
\par
Infimum of a finite set: 
\colorbox[RGB]{253,246,227}{{{{\color[RGB]{101, 123, 131} inf \{a, b, c\} f  }}}{{{\color[RGB]{181, 137, 0} = }}}{{{\color[RGB]{101, 123, 131}  f a ⊓ f b ⊓ f c }}}}\paragraph{finset.sort}
\par
\colorbox[RGB]{253,246,227}{{{{\color[RGB]{101, 123, 131} sort s }}}} constructs a sorted list from the unordered set 
\colorbox[RGB]{253,246,227}{{{{\color[RGB]{101, 123, 131} s }}}}.
(Uses merge sort algorithm.)
\paragraph{finset.Ico}
\par
\colorbox[RGB]{253,246,227}{{{{\color[RGB]{101, 123, 131} Ico n m }}}} is the set of natural numbers 
\colorbox[RGB]{253,246,227}{{{{\color[RGB]{101, 123, 131} n  }}}{{{\color[RGB]{181, 137, 0} ≤ }}}{{{\color[RGB]{101, 123, 131}  k  }}}{{{\color[RGB]{181, 137, 0} < }}}{{{\color[RGB]{101, 123, 131}  m }}}}.
\section{finsupp.lean}\paragraph{finsupp}
\par
\colorbox[RGB]{253,246,227}{{{{\color[RGB]{101, 123, 131} finsupp α β }}}}, denoted 
\colorbox[RGB]{253,246,227}{{{{\color[RGB]{101, 123, 131} α  }}}{{{\color[RGB]{133, 153, 0} → }}}{{{\color[RGB]{101, 123, 131} ₀ β }}}}, is the type of functions 
\colorbox[RGB]{253,246,227}{{{{\color[RGB]{101, 123, 131} f : α  }}}{{{\color[RGB]{133, 153, 0} → }}}{{{\color[RGB]{101, 123, 131}  β }}}} such that
\colorbox[RGB]{253,246,227}{{{{\color[RGB]{101, 123, 131} f x  }}}{{{\color[RGB]{181, 137, 0} = }}}{{{\color[RGB]{101, 123, 131}   }}}{{{\color[RGB]{108, 113, 196} 0 }}}} for all but finitely many 
\colorbox[RGB]{253,246,227}{{{{\color[RGB]{101, 123, 131} x }}}}.
\paragraph{finsupp.single}
\par
\colorbox[RGB]{253,246,227}{{{{\color[RGB]{101, 123, 131} single a b }}}} is the finitely supported function which has
value 
\colorbox[RGB]{253,246,227}{{{{\color[RGB]{101, 123, 131} b }}}} at 
\colorbox[RGB]{253,246,227}{{{{\color[RGB]{101, 123, 131} a }}}} and zero otherwise.
\paragraph{finsupp.on\_finset}
\par
\colorbox[RGB]{253,246,227}{{{{\color[RGB]{101, 123, 131} on\_finset s f hf }}}} is the finsupp function representing 
\colorbox[RGB]{253,246,227}{{{{\color[RGB]{101, 123, 131} f }}}} restricted to the set 
\colorbox[RGB]{253,246,227}{{{{\color[RGB]{101, 123, 131} s }}}}.
The function needs to be 0 outside of 
\colorbox[RGB]{253,246,227}{{{{\color[RGB]{101, 123, 131} s }}}}. Use this when the set needs filtered anyway, otherwise
often better set representation is available.
\paragraph{finsupp.map\_range}
\par
The composition of 
\colorbox[RGB]{253,246,227}{{{{\color[RGB]{101, 123, 131} f : β₁  }}}{{{\color[RGB]{133, 153, 0} → }}}{{{\color[RGB]{101, 123, 131}  β₂ }}}} and 
\colorbox[RGB]{253,246,227}{{{{\color[RGB]{101, 123, 131} g : α  }}}{{{\color[RGB]{133, 153, 0} → }}}{{{\color[RGB]{101, 123, 131} ₀ β₁ }}}} is
\colorbox[RGB]{253,246,227}{{{{\color[RGB]{101, 123, 131} map\_range f hf g : α  }}}{{{\color[RGB]{133, 153, 0} → }}}{{{\color[RGB]{101, 123, 131} ₀ β₂ }}}}, well defined when 
\colorbox[RGB]{253,246,227}{{{{\color[RGB]{101, 123, 131} f  }}}{{{\color[RGB]{108, 113, 196} 0 }}}{{{\color[RGB]{101, 123, 131}   }}}{{{\color[RGB]{181, 137, 0} = }}}{{{\color[RGB]{101, 123, 131}   }}}{{{\color[RGB]{108, 113, 196} 0 }}}}.
\paragraph{finsupp.emb\_domain}
\par
Given 
\colorbox[RGB]{253,246,227}{{{{\color[RGB]{101, 123, 131} f : α₁ ↪ α₂ }}}} and 
\colorbox[RGB]{253,246,227}{{{{\color[RGB]{101, 123, 131} v : α₁  }}}{{{\color[RGB]{133, 153, 0} → }}}{{{\color[RGB]{101, 123, 131} ₀ β }}}}, 
\colorbox[RGB]{253,246,227}{{{{\color[RGB]{101, 123, 131} emb\_domain f v : α₂  }}}{{{\color[RGB]{133, 153, 0} → }}}{{{\color[RGB]{101, 123, 131} ₀ β }}}} is the finitely supported
function whose value at 
\colorbox[RGB]{253,246,227}{{{{\color[RGB]{101, 123, 131} f a : α₂ }}}} is 
\colorbox[RGB]{253,246,227}{{{{\color[RGB]{101, 123, 131} v a }}}}. For a 
\colorbox[RGB]{253,246,227}{{{{\color[RGB]{101, 123, 131} b : α₂ }}}} outside the range of 
\colorbox[RGB]{253,246,227}{{{{\color[RGB]{101, 123, 131} f }}}} it is zero.
\paragraph{finsupp.zip\_with}
\par
\colorbox[RGB]{253,246,227}{{{{\color[RGB]{101, 123, 131} zip\_with f hf g₁ g₂ }}}} is the finitely supported function satisfying
\colorbox[RGB]{253,246,227}{{{{\color[RGB]{101, 123, 131} zip\_with f hf g₁ g₂ a  }}}{{{\color[RGB]{181, 137, 0} = }}}{{{\color[RGB]{101, 123, 131}  f (g₁ a) (g₂ a) }}}}, and well defined when 
\colorbox[RGB]{253,246,227}{{{{\color[RGB]{101, 123, 131} f  }}}{{{\color[RGB]{108, 113, 196} 0 }}}{{{\color[RGB]{101, 123, 131}   }}}{{{\color[RGB]{108, 113, 196} 0 }}}{{{\color[RGB]{101, 123, 131}   }}}{{{\color[RGB]{181, 137, 0} = }}}{{{\color[RGB]{101, 123, 131}   }}}{{{\color[RGB]{108, 113, 196} 0 }}}}.
\paragraph{finsupp.sum}
\par
\colorbox[RGB]{253,246,227}{{{{\color[RGB]{101, 123, 131} sum f g }}}} is the sum of 
\colorbox[RGB]{253,246,227}{{{{\color[RGB]{101, 123, 131} g a (f a) }}}} over the support of 
\colorbox[RGB]{253,246,227}{{{{\color[RGB]{101, 123, 131} f }}}}.
\paragraph{finsupp.prod}
\par
\colorbox[RGB]{253,246,227}{{{{\color[RGB]{101, 123, 131} prod f g }}}} is the product of 
\colorbox[RGB]{253,246,227}{{{{\color[RGB]{101, 123, 131} g a (f a) }}}} over the support of 
\colorbox[RGB]{253,246,227}{{{{\color[RGB]{101, 123, 131} f }}}}.
\paragraph{finsupp.map\_domain}
\par
Given 
\colorbox[RGB]{253,246,227}{{{{\color[RGB]{101, 123, 131} f : α₁  }}}{{{\color[RGB]{133, 153, 0} → }}}{{{\color[RGB]{101, 123, 131}  α₂ }}}} and 
\colorbox[RGB]{253,246,227}{{{{\color[RGB]{101, 123, 131} v : α₁  }}}{{{\color[RGB]{133, 153, 0} → }}}{{{\color[RGB]{101, 123, 131} ₀ β }}}}, 
\colorbox[RGB]{253,246,227}{{{{\color[RGB]{101, 123, 131} map\_domain f v : α₂  }}}{{{\color[RGB]{133, 153, 0} → }}}{{{\color[RGB]{101, 123, 131} ₀ β }}}}is the finitely supported function whose value at 
\colorbox[RGB]{253,246,227}{{{{\color[RGB]{101, 123, 131} a : α₂ }}}} is the sum
of 
\colorbox[RGB]{253,246,227}{{{{\color[RGB]{101, 123, 131} v x }}}} over all 
\colorbox[RGB]{253,246,227}{{{{\color[RGB]{101, 123, 131} x }}}} such that 
\colorbox[RGB]{253,246,227}{{{{\color[RGB]{101, 123, 131} f x  }}}{{{\color[RGB]{181, 137, 0} = }}}{{{\color[RGB]{101, 123, 131}  a }}}}.
\paragraph{finsupp.has\_mul}
\par
The product of 
\colorbox[RGB]{253,246,227}{{{{\color[RGB]{101, 123, 131} f g : α  }}}{{{\color[RGB]{133, 153, 0} → }}}{{{\color[RGB]{101, 123, 131} ₀ β }}}} is the finitely supported function
whose value at 
\colorbox[RGB]{253,246,227}{{{{\color[RGB]{101, 123, 131} a }}}} is the sum of 
\colorbox[RGB]{253,246,227}{{{{\color[RGB]{101, 123, 131} f x  }}}{{{\color[RGB]{181, 137, 0} * }}}{{{\color[RGB]{101, 123, 131}  g y }}}} over all pairs 
\colorbox[RGB]{253,246,227}{{{{\color[RGB]{101, 123, 131} x, y }}}}such that 
\colorbox[RGB]{253,246,227}{{{{\color[RGB]{101, 123, 131} x  }}}{{{\color[RGB]{181, 137, 0} + }}}{{{\color[RGB]{101, 123, 131}  y  }}}{{{\color[RGB]{181, 137, 0} = }}}{{{\color[RGB]{101, 123, 131}  a }}}}. (Think of the product of multivariate
polynomials where 
\colorbox[RGB]{253,246,227}{{{{\color[RGB]{101, 123, 131} α }}}} is the monoid of monomial exponents.)
\paragraph{finsupp.has\_one}
\par
The unit of the multiplication is 
\colorbox[RGB]{253,246,227}{{{{\color[RGB]{101, 123, 131} single  }}}{{{\color[RGB]{108, 113, 196} 0 }}}{{{\color[RGB]{101, 123, 131}   }}}{{{\color[RGB]{108, 113, 196} 1 }}}}, i.e. the function
that is 1 at 0 and zero elsewhere.
\paragraph{finsupp.filter}
\par
\colorbox[RGB]{253,246,227}{{{{\color[RGB]{101, 123, 131} filter p f }}}} is the function which is 
\colorbox[RGB]{253,246,227}{{{{\color[RGB]{101, 123, 131} f a }}}} if 
\colorbox[RGB]{253,246,227}{{{{\color[RGB]{101, 123, 131} p a }}}} is true and 0 otherwise.
\paragraph{finsupp.subtype\_domain}
\par
\colorbox[RGB]{253,246,227}{{{{\color[RGB]{101, 123, 131} subtype\_domain p f }}}} is the restriction of the finitely supported function
\colorbox[RGB]{253,246,227}{{{{\color[RGB]{101, 123, 131} f }}}} to the subtype 
\colorbox[RGB]{253,246,227}{{{{\color[RGB]{101, 123, 131} p }}}}.
\section{fintype.lean}\paragraph{fintype}
\par
\colorbox[RGB]{253,246,227}{{{{\color[RGB]{101, 123, 131} fintype α }}}} means that 
\colorbox[RGB]{253,246,227}{{{{\color[RGB]{101, 123, 131} α }}}} is finite, i.e. there are only
finitely many distinct elements of type 
\colorbox[RGB]{253,246,227}{{{{\color[RGB]{101, 123, 131} α }}}}. The evidence of this
is a finset 
\colorbox[RGB]{253,246,227}{{{{\color[RGB]{101, 123, 131} elems }}}} (a list up to permutation without duplicates),
together with a proof that everything of type 
\colorbox[RGB]{253,246,227}{{{{\color[RGB]{101, 123, 131} α }}}} is in the list.
\paragraph{finset.univ}
\par
\colorbox[RGB]{253,246,227}{{{{\color[RGB]{101, 123, 131} univ }}}} is the universal finite set of type 
\colorbox[RGB]{253,246,227}{{{{\color[RGB]{101, 123, 131} finset α }}}} implied from
the assumption 
\colorbox[RGB]{253,246,227}{{{{\color[RGB]{101, 123, 131} fintype α }}}}.
\paragraph{fintype.of\_multiset}
\par
Construct a proof of 
\colorbox[RGB]{253,246,227}{{{{\color[RGB]{101, 123, 131} fintype α }}}} from a universal multiset
\paragraph{fintype.of\_list}
\par
Construct a proof of 
\colorbox[RGB]{253,246,227}{{{{\color[RGB]{101, 123, 131} fintype α }}}} from a universal list
\paragraph{fintype.card}
\par
\colorbox[RGB]{253,246,227}{{{{\color[RGB]{101, 123, 131} card α }}}} is the number of elements in 
\colorbox[RGB]{253,246,227}{{{{\color[RGB]{101, 123, 131} α }}}}, defined when 
\colorbox[RGB]{253,246,227}{{{{\color[RGB]{101, 123, 131} α }}}} is a fintype.
\paragraph{fintype.equiv\_fin}
\par
There is (computably) a bijection between 
\colorbox[RGB]{253,246,227}{{{{\color[RGB]{101, 123, 131} α }}}} and 
\colorbox[RGB]{253,246,227}{{{{\color[RGB]{101, 123, 131} fin n }}}} where
\colorbox[RGB]{253,246,227}{{{{\color[RGB]{101, 123, 131} n  }}}{{{\color[RGB]{181, 137, 0} = }}}{{{\color[RGB]{101, 123, 131}  card α }}}}. Since it is not unique, and depends on which permutation
of the universe list is used, the bijection is wrapped in 
\colorbox[RGB]{253,246,227}{{{{\color[RGB]{101, 123, 131} trunc }}}} to
preserve computability.
\paragraph{fintype.of\_bijective}
\par
If 
\colorbox[RGB]{253,246,227}{{{{\color[RGB]{101, 123, 131} f : α  }}}{{{\color[RGB]{133, 153, 0} → }}}{{{\color[RGB]{101, 123, 131}  β }}}} is a bijection and 
\colorbox[RGB]{253,246,227}{{{{\color[RGB]{101, 123, 131} α }}}} is a fintype, then 
\colorbox[RGB]{253,246,227}{{{{\color[RGB]{101, 123, 131} β }}}} is also a fintype.
\paragraph{fintype.of\_surjective}
\par
If 
\colorbox[RGB]{253,246,227}{{{{\color[RGB]{101, 123, 131} f : α  }}}{{{\color[RGB]{133, 153, 0} → }}}{{{\color[RGB]{101, 123, 131}  β }}}} is a surjection and 
\colorbox[RGB]{253,246,227}{{{{\color[RGB]{101, 123, 131} α }}}} is a fintype, then 
\colorbox[RGB]{253,246,227}{{{{\color[RGB]{101, 123, 131} β }}}} is also a fintype.
\paragraph{fintype.of\_equiv}
\par
If 
\colorbox[RGB]{253,246,227}{{{{\color[RGB]{101, 123, 131} f : α ≃ β }}}} and 
\colorbox[RGB]{253,246,227}{{{{\color[RGB]{101, 123, 131} α }}}} is a fintype, then 
\colorbox[RGB]{253,246,227}{{{{\color[RGB]{101, 123, 131} β }}}} is also a fintype.
\paragraph{fintype.bij\_inv}
\par
\colorbox[RGB]{253,246,227}{}bij\_inv f
\colorbox[RGB]{253,246,227}{{{{\color[RGB]{101, 123, 131} is the unique inverse to a bijection }}}}f
\colorbox[RGB]{253,246,227}{{{{\color[RGB]{101, 123, 131} . This acts  }}}{{{\color[RGB]{133, 153, 0} as }}}{{{\color[RGB]{101, 123, 131}  a computable alternative to  }}}}function.inv\_fun
`
.
\section{fp/basic.lean}\section{hash\_map.lean}\paragraph{bucket\_array}
\par
\colorbox[RGB]{253,246,227}{{{{\color[RGB]{101, 123, 131} bucket\_array α β }}}} is the underlying data type for 
\colorbox[RGB]{253,246,227}{{{{\color[RGB]{101, 123, 131} hash\_map α β }}}},
an array of linked lists of key-value pairs.
\paragraph{hash\_map.mk\_idx}
\par
Make a hash\_map index from a 
\colorbox[RGB]{253,246,227}{{{{\color[RGB]{101, 123, 131} nat }}}} hash value and a (positive) buffer size
\paragraph{bucket\_array.read}
\par
Read the bucket corresponding to an element
\paragraph{bucket\_array.write}
\par
Write the bucket corresponding to an element
\paragraph{bucket\_array.modify}
\par
Modify (read, apply 
\colorbox[RGB]{253,246,227}{{{{\color[RGB]{101, 123, 131} f }}}}, and write) the bucket corresponding to an element
\paragraph{bucket\_array.as\_list}
\par
The list of all key-value pairs in the bucket list
\paragraph{bucket\_array.foldl}
\par
Fold a function 
\colorbox[RGB]{253,246,227}{{{{\color[RGB]{101, 123, 131} f }}}} over the key-value pairs in the bucket list
\paragraph{hash\_map.reinsert\_aux}
\par
Insert the pair 
\colorbox[RGB]{253,246,227}{{{{\color[RGB]{101, 123, 131} ⟨a, b⟩ }}}} into the correct location in the bucket array
(without checking for duplication)
\paragraph{hash\_map.find\_aux}
\par
Search a bucket for a key 
\colorbox[RGB]{253,246,227}{{{{\color[RGB]{101, 123, 131} a }}}} and return the value
\paragraph{hash\_map.contains\_aux}
\par
Returns 
\colorbox[RGB]{253,246,227}{{{{\color[RGB]{101, 123, 131} tt }}}} if the bucket 
\colorbox[RGB]{253,246,227}{{{{\color[RGB]{101, 123, 131} l }}}} contains the key 
\colorbox[RGB]{253,246,227}{{{{\color[RGB]{101, 123, 131} a }}}}\paragraph{hash\_map.replace\_aux}
\par
Modify a bucket to replace a value in the list. Leaves the list
unchanged if the key is not found.
\paragraph{hash\_map.erase\_aux}
\par
Modify a bucket to remove a key, if it exists.
\paragraph{hash\_map.valid}
\par
The predicate 
\colorbox[RGB]{253,246,227}{{{{\color[RGB]{101, 123, 131} valid bkts sz }}}} means that 
\colorbox[RGB]{253,246,227}{{{{\color[RGB]{101, 123, 131} bkts }}}} satisfies the 
\colorbox[RGB]{253,246,227}{{{{\color[RGB]{101, 123, 131} hash\_map }}}}invariants: There are exactly 
\colorbox[RGB]{253,246,227}{{{{\color[RGB]{101, 123, 131} sz }}}} elements in it, every pair is in the
bucket determined by its key and the hash function, and no key appears
multiple times in the list.
\paragraph{hash\_map}
\par
A hash map data structure, representing a finite key-value map
with key type 
\colorbox[RGB]{253,246,227}{{{{\color[RGB]{101, 123, 131} α }}}} and value type 
\colorbox[RGB]{253,246,227}{{{{\color[RGB]{101, 123, 131} β }}}} (which may depend on 
\colorbox[RGB]{253,246,227}{{{{\color[RGB]{101, 123, 131} α }}}}).
\paragraph{mk\_hash\_map}
\par
Construct an empty hash map with buffer size 
\colorbox[RGB]{253,246,227}{{{{\color[RGB]{101, 123, 131} nbuckets }}}} (default 8).
\paragraph{hash\_map.find}
\par
Return the value corresponding to a key, or 
\colorbox[RGB]{253,246,227}{{{{\color[RGB]{101, 123, 131} none }}}} if not found
\paragraph{hash\_map.contains}
\par
Return 
\colorbox[RGB]{253,246,227}{{{{\color[RGB]{101, 123, 131} tt }}}} if the key exists in the map
\paragraph{hash\_map.fold}
\par
Fold a function over the key-value pairs in the map
\paragraph{hash\_map.entries}
\par
The list of key-value pairs in the map
\paragraph{hash\_map.keys}
\par
The list of keys in the map
\paragraph{hash\_map.insert}
\par
Insert a key-value pair into the map. (Modifies 
\colorbox[RGB]{253,246,227}{{{{\color[RGB]{101, 123, 131} m }}}} in-place when applicable)
\paragraph{hash\_map.insert\_all}
\par
Insert a list of key-value pairs into the map. (Modifies 
\colorbox[RGB]{253,246,227}{{{{\color[RGB]{101, 123, 131} m }}}} in-place when applicable)
\paragraph{hash\_map.of\_list}
\par
Construct a hash map from a list of key-value pairs.
\paragraph{hash\_map.erase}
\par
Remove a key from the map. (Modifies 
\colorbox[RGB]{253,246,227}{{{{\color[RGB]{101, 123, 131} m }}}} in-place when applicable)
\section{holor.lean}\paragraph{holor\_index}
\par
\colorbox[RGB]{253,246,227}{{{{\color[RGB]{101, 123, 131} holor\_index ds }}}} is the type of valid index tuples to identify an entry of a holor of dimenstions 
\colorbox[RGB]{253,246,227}{{{{\color[RGB]{101, 123, 131} ds }}}}\paragraph{holor}
\par
Holor (indexed collections of tensor coefficients)
\paragraph{holor.slice}
\par
A slice is a subholor consisting of all entries with initial index i.
\paragraph{holor.slice\_eq}
\par
Two holors are equal if all their slices are equal.
\paragraph{holor.sum\_unit\_vec\_mul\_slice}
\par
The original holor can be recovered from its slices by multiplying with unit vectors and summing up.
\section{int/basic.lean}\paragraph{int.succ}
\par
Immediate successor of an integer: 
\colorbox[RGB]{253,246,227}{{{{\color[RGB]{101, 123, 131} succ n  }}}{{{\color[RGB]{181, 137, 0} = }}}{{{\color[RGB]{101, 123, 131}  n  }}}{{{\color[RGB]{181, 137, 0} + }}}{{{\color[RGB]{101, 123, 131}   }}}{{{\color[RGB]{108, 113, 196} 1 }}}}\paragraph{int.pred}
\par
Immediate predecessor of an integer: 
\colorbox[RGB]{253,246,227}{{{{\color[RGB]{101, 123, 131} pred n  }}}{{{\color[RGB]{181, 137, 0} = }}}{{{\color[RGB]{101, 123, 131}  n  }}}{{{\color[RGB]{181, 137, 0} - }}}{{{\color[RGB]{101, 123, 131}   }}}{{{\color[RGB]{108, 113, 196} 1 }}}}\paragraph{int.cast}
\par
Canonical homomorphism from the integers to any ring(-like) structure 
\colorbox[RGB]{253,246,227}{{{{\color[RGB]{101, 123, 131} α }}}}\section{int/gcd.lean}\paragraph{nat.xgcd}
\par
Use the extended GCD algorithm to generate the 
\colorbox[RGB]{253,246,227}{{{{\color[RGB]{101, 123, 131} a }}}} and 
\colorbox[RGB]{253,246,227}{{{{\color[RGB]{101, 123, 131} b }}}} values
satisfying 
\colorbox[RGB]{253,246,227}{{{{\color[RGB]{101, 123, 131} gcd x y  }}}{{{\color[RGB]{181, 137, 0} = }}}{{{\color[RGB]{101, 123, 131}  x  }}}{{{\color[RGB]{181, 137, 0} * }}}{{{\color[RGB]{101, 123, 131}  a  }}}{{{\color[RGB]{181, 137, 0} + }}}{{{\color[RGB]{101, 123, 131}  y  }}}{{{\color[RGB]{181, 137, 0} * }}}{{{\color[RGB]{101, 123, 131}  b }}}}.
\paragraph{nat.gcd\_a}
\par
The extended GCD 
\colorbox[RGB]{253,246,227}{{{{\color[RGB]{101, 123, 131} a }}}} value in the equation 
\colorbox[RGB]{253,246,227}{{{{\color[RGB]{101, 123, 131} gcd x y  }}}{{{\color[RGB]{181, 137, 0} = }}}{{{\color[RGB]{101, 123, 131}  x  }}}{{{\color[RGB]{181, 137, 0} * }}}{{{\color[RGB]{101, 123, 131}  a  }}}{{{\color[RGB]{181, 137, 0} + }}}{{{\color[RGB]{101, 123, 131}  y  }}}{{{\color[RGB]{181, 137, 0} * }}}{{{\color[RGB]{101, 123, 131}  b }}}}.
\paragraph{nat.gcd\_b}
\par
The extended GCD 
\colorbox[RGB]{253,246,227}{{{{\color[RGB]{101, 123, 131} b }}}} value in the equation 
\colorbox[RGB]{253,246,227}{{{{\color[RGB]{101, 123, 131} gcd x y  }}}{{{\color[RGB]{181, 137, 0} = }}}{{{\color[RGB]{101, 123, 131}  x  }}}{{{\color[RGB]{181, 137, 0} * }}}{{{\color[RGB]{101, 123, 131}  a  }}}{{{\color[RGB]{181, 137, 0} + }}}{{{\color[RGB]{101, 123, 131}  y  }}}{{{\color[RGB]{181, 137, 0} * }}}{{{\color[RGB]{101, 123, 131}  b }}}}.
\section{int/modeq.lean}\section{int/sqrt.lean}\paragraph{int.sqrt}
\par
\colorbox[RGB]{253,246,227}{{{{\color[RGB]{101, 123, 131} sqrt n }}}} is the square root of an integer 
\colorbox[RGB]{253,246,227}{{{{\color[RGB]{101, 123, 131} n }}}}. If 
\colorbox[RGB]{253,246,227}{{{{\color[RGB]{101, 123, 131} n }}}} is not a
perfect square, and is positive, it returns the largest 
\colorbox[RGB]{253,246,227}{{{{\color[RGB]{101, 123, 131} k:ℤ }}}} such
that 
\colorbox[RGB]{253,246,227}{{{{\color[RGB]{101, 123, 131} k }}}{{{\color[RGB]{181, 137, 0} * }}}{{{\color[RGB]{101, 123, 131} k  }}}{{{\color[RGB]{181, 137, 0} ≤ }}}{{{\color[RGB]{101, 123, 131}  n }}}}. If it is negative, it returns 0. For example,
\colorbox[RGB]{253,246,227}{{{{\color[RGB]{101, 123, 131} sqrt  }}}{{{\color[RGB]{108, 113, 196} 2 }}}{{{\color[RGB]{101, 123, 131}   }}}{{{\color[RGB]{181, 137, 0} = }}}{{{\color[RGB]{101, 123, 131}   }}}{{{\color[RGB]{108, 113, 196} 1 }}}} and 
\colorbox[RGB]{253,246,227}{{{{\color[RGB]{101, 123, 131} sqrt  }}}{{{\color[RGB]{108, 113, 196} 1 }}}{{{\color[RGB]{101, 123, 131}   }}}{{{\color[RGB]{181, 137, 0} = }}}{{{\color[RGB]{101, 123, 131}   }}}{{{\color[RGB]{108, 113, 196} 1 }}}} and 
\colorbox[RGB]{253,246,227}{{{{\color[RGB]{101, 123, 131} sqrt ( }}}{{{\color[RGB]{181, 137, 0} - }}}{{{\color[RGB]{108, 113, 196} 1 }}}{{{\color[RGB]{101, 123, 131} )  }}}{{{\color[RGB]{181, 137, 0} = }}}{{{\color[RGB]{101, 123, 131}   }}}{{{\color[RGB]{108, 113, 196} 0 }}}}\section{lazy\_list2.lean}\section{list/alist.lean}\paragraph{alist}
\par
\colorbox[RGB]{253,246,227}{{{{\color[RGB]{101, 123, 131} alist β }}}} is a key-value map stored as a 
\colorbox[RGB]{253,246,227}{{{{\color[RGB]{101, 123, 131} list }}}} (i.e. a linked list).
It is a wrapper around certain 
\colorbox[RGB]{253,246,227}{{{{\color[RGB]{101, 123, 131} list }}}} functions with the added constraint
that the list have unique keys.
\paragraph{alist.keys}
\par
The list of keys of an association list.
\paragraph{alist.has\_mem}
\par
The predicate 
\colorbox[RGB]{253,246,227}{{{{\color[RGB]{101, 123, 131} a ∈ s }}}} means that 
\colorbox[RGB]{253,246,227}{{{{\color[RGB]{101, 123, 131} s }}}} has a value associated to the key 
\colorbox[RGB]{253,246,227}{{{{\color[RGB]{101, 123, 131} a }}}}.
\paragraph{alist.has\_emptyc}
\par
The empty association list.
\paragraph{alist.singleton}
\par
The singleton association list.
\paragraph{alist.lookup}
\par
Look up the value associated to a key in an association list.
\paragraph{alist.replace}
\par
Replace a key with a given value in an association list.
If the key is not present it does nothing.
\paragraph{alist.foldl}
\par
Fold a function over the key-value pairs in the map.
\paragraph{alist.erase}
\par
Erase a key from the map. If the key is not present it does nothing.
\paragraph{alist.insert}
\par
Insert a key-value pair into an association list and erase any existing pair
with the same key.
\paragraph{alist.extract}
\par
Erase a key from the map, and return the corresponding value, if found.
\paragraph{alist.union}
\par
\colorbox[RGB]{253,246,227}{{{{\color[RGB]{101, 123, 131} s₁ ∪ s₂ }}}} is the key-based union of two association lists. It is
left-biased: if there exists an 
\colorbox[RGB]{253,246,227}{{{{\color[RGB]{101, 123, 131} a ∈ s₁ }}}}, 
\colorbox[RGB]{253,246,227}{{{{\color[RGB]{101, 123, 131} lookup a (s₁ ∪ s₂)  }}}{{{\color[RGB]{181, 137, 0} = }}}{{{\color[RGB]{101, 123, 131}  lookup a s₁ }}}}.
\section{list/basic.lean}\paragraph{list.pmap}
\par
Partial map. If 
\colorbox[RGB]{253,246,227}{{{{\color[RGB]{101, 123, 131} f : Π a, p a  }}}{{{\color[RGB]{133, 153, 0} → }}}{{{\color[RGB]{101, 123, 131}  β }}}} is a partial function defined on
\colorbox[RGB]{253,246,227}{{{{\color[RGB]{101, 123, 131} a : α }}}} satisfying 
\colorbox[RGB]{253,246,227}{{{{\color[RGB]{101, 123, 131} p }}}}, then 
\colorbox[RGB]{253,246,227}{{{{\color[RGB]{101, 123, 131} pmap f l h }}}} is essentially the same as 
\colorbox[RGB]{253,246,227}{{{{\color[RGB]{101, 123, 131} map f l }}}}but is defined only when all members of 
\colorbox[RGB]{253,246,227}{{{{\color[RGB]{101, 123, 131} l }}}} satisfy 
\colorbox[RGB]{253,246,227}{{{{\color[RGB]{101, 123, 131} p }}}}, using the proof
to apply 
\colorbox[RGB]{253,246,227}{{{{\color[RGB]{101, 123, 131} f }}}}.
\paragraph{list.attach}
\par
"Attach" the proof that the elements of 
\colorbox[RGB]{253,246,227}{{{{\color[RGB]{101, 123, 131} l }}}} are in 
\colorbox[RGB]{253,246,227}{{{{\color[RGB]{101, 123, 131} l }}}} to produce a new list
with the same elements but in the type 
\colorbox[RGB]{253,246,227}{{{{\color[RGB]{101, 123, 131} \{x  }}}{{{\color[RGB]{181, 137, 0} / }}}{{{\color[RGB]{181, 137, 0} / }}}{{{\color[RGB]{101, 123, 131}  x ∈ l\} }}}}.
\paragraph{list.Ico}
\par
\colorbox[RGB]{253,246,227}{{{{\color[RGB]{101, 123, 131} Ico n m }}}} is the list of natural numbers 
\colorbox[RGB]{253,246,227}{{{{\color[RGB]{101, 123, 131} n  }}}{{{\color[RGB]{181, 137, 0} ≤ }}}{{{\color[RGB]{101, 123, 131}  x  }}}{{{\color[RGB]{181, 137, 0} < }}}{{{\color[RGB]{101, 123, 131}  m }}}}.
(Ico stands for "interval, closed-open".)
\par
See also 
\colorbox[RGB]{253,246,227}{{{{\color[RGB]{101, 123, 131} data }}}{{{\color[RGB]{181, 137, 0} / }}}{{{\color[RGB]{101, 123, 131} set }}}{{{\color[RGB]{181, 137, 0} / }}}{{{\color[RGB]{101, 123, 131} intervals.lean }}}} for 
\colorbox[RGB]{253,246,227}{{{{\color[RGB]{101, 123, 131} set.Ico }}}}, modelling intervals in general preorders, and
\colorbox[RGB]{253,246,227}{{{{\color[RGB]{101, 123, 131} multiset.Ico }}}} and 
\colorbox[RGB]{253,246,227}{{{{\color[RGB]{101, 123, 131} finset.Ico }}}} for 
\colorbox[RGB]{253,246,227}{{{{\color[RGB]{101, 123, 131} n  }}}{{{\color[RGB]{181, 137, 0} ≤ }}}{{{\color[RGB]{101, 123, 131}  x  }}}{{{\color[RGB]{181, 137, 0} < }}}{{{\color[RGB]{101, 123, 131}  m }}}} as a multiset or as a finset.
\par
@TODO (anyone): Define 
\colorbox[RGB]{253,246,227}{{{{\color[RGB]{101, 123, 131} Ioo }}}} and 
\colorbox[RGB]{253,246,227}{{{{\color[RGB]{101, 123, 131} Icc }}}}, state basic lemmas about them.
@TODO (anyone): Prove that 
\colorbox[RGB]{253,246,227}{{{{\color[RGB]{101, 123, 131} finset.Ico }}}} and 
\colorbox[RGB]{253,246,227}{{{{\color[RGB]{101, 123, 131} set.Ico }}}} agree.
@TODO (anyone): Also do the versions for integers?
@TODO (anyone): One could generalise even further, defining
'locally finite partial orders', for which 
\colorbox[RGB]{253,246,227}{{{{\color[RGB]{101, 123, 131} set.Ico a b }}}} is 
\colorbox[RGB]{253,246,227}{{{{\color[RGB]{101, 123, 131} {[}finite{]} }}}}, and
'locally finite total orders', for which there is a list model.
\section{list/default.lean}\section{list/defs.lean}\paragraph{list.split\_at}
\par
Split a list at an index.
\\
\colorbox[RGB]{253,246,227}{\parbox{4.5in}{{{{\color[RGB]{101, 123, 131} split\_at  }}}{{{\color[RGB]{108, 113, 196} 2 }}}{{{\color[RGB]{101, 123, 131}  {[}a, b, c{]}  }}}{{{\color[RGB]{181, 137, 0} = }}}{{{\color[RGB]{101, 123, 131}  ({[}a, b{]}, {[}c{]}) }}}\\

}}\paragraph{list.split\_on\_p}
\par
Split a list at every element satisfying a predicate.
\paragraph{list.split\_on}
\par
Split a list at every occurrence of an element.
\par
{[}
1,1,2,3,2,4,4
{]}
.split\_on 2 = 
{[}
{[}
1,1
{]}
,
{[}
3
{]}
,
{[}
4,4
{]}
{]}
\paragraph{list.concat}
\par
Concatenate an element at the end of a list.
\\
\colorbox[RGB]{253,246,227}{\parbox{4.5in}{{{{\color[RGB]{101, 123, 131} concat {[}a, b{]} c  }}}{{{\color[RGB]{181, 137, 0} = }}}{{{\color[RGB]{101, 123, 131}  {[}a, b, c{]} }}}\\

}}\paragraph{list.to\_array}
\par
Convert a list into an array (whose length is the length of 
\colorbox[RGB]{253,246,227}{{{{\color[RGB]{101, 123, 131} l }}}}).
\paragraph{list.inth}
\par
"inhabited" 
\colorbox[RGB]{253,246,227}{{{{\color[RGB]{101, 123, 131} nth }}}} function: returns 
\colorbox[RGB]{253,246,227}{{{{\color[RGB]{101, 123, 131} default }}}} instead of 
\colorbox[RGB]{253,246,227}{{{{\color[RGB]{101, 123, 131} none }}}} in the case
that the index is out of bounds.
\paragraph{list.modify\_nth\_tail}
\par
Apply a function to the nth tail of 
\colorbox[RGB]{253,246,227}{{{{\color[RGB]{101, 123, 131} l }}}}. Returns the input without
using 
\colorbox[RGB]{253,246,227}{{{{\color[RGB]{101, 123, 131} f }}}} if the index is larger than the length of the list.
\\
\colorbox[RGB]{253,246,227}{\parbox{4.5in}{{{{\color[RGB]{101, 123, 131} modify\_nth\_tail f  }}}{{{\color[RGB]{108, 113, 196} 2 }}}{{{\color[RGB]{101, 123, 131}  {[}a, b, c{]}  }}}{{{\color[RGB]{181, 137, 0} = }}}{{{\color[RGB]{101, 123, 131}  {[}a, b{]}  }}}{{{\color[RGB]{181, 137, 0} + }}}{{{\color[RGB]{181, 137, 0} + }}}{{{\color[RGB]{101, 123, 131}  f {[}c{]} }}}\\

}}\paragraph{list.modify\_head}
\par
Apply 
\colorbox[RGB]{253,246,227}{{{{\color[RGB]{101, 123, 131} f }}}} to the head of the list, if it exists.
\paragraph{list.modify\_nth}
\par
Apply 
\colorbox[RGB]{253,246,227}{{{{\color[RGB]{101, 123, 131} f }}}} to the nth element of the list, if it exists.
\paragraph{list.take\_while}
\par
Get the longest initial segment of the list whose members all satisfy 
\colorbox[RGB]{253,246,227}{{{{\color[RGB]{101, 123, 131} p }}}}.
\\
\colorbox[RGB]{253,246,227}{\parbox{4.5in}{{{{\color[RGB]{101, 123, 131} take\_while (λ x, x  }}}{{{\color[RGB]{181, 137, 0} < }}}{{{\color[RGB]{101, 123, 131}   }}}{{{\color[RGB]{108, 113, 196} 3 }}}{{{\color[RGB]{101, 123, 131} ) {[} }}}{{{\color[RGB]{108, 113, 196} 0 }}}{{{\color[RGB]{101, 123, 131} ,  }}}{{{\color[RGB]{108, 113, 196} 2 }}}{{{\color[RGB]{101, 123, 131} ,  }}}{{{\color[RGB]{108, 113, 196} 5 }}}{{{\color[RGB]{101, 123, 131} ,  }}}{{{\color[RGB]{108, 113, 196} 1 }}}{{{\color[RGB]{101, 123, 131} {]}  }}}{{{\color[RGB]{181, 137, 0} = }}}{{{\color[RGB]{101, 123, 131}  {[} }}}{{{\color[RGB]{108, 113, 196} 0 }}}{{{\color[RGB]{101, 123, 131} ,  }}}{{{\color[RGB]{108, 113, 196} 2 }}}{{{\color[RGB]{101, 123, 131} {]} }}}\\

}}\paragraph{list.scanl}
\par
Fold a function 
\colorbox[RGB]{253,246,227}{{{{\color[RGB]{101, 123, 131} f }}}} over the list from the left, returning the list
of partial results.
\\
\colorbox[RGB]{253,246,227}{\parbox{4.5in}{{{{\color[RGB]{101, 123, 131} scanl ( }}}{{{\color[RGB]{181, 137, 0} + }}}{{{\color[RGB]{101, 123, 131} )  }}}{{{\color[RGB]{108, 113, 196} 0 }}}{{{\color[RGB]{101, 123, 131}  {[} }}}{{{\color[RGB]{108, 113, 196} 1 }}}{{{\color[RGB]{101, 123, 131} ,  }}}{{{\color[RGB]{108, 113, 196} 2 }}}{{{\color[RGB]{101, 123, 131} ,  }}}{{{\color[RGB]{108, 113, 196} 3 }}}{{{\color[RGB]{101, 123, 131} {]}  }}}{{{\color[RGB]{181, 137, 0} = }}}{{{\color[RGB]{101, 123, 131}  {[} }}}{{{\color[RGB]{108, 113, 196} 0 }}}{{{\color[RGB]{101, 123, 131} ,  }}}{{{\color[RGB]{108, 113, 196} 1 }}}{{{\color[RGB]{101, 123, 131} ,  }}}{{{\color[RGB]{108, 113, 196} 3 }}}{{{\color[RGB]{101, 123, 131} ,  }}}{{{\color[RGB]{108, 113, 196} 6 }}}{{{\color[RGB]{101, 123, 131} {]} }}}\\

}}\paragraph{list.scanr}
\par
Fold a function 
\colorbox[RGB]{253,246,227}{{{{\color[RGB]{101, 123, 131} f }}}} over the list from the right, returning the list
of partial results.
\\
\colorbox[RGB]{253,246,227}{\parbox{4.5in}{{{{\color[RGB]{101, 123, 131} scanr ( }}}{{{\color[RGB]{181, 137, 0} + }}}{{{\color[RGB]{101, 123, 131} )  }}}{{{\color[RGB]{108, 113, 196} 0 }}}{{{\color[RGB]{101, 123, 131}  {[} }}}{{{\color[RGB]{108, 113, 196} 1 }}}{{{\color[RGB]{101, 123, 131} ,  }}}{{{\color[RGB]{108, 113, 196} 2 }}}{{{\color[RGB]{101, 123, 131} ,  }}}{{{\color[RGB]{108, 113, 196} 3 }}}{{{\color[RGB]{101, 123, 131} {]}  }}}{{{\color[RGB]{181, 137, 0} = }}}{{{\color[RGB]{101, 123, 131}  {[} }}}{{{\color[RGB]{108, 113, 196} 6 }}}{{{\color[RGB]{101, 123, 131} ,  }}}{{{\color[RGB]{108, 113, 196} 5 }}}{{{\color[RGB]{101, 123, 131} ,  }}}{{{\color[RGB]{108, 113, 196} 3 }}}{{{\color[RGB]{101, 123, 131} ,  }}}{{{\color[RGB]{108, 113, 196} 0 }}}{{{\color[RGB]{101, 123, 131} {]} }}}\\

}}\paragraph{list.prod}
\par
Product of a list.
\\
\colorbox[RGB]{253,246,227}{\parbox{4.5in}{{{{\color[RGB]{101, 123, 131} prod {[}a, b, c{]}  }}}{{{\color[RGB]{181, 137, 0} = }}}{{{\color[RGB]{101, 123, 131}  (( }}}{{{\color[RGB]{108, 113, 196} 1 }}}{{{\color[RGB]{101, 123, 131}   }}}{{{\color[RGB]{181, 137, 0} * }}}{{{\color[RGB]{101, 123, 131}  a)  }}}{{{\color[RGB]{181, 137, 0} * }}}{{{\color[RGB]{101, 123, 131}  b)  }}}{{{\color[RGB]{181, 137, 0} * }}}{{{\color[RGB]{101, 123, 131}  c }}}\\

}}\paragraph{list.find}
\par
\colorbox[RGB]{253,246,227}{{{{\color[RGB]{101, 123, 131} find p l }}}} is the first element of 
\colorbox[RGB]{253,246,227}{{{{\color[RGB]{101, 123, 131} l }}}} satisfying 
\colorbox[RGB]{253,246,227}{{{{\color[RGB]{101, 123, 131} p }}}}, or 
\colorbox[RGB]{253,246,227}{{{{\color[RGB]{101, 123, 131} none }}}} if no such
element exists.
\paragraph{list.find\_indexes}
\par
\colorbox[RGB]{253,246,227}{{{{\color[RGB]{101, 123, 131} find\_indexes p l }}}} is the list of indexes of elements of 
\colorbox[RGB]{253,246,227}{{{{\color[RGB]{101, 123, 131} l }}}} that satisfy 
\colorbox[RGB]{253,246,227}{{{{\color[RGB]{101, 123, 131} p }}}}.
\paragraph{list.lookmap}
\par
\colorbox[RGB]{253,246,227}{{{{\color[RGB]{101, 123, 131} lookmap }}}} is a combination of 
\colorbox[RGB]{253,246,227}{{{{\color[RGB]{101, 123, 131} lookup }}}} and 
\colorbox[RGB]{253,246,227}{{{{\color[RGB]{101, 123, 131} filter\_map }}}}.
\colorbox[RGB]{253,246,227}{{{{\color[RGB]{101, 123, 131} lookmap f l }}}} will apply 
\colorbox[RGB]{253,246,227}{{{{\color[RGB]{101, 123, 131} f : α  }}}{{{\color[RGB]{133, 153, 0} → }}}{{{\color[RGB]{101, 123, 131}  option α }}}} to each element of the list,
replacing 
\colorbox[RGB]{253,246,227}{{{{\color[RGB]{101, 123, 131} a  }}}{{{\color[RGB]{181, 137, 0} -> }}}{{{\color[RGB]{101, 123, 131}  b }}}} at the first value 
\colorbox[RGB]{253,246,227}{{{{\color[RGB]{101, 123, 131} a }}}} in the list such that 
\colorbox[RGB]{253,246,227}{{{{\color[RGB]{101, 123, 131} f a  }}}{{{\color[RGB]{181, 137, 0} = }}}{{{\color[RGB]{101, 123, 131}  some b }}}}.
\paragraph{list.indexes\_of}
\par
\colorbox[RGB]{253,246,227}{{{{\color[RGB]{101, 123, 131} indexes\_of a l }}}} is the list of all indexes of 
\colorbox[RGB]{253,246,227}{{{{\color[RGB]{101, 123, 131} a }}}} in 
\colorbox[RGB]{253,246,227}{{{{\color[RGB]{101, 123, 131} l }}}}.
\\
\colorbox[RGB]{253,246,227}{\parbox{4.5in}{{{{\color[RGB]{101, 123, 131} indexes\_of a {[}a, b, a, a{]}  }}}{{{\color[RGB]{181, 137, 0} = }}}{{{\color[RGB]{101, 123, 131}  {[} }}}{{{\color[RGB]{108, 113, 196} 0 }}}{{{\color[RGB]{101, 123, 131} ,  }}}{{{\color[RGB]{108, 113, 196} 2 }}}{{{\color[RGB]{101, 123, 131} ,  }}}{{{\color[RGB]{108, 113, 196} 3 }}}{{{\color[RGB]{101, 123, 131} {]} }}}\\

}}\paragraph{list.countp}
\par
\colorbox[RGB]{253,246,227}{{{{\color[RGB]{101, 123, 131} countp p l }}}} is the number of elements of 
\colorbox[RGB]{253,246,227}{{{{\color[RGB]{101, 123, 131} l }}}} that satisfy 
\colorbox[RGB]{253,246,227}{{{{\color[RGB]{101, 123, 131} p }}}}.
\paragraph{list.count}
\par
\colorbox[RGB]{253,246,227}{{{{\color[RGB]{101, 123, 131} count a l }}}} is the number of occurrences of 
\colorbox[RGB]{253,246,227}{{{{\color[RGB]{101, 123, 131} a }}}} in 
\colorbox[RGB]{253,246,227}{{{{\color[RGB]{101, 123, 131} l }}}}.
\paragraph{list.is\_prefix}
\par
\colorbox[RGB]{253,246,227}{{{{\color[RGB]{101, 123, 131} is\_prefix l₁ l₂ }}}}, or 
\colorbox[RGB]{253,246,227}{{{{\color[RGB]{101, 123, 131} l₁  }}}{{{\color[RGB]{181, 137, 0} < }}}{{{\color[RGB]{181, 137, 0} + }}}{{{\color[RGB]{101, 123, 131} : l₂ }}}}, means that 
\colorbox[RGB]{253,246,227}{{{{\color[RGB]{101, 123, 131} l₁ }}}} is a prefix of 
\colorbox[RGB]{253,246,227}{{{{\color[RGB]{101, 123, 131} l₂ }}}},
that is, 
\colorbox[RGB]{253,246,227}{{{{\color[RGB]{101, 123, 131} l₂ }}}} has the form 
\colorbox[RGB]{253,246,227}{{{{\color[RGB]{101, 123, 131} l₁  }}}{{{\color[RGB]{181, 137, 0} + }}}{{{\color[RGB]{181, 137, 0} + }}}{{{\color[RGB]{101, 123, 131}  t }}}} for some 
\colorbox[RGB]{253,246,227}{{{{\color[RGB]{101, 123, 131} t }}}}.
\paragraph{list.is\_suffix}
\par
\colorbox[RGB]{253,246,227}{{{{\color[RGB]{101, 123, 131} is\_suffix l₁ l₂ }}}}, or 
\colorbox[RGB]{253,246,227}{{{{\color[RGB]{101, 123, 131} l₁  }}}{{{\color[RGB]{181, 137, 0} < }}}{{{\color[RGB]{101, 123, 131} : }}}{{{\color[RGB]{181, 137, 0} + }}}{{{\color[RGB]{101, 123, 131}  l₂ }}}}, means that 
\colorbox[RGB]{253,246,227}{{{{\color[RGB]{101, 123, 131} l₁ }}}} is a suffix of 
\colorbox[RGB]{253,246,227}{{{{\color[RGB]{101, 123, 131} l₂ }}}},
that is, 
\colorbox[RGB]{253,246,227}{{{{\color[RGB]{101, 123, 131} l₂ }}}} has the form 
\colorbox[RGB]{253,246,227}{{{{\color[RGB]{101, 123, 131} t  }}}{{{\color[RGB]{181, 137, 0} + }}}{{{\color[RGB]{181, 137, 0} + }}}{{{\color[RGB]{101, 123, 131}  l₁ }}}} for some 
\colorbox[RGB]{253,246,227}{{{{\color[RGB]{101, 123, 131} t }}}}.
\paragraph{list.is\_infix}
\par
\colorbox[RGB]{253,246,227}{{{{\color[RGB]{101, 123, 131} is\_infix l₁ l₂ }}}}, or 
\colorbox[RGB]{253,246,227}{{{{\color[RGB]{101, 123, 131} l₁  }}}{{{\color[RGB]{181, 137, 0} < }}}{{{\color[RGB]{101, 123, 131} : }}}{{{\color[RGB]{181, 137, 0} + }}}{{{\color[RGB]{101, 123, 131} : l₂ }}}}, means that 
\colorbox[RGB]{253,246,227}{{{{\color[RGB]{101, 123, 131} l₁ }}}} is a contiguous
substring of 
\colorbox[RGB]{253,246,227}{{{{\color[RGB]{101, 123, 131} l₂ }}}}, that is, 
\colorbox[RGB]{253,246,227}{{{{\color[RGB]{101, 123, 131} l₂ }}}} has the form 
\colorbox[RGB]{253,246,227}{{{{\color[RGB]{101, 123, 131} s  }}}{{{\color[RGB]{181, 137, 0} + }}}{{{\color[RGB]{181, 137, 0} + }}}{{{\color[RGB]{101, 123, 131}  l₁  }}}{{{\color[RGB]{181, 137, 0} + }}}{{{\color[RGB]{181, 137, 0} + }}}{{{\color[RGB]{101, 123, 131}  t }}}} for some 
\colorbox[RGB]{253,246,227}{{{{\color[RGB]{101, 123, 131} s, t }}}}.
\paragraph{list.inits}
\par
\colorbox[RGB]{253,246,227}{{{{\color[RGB]{101, 123, 131} inits l }}}} is the list of initial segments of 
\colorbox[RGB]{253,246,227}{{{{\color[RGB]{101, 123, 131} l }}}}.
\\
\colorbox[RGB]{253,246,227}{\parbox{4.5in}{{{{\color[RGB]{101, 123, 131} inits {[} }}}{{{\color[RGB]{108, 113, 196} 1 }}}{{{\color[RGB]{101, 123, 131} ,  }}}{{{\color[RGB]{108, 113, 196} 2 }}}{{{\color[RGB]{101, 123, 131} ,  }}}{{{\color[RGB]{108, 113, 196} 3 }}}{{{\color[RGB]{101, 123, 131} {]}  }}}{{{\color[RGB]{181, 137, 0} = }}}{{{\color[RGB]{101, 123, 131}  {[}{[}{]}, {[} }}}{{{\color[RGB]{108, 113, 196} 1 }}}{{{\color[RGB]{101, 123, 131} {]}, {[} }}}{{{\color[RGB]{108, 113, 196} 1 }}}{{{\color[RGB]{101, 123, 131} ,  }}}{{{\color[RGB]{108, 113, 196} 2 }}}{{{\color[RGB]{101, 123, 131} {]}, {[} }}}{{{\color[RGB]{108, 113, 196} 1 }}}{{{\color[RGB]{101, 123, 131} ,  }}}{{{\color[RGB]{108, 113, 196} 2 }}}{{{\color[RGB]{101, 123, 131} ,  }}}{{{\color[RGB]{108, 113, 196} 3 }}}{{{\color[RGB]{101, 123, 131} {]}{]} }}}\\

}}\paragraph{list.tails}
\par
\colorbox[RGB]{253,246,227}{{{{\color[RGB]{101, 123, 131} tails l }}}} is the list of terminal segments of 
\colorbox[RGB]{253,246,227}{{{{\color[RGB]{101, 123, 131} l }}}}.
\\
\colorbox[RGB]{253,246,227}{\parbox{4.5in}{{{{\color[RGB]{101, 123, 131} tails {[} }}}{{{\color[RGB]{108, 113, 196} 1 }}}{{{\color[RGB]{101, 123, 131} ,  }}}{{{\color[RGB]{108, 113, 196} 2 }}}{{{\color[RGB]{101, 123, 131} ,  }}}{{{\color[RGB]{108, 113, 196} 3 }}}{{{\color[RGB]{101, 123, 131} {]}  }}}{{{\color[RGB]{181, 137, 0} = }}}{{{\color[RGB]{101, 123, 131}  {[}{[} }}}{{{\color[RGB]{108, 113, 196} 1 }}}{{{\color[RGB]{101, 123, 131} ,  }}}{{{\color[RGB]{108, 113, 196} 2 }}}{{{\color[RGB]{101, 123, 131} ,  }}}{{{\color[RGB]{108, 113, 196} 3 }}}{{{\color[RGB]{101, 123, 131} {]}, {[} }}}{{{\color[RGB]{108, 113, 196} 2 }}}{{{\color[RGB]{101, 123, 131} ,  }}}{{{\color[RGB]{108, 113, 196} 3 }}}{{{\color[RGB]{101, 123, 131} {]}, {[} }}}{{{\color[RGB]{108, 113, 196} 3 }}}{{{\color[RGB]{101, 123, 131} {]}, {[}{]}{]} }}}\\

}}\paragraph{list.sublists'}
\par
\colorbox[RGB]{253,246,227}{{{{\color[RGB]{101, 123, 131} sublists' l }}}} is the list of all (non-contiguous) sublists of 
\colorbox[RGB]{253,246,227}{{{{\color[RGB]{101, 123, 131} l }}}}.
It differs from 
\colorbox[RGB]{253,246,227}{{{{\color[RGB]{101, 123, 131} sublists }}}} only in the order of appearance of the sublists;
\colorbox[RGB]{253,246,227}{{{{\color[RGB]{101, 123, 131} sublists' }}}} uses the first element of the list as the MSB,
\colorbox[RGB]{253,246,227}{{{{\color[RGB]{101, 123, 131} sublists }}}} uses the first element of the list as the LSB.
\\
\colorbox[RGB]{253,246,227}{\parbox{4.5in}{{{{\color[RGB]{101, 123, 131} sublists' {[} }}}{{{\color[RGB]{108, 113, 196} 1 }}}{{{\color[RGB]{101, 123, 131} ,  }}}{{{\color[RGB]{108, 113, 196} 2 }}}{{{\color[RGB]{101, 123, 131} ,  }}}{{{\color[RGB]{108, 113, 196} 3 }}}{{{\color[RGB]{101, 123, 131} {]}  }}}{{{\color[RGB]{181, 137, 0} = }}}{{{\color[RGB]{101, 123, 131}  {[}{[}{]}, {[} }}}{{{\color[RGB]{108, 113, 196} 3 }}}{{{\color[RGB]{101, 123, 131} {]}, {[} }}}{{{\color[RGB]{108, 113, 196} 2 }}}{{{\color[RGB]{101, 123, 131} {]}, {[} }}}{{{\color[RGB]{108, 113, 196} 2 }}}{{{\color[RGB]{101, 123, 131} ,  }}}{{{\color[RGB]{108, 113, 196} 3 }}}{{{\color[RGB]{101, 123, 131} {]}, {[} }}}{{{\color[RGB]{108, 113, 196} 1 }}}{{{\color[RGB]{101, 123, 131} {]}, {[} }}}{{{\color[RGB]{108, 113, 196} 1 }}}{{{\color[RGB]{101, 123, 131} ,  }}}{{{\color[RGB]{108, 113, 196} 3 }}}{{{\color[RGB]{101, 123, 131} {]}, {[} }}}{{{\color[RGB]{108, 113, 196} 1 }}}{{{\color[RGB]{101, 123, 131} ,  }}}{{{\color[RGB]{108, 113, 196} 2 }}}{{{\color[RGB]{101, 123, 131} {]}, {[} }}}{{{\color[RGB]{108, 113, 196} 1 }}}{{{\color[RGB]{101, 123, 131} ,  }}}{{{\color[RGB]{108, 113, 196} 2 }}}{{{\color[RGB]{101, 123, 131} ,  }}}{{{\color[RGB]{108, 113, 196} 3 }}}{{{\color[RGB]{101, 123, 131} {]}{]} }}}\\

}}\paragraph{list.sublists}
\par
\colorbox[RGB]{253,246,227}{{{{\color[RGB]{101, 123, 131} sublists l }}}} is the list of all (non-contiguous) sublists of 
\colorbox[RGB]{253,246,227}{{{{\color[RGB]{101, 123, 131} l }}}}; cf. 
\colorbox[RGB]{253,246,227}{{{{\color[RGB]{101, 123, 131} sublists' }}}}for a different ordering.
\\
\colorbox[RGB]{253,246,227}{\parbox{4.5in}{{{{\color[RGB]{101, 123, 131} sublists {[} }}}{{{\color[RGB]{108, 113, 196} 1 }}}{{{\color[RGB]{101, 123, 131} ,  }}}{{{\color[RGB]{108, 113, 196} 2 }}}{{{\color[RGB]{101, 123, 131} ,  }}}{{{\color[RGB]{108, 113, 196} 3 }}}{{{\color[RGB]{101, 123, 131} {]}  }}}{{{\color[RGB]{181, 137, 0} = }}}{{{\color[RGB]{101, 123, 131}  {[}{[}{]}, {[} }}}{{{\color[RGB]{108, 113, 196} 1 }}}{{{\color[RGB]{101, 123, 131} {]}, {[} }}}{{{\color[RGB]{108, 113, 196} 2 }}}{{{\color[RGB]{101, 123, 131} {]}, {[} }}}{{{\color[RGB]{108, 113, 196} 1 }}}{{{\color[RGB]{101, 123, 131} ,  }}}{{{\color[RGB]{108, 113, 196} 2 }}}{{{\color[RGB]{101, 123, 131} {]}, {[} }}}{{{\color[RGB]{108, 113, 196} 3 }}}{{{\color[RGB]{101, 123, 131} {]}, {[} }}}{{{\color[RGB]{108, 113, 196} 1 }}}{{{\color[RGB]{101, 123, 131} ,  }}}{{{\color[RGB]{108, 113, 196} 3 }}}{{{\color[RGB]{101, 123, 131} {]}, {[} }}}{{{\color[RGB]{108, 113, 196} 2 }}}{{{\color[RGB]{101, 123, 131} ,  }}}{{{\color[RGB]{108, 113, 196} 3 }}}{{{\color[RGB]{101, 123, 131} {]}, {[} }}}{{{\color[RGB]{108, 113, 196} 1 }}}{{{\color[RGB]{101, 123, 131} ,  }}}{{{\color[RGB]{108, 113, 196} 2 }}}{{{\color[RGB]{101, 123, 131} ,  }}}{{{\color[RGB]{108, 113, 196} 3 }}}{{{\color[RGB]{101, 123, 131} {]}{]} }}}\\

}}\paragraph{list.transpose}
\par
transpose of a list of lists, treated as a matrix.
\\
\colorbox[RGB]{253,246,227}{\parbox{4.5in}{{{{\color[RGB]{101, 123, 131} transpose {[}{[} }}}{{{\color[RGB]{108, 113, 196} 1 }}}{{{\color[RGB]{101, 123, 131} ,  }}}{{{\color[RGB]{108, 113, 196} 2 }}}{{{\color[RGB]{101, 123, 131} {]}, {[} }}}{{{\color[RGB]{108, 113, 196} 3 }}}{{{\color[RGB]{101, 123, 131} ,  }}}{{{\color[RGB]{108, 113, 196} 4 }}}{{{\color[RGB]{101, 123, 131} {]}, {[} }}}{{{\color[RGB]{108, 113, 196} 5 }}}{{{\color[RGB]{101, 123, 131} ,  }}}{{{\color[RGB]{108, 113, 196} 6 }}}{{{\color[RGB]{101, 123, 131} {]}{]}  }}}{{{\color[RGB]{181, 137, 0} = }}}{{{\color[RGB]{101, 123, 131}  {[}{[} }}}{{{\color[RGB]{108, 113, 196} 1 }}}{{{\color[RGB]{101, 123, 131} ,  }}}{{{\color[RGB]{108, 113, 196} 3 }}}{{{\color[RGB]{101, 123, 131} ,  }}}{{{\color[RGB]{108, 113, 196} 5 }}}{{{\color[RGB]{101, 123, 131} {]}, {[} }}}{{{\color[RGB]{108, 113, 196} 2 }}}{{{\color[RGB]{101, 123, 131} ,  }}}{{{\color[RGB]{108, 113, 196} 4 }}}{{{\color[RGB]{101, 123, 131} ,  }}}{{{\color[RGB]{108, 113, 196} 6 }}}{{{\color[RGB]{101, 123, 131} {]}{]} }}}\\

}}\paragraph{list.sections}
\par
List of all sections through a list of lists. A section
of 
\colorbox[RGB]{253,246,227}{{{{\color[RGB]{101, 123, 131} {[}L₁, L₂, ..., Lₙ{]} }}}} is a list whose first element comes from
\colorbox[RGB]{253,246,227}{{{{\color[RGB]{101, 123, 131} L₁ }}}}, whose second element comes from 
\colorbox[RGB]{253,246,227}{{{{\color[RGB]{101, 123, 131} L₂ }}}}, and so on.
\paragraph{list.permutations}
\par
List of all permutations of 
\colorbox[RGB]{253,246,227}{{{{\color[RGB]{101, 123, 131} l }}}}.
\\
\colorbox[RGB]{253,246,227}{\parbox{4.5in}{{{{\color[RGB]{101, 123, 131} permutations {[} }}}{{{\color[RGB]{108, 113, 196} 1 }}}{{{\color[RGB]{101, 123, 131} ,  }}}{{{\color[RGB]{108, 113, 196} 2 }}}{{{\color[RGB]{101, 123, 131} ,  }}}{{{\color[RGB]{108, 113, 196} 3 }}}{{{\color[RGB]{101, 123, 131} {]}  }}}{{{\color[RGB]{181, 137, 0} = }}}{{{\color[RGB]{101, 123, 131} 
 }}}\\

{{{\color[RGB]{101, 123, 131}   {[}{[} }}}{{{\color[RGB]{108, 113, 196} 1 }}}{{{\color[RGB]{101, 123, 131} ,  }}}{{{\color[RGB]{108, 113, 196} 2 }}}{{{\color[RGB]{101, 123, 131} ,  }}}{{{\color[RGB]{108, 113, 196} 3 }}}{{{\color[RGB]{101, 123, 131} {]}, {[} }}}{{{\color[RGB]{108, 113, 196} 2 }}}{{{\color[RGB]{101, 123, 131} ,  }}}{{{\color[RGB]{108, 113, 196} 1 }}}{{{\color[RGB]{101, 123, 131} ,  }}}{{{\color[RGB]{108, 113, 196} 3 }}}{{{\color[RGB]{101, 123, 131} {]}, {[} }}}{{{\color[RGB]{108, 113, 196} 3 }}}{{{\color[RGB]{101, 123, 131} ,  }}}{{{\color[RGB]{108, 113, 196} 2 }}}{{{\color[RGB]{101, 123, 131} ,  }}}{{{\color[RGB]{108, 113, 196} 1 }}}{{{\color[RGB]{101, 123, 131} {]},
 }}}\\

{{{\color[RGB]{101, 123, 131}    {[} }}}{{{\color[RGB]{108, 113, 196} 2 }}}{{{\color[RGB]{101, 123, 131} ,  }}}{{{\color[RGB]{108, 113, 196} 3 }}}{{{\color[RGB]{101, 123, 131} ,  }}}{{{\color[RGB]{108, 113, 196} 1 }}}{{{\color[RGB]{101, 123, 131} {]}, {[} }}}{{{\color[RGB]{108, 113, 196} 3 }}}{{{\color[RGB]{101, 123, 131} ,  }}}{{{\color[RGB]{108, 113, 196} 1 }}}{{{\color[RGB]{101, 123, 131} ,  }}}{{{\color[RGB]{108, 113, 196} 2 }}}{{{\color[RGB]{101, 123, 131} {]}, {[} }}}{{{\color[RGB]{108, 113, 196} 1 }}}{{{\color[RGB]{101, 123, 131} ,  }}}{{{\color[RGB]{108, 113, 196} 3 }}}{{{\color[RGB]{101, 123, 131} ,  }}}{{{\color[RGB]{108, 113, 196} 2 }}}{{{\color[RGB]{101, 123, 131} {]}{]} }}}\\

}}\paragraph{list.product}
\par
\colorbox[RGB]{253,246,227}{{{{\color[RGB]{101, 123, 131} product l₁ l₂ }}}} is the list of pairs 
\colorbox[RGB]{253,246,227}{{{{\color[RGB]{101, 123, 131} (a, b) }}}} where 
\colorbox[RGB]{253,246,227}{{{{\color[RGB]{101, 123, 131} a ∈ l₁ }}}} and 
\colorbox[RGB]{253,246,227}{{{{\color[RGB]{101, 123, 131} b ∈ l₂ }}}}.
\\
\colorbox[RGB]{253,246,227}{\parbox{4.5in}{{{{\color[RGB]{101, 123, 131} product {[} }}}{{{\color[RGB]{108, 113, 196} 1 }}}{{{\color[RGB]{101, 123, 131} ,  }}}{{{\color[RGB]{108, 113, 196} 2 }}}{{{\color[RGB]{101, 123, 131} {]} {[} }}}{{{\color[RGB]{108, 113, 196} 5 }}}{{{\color[RGB]{101, 123, 131} ,  }}}{{{\color[RGB]{108, 113, 196} 6 }}}{{{\color[RGB]{101, 123, 131} {]}  }}}{{{\color[RGB]{181, 137, 0} = }}}{{{\color[RGB]{101, 123, 131}  {[}( }}}{{{\color[RGB]{108, 113, 196} 1 }}}{{{\color[RGB]{101, 123, 131} ,  }}}{{{\color[RGB]{108, 113, 196} 5 }}}{{{\color[RGB]{101, 123, 131} ), ( }}}{{{\color[RGB]{108, 113, 196} 1 }}}{{{\color[RGB]{101, 123, 131} ,  }}}{{{\color[RGB]{108, 113, 196} 6 }}}{{{\color[RGB]{101, 123, 131} ), ( }}}{{{\color[RGB]{108, 113, 196} 2 }}}{{{\color[RGB]{101, 123, 131} ,  }}}{{{\color[RGB]{108, 113, 196} 5 }}}{{{\color[RGB]{101, 123, 131} ), ( }}}{{{\color[RGB]{108, 113, 196} 2 }}}{{{\color[RGB]{101, 123, 131} ,  }}}{{{\color[RGB]{108, 113, 196} 6 }}}{{{\color[RGB]{101, 123, 131} ){]} }}}\\

}}\paragraph{list.sigma}
\par
\colorbox[RGB]{253,246,227}{{{{\color[RGB]{101, 123, 131} sigma l₁ l₂ }}}} is the list of dependent pairs 
\colorbox[RGB]{253,246,227}{{{{\color[RGB]{101, 123, 131} (a, b) }}}} where 
\colorbox[RGB]{253,246,227}{{{{\color[RGB]{101, 123, 131} a ∈ l₁ }}}} and 
\colorbox[RGB]{253,246,227}{{{{\color[RGB]{101, 123, 131} b ∈ l₂ a }}}}.
\\
\colorbox[RGB]{253,246,227}{\parbox{4.5in}{{{{\color[RGB]{101, 123, 131} sigma {[} }}}{{{\color[RGB]{108, 113, 196} 1 }}}{{{\color[RGB]{101, 123, 131} ,  }}}{{{\color[RGB]{108, 113, 196} 2 }}}{{{\color[RGB]{101, 123, 131} {]} (λ\_, {[}( }}}{{{\color[RGB]{108, 113, 196} 5 }}}{{{\color[RGB]{101, 123, 131}  : ℕ),  }}}{{{\color[RGB]{108, 113, 196} 6 }}}{{{\color[RGB]{101, 123, 131} {]})  }}}{{{\color[RGB]{181, 137, 0} = }}}{{{\color[RGB]{101, 123, 131}  {[}( }}}{{{\color[RGB]{108, 113, 196} 1 }}}{{{\color[RGB]{101, 123, 131} ,  }}}{{{\color[RGB]{108, 113, 196} 5 }}}{{{\color[RGB]{101, 123, 131} ), ( }}}{{{\color[RGB]{108, 113, 196} 1 }}}{{{\color[RGB]{101, 123, 131} ,  }}}{{{\color[RGB]{108, 113, 196} 6 }}}{{{\color[RGB]{101, 123, 131} ), ( }}}{{{\color[RGB]{108, 113, 196} 2 }}}{{{\color[RGB]{101, 123, 131} ,  }}}{{{\color[RGB]{108, 113, 196} 5 }}}{{{\color[RGB]{101, 123, 131} ), ( }}}{{{\color[RGB]{108, 113, 196} 2 }}}{{{\color[RGB]{101, 123, 131} ,  }}}{{{\color[RGB]{108, 113, 196} 6 }}}{{{\color[RGB]{101, 123, 131} ){]} }}}\\

}}\paragraph{list.disjoint}
\par
\colorbox[RGB]{253,246,227}{{{{\color[RGB]{101, 123, 131} disjoint l₁ l₂ }}}} means that 
\colorbox[RGB]{253,246,227}{{{{\color[RGB]{101, 123, 131} l₁ }}}} and 
\colorbox[RGB]{253,246,227}{{{{\color[RGB]{101, 123, 131} l₂ }}}} have no elements in common.
\paragraph{list.pairwise}
\par
\colorbox[RGB]{253,246,227}{{{{\color[RGB]{101, 123, 131} pairwise R l }}}} means that all the elements with earlier indexes are
\colorbox[RGB]{253,246,227}{{{{\color[RGB]{101, 123, 131} R }}}}-related to all the elements with later indexes.
\\
\colorbox[RGB]{253,246,227}{\parbox{4.5in}{{{{\color[RGB]{101, 123, 131} pairwise R {[} }}}{{{\color[RGB]{108, 113, 196} 1 }}}{{{\color[RGB]{101, 123, 131} ,  }}}{{{\color[RGB]{108, 113, 196} 2 }}}{{{\color[RGB]{101, 123, 131} ,  }}}{{{\color[RGB]{108, 113, 196} 3 }}}{{{\color[RGB]{101, 123, 131} {]}  }}}{{{\color[RGB]{181, 137, 0} ↔ }}}{{{\color[RGB]{101, 123, 131}  R  }}}{{{\color[RGB]{108, 113, 196} 1 }}}{{{\color[RGB]{101, 123, 131}   }}}{{{\color[RGB]{108, 113, 196} 2 }}}{{{\color[RGB]{101, 123, 131}   }}}{{{\color[RGB]{181, 137, 0} ∧ }}}{{{\color[RGB]{101, 123, 131}  R  }}}{{{\color[RGB]{108, 113, 196} 1 }}}{{{\color[RGB]{101, 123, 131}   }}}{{{\color[RGB]{108, 113, 196} 3 }}}{{{\color[RGB]{101, 123, 131}   }}}{{{\color[RGB]{181, 137, 0} ∧ }}}{{{\color[RGB]{101, 123, 131}  R  }}}{{{\color[RGB]{108, 113, 196} 2 }}}{{{\color[RGB]{101, 123, 131}   }}}{{{\color[RGB]{108, 113, 196} 3 }}}{{{\color[RGB]{101, 123, 131} 
 }}}\\

}}\par
For example if 
\colorbox[RGB]{253,246,227}{{{{\color[RGB]{101, 123, 131} R  }}}{{{\color[RGB]{181, 137, 0} = }}}{{{\color[RGB]{101, 123, 131}  ( }}}{{{\color[RGB]{181, 137, 0} ≠ }}}{{{\color[RGB]{101, 123, 131} ) }}}} then it asserts 
\colorbox[RGB]{253,246,227}{{{{\color[RGB]{101, 123, 131} l }}}} has no duplicates,
and if 
\colorbox[RGB]{253,246,227}{{{{\color[RGB]{101, 123, 131} R  }}}{{{\color[RGB]{181, 137, 0} = }}}{{{\color[RGB]{101, 123, 131}  ( }}}{{{\color[RGB]{181, 137, 0} < }}}{{{\color[RGB]{101, 123, 131} ) }}}} then it asserts that 
\colorbox[RGB]{253,246,227}{{{{\color[RGB]{101, 123, 131} l }}}} is (strictly) sorted.
\paragraph{list.pw\_filter}
\par
\colorbox[RGB]{253,246,227}{{{{\color[RGB]{101, 123, 131} pw\_filter R l }}}} is a maximal sublist of 
\colorbox[RGB]{253,246,227}{{{{\color[RGB]{101, 123, 131} l }}}} which is 
\colorbox[RGB]{253,246,227}{{{{\color[RGB]{101, 123, 131} pairwise R }}}}.
\colorbox[RGB]{253,246,227}{{{{\color[RGB]{101, 123, 131} pw\_filter ( }}}{{{\color[RGB]{181, 137, 0} ≠ }}}{{{\color[RGB]{101, 123, 131} ) }}}} is the erase duplicates function (cf. 
\colorbox[RGB]{253,246,227}{{{{\color[RGB]{101, 123, 131} erase\_dup }}}}), and 
\colorbox[RGB]{253,246,227}{{{{\color[RGB]{101, 123, 131} pw\_filter ( }}}{{{\color[RGB]{181, 137, 0} < }}}{{{\color[RGB]{101, 123, 131} ) }}}} finds
a maximal increasing subsequence in 
\colorbox[RGB]{253,246,227}{{{{\color[RGB]{101, 123, 131} l }}}}. For example,
\\
\colorbox[RGB]{253,246,227}{\parbox{4.5in}{{{{\color[RGB]{101, 123, 131} pw\_filter ( }}}{{{\color[RGB]{181, 137, 0} < }}}{{{\color[RGB]{101, 123, 131} ) {[} }}}{{{\color[RGB]{108, 113, 196} 0 }}}{{{\color[RGB]{101, 123, 131} ,  }}}{{{\color[RGB]{108, 113, 196} 1 }}}{{{\color[RGB]{101, 123, 131} ,  }}}{{{\color[RGB]{108, 113, 196} 5 }}}{{{\color[RGB]{101, 123, 131} ,  }}}{{{\color[RGB]{108, 113, 196} 2 }}}{{{\color[RGB]{101, 123, 131} ,  }}}{{{\color[RGB]{108, 113, 196} 6 }}}{{{\color[RGB]{101, 123, 131} ,  }}}{{{\color[RGB]{108, 113, 196} 3 }}}{{{\color[RGB]{101, 123, 131} ,  }}}{{{\color[RGB]{108, 113, 196} 4 }}}{{{\color[RGB]{101, 123, 131} {]}  }}}{{{\color[RGB]{181, 137, 0} = }}}{{{\color[RGB]{101, 123, 131}  {[} }}}{{{\color[RGB]{108, 113, 196} 0 }}}{{{\color[RGB]{101, 123, 131} ,  }}}{{{\color[RGB]{108, 113, 196} 1 }}}{{{\color[RGB]{101, 123, 131} ,  }}}{{{\color[RGB]{108, 113, 196} 2 }}}{{{\color[RGB]{101, 123, 131} ,  }}}{{{\color[RGB]{108, 113, 196} 3 }}}{{{\color[RGB]{101, 123, 131} ,  }}}{{{\color[RGB]{108, 113, 196} 4 }}}{{{\color[RGB]{101, 123, 131} {]} }}}\\

}}\paragraph{list.chain}
\par
\colorbox[RGB]{253,246,227}{{{{\color[RGB]{101, 123, 131} chain R a l }}}} means that 
\colorbox[RGB]{253,246,227}{{{{\color[RGB]{101, 123, 131} R }}}} holds between adjacent elements of 
\colorbox[RGB]{253,246,227}{{{{\color[RGB]{101, 123, 131} a::l }}}}.
\\
\colorbox[RGB]{253,246,227}{\parbox{4.5in}{{{{\color[RGB]{101, 123, 131} chain R a {[}b, c, d{]}  }}}{{{\color[RGB]{181, 137, 0} ↔ }}}{{{\color[RGB]{101, 123, 131}  R a b  }}}{{{\color[RGB]{181, 137, 0} ∧ }}}{{{\color[RGB]{101, 123, 131}  R b c  }}}{{{\color[RGB]{181, 137, 0} ∧ }}}{{{\color[RGB]{101, 123, 131}  R c d }}}\\

}}\paragraph{list.chain'}
\par
\colorbox[RGB]{253,246,227}{{{{\color[RGB]{101, 123, 131} chain' R l }}}} means that 
\colorbox[RGB]{253,246,227}{{{{\color[RGB]{101, 123, 131} R }}}} holds between adjacent elements of 
\colorbox[RGB]{253,246,227}{{{{\color[RGB]{101, 123, 131} l }}}}.
\\
\colorbox[RGB]{253,246,227}{\parbox{4.5in}{{{{\color[RGB]{101, 123, 131} chain' R {[}a, b, c, d{]}  }}}{{{\color[RGB]{181, 137, 0} ↔ }}}{{{\color[RGB]{101, 123, 131}  R a b  }}}{{{\color[RGB]{181, 137, 0} ∧ }}}{{{\color[RGB]{101, 123, 131}  R b c  }}}{{{\color[RGB]{181, 137, 0} ∧ }}}{{{\color[RGB]{101, 123, 131}  R c d }}}\\

}}\paragraph{list.nodup}
\par
\colorbox[RGB]{253,246,227}{{{{\color[RGB]{101, 123, 131} nodup l }}}} means that 
\colorbox[RGB]{253,246,227}{{{{\color[RGB]{101, 123, 131} l }}}} has no duplicates, that is, any element appears at most
once in the list. It is defined as 
\colorbox[RGB]{253,246,227}{{{{\color[RGB]{101, 123, 131} pairwise ( }}}{{{\color[RGB]{181, 137, 0} ≠ }}}{{{\color[RGB]{101, 123, 131} ) }}}}.
\paragraph{list.erase\_dup}
\par
\colorbox[RGB]{253,246,227}{{{{\color[RGB]{101, 123, 131} erase\_dup l }}}} removes duplicates from 
\colorbox[RGB]{253,246,227}{{{{\color[RGB]{101, 123, 131} l }}}} (taking only the first occurrence).
Defined as 
\colorbox[RGB]{253,246,227}{{{{\color[RGB]{101, 123, 131} pw\_filter ( }}}{{{\color[RGB]{181, 137, 0} ≠ }}}{{{\color[RGB]{101, 123, 131} ) }}}}.
\\
\colorbox[RGB]{253,246,227}{\parbox{4.5in}{{{{\color[RGB]{101, 123, 131} erase\_dup {[} }}}{{{\color[RGB]{108, 113, 196} 1 }}}{{{\color[RGB]{101, 123, 131} ,  }}}{{{\color[RGB]{108, 113, 196} 0 }}}{{{\color[RGB]{101, 123, 131} ,  }}}{{{\color[RGB]{108, 113, 196} 2 }}}{{{\color[RGB]{101, 123, 131} ,  }}}{{{\color[RGB]{108, 113, 196} 2 }}}{{{\color[RGB]{101, 123, 131} ,  }}}{{{\color[RGB]{108, 113, 196} 1 }}}{{{\color[RGB]{101, 123, 131} {]}  }}}{{{\color[RGB]{181, 137, 0} = }}}{{{\color[RGB]{101, 123, 131}  {[} }}}{{{\color[RGB]{108, 113, 196} 0 }}}{{{\color[RGB]{101, 123, 131} ,  }}}{{{\color[RGB]{108, 113, 196} 2 }}}{{{\color[RGB]{101, 123, 131} ,  }}}{{{\color[RGB]{108, 113, 196} 1 }}}{{{\color[RGB]{101, 123, 131} {]} }}}\\

}}\paragraph{list.range'}
\par
\colorbox[RGB]{253,246,227}{{{{\color[RGB]{101, 123, 131} range' s n }}}} is the list of numbers 
\colorbox[RGB]{253,246,227}{{{{\color[RGB]{101, 123, 131} {[}s, s }}}{{{\color[RGB]{181, 137, 0} + }}}{{{\color[RGB]{108, 113, 196} 1 }}}{{{\color[RGB]{101, 123, 131} , ..., s }}}{{{\color[RGB]{181, 137, 0} + }}}{{{\color[RGB]{101, 123, 131} n }}}{{{\color[RGB]{181, 137, 0} - }}}{{{\color[RGB]{108, 113, 196} 1 }}}{{{\color[RGB]{101, 123, 131} {]} }}}}.
It is intended mainly for proving properties of 
\colorbox[RGB]{253,246,227}{{{{\color[RGB]{101, 123, 131} range }}}} and 
\colorbox[RGB]{253,246,227}{{{{\color[RGB]{101, 123, 131} iota }}}}.
\paragraph{list.rotate}
\par
\colorbox[RGB]{253,246,227}{{{{\color[RGB]{101, 123, 131} rotate l n }}}} rotates the elements of 
\colorbox[RGB]{253,246,227}{{{{\color[RGB]{101, 123, 131} l }}}} to the left by 
\colorbox[RGB]{253,246,227}{{{{\color[RGB]{101, 123, 131} n }}}}\\
\colorbox[RGB]{253,246,227}{\parbox{4.5in}{{{{\color[RGB]{101, 123, 131} rotate {[} }}}{{{\color[RGB]{108, 113, 196} 0 }}}{{{\color[RGB]{101, 123, 131} ,  }}}{{{\color[RGB]{108, 113, 196} 1 }}}{{{\color[RGB]{101, 123, 131} ,  }}}{{{\color[RGB]{108, 113, 196} 2 }}}{{{\color[RGB]{101, 123, 131} ,  }}}{{{\color[RGB]{108, 113, 196} 3 }}}{{{\color[RGB]{101, 123, 131} ,  }}}{{{\color[RGB]{108, 113, 196} 4 }}}{{{\color[RGB]{101, 123, 131} ,  }}}{{{\color[RGB]{108, 113, 196} 5 }}}{{{\color[RGB]{101, 123, 131} {]}  }}}{{{\color[RGB]{108, 113, 196} 2 }}}{{{\color[RGB]{101, 123, 131}   }}}{{{\color[RGB]{181, 137, 0} = }}}{{{\color[RGB]{101, 123, 131}  {[} }}}{{{\color[RGB]{108, 113, 196} 2 }}}{{{\color[RGB]{101, 123, 131} ,  }}}{{{\color[RGB]{108, 113, 196} 3 }}}{{{\color[RGB]{101, 123, 131} ,  }}}{{{\color[RGB]{108, 113, 196} 4 }}}{{{\color[RGB]{101, 123, 131} ,  }}}{{{\color[RGB]{108, 113, 196} 5 }}}{{{\color[RGB]{101, 123, 131} ,  }}}{{{\color[RGB]{108, 113, 196} 0 }}}{{{\color[RGB]{101, 123, 131} ,  }}}{{{\color[RGB]{108, 113, 196} 1 }}}{{{\color[RGB]{101, 123, 131} {]} }}}\\

}}\paragraph{list.rotate'}
\par
rotate' is the same as 
\colorbox[RGB]{253,246,227}{{{{\color[RGB]{101, 123, 131} rotate }}}}, but slower. Used for proofs about 
\colorbox[RGB]{253,246,227}{{{{\color[RGB]{101, 123, 131} rotate }}}}\section{list/min\_max.lean}\section{list/perm.lean}\paragraph{list.perm}
\par
\colorbox[RGB]{253,246,227}{{{{\color[RGB]{101, 123, 131} perm l₁ l₂ }}}} or 
\colorbox[RGB]{253,246,227}{{{{\color[RGB]{101, 123, 131} l₁ \textasciitilde{} l₂ }}}} asserts that 
\colorbox[RGB]{253,246,227}{{{{\color[RGB]{101, 123, 131} l₁ }}}} and 
\colorbox[RGB]{253,246,227}{{{{\color[RGB]{101, 123, 131} l₂ }}}} are permutations
of each other. This is defined by induction using pairwise swaps.
\paragraph{list.subperm}
\par
\colorbox[RGB]{253,246,227}{{{{\color[RGB]{101, 123, 131} subperm l₁ l₂ }}}}, denoted 
\colorbox[RGB]{253,246,227}{{{{\color[RGB]{101, 123, 131} l₁  }}}{{{\color[RGB]{181, 137, 0} < }}}{{{\color[RGB]{181, 137, 0} + }}}{{{\color[RGB]{101, 123, 131} \textasciitilde{} l₂ }}}}, means that 
\colorbox[RGB]{253,246,227}{{{{\color[RGB]{101, 123, 131} l₁ }}}} is a sublist of
a permutation of 
\colorbox[RGB]{253,246,227}{{{{\color[RGB]{101, 123, 131} l₂ }}}}. This is an analogue of 
\colorbox[RGB]{253,246,227}{{{{\color[RGB]{101, 123, 131} l₁ ⊆ l₂ }}}} which respects
multiplicities of elements, and is used for the 
\colorbox[RGB]{253,246,227}{{{{\color[RGB]{181, 137, 0} ≤ }}}} relation on multisets.
\section{list/sigma.lean}\paragraph{list.keys}
\par
List of keys from a list of key-value pairs
\paragraph{list.lookup}
\par
\colorbox[RGB]{253,246,227}{{{{\color[RGB]{101, 123, 131} lookup a l }}}} is the first value in 
\colorbox[RGB]{253,246,227}{{{{\color[RGB]{101, 123, 131} l }}}} corresponding to the key 
\colorbox[RGB]{253,246,227}{{{{\color[RGB]{101, 123, 131} a }}}},
or 
\colorbox[RGB]{253,246,227}{{{{\color[RGB]{101, 123, 131} none }}}} if no such element exists.
\paragraph{list.lookup\_all}
\par
\colorbox[RGB]{253,246,227}{{{{\color[RGB]{101, 123, 131} lookup\_all a l }}}} is the list of all values in 
\colorbox[RGB]{253,246,227}{{{{\color[RGB]{101, 123, 131} l }}}} corresponding to the key 
\colorbox[RGB]{253,246,227}{{{{\color[RGB]{101, 123, 131} a }}}}.
\paragraph{list.kerase}
\par
Remove the first pair with the key 
\colorbox[RGB]{253,246,227}{{{{\color[RGB]{101, 123, 131} a }}}}.
\paragraph{list.kinsert}
\par
Insert the pair 
\colorbox[RGB]{253,246,227}{{{{\color[RGB]{101, 123, 131} ⟨a, b⟩ }}}} and erase the first pair with the key 
\colorbox[RGB]{253,246,227}{{{{\color[RGB]{101, 123, 131} a }}}}.
\paragraph{list.kunion}
\par
\colorbox[RGB]{253,246,227}{{{{\color[RGB]{101, 123, 131} kunion l₁ l₂ }}}} is the append to l₁ of l₂ after, for each key in l₁, the
first matching pair in l₂ is erased.
\section{list/sort.lean}\paragraph{list.sorted}
\par
\colorbox[RGB]{253,246,227}{{{{\color[RGB]{101, 123, 131} sorted r l }}}} is the same as 
\colorbox[RGB]{253,246,227}{{{{\color[RGB]{101, 123, 131} pairwise r l }}}}, preferred in the case that 
\colorbox[RGB]{253,246,227}{{{{\color[RGB]{101, 123, 131} r }}}}is a 
\colorbox[RGB]{253,246,227}{{{{\color[RGB]{181, 137, 0} < }}}} or 
\colorbox[RGB]{253,246,227}{{{{\color[RGB]{181, 137, 0} ≤ }}}}-like relation (transitive and antisymmetric or asymmetric)
\paragraph{list.ordered\_insert}
\par
\colorbox[RGB]{253,246,227}{{{{\color[RGB]{101, 123, 131} ordered\_insert a l }}}} inserts 
\colorbox[RGB]{253,246,227}{{{{\color[RGB]{101, 123, 131} a }}}} into 
\colorbox[RGB]{253,246,227}{{{{\color[RGB]{101, 123, 131} l }}}} at such that
\colorbox[RGB]{253,246,227}{{{{\color[RGB]{101, 123, 131} ordered\_insert a l }}}} is sorted if 
\colorbox[RGB]{253,246,227}{{{{\color[RGB]{101, 123, 131} l }}}} is.
\paragraph{list.insertion\_sort}
\par
\colorbox[RGB]{253,246,227}{{{{\color[RGB]{101, 123, 131} insertion\_sort l }}}} returns 
\colorbox[RGB]{253,246,227}{{{{\color[RGB]{101, 123, 131} l }}}} sorted using the insertion sort algorithm.
\paragraph{list.split}
\par
Split 
\colorbox[RGB]{253,246,227}{{{{\color[RGB]{101, 123, 131} l }}}} into two lists of approximately equal length.
\\
\colorbox[RGB]{253,246,227}{\parbox{4.5in}{{{{\color[RGB]{101, 123, 131} split {[} }}}{{{\color[RGB]{108, 113, 196} 1 }}}{{{\color[RGB]{101, 123, 131} ,  }}}{{{\color[RGB]{108, 113, 196} 2 }}}{{{\color[RGB]{101, 123, 131} ,  }}}{{{\color[RGB]{108, 113, 196} 3 }}}{{{\color[RGB]{101, 123, 131} ,  }}}{{{\color[RGB]{108, 113, 196} 4 }}}{{{\color[RGB]{101, 123, 131} ,  }}}{{{\color[RGB]{108, 113, 196} 5 }}}{{{\color[RGB]{101, 123, 131} {]}  }}}{{{\color[RGB]{181, 137, 0} = }}}{{{\color[RGB]{101, 123, 131}  ({[} }}}{{{\color[RGB]{108, 113, 196} 1 }}}{{{\color[RGB]{101, 123, 131} ,  }}}{{{\color[RGB]{108, 113, 196} 3 }}}{{{\color[RGB]{101, 123, 131} ,  }}}{{{\color[RGB]{108, 113, 196} 5 }}}{{{\color[RGB]{101, 123, 131} {]}, {[} }}}{{{\color[RGB]{108, 113, 196} 2 }}}{{{\color[RGB]{101, 123, 131} ,  }}}{{{\color[RGB]{108, 113, 196} 4 }}}{{{\color[RGB]{101, 123, 131} {]}) }}}\\

}}\paragraph{list.merge}
\par
Merge two sorted lists into one in linear time.
\\
\colorbox[RGB]{253,246,227}{\parbox{4.5in}{{{{\color[RGB]{101, 123, 131} merge {[} }}}{{{\color[RGB]{108, 113, 196} 1 }}}{{{\color[RGB]{101, 123, 131} ,  }}}{{{\color[RGB]{108, 113, 196} 2 }}}{{{\color[RGB]{101, 123, 131} ,  }}}{{{\color[RGB]{108, 113, 196} 4 }}}{{{\color[RGB]{101, 123, 131} ,  }}}{{{\color[RGB]{108, 113, 196} 5 }}}{{{\color[RGB]{101, 123, 131} {]} {[} }}}{{{\color[RGB]{108, 113, 196} 0 }}}{{{\color[RGB]{101, 123, 131} ,  }}}{{{\color[RGB]{108, 113, 196} 1 }}}{{{\color[RGB]{101, 123, 131} ,  }}}{{{\color[RGB]{108, 113, 196} 3 }}}{{{\color[RGB]{101, 123, 131} ,  }}}{{{\color[RGB]{108, 113, 196} 4 }}}{{{\color[RGB]{101, 123, 131} {]}  }}}{{{\color[RGB]{181, 137, 0} = }}}{{{\color[RGB]{101, 123, 131}  {[} }}}{{{\color[RGB]{108, 113, 196} 0 }}}{{{\color[RGB]{101, 123, 131} ,  }}}{{{\color[RGB]{108, 113, 196} 1 }}}{{{\color[RGB]{101, 123, 131} ,  }}}{{{\color[RGB]{108, 113, 196} 1 }}}{{{\color[RGB]{101, 123, 131} ,  }}}{{{\color[RGB]{108, 113, 196} 2 }}}{{{\color[RGB]{101, 123, 131} ,  }}}{{{\color[RGB]{108, 113, 196} 3 }}}{{{\color[RGB]{101, 123, 131} ,  }}}{{{\color[RGB]{108, 113, 196} 4 }}}{{{\color[RGB]{101, 123, 131} ,  }}}{{{\color[RGB]{108, 113, 196} 4 }}}{{{\color[RGB]{101, 123, 131} ,  }}}{{{\color[RGB]{108, 113, 196} 5 }}}{{{\color[RGB]{101, 123, 131} {]} }}}\\

}}\paragraph{list.merge\_sort}
\par
Implementation of a merge sort algorithm to sort a list.
\section{matrix.lean}\section{mllist.lean}\section{multiset.lean}\paragraph{multiset}
\par
\colorbox[RGB]{253,246,227}{{{{\color[RGB]{101, 123, 131} multiset α }}}} is the quotient of 
\colorbox[RGB]{253,246,227}{{{{\color[RGB]{101, 123, 131} list α }}}} by list permutation. The result
is a type of finite sets with duplicates allowed.
\paragraph{multiset.zero}
\par
\colorbox[RGB]{253,246,227}{{{{\color[RGB]{108, 113, 196} 0 }}}{{{\color[RGB]{101, 123, 131}  : multiset α }}}} is the empty set
\paragraph{multiset.cons}
\par
\colorbox[RGB]{253,246,227}{{{{\color[RGB]{101, 123, 131} cons a s }}}} is the multiset which contains 
\colorbox[RGB]{253,246,227}{{{{\color[RGB]{101, 123, 131} s }}}} plus one more
instance of 
\colorbox[RGB]{253,246,227}{{{{\color[RGB]{101, 123, 131} a }}}}.
\paragraph{multiset.rec}
\par
Dependent recursor on multisets.
\par
TODO: should be @
{[}
recursor 6
{]}
, but then the definition of 
\colorbox[RGB]{253,246,227}{{{{\color[RGB]{101, 123, 131} multiset.pi }}}} failes with a stack
overflow in 
\colorbox[RGB]{253,246,227}{{{{\color[RGB]{101, 123, 131} whnf }}}}.
\paragraph{multiset.mem}
\par
\colorbox[RGB]{253,246,227}{{{{\color[RGB]{101, 123, 131} a ∈ s }}}} means that 
\colorbox[RGB]{253,246,227}{{{{\color[RGB]{101, 123, 131} a }}}} has nonzero multiplicity in 
\colorbox[RGB]{253,246,227}{{{{\color[RGB]{101, 123, 131} s }}}}.
\paragraph{multiset.subset}
\par
\colorbox[RGB]{253,246,227}{{{{\color[RGB]{101, 123, 131} s ⊆ t }}}} is the lift of the list subset relation. It means that any
element with nonzero multiplicity in 
\colorbox[RGB]{253,246,227}{{{{\color[RGB]{101, 123, 131} s }}}} has nonzero multiplicity in 
\colorbox[RGB]{253,246,227}{{{{\color[RGB]{101, 123, 131} t }}}},
but it does not imply that the multiplicity of 
\colorbox[RGB]{253,246,227}{{{{\color[RGB]{101, 123, 131} a }}}} in 
\colorbox[RGB]{253,246,227}{{{{\color[RGB]{101, 123, 131} s }}}} is less or equal than in 
\colorbox[RGB]{253,246,227}{{{{\color[RGB]{101, 123, 131} t }}}};
see 
\colorbox[RGB]{253,246,227}{{{{\color[RGB]{101, 123, 131} s  }}}{{{\color[RGB]{181, 137, 0} ≤ }}}{{{\color[RGB]{101, 123, 131}  t }}}} for this relation.
\paragraph{multiset.le}
\par
\colorbox[RGB]{253,246,227}{{{{\color[RGB]{101, 123, 131} s  }}}{{{\color[RGB]{181, 137, 0} ≤ }}}{{{\color[RGB]{101, 123, 131}  t }}}} means that 
\colorbox[RGB]{253,246,227}{{{{\color[RGB]{101, 123, 131} s }}}} is a sublist of 
\colorbox[RGB]{253,246,227}{{{{\color[RGB]{101, 123, 131} t }}}} (up to permutation).
Equivalently, 
\colorbox[RGB]{253,246,227}{{{{\color[RGB]{101, 123, 131} s  }}}{{{\color[RGB]{181, 137, 0} ≤ }}}{{{\color[RGB]{101, 123, 131}  t }}}} means that 
\colorbox[RGB]{253,246,227}{{{{\color[RGB]{101, 123, 131} count a s  }}}{{{\color[RGB]{181, 137, 0} ≤ }}}{{{\color[RGB]{101, 123, 131}  count a t }}}} for all 
\colorbox[RGB]{253,246,227}{{{{\color[RGB]{101, 123, 131} a }}}}.
\paragraph{multiset.card}
\par
The cardinality of a multiset is the sum of the multiplicities
of all its elements, or simply the length of the underlying list.
\paragraph{multiset.add}
\par
The sum of two multisets is the lift of the list append operation.
This adds the multiplicities of each element,
i.e. 
\colorbox[RGB]{253,246,227}{{{{\color[RGB]{101, 123, 131} count a (s  }}}{{{\color[RGB]{181, 137, 0} + }}}{{{\color[RGB]{101, 123, 131}  t)  }}}{{{\color[RGB]{181, 137, 0} = }}}{{{\color[RGB]{101, 123, 131}  count a s  }}}{{{\color[RGB]{181, 137, 0} + }}}{{{\color[RGB]{101, 123, 131}  count a t }}}}.
\paragraph{multiset.repeat}
\par
\colorbox[RGB]{253,246,227}{{{{\color[RGB]{101, 123, 131} repeat a n }}}} is the multiset containing only 
\colorbox[RGB]{253,246,227}{{{{\color[RGB]{101, 123, 131} a }}}} with multiplicity 
\colorbox[RGB]{253,246,227}{{{{\color[RGB]{101, 123, 131} n }}}}.
\paragraph{multiset.range}
\par
\colorbox[RGB]{253,246,227}{{{{\color[RGB]{101, 123, 131} range n }}}} is the multiset lifted from the list 
\colorbox[RGB]{253,246,227}{{{{\color[RGB]{101, 123, 131} range n }}}},
that is, the set 
\colorbox[RGB]{253,246,227}{{{{\color[RGB]{101, 123, 131} \{ }}}{{{\color[RGB]{108, 113, 196} 0 }}}{{{\color[RGB]{101, 123, 131} ,  }}}{{{\color[RGB]{108, 113, 196} 1 }}}{{{\color[RGB]{101, 123, 131} , ..., n }}}{{{\color[RGB]{181, 137, 0} - }}}{{{\color[RGB]{108, 113, 196} 1 }}}{{{\color[RGB]{101, 123, 131} \} }}}}.
\paragraph{multiset.erase}
\par
\colorbox[RGB]{253,246,227}{{{{\color[RGB]{101, 123, 131} erase s a }}}} is the multiset that subtracts 1 from the
multiplicity of 
\colorbox[RGB]{253,246,227}{{{{\color[RGB]{101, 123, 131} a }}}}.
\paragraph{multiset.map}
\par
\colorbox[RGB]{253,246,227}{{{{\color[RGB]{101, 123, 131} map f s }}}} is the lift of the list 
\colorbox[RGB]{253,246,227}{{{{\color[RGB]{101, 123, 131} map }}}} operation. The multiplicity
of 
\colorbox[RGB]{253,246,227}{{{{\color[RGB]{101, 123, 131} b }}}} in 
\colorbox[RGB]{253,246,227}{{{{\color[RGB]{101, 123, 131} map f s }}}} is the number of 
\colorbox[RGB]{253,246,227}{{{{\color[RGB]{101, 123, 131} a ∈ s }}}} (counting multiplicity)
such that 
\colorbox[RGB]{253,246,227}{{{{\color[RGB]{101, 123, 131} f a  }}}{{{\color[RGB]{181, 137, 0} = }}}{{{\color[RGB]{101, 123, 131}  b }}}}.
\paragraph{multiset.foldl}
\par
\colorbox[RGB]{253,246,227}{{{{\color[RGB]{101, 123, 131} foldl f H b s }}}} is the lift of the list operation 
\colorbox[RGB]{253,246,227}{{{{\color[RGB]{101, 123, 131} foldl f b l }}}},
which folds 
\colorbox[RGB]{253,246,227}{{{{\color[RGB]{101, 123, 131} f }}}} over the multiset. It is well defined when 
\colorbox[RGB]{253,246,227}{{{{\color[RGB]{101, 123, 131} f }}}} is right-commutative,
that is, 
\colorbox[RGB]{253,246,227}{{{{\color[RGB]{101, 123, 131} f (f b a₁) a₂  }}}{{{\color[RGB]{181, 137, 0} = }}}{{{\color[RGB]{101, 123, 131}  f (f b a₂) a₁ }}}}.
\paragraph{multiset.foldr}
\par
\colorbox[RGB]{253,246,227}{{{{\color[RGB]{101, 123, 131} foldr f H b s }}}} is the lift of the list operation 
\colorbox[RGB]{253,246,227}{{{{\color[RGB]{101, 123, 131} foldr f b l }}}},
which folds 
\colorbox[RGB]{253,246,227}{{{{\color[RGB]{101, 123, 131} f }}}} over the multiset. It is well defined when 
\colorbox[RGB]{253,246,227}{{{{\color[RGB]{101, 123, 131} f }}}} is left-commutative,
that is, 
\colorbox[RGB]{253,246,227}{{{{\color[RGB]{101, 123, 131} f a₁ (f a₂ b)  }}}{{{\color[RGB]{181, 137, 0} = }}}{{{\color[RGB]{101, 123, 131}  f a₂ (f a₁ b) }}}}.
\paragraph{multiset.prod}
\par
Product of a multiset given a commutative monoid structure on 
\colorbox[RGB]{253,246,227}{{{{\color[RGB]{101, 123, 131} α }}}}.
\colorbox[RGB]{253,246,227}{{{{\color[RGB]{101, 123, 131} prod \{a, b, c\}  }}}{{{\color[RGB]{181, 137, 0} = }}}{{{\color[RGB]{101, 123, 131}  a  }}}{{{\color[RGB]{181, 137, 0} * }}}{{{\color[RGB]{101, 123, 131}  b  }}}{{{\color[RGB]{181, 137, 0} * }}}{{{\color[RGB]{101, 123, 131}  c }}}}\paragraph{multiset.join}
\par
\colorbox[RGB]{253,246,227}{{{{\color[RGB]{101, 123, 131} join S }}}}, where 
\colorbox[RGB]{253,246,227}{{{{\color[RGB]{101, 123, 131} S }}}} is a multiset of multisets, is the lift of the list join
operation, that is, the union of all the sets.
\\
\colorbox[RGB]{253,246,227}{\parbox{4.5in}{{{{\color[RGB]{101, 123, 131} join \{\{ }}}{{{\color[RGB]{108, 113, 196} 1 }}}{{{\color[RGB]{101, 123, 131} ,  }}}{{{\color[RGB]{108, 113, 196} 2 }}}{{{\color[RGB]{101, 123, 131} \}, \{ }}}{{{\color[RGB]{108, 113, 196} 1 }}}{{{\color[RGB]{101, 123, 131} ,  }}}{{{\color[RGB]{108, 113, 196} 2 }}}{{{\color[RGB]{101, 123, 131} \}, \{ }}}{{{\color[RGB]{108, 113, 196} 0 }}}{{{\color[RGB]{101, 123, 131} ,  }}}{{{\color[RGB]{108, 113, 196} 1 }}}{{{\color[RGB]{101, 123, 131} \}\}  }}}{{{\color[RGB]{181, 137, 0} = }}}{{{\color[RGB]{101, 123, 131}  \{ }}}{{{\color[RGB]{108, 113, 196} 0 }}}{{{\color[RGB]{101, 123, 131} ,  }}}{{{\color[RGB]{108, 113, 196} 1 }}}{{{\color[RGB]{101, 123, 131} ,  }}}{{{\color[RGB]{108, 113, 196} 1 }}}{{{\color[RGB]{101, 123, 131} ,  }}}{{{\color[RGB]{108, 113, 196} 1 }}}{{{\color[RGB]{101, 123, 131} ,  }}}{{{\color[RGB]{108, 113, 196} 2 }}}{{{\color[RGB]{101, 123, 131} ,  }}}{{{\color[RGB]{108, 113, 196} 2 }}}{{{\color[RGB]{101, 123, 131} \} }}}\\

}}\paragraph{multiset.bind}
\par
\colorbox[RGB]{253,246,227}{{{{\color[RGB]{101, 123, 131} bind s f }}}} is the monad bind operation, defined as 
\colorbox[RGB]{253,246,227}{{{{\color[RGB]{101, 123, 131} join (map f s) }}}}.
It is the union of 
\colorbox[RGB]{253,246,227}{{{{\color[RGB]{101, 123, 131} f a }}}} as 
\colorbox[RGB]{253,246,227}{{{{\color[RGB]{101, 123, 131} a }}}} ranges over 
\colorbox[RGB]{253,246,227}{{{{\color[RGB]{101, 123, 131} s }}}}.
\paragraph{multiset.product}
\par
The multiplicity of 
\colorbox[RGB]{253,246,227}{{{{\color[RGB]{101, 123, 131} (a, b) }}}} in 
\colorbox[RGB]{253,246,227}{{{{\color[RGB]{101, 123, 131} product s t }}}} is
the product of the multiplicity of 
\colorbox[RGB]{253,246,227}{{{{\color[RGB]{101, 123, 131} a }}}} in 
\colorbox[RGB]{253,246,227}{{{{\color[RGB]{101, 123, 131} s }}}} and 
\colorbox[RGB]{253,246,227}{{{{\color[RGB]{101, 123, 131} b }}}} in 
\colorbox[RGB]{253,246,227}{{{{\color[RGB]{101, 123, 131} t }}}}.
\paragraph{multiset.sigma}
\par
\colorbox[RGB]{253,246,227}{{{{\color[RGB]{101, 123, 131} sigma s t }}}} is the dependent version of 
\colorbox[RGB]{253,246,227}{{{{\color[RGB]{101, 123, 131} product }}}}. It is the sum of
\colorbox[RGB]{253,246,227}{{{{\color[RGB]{101, 123, 131} (a, b) }}}} as 
\colorbox[RGB]{253,246,227}{{{{\color[RGB]{101, 123, 131} a }}}} ranges over 
\colorbox[RGB]{253,246,227}{{{{\color[RGB]{101, 123, 131} s }}}} and 
\colorbox[RGB]{253,246,227}{{{{\color[RGB]{101, 123, 131} b }}}} ranges over 
\colorbox[RGB]{253,246,227}{{{{\color[RGB]{101, 123, 131} t a }}}}.
\paragraph{multiset.pmap}
\par
Lift of the list 
\colorbox[RGB]{253,246,227}{{{{\color[RGB]{101, 123, 131} pmap }}}} operation. Map a partial function 
\colorbox[RGB]{253,246,227}{{{{\color[RGB]{101, 123, 131} f }}}} over a multiset
\colorbox[RGB]{253,246,227}{{{{\color[RGB]{101, 123, 131} s }}}} whose elements are all in the domain of 
\colorbox[RGB]{253,246,227}{{{{\color[RGB]{101, 123, 131} f }}}}.
\paragraph{multiset.attach}
\par
"Attach" a proof that 
\colorbox[RGB]{253,246,227}{{{{\color[RGB]{101, 123, 131} a ∈ s }}}} to each element 
\colorbox[RGB]{253,246,227}{{{{\color[RGB]{101, 123, 131} a }}}} in 
\colorbox[RGB]{253,246,227}{{{{\color[RGB]{101, 123, 131} s }}}} to produce
a multiset on 
\colorbox[RGB]{253,246,227}{{{{\color[RGB]{101, 123, 131} \{x  }}}{{{\color[RGB]{181, 137, 0} / }}}{{{\color[RGB]{181, 137, 0} / }}}{{{\color[RGB]{101, 123, 131}  x ∈ s\} }}}}.
\paragraph{multiset.decidable\_eq\_pi\_multiset}
\par
decidable equality for functions whose domain is bounded by multisets
\paragraph{multiset.sub}
\par
\colorbox[RGB]{253,246,227}{{{{\color[RGB]{101, 123, 131} s  }}}{{{\color[RGB]{181, 137, 0} - }}}{{{\color[RGB]{101, 123, 131}  t }}}} is the multiset such that
\colorbox[RGB]{253,246,227}{{{{\color[RGB]{101, 123, 131} count a (s  }}}{{{\color[RGB]{181, 137, 0} - }}}{{{\color[RGB]{101, 123, 131}  t)  }}}{{{\color[RGB]{181, 137, 0} = }}}{{{\color[RGB]{101, 123, 131}  count a s  }}}{{{\color[RGB]{181, 137, 0} - }}}{{{\color[RGB]{101, 123, 131}  count a t }}}} for all 
\colorbox[RGB]{253,246,227}{{{{\color[RGB]{101, 123, 131} a }}}}.
\paragraph{multiset.union}
\par
\colorbox[RGB]{253,246,227}{{{{\color[RGB]{101, 123, 131} s ∪ t }}}} is the lattice join operation with respect to the
multiset 
\colorbox[RGB]{253,246,227}{{{{\color[RGB]{181, 137, 0} ≤ }}}}. The multiplicity of 
\colorbox[RGB]{253,246,227}{{{{\color[RGB]{101, 123, 131} a }}}} in 
\colorbox[RGB]{253,246,227}{{{{\color[RGB]{101, 123, 131} s ∪ t }}}} is the maximum
of the multiplicities in 
\colorbox[RGB]{253,246,227}{{{{\color[RGB]{101, 123, 131} s }}}} and 
\colorbox[RGB]{253,246,227}{{{{\color[RGB]{101, 123, 131} t }}}}.
\paragraph{multiset.inter}
\par
\colorbox[RGB]{253,246,227}{{{{\color[RGB]{101, 123, 131} s ∩ t }}}} is the lattice meet operation with respect to the
multiset 
\colorbox[RGB]{253,246,227}{{{{\color[RGB]{181, 137, 0} ≤ }}}}. The multiplicity of 
\colorbox[RGB]{253,246,227}{{{{\color[RGB]{101, 123, 131} a }}}} in 
\colorbox[RGB]{253,246,227}{{{{\color[RGB]{101, 123, 131} s ∩ t }}}} is the minimum
of the multiplicities in 
\colorbox[RGB]{253,246,227}{{{{\color[RGB]{101, 123, 131} s }}}} and 
\colorbox[RGB]{253,246,227}{{{{\color[RGB]{101, 123, 131} t }}}}.
\paragraph{multiset.filter}
\par
\colorbox[RGB]{253,246,227}{{{{\color[RGB]{101, 123, 131} filter p s }}}} returns the elements in 
\colorbox[RGB]{253,246,227}{{{{\color[RGB]{101, 123, 131} s }}}} (with the same multiplicities)
which satisfy 
\colorbox[RGB]{253,246,227}{{{{\color[RGB]{101, 123, 131} p }}}}, and removes the rest.
\paragraph{multiset.filter\_map}
\par
\colorbox[RGB]{253,246,227}{{{{\color[RGB]{101, 123, 131} filter\_map f s }}}} is a combination filter/map operation on 
\colorbox[RGB]{253,246,227}{{{{\color[RGB]{101, 123, 131} s }}}}.
The function 
\colorbox[RGB]{253,246,227}{{{{\color[RGB]{101, 123, 131} f : α  }}}{{{\color[RGB]{133, 153, 0} → }}}{{{\color[RGB]{101, 123, 131}  option β }}}} is applied to each element of 
\colorbox[RGB]{253,246,227}{{{{\color[RGB]{101, 123, 131} s }}}};
if 
\colorbox[RGB]{253,246,227}{{{{\color[RGB]{101, 123, 131} f a }}}} is 
\colorbox[RGB]{253,246,227}{{{{\color[RGB]{101, 123, 131} some b }}}} then 
\colorbox[RGB]{253,246,227}{{{{\color[RGB]{101, 123, 131} b }}}} is added to the result, otherwise
\colorbox[RGB]{253,246,227}{{{{\color[RGB]{101, 123, 131} a }}}} is removed from the resulting multiset.
\paragraph{multiset.countp}
\par
\colorbox[RGB]{253,246,227}{{{{\color[RGB]{101, 123, 131} countp p s }}}} counts the number of elements of 
\colorbox[RGB]{253,246,227}{{{{\color[RGB]{101, 123, 131} s }}}} (with multiplicity) that
satisfy 
\colorbox[RGB]{253,246,227}{{{{\color[RGB]{101, 123, 131} p }}}}.
\paragraph{multiset.count}
\par
\colorbox[RGB]{253,246,227}{{{{\color[RGB]{101, 123, 131} count a s }}}} is the multiplicity of 
\colorbox[RGB]{253,246,227}{{{{\color[RGB]{101, 123, 131} a }}}} in 
\colorbox[RGB]{253,246,227}{{{{\color[RGB]{101, 123, 131} s }}}}.
\paragraph{multiset.rel}
\par
\colorbox[RGB]{253,246,227}{{{{\color[RGB]{101, 123, 131} rel r s t }}}} -{}- lift the relation 
\colorbox[RGB]{253,246,227}{{{{\color[RGB]{101, 123, 131} r }}}} between two elements to a relation between 
\colorbox[RGB]{253,246,227}{{{{\color[RGB]{101, 123, 131} s }}}} and 
\colorbox[RGB]{253,246,227}{{{{\color[RGB]{101, 123, 131} t }}}},
s.t. there is a one-to-one mapping betweem elements in 
\colorbox[RGB]{253,246,227}{{{{\color[RGB]{101, 123, 131} s }}}} and 
\colorbox[RGB]{253,246,227}{{{{\color[RGB]{101, 123, 131} t }}}} following 
\colorbox[RGB]{253,246,227}{{{{\color[RGB]{101, 123, 131} r }}}}.
\paragraph{multiset.disjoint}
\par
\colorbox[RGB]{253,246,227}{{{{\color[RGB]{101, 123, 131} disjoint s t }}}} means that 
\colorbox[RGB]{253,246,227}{{{{\color[RGB]{101, 123, 131} s }}}} and 
\colorbox[RGB]{253,246,227}{{{{\color[RGB]{101, 123, 131} t }}}} have no elements in common.
\paragraph{multiset.pairwise}
\par
\colorbox[RGB]{253,246,227}{{{{\color[RGB]{101, 123, 131} pairwise r m }}}} states that there exists a list of the elements s.t. 
\colorbox[RGB]{253,246,227}{{{{\color[RGB]{101, 123, 131} r }}}} holds pairwise on this list.
\paragraph{multiset.nodup}
\par
\colorbox[RGB]{253,246,227}{{{{\color[RGB]{101, 123, 131} nodup s }}}} means that 
\colorbox[RGB]{253,246,227}{{{{\color[RGB]{101, 123, 131} s }}}} has no duplicates, i.e. the multiplicity of
any element is at most 1.
\paragraph{multiset.erase\_dup}
\par
\colorbox[RGB]{253,246,227}{{{{\color[RGB]{101, 123, 131} erase\_dup s }}}} removes duplicates from 
\colorbox[RGB]{253,246,227}{{{{\color[RGB]{101, 123, 131} s }}}}, yielding a 
\colorbox[RGB]{253,246,227}{{{{\color[RGB]{101, 123, 131} nodup }}}} multiset.
\paragraph{multiset.ndinsert}
\par
\colorbox[RGB]{253,246,227}{{{{\color[RGB]{101, 123, 131} ndinsert a s }}}} is the lift of the list 
\colorbox[RGB]{253,246,227}{{{{\color[RGB]{101, 123, 131} insert }}}} operation. This operation
does not respect multiplicities, unlike 
\colorbox[RGB]{253,246,227}{{{{\color[RGB]{101, 123, 131} cons }}}}, but it is suitable as
an insert operation on 
\colorbox[RGB]{253,246,227}{{{{\color[RGB]{101, 123, 131} finset }}}}.
\paragraph{multiset.ndunion}
\par
\colorbox[RGB]{253,246,227}{{{{\color[RGB]{101, 123, 131} ndunion s t }}}} is the lift of the list 
\colorbox[RGB]{253,246,227}{{{{\color[RGB]{101, 123, 131} union }}}} operation. This operation
does not respect multiplicities, unlike 
\colorbox[RGB]{253,246,227}{{{{\color[RGB]{101, 123, 131} s ∪ t }}}}, but it is suitable as
a union operation on 
\colorbox[RGB]{253,246,227}{{{{\color[RGB]{101, 123, 131} finset }}}}. (
\colorbox[RGB]{253,246,227}{{{{\color[RGB]{101, 123, 131} s ∪ t }}}} would also work as a union operation
on finset, but this is more efficient.)
\paragraph{multiset.ndinter}
\par
\colorbox[RGB]{253,246,227}{{{{\color[RGB]{101, 123, 131} ndinter s t }}}} is the lift of the list 
\colorbox[RGB]{253,246,227}{{{{\color[RGB]{101, 123, 131} ∩ }}}} operation. This operation
does not respect multiplicities, unlike 
\colorbox[RGB]{253,246,227}{{{{\color[RGB]{101, 123, 131} s ∩ t }}}}, but it is suitable as
an intersection operation on 
\colorbox[RGB]{253,246,227}{{{{\color[RGB]{101, 123, 131} finset }}}}. (
\colorbox[RGB]{253,246,227}{{{{\color[RGB]{101, 123, 131} s ∩ t }}}} would also work as a union operation
on finset, but this is more efficient.)
\paragraph{multiset.fold}
\par
\colorbox[RGB]{253,246,227}{{{{\color[RGB]{101, 123, 131} fold op b s }}}} folds a commutative associative operation 
\colorbox[RGB]{253,246,227}{{{{\color[RGB]{101, 123, 131} op }}}} over
the multiset 
\colorbox[RGB]{253,246,227}{{{{\color[RGB]{101, 123, 131} s }}}}.
\paragraph{multiset.sup}
\par
Supremum of a multiset: 
\colorbox[RGB]{253,246,227}{{{{\color[RGB]{101, 123, 131} sup \{a, b, c\}  }}}{{{\color[RGB]{181, 137, 0} = }}}{{{\color[RGB]{101, 123, 131}  a ⊔ b ⊔ c }}}}\paragraph{multiset.inf}
\par
Infimum of a multiset: 
\colorbox[RGB]{253,246,227}{{{{\color[RGB]{101, 123, 131} inf \{a, b, c\}  }}}{{{\color[RGB]{181, 137, 0} = }}}{{{\color[RGB]{101, 123, 131}  a ⊓ b ⊓ c }}}}\paragraph{multiset.sort}
\par
\colorbox[RGB]{253,246,227}{{{{\color[RGB]{101, 123, 131} sort s }}}} constructs a sorted list from the multiset 
\colorbox[RGB]{253,246,227}{{{{\color[RGB]{101, 123, 131} s }}}}.
(Uses merge sort algorithm.)
\paragraph{multiset.pi}
\par
\colorbox[RGB]{253,246,227}{{{{\color[RGB]{101, 123, 131} pi m t }}}} constructs the Cartesian product over 
\colorbox[RGB]{253,246,227}{{{{\color[RGB]{101, 123, 131} t }}}} indexed by 
\colorbox[RGB]{253,246,227}{{{{\color[RGB]{101, 123, 131} m }}}}.
\paragraph{multiset.Ico}
\par
\colorbox[RGB]{253,246,227}{{{{\color[RGB]{101, 123, 131} Ico n m }}}} is the multiset lifted from the list 
\colorbox[RGB]{253,246,227}{{{{\color[RGB]{101, 123, 131} Ico n m }}}}, e.g. the set 
\colorbox[RGB]{253,246,227}{{{{\color[RGB]{101, 123, 131} \{n, n }}}{{{\color[RGB]{181, 137, 0} + }}}{{{\color[RGB]{108, 113, 196} 1 }}}{{{\color[RGB]{101, 123, 131} , ..., m }}}{{{\color[RGB]{181, 137, 0} - }}}{{{\color[RGB]{108, 113, 196} 1 }}}{{{\color[RGB]{101, 123, 131} \} }}}}.
\section{mv\_polynomial.lean}\paragraph{mv\_polynomial}
\par
Multivariate polynomial, where 
\colorbox[RGB]{253,246,227}{{{{\color[RGB]{101, 123, 131} σ }}}} is the index set of the variables and
\colorbox[RGB]{253,246,227}{{{{\color[RGB]{101, 123, 131} α }}}} is the coefficient ring
\paragraph{mv\_polynomial.monomial}
\par
\colorbox[RGB]{253,246,227}{{{{\color[RGB]{101, 123, 131} monomial s a }}}} is the monomial 
\colorbox[RGB]{253,246,227}{{{{\color[RGB]{101, 123, 131} a  }}}{{{\color[RGB]{181, 137, 0} * }}}{{{\color[RGB]{101, 123, 131}  X\textasciicircum{}s }}}}\paragraph{mv\_polynomial.C}
\par
\colorbox[RGB]{253,246,227}{{{{\color[RGB]{101, 123, 131} C a }}}} is the constant polynomial with value 
\colorbox[RGB]{253,246,227}{{{{\color[RGB]{101, 123, 131} a }}}}\paragraph{mv\_polynomial.X}
\par
\colorbox[RGB]{253,246,227}{{{{\color[RGB]{101, 123, 131} X n }}}} is the polynomial with value X\_n
\paragraph{mv\_polynomial.eval₂}
\par
Evaluate a polynomial 
\colorbox[RGB]{253,246,227}{{{{\color[RGB]{101, 123, 131} p }}}} given a valuation 
\colorbox[RGB]{253,246,227}{{{{\color[RGB]{101, 123, 131} g }}}} of all the variables
and a ring hom 
\colorbox[RGB]{253,246,227}{{{{\color[RGB]{101, 123, 131} f }}}} from the scalar ring to the target
\paragraph{mv\_polynomial.eval}
\par
Evaluate a polynomial 
\colorbox[RGB]{253,246,227}{{{{\color[RGB]{101, 123, 131} p }}}} given a valuation 
\colorbox[RGB]{253,246,227}{{{{\color[RGB]{101, 123, 131} f }}}} of all the variables
\paragraph{mv\_polynomial.map}
\par
\colorbox[RGB]{253,246,227}{{{{\color[RGB]{101, 123, 131} map f p }}}} maps a polynomial 
\colorbox[RGB]{253,246,227}{{{{\color[RGB]{101, 123, 131} p }}}} across a ring hom 
\colorbox[RGB]{253,246,227}{{{{\color[RGB]{101, 123, 131} f }}}}\paragraph{mv\_polynomial.vars}
\par
\colorbox[RGB]{253,246,227}{{{{\color[RGB]{101, 123, 131} vars p }}}} is the set of variables appearing in the polynomial 
\colorbox[RGB]{253,246,227}{{{{\color[RGB]{101, 123, 131} p }}}}\paragraph{mv\_polynomial.degree\_of}
\par
\colorbox[RGB]{253,246,227}{{{{\color[RGB]{101, 123, 131} degree\_of n p }}}} gives the highest power of X\_n that appears in 
\colorbox[RGB]{253,246,227}{{{{\color[RGB]{101, 123, 131} p }}}}\paragraph{mv\_polynomial.total\_degree}
\par
\colorbox[RGB]{253,246,227}{{{{\color[RGB]{101, 123, 131} total\_degree p }}}} gives the maximum |s| over the monomials X\textasciicircum{}s in 
\colorbox[RGB]{253,246,227}{{{{\color[RGB]{101, 123, 131} p }}}}\paragraph{mv\_polynomial.eval₂\_hom\_X}
\par
A ring homomorphism f : Z
{[}
X\_1, X\_2, ...
{]}
 -> R
is determined by the evaluations f(X\_1), f(X\_2), ...
\section{nat/basic.lean}\paragraph{nat.pred\_one\_add}
\par
This ensures that 
\colorbox[RGB]{253,246,227}{{{{\color[RGB]{101, 123, 131} simp }}}} succeeds on 
\colorbox[RGB]{253,246,227}{{{{\color[RGB]{101, 123, 131} pred (n  }}}{{{\color[RGB]{181, 137, 0} + }}}{{{\color[RGB]{101, 123, 131}   }}}{{{\color[RGB]{108, 113, 196} 1 }}}{{{\color[RGB]{101, 123, 131} )  }}}{{{\color[RGB]{181, 137, 0} = }}}{{{\color[RGB]{101, 123, 131}  n }}}}.
\paragraph{nat.ppred}
\par
Partial predecessor operation. Returns 
\colorbox[RGB]{253,246,227}{{{{\color[RGB]{101, 123, 131} ppred n  }}}{{{\color[RGB]{181, 137, 0} = }}}{{{\color[RGB]{101, 123, 131}  some m }}}}if 
\colorbox[RGB]{253,246,227}{{{{\color[RGB]{101, 123, 131} n  }}}{{{\color[RGB]{181, 137, 0} = }}}{{{\color[RGB]{101, 123, 131}  m  }}}{{{\color[RGB]{181, 137, 0} + }}}{{{\color[RGB]{101, 123, 131}   }}}{{{\color[RGB]{108, 113, 196} 1 }}}}, otherwise 
\colorbox[RGB]{253,246,227}{{{{\color[RGB]{101, 123, 131} none }}}}.
\paragraph{nat.psub}
\par
Partial subtraction operation. Returns 
\colorbox[RGB]{253,246,227}{{{{\color[RGB]{101, 123, 131} psub m n  }}}{{{\color[RGB]{181, 137, 0} = }}}{{{\color[RGB]{101, 123, 131}  some k }}}}if 
\colorbox[RGB]{253,246,227}{{{{\color[RGB]{101, 123, 131} m  }}}{{{\color[RGB]{181, 137, 0} = }}}{{{\color[RGB]{101, 123, 131}  n  }}}{{{\color[RGB]{181, 137, 0} + }}}{{{\color[RGB]{101, 123, 131}  k }}}}, otherwise 
\colorbox[RGB]{253,246,227}{{{{\color[RGB]{101, 123, 131} none }}}}.
\paragraph{nat.fact}
\par
\colorbox[RGB]{253,246,227}{{{{\color[RGB]{101, 123, 131} fact n }}}} is the factorial of 
\colorbox[RGB]{253,246,227}{{{{\color[RGB]{101, 123, 131} n }}}}.
\paragraph{nat.find\_greatest}
\par
\colorbox[RGB]{253,246,227}{{{{\color[RGB]{101, 123, 131} find\_greatest P b }}}} is the largest 
\colorbox[RGB]{253,246,227}{{{{\color[RGB]{101, 123, 131} i  }}}{{{\color[RGB]{181, 137, 0} ≤ }}}{{{\color[RGB]{101, 123, 131}  bound }}}} such that 
\colorbox[RGB]{253,246,227}{{{{\color[RGB]{101, 123, 131} P i }}}} holds, or 
\colorbox[RGB]{253,246,227}{{{{\color[RGB]{108, 113, 196} 0 }}}} if no such 
\colorbox[RGB]{253,246,227}{{{{\color[RGB]{101, 123, 131} i }}}}exists
\section{nat/cast.lean}\paragraph{nat.cast}
\par
Canonical homomorphism from 
\colorbox[RGB]{253,246,227}{{{{\color[RGB]{101, 123, 131} ℕ }}}} to a type 
\colorbox[RGB]{253,246,227}{{{{\color[RGB]{101, 123, 131} α }}}} with 
\colorbox[RGB]{253,246,227}{{{{\color[RGB]{108, 113, 196} 0 }}}}, 
\colorbox[RGB]{253,246,227}{{{{\color[RGB]{108, 113, 196} 1 }}}} and 
\colorbox[RGB]{253,246,227}{{{{\color[RGB]{181, 137, 0} + }}}}.
\section{nat/choose.lean}\paragraph{add\_pow}
\par
The binomial theorem
\section{nat/dist.lean}\paragraph{nat.dist}
\par
Distance (absolute value of difference) between natural numbers.
\section{nat/enat.lean}\section{nat/gcd.lean}\section{nat/modeq.lean}\paragraph{nat.modeq}
\par
Modular equality. 
\colorbox[RGB]{253,246,227}{{{{\color[RGB]{101, 123, 131} modeq n a b }}}}, or 
\colorbox[RGB]{253,246,227}{{{{\color[RGB]{101, 123, 131} a ≡ b {[}MOD n{]} }}}}, means
that 
\colorbox[RGB]{253,246,227}{{{{\color[RGB]{101, 123, 131} a  }}}{{{\color[RGB]{181, 137, 0} - }}}{{{\color[RGB]{101, 123, 131}  b }}}} is a multiple of 
\colorbox[RGB]{253,246,227}{{{{\color[RGB]{101, 123, 131} n }}}}.
\section{nat/pairing.lean}\paragraph{nat.mkpair}
\par
Pairing function for the natural numbers.
\paragraph{nat.unpair}
\par
Unpairing function for the natural numbers.
\section{nat/prime.lean}\paragraph{nat.prime}
\par
\colorbox[RGB]{253,246,227}{{{{\color[RGB]{101, 123, 131} prime p }}}} means that 
\colorbox[RGB]{253,246,227}{{{{\color[RGB]{101, 123, 131} p }}}} is a prime number, that is, a natural number
at least 2 whose only divisors are 
\colorbox[RGB]{253,246,227}{{{{\color[RGB]{101, 123, 131} p }}}} and 
\colorbox[RGB]{253,246,227}{{{{\color[RGB]{108, 113, 196} 1 }}}}.
\paragraph{nat.decidable\_prime\_1}
\par
This instance is slower than the instance 
\colorbox[RGB]{253,246,227}{{{{\color[RGB]{101, 123, 131} decidable\_prime }}}} defined below,
but has the advantage that it works in the kernel.
\par
If you need to prove that a particular number is prime, in any case
you should not use 
\colorbox[RGB]{253,246,227}{{{{\color[RGB]{101, 123, 131} dec\_trivial }}}}, but rather 
\colorbox[RGB]{253,246,227}{{{{\color[RGB]{133, 153, 0} by }}}{{{\color[RGB]{101, 123, 131}  norm\_num }}}}, which is
much faster.
\paragraph{nat.min\_fac}
\par
Returns the smallest prime factor of 
\colorbox[RGB]{253,246,227}{{{{\color[RGB]{101, 123, 131} n  }}}{{{\color[RGB]{181, 137, 0} ≠ }}}{{{\color[RGB]{101, 123, 131}   }}}{{{\color[RGB]{108, 113, 196} 1 }}}}.
\paragraph{nat.decidable\_prime}
\par
This instance is faster in the virtual machine than 
\colorbox[RGB]{253,246,227}{{{{\color[RGB]{101, 123, 131} decidable\_prime\_1 }}}},
but slower in the kernel.
\par
If you need to prove that a particular number is prime, in any case
you should not use 
\colorbox[RGB]{253,246,227}{{{{\color[RGB]{101, 123, 131} dec\_trivial }}}}, but rather 
\colorbox[RGB]{253,246,227}{{{{\color[RGB]{133, 153, 0} by }}}{{{\color[RGB]{101, 123, 131}  norm\_num }}}}, which is
much faster.
\paragraph{nat.factors}
\par
\colorbox[RGB]{253,246,227}{{{{\color[RGB]{101, 123, 131} factors n }}}} is the prime factorization of 
\colorbox[RGB]{253,246,227}{{{{\color[RGB]{101, 123, 131} n }}}}, listed in increasing order.
\paragraph{nat.primes}
\par
The type of prime numbers
\section{nat/sqrt.lean}\paragraph{nat.sqrt}
\par
\colorbox[RGB]{253,246,227}{{{{\color[RGB]{101, 123, 131} sqrt n }}}} is the square root of a natural number 
\colorbox[RGB]{253,246,227}{{{{\color[RGB]{101, 123, 131} n }}}}. If 
\colorbox[RGB]{253,246,227}{{{{\color[RGB]{101, 123, 131} n }}}} is not a
perfect square, it returns the largest 
\colorbox[RGB]{253,246,227}{{{{\color[RGB]{101, 123, 131} k:ℕ }}}} such that 
\colorbox[RGB]{253,246,227}{{{{\color[RGB]{101, 123, 131} k }}}{{{\color[RGB]{181, 137, 0} * }}}{{{\color[RGB]{101, 123, 131} k  }}}{{{\color[RGB]{181, 137, 0} ≤ }}}{{{\color[RGB]{101, 123, 131}  n }}}}.
\section{nat/totient.lean}\section{num/basic.lean}\paragraph{pos\_num}
\par
The type of positive binary numbers.
\\
\colorbox[RGB]{253,246,227}{\parbox{4.5in}{{{{\color[RGB]{108, 113, 196} 13 }}}{{{\color[RGB]{101, 123, 131}   }}}{{{\color[RGB]{181, 137, 0} = }}}{{{\color[RGB]{101, 123, 131}   }}}{{{\color[RGB]{108, 113, 196} 1101 }}}{{{\color[RGB]{101, 123, 131} (base  }}}{{{\color[RGB]{108, 113, 196} 2 }}}{{{\color[RGB]{101, 123, 131} )  }}}{{{\color[RGB]{181, 137, 0} = }}}{{{\color[RGB]{101, 123, 131}  bit1 (bit0 (bit1 one)) }}}\\

}}\paragraph{num}
\par
The type of nonnegative binary numbers, using 
\colorbox[RGB]{253,246,227}{{{{\color[RGB]{101, 123, 131} pos\_num }}}}.
\\
\colorbox[RGB]{253,246,227}{\parbox{4.5in}{{{{\color[RGB]{108, 113, 196} 13 }}}{{{\color[RGB]{101, 123, 131}   }}}{{{\color[RGB]{181, 137, 0} = }}}{{{\color[RGB]{101, 123, 131}   }}}{{{\color[RGB]{108, 113, 196} 1101 }}}{{{\color[RGB]{101, 123, 131} (base  }}}{{{\color[RGB]{108, 113, 196} 2 }}}{{{\color[RGB]{101, 123, 131} )  }}}{{{\color[RGB]{181, 137, 0} = }}}{{{\color[RGB]{101, 123, 131}  pos (bit1 (bit0 (bit1 one))) }}}\\

}}\paragraph{znum}
\par
Representation of integers using trichotomy around zero.
\\
\colorbox[RGB]{253,246,227}{\parbox{4.5in}{{{{\color[RGB]{108, 113, 196} 13 }}}{{{\color[RGB]{101, 123, 131}   }}}{{{\color[RGB]{181, 137, 0} = }}}{{{\color[RGB]{101, 123, 131}   }}}{{{\color[RGB]{108, 113, 196} 1101 }}}{{{\color[RGB]{101, 123, 131} (base  }}}{{{\color[RGB]{108, 113, 196} 2 }}}{{{\color[RGB]{101, 123, 131} )  }}}{{{\color[RGB]{181, 137, 0} = }}}{{{\color[RGB]{101, 123, 131}  pos (bit1 (bit0 (bit1 one)))
 }}}\\

{{{\color[RGB]{181, 137, 0} - }}}{{{\color[RGB]{108, 113, 196} 13 }}}{{{\color[RGB]{101, 123, 131}   }}}{{{\color[RGB]{181, 137, 0} = }}}{{{\color[RGB]{101, 123, 131}   }}}{{{\color[RGB]{181, 137, 0} - }}}{{{\color[RGB]{108, 113, 196} 1101 }}}{{{\color[RGB]{101, 123, 131} (base  }}}{{{\color[RGB]{108, 113, 196} 2 }}}{{{\color[RGB]{101, 123, 131} )  }}}{{{\color[RGB]{181, 137, 0} = }}}{{{\color[RGB]{101, 123, 131}  neg (bit1 (bit0 (bit1 one))) }}}\\

}}\paragraph{nzsnum}
\par
See 
\colorbox[RGB]{253,246,227}{{{{\color[RGB]{101, 123, 131} snum }}}}.
\paragraph{snum}
\par
Alternative representation of integers using a sign bit at the end.
The convention on sign here is to have the argument to 
\colorbox[RGB]{253,246,227}{{{{\color[RGB]{101, 123, 131} msb }}}} denote
the sign of the MSB itself, with all higher bits set to the negation
of this sign. The result is interpreted in two's complement.
\\
\colorbox[RGB]{253,246,227}{\parbox{4.5in}{{{{\color[RGB]{108, 113, 196} 13 }}}{{{\color[RGB]{101, 123, 131}    }}}{{{\color[RGB]{181, 137, 0} = }}}{{{\color[RGB]{101, 123, 131}  .. }}}{{{\color[RGB]{108, 113, 196} 0001101 }}}{{{\color[RGB]{101, 123, 131} (base  }}}{{{\color[RGB]{108, 113, 196} 2 }}}{{{\color[RGB]{101, 123, 131} )  }}}{{{\color[RGB]{181, 137, 0} = }}}{{{\color[RGB]{101, 123, 131}  nz (bit1 (bit0 (bit1 (msb tt))))
 }}}\\

{{{\color[RGB]{181, 137, 0} - }}}{{{\color[RGB]{108, 113, 196} 13 }}}{{{\color[RGB]{101, 123, 131}   }}}{{{\color[RGB]{181, 137, 0} = }}}{{{\color[RGB]{101, 123, 131}  .. }}}{{{\color[RGB]{108, 113, 196} 1110011 }}}{{{\color[RGB]{101, 123, 131} (base  }}}{{{\color[RGB]{108, 113, 196} 2 }}}{{{\color[RGB]{101, 123, 131} )  }}}{{{\color[RGB]{181, 137, 0} = }}}{{{\color[RGB]{101, 123, 131}  nz (bit1 (bit1 (bit0 (msb ff))))
 }}}\\

}}\par
As with 
\colorbox[RGB]{253,246,227}{{{{\color[RGB]{101, 123, 131} num }}}}, a special case must be added for zero, which has no msb,
but by two's complement symmetry there is a second special case for -1.
Here the 
\colorbox[RGB]{253,246,227}{{{{\color[RGB]{101, 123, 131} bool }}}} field indicates the sign of the number.
\\
\colorbox[RGB]{253,246,227}{\parbox{4.5in}{{{{\color[RGB]{108, 113, 196} 0 }}}{{{\color[RGB]{101, 123, 131}    }}}{{{\color[RGB]{181, 137, 0} = }}}{{{\color[RGB]{101, 123, 131}  .. }}}{{{\color[RGB]{108, 113, 196} 0000000 }}}{{{\color[RGB]{101, 123, 131} (base  }}}{{{\color[RGB]{108, 113, 196} 2 }}}{{{\color[RGB]{101, 123, 131} )  }}}{{{\color[RGB]{181, 137, 0} = }}}{{{\color[RGB]{101, 123, 131}  zero ff
 }}}\\

{{{\color[RGB]{181, 137, 0} - }}}{{{\color[RGB]{108, 113, 196} 1 }}}{{{\color[RGB]{101, 123, 131}   }}}{{{\color[RGB]{181, 137, 0} = }}}{{{\color[RGB]{101, 123, 131}  .. }}}{{{\color[RGB]{108, 113, 196} 1111111 }}}{{{\color[RGB]{101, 123, 131} (base  }}}{{{\color[RGB]{108, 113, 196} 2 }}}{{{\color[RGB]{101, 123, 131} )  }}}{{{\color[RGB]{181, 137, 0} = }}}{{{\color[RGB]{101, 123, 131}  zero tt }}}\\

}}\section{num/bitwise.lean}\section{num/lemmas.lean}\section{opposite.lean}\paragraph{opposite.opposite}
\par
The type of objects of the opposite of 
\colorbox[RGB]{253,246,227}{{{{\color[RGB]{101, 123, 131} α }}}}; used to defined opposite category/group/...
\par
In order to avoid confusion between 
\colorbox[RGB]{253,246,227}{{{{\color[RGB]{101, 123, 131} α }}}} and its opposite type, we
set up the type of objects 
\colorbox[RGB]{253,246,227}{{{{\color[RGB]{101, 123, 131} opposite α }}}} using the following pattern,
which will be repeated later for the morphisms.
\begin{enumerate}[1]
\item Define 
\colorbox[RGB]{253,246,227}{{{{\color[RGB]{101, 123, 131} opposite α  }}}{{{\color[RGB]{181, 137, 0} := }}}{{{\color[RGB]{101, 123, 131}  α }}}}.

\item Define the isomorphisms 
\colorbox[RGB]{253,246,227}{{{{\color[RGB]{101, 123, 131} op : α  }}}{{{\color[RGB]{133, 153, 0} → }}}{{{\color[RGB]{101, 123, 131}  opposite α }}}}, 
\colorbox[RGB]{253,246,227}{{{{\color[RGB]{101, 123, 131} unop : opposite α  }}}{{{\color[RGB]{133, 153, 0} → }}}{{{\color[RGB]{101, 123, 131}  α }}}}.

\item Make the definition 
\colorbox[RGB]{253,246,227}{{{{\color[RGB]{101, 123, 131} opposite }}}} irreducible.

\end{enumerate}\par
This has the following consequences.
\begin{itemize}\item \colorbox[RGB]{253,246,227}{{{{\color[RGB]{101, 123, 131} opposite α }}}} and 
\colorbox[RGB]{253,246,227}{{{{\color[RGB]{101, 123, 131} α }}}} are distinct types in the elaborator, so you
must use 
\colorbox[RGB]{253,246,227}{{{{\color[RGB]{101, 123, 131} op }}}} and 
\colorbox[RGB]{253,246,227}{{{{\color[RGB]{101, 123, 131} unop }}}} explicitly to convert between them.

\item Both 
\colorbox[RGB]{253,246,227}{{{{\color[RGB]{101, 123, 131} unop (op X)  }}}{{{\color[RGB]{181, 137, 0} = }}}{{{\color[RGB]{101, 123, 131}  X }}}} and 
\colorbox[RGB]{253,246,227}{{{{\color[RGB]{101, 123, 131} op (unop X)  }}}{{{\color[RGB]{181, 137, 0} = }}}{{{\color[RGB]{101, 123, 131}  X }}}} are definitional
equalities. Notably, every object of the opposite category is
definitionally of the form 
\colorbox[RGB]{253,246,227}{{{{\color[RGB]{101, 123, 131} op X }}}}, which greatly simplifies the
definition of the structure of the opposite category, for example.

\end{itemize}\par
(If Lean supported definitional eta equality for records, we could
achieve the same goals using a structure with one field.)
\section{option/basic.lean}\paragraph{option.injective\_map}
\par
\colorbox[RGB]{253,246,227}{{{{\color[RGB]{101, 123, 131} option.map f }}}} is injective if 
\colorbox[RGB]{253,246,227}{{{{\color[RGB]{101, 123, 131} f }}}} is injective.
\section{option/defs.lean}\paragraph{option.iget}
\par
inhabited 
\colorbox[RGB]{253,246,227}{{{{\color[RGB]{101, 123, 131} get }}}} function. Returns 
\colorbox[RGB]{253,246,227}{{{{\color[RGB]{101, 123, 131} a }}}} if the input is 
\colorbox[RGB]{253,246,227}{{{{\color[RGB]{101, 123, 131} some a }}}},
otherwise returns 
\colorbox[RGB]{253,246,227}{{{{\color[RGB]{101, 123, 131} default }}}}.
\paragraph{option.guard}
\par
\colorbox[RGB]{253,246,227}{{{{\color[RGB]{101, 123, 131} guard p a }}}} returns 
\colorbox[RGB]{253,246,227}{{{{\color[RGB]{101, 123, 131} some a }}}} if 
\colorbox[RGB]{253,246,227}{{{{\color[RGB]{101, 123, 131} p a }}}} holds, otherwise 
\colorbox[RGB]{253,246,227}{{{{\color[RGB]{101, 123, 131} none }}}}.
\paragraph{option.filter}
\par
\colorbox[RGB]{253,246,227}{{{{\color[RGB]{101, 123, 131} filter p o }}}} returns 
\colorbox[RGB]{253,246,227}{{{{\color[RGB]{101, 123, 131} some a }}}} if 
\colorbox[RGB]{253,246,227}{{{{\color[RGB]{101, 123, 131} o }}}} is 
\colorbox[RGB]{253,246,227}{{{{\color[RGB]{101, 123, 131} some a }}}}and 
\colorbox[RGB]{253,246,227}{{{{\color[RGB]{101, 123, 131} p a }}}} holds, otherwise 
\colorbox[RGB]{253,246,227}{{{{\color[RGB]{101, 123, 131} none }}}}.
\section{padics/default.lean}\section{padics/hensel.lean}\section{padics/padic\_integers.lean}\section{padics/padic\_norm.lean}\section{padics/padic\_numbers.lean}\paragraph{tactic.interactive.padic\_index\_simp}
\par
This is a special-purpose tactic that lifts padic\_norm (f (stationary\_point f)) to
padic\_norm (f (max \_ \_ 
\_
)).
\section{pfun.lean}\paragraph{roption}
\par
\colorbox[RGB]{253,246,227}{{{{\color[RGB]{101, 123, 131} roption α }}}} is the type of "partial values" of type 
\colorbox[RGB]{253,246,227}{{{{\color[RGB]{101, 123, 131} α }}}}. It
is similar to 
\colorbox[RGB]{253,246,227}{{{{\color[RGB]{101, 123, 131} option α }}}} except the domain condition can be an
arbitrary proposition, not necessarily decidable.
\paragraph{roption.to\_option}
\par
Convert an 
\colorbox[RGB]{253,246,227}{{{{\color[RGB]{101, 123, 131} roption α }}}} with a decidable domain to an option
\paragraph{roption.ext'}
\par
\colorbox[RGB]{253,246,227}{{{{\color[RGB]{101, 123, 131} roption }}}} extensionality
\paragraph{roption.eta}
\par
\colorbox[RGB]{253,246,227}{{{{\color[RGB]{101, 123, 131} roption }}}} eta expansion
\paragraph{roption.mem}
\par
\colorbox[RGB]{253,246,227}{{{{\color[RGB]{101, 123, 131} a ∈ o }}}} means that 
\colorbox[RGB]{253,246,227}{{{{\color[RGB]{101, 123, 131} o }}}} is defined and equal to 
\colorbox[RGB]{253,246,227}{{{{\color[RGB]{101, 123, 131} a }}}}\paragraph{roption.ext}
\par
\colorbox[RGB]{253,246,227}{{{{\color[RGB]{101, 123, 131} roption }}}} extensionality
\paragraph{roption.none}
\par
The 
\colorbox[RGB]{253,246,227}{{{{\color[RGB]{101, 123, 131} none }}}} value in 
\colorbox[RGB]{253,246,227}{{{{\color[RGB]{101, 123, 131} roption }}}} has a 
\colorbox[RGB]{253,246,227}{{{{\color[RGB]{101, 123, 131} false }}}} domain and an empty function.
\paragraph{roption.some}
\par
The 
\colorbox[RGB]{253,246,227}{{{{\color[RGB]{101, 123, 131} some a }}}} value in 
\colorbox[RGB]{253,246,227}{{{{\color[RGB]{101, 123, 131} roption }}}} has a 
\colorbox[RGB]{253,246,227}{{{{\color[RGB]{101, 123, 131} true }}}} domain and the
function returns 
\colorbox[RGB]{253,246,227}{{{{\color[RGB]{101, 123, 131} a }}}}.
\paragraph{roption.of\_option}
\par
Convert an 
\colorbox[RGB]{253,246,227}{{{{\color[RGB]{101, 123, 131} option α }}}} into an 
\colorbox[RGB]{253,246,227}{{{{\color[RGB]{101, 123, 131} roption α }}}}\paragraph{roption.assert}
\par
\colorbox[RGB]{253,246,227}{{{{\color[RGB]{133, 153, 0} assert }}}{{{\color[RGB]{101, 123, 131}  p f }}}} is a bind-like operation which appends an additional condition
\colorbox[RGB]{253,246,227}{{{{\color[RGB]{101, 123, 131} p }}}} to the domain and uses 
\colorbox[RGB]{253,246,227}{{{{\color[RGB]{101, 123, 131} f }}}} to produce the value.
\paragraph{roption.bind}
\par
The bind operation has value 
\colorbox[RGB]{253,246,227}{{{{\color[RGB]{101, 123, 131} g (f.get) }}}}, and is defined when all the
parts are defined.
\paragraph{roption.map}
\par
The map operation for 
\colorbox[RGB]{253,246,227}{{{{\color[RGB]{101, 123, 131} roption }}}} just maps the value and maintains the same domain.
\paragraph{roption.unwrap}
\par
\colorbox[RGB]{253,246,227}{{{{\color[RGB]{101, 123, 131} unwrap o }}}} gets the value at 
\colorbox[RGB]{253,246,227}{{{{\color[RGB]{101, 123, 131} o }}}}, ignoring the condition.
(This function is unsound.)
\paragraph{pfun}
\par
\colorbox[RGB]{253,246,227}{{{{\color[RGB]{101, 123, 131} pfun α β }}}}, or 
\colorbox[RGB]{253,246,227}{{{{\color[RGB]{101, 123, 131} α  }}}{{{\color[RGB]{133, 153, 0} → }}}{{{\color[RGB]{101, 123, 131} . β }}}}, is the type of partial functions from
\colorbox[RGB]{253,246,227}{{{{\color[RGB]{101, 123, 131} α }}}} to 
\colorbox[RGB]{253,246,227}{{{{\color[RGB]{101, 123, 131} β }}}}. It is defined as 
\colorbox[RGB]{253,246,227}{{{{\color[RGB]{101, 123, 131} α  }}}{{{\color[RGB]{133, 153, 0} → }}}{{{\color[RGB]{101, 123, 131}  roption β }}}}.
\paragraph{pfun.dom}
\par
The domain of a partial function
\paragraph{pfun.fn}
\par
Evaluate a partial function
\paragraph{pfun.eval\_opt}
\par
Evaluate a partial function to return an 
\colorbox[RGB]{253,246,227}{{{{\color[RGB]{101, 123, 131} option }}}}\paragraph{pfun.ext'}
\par
Partial function extensionality
\paragraph{pfun.as\_subtype}
\par
Turn a partial function into a function out of a subtype
\paragraph{pfun.lift}
\par
Turn a total function into a partial function
\paragraph{pfun.graph}
\par
The graph of a partial function is the set of pairs
\colorbox[RGB]{253,246,227}{{{{\color[RGB]{101, 123, 131} (x, f x) }}}} where 
\colorbox[RGB]{253,246,227}{{{{\color[RGB]{101, 123, 131} x }}}} is in the domain of 
\colorbox[RGB]{253,246,227}{{{{\color[RGB]{101, 123, 131} f }}}}.
\paragraph{pfun.ran}
\par
The range of a partial function is the set of values
\colorbox[RGB]{253,246,227}{{{{\color[RGB]{101, 123, 131} f x }}}} where 
\colorbox[RGB]{253,246,227}{{{{\color[RGB]{101, 123, 131} x }}}} is in the domain of 
\colorbox[RGB]{253,246,227}{{{{\color[RGB]{101, 123, 131} f }}}}.
\paragraph{pfun.restrict}
\par
Restrict a partial function to a smaller domain.
\paragraph{pfun.pure}
\par
The monad 
\colorbox[RGB]{253,246,227}{{{{\color[RGB]{101, 123, 131} pure }}}} function, the total constant 
\colorbox[RGB]{253,246,227}{{{{\color[RGB]{101, 123, 131} x }}}} function
\paragraph{pfun.bind}
\par
The monad 
\colorbox[RGB]{253,246,227}{{{{\color[RGB]{101, 123, 131} bind }}}} function, pointwise 
\colorbox[RGB]{253,246,227}{{{{\color[RGB]{101, 123, 131} roption.bind }}}}\paragraph{pfun.map}
\par
The monad 
\colorbox[RGB]{253,246,227}{{{{\color[RGB]{101, 123, 131} map }}}} function, pointwise 
\colorbox[RGB]{253,246,227}{{{{\color[RGB]{101, 123, 131} roption.map }}}}\section{pnat/basic.lean}\paragraph{pnat}
\par
\colorbox[RGB]{253,246,227}{{{{\color[RGB]{101, 123, 131} ℕ }}}{{{\color[RGB]{181, 137, 0} + }}}} is the type of positive natural numbers. It is defined as a subtype,
and the VM representation of 
\colorbox[RGB]{253,246,227}{{{{\color[RGB]{101, 123, 131} ℕ }}}{{{\color[RGB]{181, 137, 0} + }}}} is the same as 
\colorbox[RGB]{253,246,227}{{{{\color[RGB]{101, 123, 131} ℕ }}}} because the proof
is not stored.
\paragraph{nat.to\_pnat}
\par
Convert a natural number to a positive natural number. The
positivity assumption is inferred by 
\colorbox[RGB]{253,246,227}{{{{\color[RGB]{101, 123, 131} dec\_trivial }}}}.
\paragraph{nat.succ\_pnat}
\par
Write a successor as an element of 
\colorbox[RGB]{253,246,227}{{{{\color[RGB]{101, 123, 131} ℕ }}}{{{\color[RGB]{181, 137, 0} + }}}}.
\paragraph{nat.to\_pnat'}
\par
Convert a natural number to a pnat. 
\colorbox[RGB]{253,246,227}{{{{\color[RGB]{101, 123, 131} n }}}{{{\color[RGB]{181, 137, 0} + }}}{{{\color[RGB]{108, 113, 196} 1 }}}} is mapped to itself,
and 
\colorbox[RGB]{253,246,227}{{{{\color[RGB]{108, 113, 196} 0 }}}} becomes 
\colorbox[RGB]{253,246,227}{{{{\color[RGB]{108, 113, 196} 1 }}}}.
\paragraph{pnat.decidable\_eq}
\par
We now define a long list of structures on ℕ+ induced by
similar structures on ℕ. Most of these behave in a completely
obvious way, but there are a few things to be said about
subtraction, division and powers.
\paragraph{pnat.has\_sub}
\par
Subtraction a - b is defined in the obvious way when
a > b, and by a - b = 1 if a ≤ b.
\paragraph{pnat.mod\_div\_aux}
\par
We define m \% k and m / k in the same way as for nat
except that when m = n * k we take m \% k = k and
m / k = n - 1.  This ensures that m \% k is always positive
and m = (m \% k) + k * (m / k) in all cases.  Later we
define a function div\_exact which gives the usual m / k
in the case where k divides m.
\section{pnat/factors.lean}\paragraph{prime\_multiset}
\par
The type of multisets of prime numbers.  Unique factorization
gives an equivalence between this set and ℕ+, as we will formalize
below.
\paragraph{prime\_multiset.of\_prime}
\par
The multiset consisting of a single prime
\paragraph{prime\_multiset.to\_nat\_multiset}
\par
We can forget the primality property and regard a multiset
of primes as just a multiset of positive integers, or a multiset
of natural numbers.  In the opposite direction, if we have a
multiset of positive integers or natural numbers, together with
a proof that all the elements are prime, then we can regard it
as a multiset of primes.  The next block of results records
obvious properties of these coercions.
\paragraph{prime\_multiset.of\_nat\_list}
\par
Lists can be coerced to multisets; here we have some results
about how this interacts with our constructions on multisets.
\paragraph{prime\_multiset.prod\_zero}
\par
The product map gives a homomorphism from the additive monoid
of multisets to the multiplicative monoid ℕ+.
\paragraph{pnat.factor\_multiset}
\par
The prime factors of n, regarded as a multiset
\paragraph{pnat.prod\_factor\_multiset}
\par
The product of the factors is the original number
\paragraph{prime\_multiset.factor\_multiset\_prod}
\par
If we start with a multiset of primes, take the product and
then factor it, we get back the original multiset.
\paragraph{pnat.factor\_multiset\_equiv}
\par
Positive integers biject with multisets of primes.
\paragraph{pnat.factor\_multiset\_one}
\par
Factoring gives a homomorphism from the multiplicative
monoid ℕ+ to the additive monoid of multisets.
\paragraph{pnat.factor\_multiset\_of\_prime}
\par
Factoring a prime gives the corresponding one-element multiset.
\paragraph{pnat.factor\_multiset\_le\_iff}
\par
We now have four different results that all encode the
idea that inequality of multisets corresponds to divisibility
of positive integers.
\paragraph{pnat.factor\_multiset\_gcd}
\par
The gcd and lcm operations on positive integers correspond
to the inf and sup operations on multisets.
\paragraph{pnat.count\_factor\_multiset}
\par
The number of occurrences of p in the factor multiset of m
is the same as the p-adic valuation of m.
\section{pnat/xgcd.lean}\paragraph{pnat.xgcd\_type}
\par
A term of xgcd\_type is a system of six naturals.  They should
be thought of as representing the matrix
{[}
{[}
w, x
{]}
, 
{[}
y, z
{]}
{]}
 = 
{[}
{[}
wp + 1, x
{]}
, 
{[}
y, zp + 1
{]}
{]}
together with the vector 
{[}
a, b
{]}
 = 
{[}
ap + 1, bp + 1
{]}
.
\paragraph{pnat.xgcd\_type.has\_repr}
\par
The has\_repr instance converts terms to strings in a way that
reflects the matrix/vector interpretation as above.
\paragraph{pnat.xgcd\_type.vp}
\par
The map v gives the product of the matrix
{[}
{[}
w, x
{]}
, 
{[}
y, z
{]}
{]}
 = 
{[}
{[}
wp + 1, x
{]}
, 
{[}
y, zp + 1
{]}
{]}
and the vector 
{[}
a, b
{]}
 = 
{[}
ap + 1, bp + 1
{]}
.  The map
vp gives 
{[}
sp, tp
{]}
 such that v = 
{[}
sp + 1, tp + 1
{]}
.
\paragraph{pnat.xgcd\_type.is\_special}
\par
is\_special holds if the matrix has determinant one.
\paragraph{pnat.xgcd\_type.is\_reduced}
\par
is\_reduced holds if the two entries in the vector are the
same.  The reduction algorithm will produce a system with this
property, whose product vector is the same as for the original
system.
\paragraph{pnat.xgcd\_type.rq\_eq}
\par
Properties of division with remainder for a / b.
\paragraph{pnat.xgcd\_type.start}
\par
The following function provides the starting point for
our algorithm.  We will apply an iterative reduction process
to it, which will produce a system satisfying is\_reduced.
The gcd can be read off from this final system.
\paragraph{pnat.xgcd\_type.step}
\par
This is the main reduction step, which is used when u.r ≠ 0, or
equivalently b does not divide a.
\paragraph{pnat.xgcd\_type.step\_wf}
\par
We will apply the above step recursively.  The following result
is used to ensure that the process terminates.
\paragraph{pnat.xgcd\_type.step\_v}
\par
The reduction step does not change the product vector.
\paragraph{pnat.xgcd\_type.reduce}
\par
We can now define the full reduction function, which applies
step as long as possible, and then applies finish. Note that the
"have" statement puts a fact in the local context, and the
equation compiler uses this fact to help construct the full
definition in terms of well-founded recursion.  The same fact
needs to be introduced in all the inductive proofs of properties
given below.
\section{polynomial.lean}\paragraph{polynomial}
\par
\colorbox[RGB]{253,246,227}{{{{\color[RGB]{101, 123, 131} polynomial α }}}} is the type of univariate polynomials over 
\colorbox[RGB]{253,246,227}{{{{\color[RGB]{101, 123, 131} α }}}}.
\par
Polynomials should be seen as (semi-)rings with the additional constructor 
\colorbox[RGB]{253,246,227}{{{{\color[RGB]{101, 123, 131} X }}}}.
The embedding from α is called 
\colorbox[RGB]{253,246,227}{{{{\color[RGB]{101, 123, 131} C }}}}.
\paragraph{polynomial.C}
\par
\colorbox[RGB]{253,246,227}{{{{\color[RGB]{101, 123, 131} C a }}}} is the constant polynomial 
\colorbox[RGB]{253,246,227}{{{{\color[RGB]{101, 123, 131} a }}}}.
\paragraph{polynomial.X}
\par
\colorbox[RGB]{253,246,227}{{{{\color[RGB]{101, 123, 131} X }}}} is the polynomial variable (aka indeterminant).
\paragraph{polynomial.coeff}
\par
coeff p n is the coefficient of X\textasciicircum{}n in p
\paragraph{polynomial.degree}
\par
\colorbox[RGB]{253,246,227}{{{{\color[RGB]{101, 123, 131} degree p }}}} is the degree of the polynomial 
\colorbox[RGB]{253,246,227}{{{{\color[RGB]{101, 123, 131} p }}}}, i.e. the largest 
\colorbox[RGB]{253,246,227}{{{{\color[RGB]{101, 123, 131} X }}}}-exponent in 
\colorbox[RGB]{253,246,227}{{{{\color[RGB]{101, 123, 131} p }}}}.
\colorbox[RGB]{253,246,227}{{{{\color[RGB]{101, 123, 131} degree p  }}}{{{\color[RGB]{181, 137, 0} = }}}{{{\color[RGB]{101, 123, 131}  some n }}}} when 
\colorbox[RGB]{253,246,227}{{{{\color[RGB]{101, 123, 131} p  }}}{{{\color[RGB]{181, 137, 0} ≠ }}}{{{\color[RGB]{101, 123, 131}   }}}{{{\color[RGB]{108, 113, 196} 0 }}}} and 
\colorbox[RGB]{253,246,227}{{{{\color[RGB]{101, 123, 131} n }}}} is the highest power of 
\colorbox[RGB]{253,246,227}{{{{\color[RGB]{101, 123, 131} X }}}} that appears in 
\colorbox[RGB]{253,246,227}{{{{\color[RGB]{101, 123, 131} p }}}}, otherwise
\colorbox[RGB]{253,246,227}{{{{\color[RGB]{101, 123, 131} degree  }}}{{{\color[RGB]{108, 113, 196} 0 }}}{{{\color[RGB]{101, 123, 131}   }}}{{{\color[RGB]{181, 137, 0} = }}}{{{\color[RGB]{101, 123, 131}  ⊥ }}}}.
\paragraph{polynomial.nat\_degree}
\par
\colorbox[RGB]{253,246,227}{{{{\color[RGB]{101, 123, 131} nat\_degree p }}}} forces 
\colorbox[RGB]{253,246,227}{{{{\color[RGB]{101, 123, 131} degree p }}}} to ℕ, by defining nat\_degree 0 = 0.
\paragraph{polynomial.eval₂}
\par
Evaluate a polynomial 
\colorbox[RGB]{253,246,227}{{{{\color[RGB]{101, 123, 131} p }}}} given a ring hom 
\colorbox[RGB]{253,246,227}{{{{\color[RGB]{101, 123, 131} f }}}} from the scalar ring
to the target and a value 
\colorbox[RGB]{253,246,227}{{{{\color[RGB]{101, 123, 131} x }}}} for the variable in the target
\paragraph{polynomial.eval}
\par
\colorbox[RGB]{253,246,227}{{{{\color[RGB]{133, 153, 0} eval }}}{{{\color[RGB]{101, 123, 131}  x p }}}} is the evaluation of the polynomial 
\colorbox[RGB]{253,246,227}{{{{\color[RGB]{101, 123, 131} p }}}} at 
\colorbox[RGB]{253,246,227}{{{{\color[RGB]{101, 123, 131} x }}}}\paragraph{polynomial.is\_root}
\par
\colorbox[RGB]{253,246,227}{{{{\color[RGB]{101, 123, 131} is\_root p x }}}} implies 
\colorbox[RGB]{253,246,227}{{{{\color[RGB]{101, 123, 131} x }}}} is a root of 
\colorbox[RGB]{253,246,227}{{{{\color[RGB]{101, 123, 131} p }}}}. The evaluation of 
\colorbox[RGB]{253,246,227}{{{{\color[RGB]{101, 123, 131} p }}}} at 
\colorbox[RGB]{253,246,227}{{{{\color[RGB]{101, 123, 131} x }}}} is zero
\paragraph{polynomial.map}
\par
\colorbox[RGB]{253,246,227}{{{{\color[RGB]{101, 123, 131} map f p }}}} maps a polynomial 
\colorbox[RGB]{253,246,227}{{{{\color[RGB]{101, 123, 131} p }}}} across a ring hom 
\colorbox[RGB]{253,246,227}{{{{\color[RGB]{101, 123, 131} f }}}}\paragraph{polynomial.leading\_coeff}
\par
\colorbox[RGB]{253,246,227}{{{{\color[RGB]{101, 123, 131} leading\_coeff p }}}} gives the coefficient of the highest power of 
\colorbox[RGB]{253,246,227}{{{{\color[RGB]{101, 123, 131} X }}}} in 
\colorbox[RGB]{253,246,227}{{{{\color[RGB]{101, 123, 131} p }}}}\paragraph{polynomial.monic}
\par
a polynomial is 
\colorbox[RGB]{253,246,227}{{{{\color[RGB]{101, 123, 131} monic }}}} if its leading coefficient is 1
\paragraph{polynomial.div\_X}
\par
\colorbox[RGB]{253,246,227}{{{{\color[RGB]{101, 123, 131} dix\_X p }}}} return a polynomial 
\colorbox[RGB]{253,246,227}{{{{\color[RGB]{101, 123, 131} q }}}} such that 
\colorbox[RGB]{253,246,227}{{{{\color[RGB]{101, 123, 131} q  }}}{{{\color[RGB]{181, 137, 0} * }}}{{{\color[RGB]{101, 123, 131}  X  }}}{{{\color[RGB]{181, 137, 0} + }}}{{{\color[RGB]{101, 123, 131}  C (p.coeff  }}}{{{\color[RGB]{108, 113, 196} 0 }}}{{{\color[RGB]{101, 123, 131} )  }}}{{{\color[RGB]{181, 137, 0} = }}}{{{\color[RGB]{101, 123, 131}  p }}}}.
It can be used in a semiring where the usual division algorithm is not possible
\paragraph{polynomial.div\_by\_monic}
\par
\colorbox[RGB]{253,246,227}{{{{\color[RGB]{101, 123, 131} div\_by\_monic }}}} gives the quotient of 
\colorbox[RGB]{253,246,227}{{{{\color[RGB]{101, 123, 131} p }}}} by a monic polynomial 
\colorbox[RGB]{253,246,227}{{{{\color[RGB]{101, 123, 131} q }}}}.
\paragraph{polynomial.mod\_by\_monic}
\par
\colorbox[RGB]{253,246,227}{{{{\color[RGB]{101, 123, 131} mod\_by\_monic }}}} gives the remainder of 
\colorbox[RGB]{253,246,227}{{{{\color[RGB]{101, 123, 131} p }}}} by a monic polynomial 
\colorbox[RGB]{253,246,227}{{{{\color[RGB]{101, 123, 131} q }}}}.
\paragraph{polynomial.roots}
\par
\colorbox[RGB]{253,246,227}{{{{\color[RGB]{101, 123, 131} roots p }}}} noncomputably gives a finset containing all the roots of 
\colorbox[RGB]{253,246,227}{{{{\color[RGB]{101, 123, 131} p }}}}\paragraph{polynomial.nth\_roots}
\par
\colorbox[RGB]{253,246,227}{{{{\color[RGB]{101, 123, 131} nth\_roots n a }}}} noncomputably returns the solutions to 
\colorbox[RGB]{253,246,227}{{{{\color[RGB]{101, 123, 131} x \textasciicircum{} n  }}}{{{\color[RGB]{181, 137, 0} = }}}{{{\color[RGB]{101, 123, 131}  a }}}}\paragraph{polynomial.derivative}
\par
\colorbox[RGB]{253,246,227}{{{{\color[RGB]{101, 123, 131} derivative p }}}} formal derivative of the polynomial 
\colorbox[RGB]{253,246,227}{{{{\color[RGB]{101, 123, 131} p }}}}\section{prod.lean}\paragraph{prod.swap}
\par
Swap the factors of a product. 
\colorbox[RGB]{253,246,227}{{{{\color[RGB]{101, 123, 131} swap (a, b)  }}}{{{\color[RGB]{181, 137, 0} = }}}{{{\color[RGB]{101, 123, 131}  (b, a) }}}}\section{quot.lean}\paragraph{quot.out}
\par
Choose an element of the equivalence class using the axiom of choice.
Sound but noncomputable.
\paragraph{quot.unquot}
\par
Unwrap the VM representation of a quotient to obtain an element of the equivalence class.
Computable but unsound.
\paragraph{quotient.out}
\par
Choose an element of the equivalence class using the axiom of choice.
Sound but noncomputable.
\paragraph{trunc}
\par
\colorbox[RGB]{253,246,227}{{{{\color[RGB]{101, 123, 131} trunc α }}}} is the quotient of 
\colorbox[RGB]{253,246,227}{{{{\color[RGB]{101, 123, 131} α }}}} by the always-true relation. This
is related to the propositional truncation in HoTT, and is similar
in effect to 
\colorbox[RGB]{253,246,227}{{{{\color[RGB]{101, 123, 131} nonempty α }}}}, but unlike 
\colorbox[RGB]{253,246,227}{{{{\color[RGB]{101, 123, 131} nonempty α }}}}, 
\colorbox[RGB]{253,246,227}{{{{\color[RGB]{101, 123, 131} trunc α }}}} is data,
so the VM representation is the same as 
\colorbox[RGB]{253,246,227}{{{{\color[RGB]{101, 123, 131} α }}}}, and so this can be used to
maintain computability.
\paragraph{trunc.mk}
\par
Constructor for 
\colorbox[RGB]{253,246,227}{{{{\color[RGB]{101, 123, 131} trunc α }}}}\paragraph{trunc.lift}
\par
Any constant function lifts to a function out of the truncation
\paragraph{trunc.out}
\par
Noncomputably extract a representative of 
\colorbox[RGB]{253,246,227}{{{{\color[RGB]{101, 123, 131} trunc α }}}} (using the axiom of choice).
\section{rat.lean}\paragraph{rat}
\par
\colorbox[RGB]{253,246,227}{{{{\color[RGB]{101, 123, 131} rat }}}}, or 
\colorbox[RGB]{253,246,227}{{{{\color[RGB]{101, 123, 131} ℚ }}}}, is the type of rational numbers. It is defined
as the set of pairs ⟨n, d⟩ of integers such that 
\colorbox[RGB]{253,246,227}{{{{\color[RGB]{101, 123, 131} d }}}} is positive and 
\colorbox[RGB]{253,246,227}{{{{\color[RGB]{101, 123, 131} n }}}} and
\colorbox[RGB]{253,246,227}{{{{\color[RGB]{101, 123, 131} d }}}} are coprime. This representation is preferred to the quotient
because without periodic reduction, the numerator and denominator can grow
exponentially (for example, adding 1/2 to itself repeatedly).
\paragraph{rat.of\_int}
\par
Embed an integer as a rational number
\paragraph{rat.mk\_pnat}
\par
Form the quotient 
\colorbox[RGB]{253,246,227}{{{{\color[RGB]{101, 123, 131} n  }}}{{{\color[RGB]{181, 137, 0} / }}}{{{\color[RGB]{101, 123, 131}  d }}}} where 
\colorbox[RGB]{253,246,227}{{{{\color[RGB]{101, 123, 131} n:ℤ }}}} and 
\colorbox[RGB]{253,246,227}{{{{\color[RGB]{101, 123, 131} d:ℕ }}}{{{\color[RGB]{181, 137, 0} + }}}} (not necessarily coprime)
\paragraph{rat.mk\_nat}
\par
Form the quotient 
\colorbox[RGB]{253,246,227}{{{{\color[RGB]{101, 123, 131} n  }}}{{{\color[RGB]{181, 137, 0} / }}}{{{\color[RGB]{101, 123, 131}  d }}}} where 
\colorbox[RGB]{253,246,227}{{{{\color[RGB]{101, 123, 131} n:ℤ }}}} and 
\colorbox[RGB]{253,246,227}{{{{\color[RGB]{101, 123, 131} d:ℕ }}}}. In the case 
\colorbox[RGB]{253,246,227}{{{{\color[RGB]{101, 123, 131} d  }}}{{{\color[RGB]{181, 137, 0} = }}}{{{\color[RGB]{101, 123, 131}   }}}{{{\color[RGB]{108, 113, 196} 0 }}}}, we
define 
\colorbox[RGB]{253,246,227}{{{{\color[RGB]{101, 123, 131} n  }}}{{{\color[RGB]{181, 137, 0} / }}}{{{\color[RGB]{101, 123, 131}   }}}{{{\color[RGB]{108, 113, 196} 0 }}}{{{\color[RGB]{101, 123, 131}   }}}{{{\color[RGB]{181, 137, 0} = }}}{{{\color[RGB]{101, 123, 131}   }}}{{{\color[RGB]{108, 113, 196} 0 }}}} by convention.
\paragraph{rat.mk}
\par
Form the quotient 
\colorbox[RGB]{253,246,227}{{{{\color[RGB]{101, 123, 131} n  }}}{{{\color[RGB]{181, 137, 0} / }}}{{{\color[RGB]{101, 123, 131}  d }}}} where 
\colorbox[RGB]{253,246,227}{{{{\color[RGB]{101, 123, 131} n d : ℤ }}}}.
\paragraph{rat.floor}
\par
\colorbox[RGB]{253,246,227}{{{{\color[RGB]{101, 123, 131} floor q }}}} is the largest integer 
\colorbox[RGB]{253,246,227}{{{{\color[RGB]{101, 123, 131} z }}}} such that 
\colorbox[RGB]{253,246,227}{{{{\color[RGB]{101, 123, 131} z  }}}{{{\color[RGB]{181, 137, 0} ≤ }}}{{{\color[RGB]{101, 123, 131}  q }}}}\paragraph{rat.ceil}
\par
\colorbox[RGB]{253,246,227}{{{{\color[RGB]{101, 123, 131} ceil q }}}} is the smallest integer 
\colorbox[RGB]{253,246,227}{{{{\color[RGB]{101, 123, 131} z }}}} such that 
\colorbox[RGB]{253,246,227}{{{{\color[RGB]{101, 123, 131} q  }}}{{{\color[RGB]{181, 137, 0} ≤ }}}{{{\color[RGB]{101, 123, 131}  z }}}}\paragraph{rat.cast}
\par
Construct the canonical injection from 
\colorbox[RGB]{253,246,227}{{{{\color[RGB]{101, 123, 131} ℚ }}}} into an arbitrary
division ring. If the field has positive characteristic 
\colorbox[RGB]{253,246,227}{{{{\color[RGB]{101, 123, 131} p }}}},
we define 
\colorbox[RGB]{253,246,227}{{{{\color[RGB]{108, 113, 196} 1 }}}{{{\color[RGB]{101, 123, 131}   }}}{{{\color[RGB]{181, 137, 0} / }}}{{{\color[RGB]{101, 123, 131}  p  }}}{{{\color[RGB]{181, 137, 0} = }}}{{{\color[RGB]{101, 123, 131}   }}}{{{\color[RGB]{108, 113, 196} 1 }}}{{{\color[RGB]{101, 123, 131}   }}}{{{\color[RGB]{181, 137, 0} / }}}{{{\color[RGB]{101, 123, 131}   }}}{{{\color[RGB]{108, 113, 196} 0 }}}{{{\color[RGB]{101, 123, 131}   }}}{{{\color[RGB]{181, 137, 0} = }}}{{{\color[RGB]{101, 123, 131}   }}}{{{\color[RGB]{108, 113, 196} 0 }}}} for consistency with our
division by zero convention.
\paragraph{rat.nat\_ceil}
\par
\colorbox[RGB]{253,246,227}{{{{\color[RGB]{101, 123, 131} nat\_ceil q }}}} is the smallest nonnegative integer 
\colorbox[RGB]{253,246,227}{{{{\color[RGB]{101, 123, 131} n }}}} with 
\colorbox[RGB]{253,246,227}{{{{\color[RGB]{101, 123, 131} q  }}}{{{\color[RGB]{181, 137, 0} ≤ }}}{{{\color[RGB]{101, 123, 131}  n }}}}.
It is the same as 
\colorbox[RGB]{253,246,227}{{{{\color[RGB]{101, 123, 131} ceil q }}}} when 
\colorbox[RGB]{253,246,227}{{{{\color[RGB]{101, 123, 131} q  }}}{{{\color[RGB]{181, 137, 0} ≥ }}}{{{\color[RGB]{101, 123, 131}   }}}{{{\color[RGB]{108, 113, 196} 0 }}}}, otherwise it is 
\colorbox[RGB]{253,246,227}{{{{\color[RGB]{108, 113, 196} 0 }}}}.
\section{real/basic.lean}\section{real/cau\_seq.lean}\section{real/cau\_seq\_completion.lean}\section{real/ennreal.lean}\paragraph{ennreal}
\par
The extended nonnegative real numbers. This is usually denoted 
{[}
0, ∞
{]}
,
and is relevant as the codomain of a measure.
\paragraph{ennreal.to\_nnreal}
\par
\colorbox[RGB]{253,246,227}{{{{\color[RGB]{101, 123, 131} to\_nnreal x }}}} returns 
\colorbox[RGB]{253,246,227}{{{{\color[RGB]{101, 123, 131} x }}}} if it is real, otherwise 0.
\paragraph{ennreal.to\_real}
\par
\colorbox[RGB]{253,246,227}{{{{\color[RGB]{101, 123, 131} to\_real x }}}} returns 
\colorbox[RGB]{253,246,227}{{{{\color[RGB]{101, 123, 131} x }}}} if it is real, 
\colorbox[RGB]{253,246,227}{{{{\color[RGB]{108, 113, 196} 0 }}}} otherwise.
\paragraph{ennreal.of\_real}
\par
\colorbox[RGB]{253,246,227}{{{{\color[RGB]{101, 123, 131} of\_real x }}}} returns 
\colorbox[RGB]{253,246,227}{{{{\color[RGB]{101, 123, 131} x }}}} if it is nonnegative, 
\colorbox[RGB]{253,246,227}{{{{\color[RGB]{108, 113, 196} 0 }}}} otherwise.
\section{real/hyperreal.lean}\paragraph{hyperreal}
\par
Hyperreal numbers on the ultrafilter extending the cofinite filter
\paragraph{hyperreal.epsilon}
\par
A sample infinitesimal hyperreal
\paragraph{hyperreal.omega}
\par
A sample infinite hyperreal
\paragraph{hyperreal.is\_st}
\par
Standard part predicate
\paragraph{hyperreal.st}
\par
Standard part function: like a "round" to ℝ instead of ℤ
\paragraph{hyperreal.infinitesimal}
\par
A hyperreal number is infinitesimal if its standard part is 0
\paragraph{hyperreal.infinite\_pos}
\par
A hyperreal number is positive infinite if it is larger than all real numbers
\paragraph{hyperreal.infinite\_neg}
\par
A hyperreal number is negative infinite if it is smaller than all real numbers
\paragraph{hyperreal.infinite}
\par
A hyperreal number is infinite if it is infinite positive or infinite negative
\section{real/irrational.lean}\section{real/nnreal.lean}\section{real/pi.lean}\paragraph{real.sqrt\_two\_add\_series}
\par
the series 
\colorbox[RGB]{253,246,227}{{{{\color[RGB]{101, 123, 131} sqrt\_two\_add\_series x n }}}} is 
\colorbox[RGB]{253,246,227}{{{{\color[RGB]{101, 123, 131} sqrt( }}}{{{\color[RGB]{108, 113, 196} 2 }}}{{{\color[RGB]{101, 123, 131}   }}}{{{\color[RGB]{181, 137, 0} + }}}{{{\color[RGB]{101, 123, 131}  sqrt( }}}{{{\color[RGB]{108, 113, 196} 2 }}}{{{\color[RGB]{101, 123, 131}   }}}{{{\color[RGB]{181, 137, 0} + }}}{{{\color[RGB]{101, 123, 131}  ... )) }}}} with 
\colorbox[RGB]{253,246,227}{{{{\color[RGB]{101, 123, 131} n }}}} square roots,
starting with 
\colorbox[RGB]{253,246,227}{{{{\color[RGB]{101, 123, 131} x }}}}. We define it here because 
\colorbox[RGB]{253,246,227}{{{{\color[RGB]{101, 123, 131} cos (pi  }}}{{{\color[RGB]{181, 137, 0} / }}}{{{\color[RGB]{101, 123, 131}   }}}{{{\color[RGB]{108, 113, 196} 2 }}}{{{\color[RGB]{101, 123, 131}  \textasciicircum{} (n }}}{{{\color[RGB]{181, 137, 0} + }}}{{{\color[RGB]{108, 113, 196} 1 }}}{{{\color[RGB]{101, 123, 131} ))  }}}{{{\color[RGB]{181, 137, 0} = }}}{{{\color[RGB]{101, 123, 131}  sqrt\_two\_add\_series  }}}{{{\color[RGB]{108, 113, 196} 0 }}}{{{\color[RGB]{101, 123, 131}  n  }}}{{{\color[RGB]{181, 137, 0} / }}}{{{\color[RGB]{101, 123, 131}   }}}{{{\color[RGB]{108, 113, 196} 2 }}}}\section{rel.lean}\section{semiquot.lean}\paragraph{semiquot}
\par
A member of 
\colorbox[RGB]{253,246,227}{{{{\color[RGB]{101, 123, 131} semiquot α }}}} is classically a nonempty 
\colorbox[RGB]{253,246,227}{{{{\color[RGB]{101, 123, 131} set α }}}},
and in the VM is represented by an element of 
\colorbox[RGB]{253,246,227}{{{{\color[RGB]{101, 123, 131} α }}}}; the relation
between these is that the VM element is required to be a member
of the set 
\colorbox[RGB]{253,246,227}{{{{\color[RGB]{101, 123, 131} s }}}}. The specific element of 
\colorbox[RGB]{253,246,227}{{{{\color[RGB]{101, 123, 131} s }}}} that the VM computes
is hidden by a quotient construction, allowing for the representation
of nondeterministic functions.
\section{seq/computation.lean}\paragraph{computation}
\par
\colorbox[RGB]{253,246,227}{{{{\color[RGB]{101, 123, 131} computation α }}}} is the type of unbounded computations returning 
\colorbox[RGB]{253,246,227}{{{{\color[RGB]{101, 123, 131} α }}}}.
An element of 
\colorbox[RGB]{253,246,227}{{{{\color[RGB]{101, 123, 131} computation α }}}} is an infinite sequence of 
\colorbox[RGB]{253,246,227}{{{{\color[RGB]{101, 123, 131} option α }}}} such
that if 
\colorbox[RGB]{253,246,227}{{{{\color[RGB]{101, 123, 131} f n  }}}{{{\color[RGB]{181, 137, 0} = }}}{{{\color[RGB]{101, 123, 131}  some a }}}} for some 
\colorbox[RGB]{253,246,227}{{{{\color[RGB]{101, 123, 131} n }}}} then it is constantly 
\colorbox[RGB]{253,246,227}{{{{\color[RGB]{101, 123, 131} some a }}}} after that.
\paragraph{computation.return}
\par
\colorbox[RGB]{253,246,227}{{{{\color[RGB]{101, 123, 131} return a }}}} is the computation that immediately terminates with result 
\colorbox[RGB]{253,246,227}{{{{\color[RGB]{101, 123, 131} a }}}}.
\paragraph{computation.think}
\par
\colorbox[RGB]{253,246,227}{{{{\color[RGB]{101, 123, 131} think c }}}} is the computation that delays for one "tick" and then performs
computation 
\colorbox[RGB]{253,246,227}{{{{\color[RGB]{101, 123, 131} c }}}}.
\paragraph{computation.thinkN}
\par
\colorbox[RGB]{253,246,227}{{{{\color[RGB]{101, 123, 131} thinkN c n }}}} is the computation that delays for 
\colorbox[RGB]{253,246,227}{{{{\color[RGB]{101, 123, 131} n }}}} ticks and then performs
computation 
\colorbox[RGB]{253,246,227}{{{{\color[RGB]{101, 123, 131} c }}}}.
\paragraph{computation.head}
\par
\colorbox[RGB]{253,246,227}{{{{\color[RGB]{101, 123, 131} head c }}}} is the first step of computation, either 
\colorbox[RGB]{253,246,227}{{{{\color[RGB]{101, 123, 131} some a }}}} if 
\colorbox[RGB]{253,246,227}{{{{\color[RGB]{101, 123, 131} c  }}}{{{\color[RGB]{181, 137, 0} = }}}{{{\color[RGB]{101, 123, 131}  return a }}}}or 
\colorbox[RGB]{253,246,227}{{{{\color[RGB]{101, 123, 131} none }}}} if 
\colorbox[RGB]{253,246,227}{{{{\color[RGB]{101, 123, 131} c  }}}{{{\color[RGB]{181, 137, 0} = }}}{{{\color[RGB]{101, 123, 131}  think c' }}}}.
\paragraph{computation.tail}
\par
\colorbox[RGB]{253,246,227}{{{{\color[RGB]{101, 123, 131} tail c }}}} is the remainder of computation, either 
\colorbox[RGB]{253,246,227}{{{{\color[RGB]{101, 123, 131} c }}}} if 
\colorbox[RGB]{253,246,227}{{{{\color[RGB]{101, 123, 131} c  }}}{{{\color[RGB]{181, 137, 0} = }}}{{{\color[RGB]{101, 123, 131}  return a }}}}or 
\colorbox[RGB]{253,246,227}{{{{\color[RGB]{101, 123, 131} c' }}}} if 
\colorbox[RGB]{253,246,227}{{{{\color[RGB]{101, 123, 131} c  }}}{{{\color[RGB]{181, 137, 0} = }}}{{{\color[RGB]{101, 123, 131}  think c' }}}}.
\paragraph{computation.empty}
\par
\colorbox[RGB]{253,246,227}{{{{\color[RGB]{101, 123, 131} empty α }}}} is the computation that never returns, an infinite sequence of
\colorbox[RGB]{253,246,227}{{{{\color[RGB]{101, 123, 131} think }}}}s.
\paragraph{computation.run\_for}
\par
\colorbox[RGB]{253,246,227}{{{{\color[RGB]{101, 123, 131} run\_for c n }}}} evaluates 
\colorbox[RGB]{253,246,227}{{{{\color[RGB]{101, 123, 131} c }}}} for 
\colorbox[RGB]{253,246,227}{{{{\color[RGB]{101, 123, 131} n }}}} steps and returns the result, or 
\colorbox[RGB]{253,246,227}{{{{\color[RGB]{101, 123, 131} none }}}}if it did not terminate after 
\colorbox[RGB]{253,246,227}{{{{\color[RGB]{101, 123, 131} n }}}} steps.
\paragraph{computation.destruct}
\par
\colorbox[RGB]{253,246,227}{{{{\color[RGB]{101, 123, 131} destruct c }}}} is the destructor for 
\colorbox[RGB]{253,246,227}{{{{\color[RGB]{101, 123, 131} computation α }}}} as a coinductive type.
It returns 
\colorbox[RGB]{253,246,227}{{{{\color[RGB]{101, 123, 131} inl a }}}} if 
\colorbox[RGB]{253,246,227}{{{{\color[RGB]{101, 123, 131} c  }}}{{{\color[RGB]{181, 137, 0} = }}}{{{\color[RGB]{101, 123, 131}  return a }}}} and 
\colorbox[RGB]{253,246,227}{{{{\color[RGB]{101, 123, 131} inr c' }}}} if 
\colorbox[RGB]{253,246,227}{{{{\color[RGB]{101, 123, 131} c  }}}{{{\color[RGB]{181, 137, 0} = }}}{{{\color[RGB]{101, 123, 131}  think c' }}}}.
\paragraph{computation.run}
\par
\colorbox[RGB]{253,246,227}{{{{\color[RGB]{101, 123, 131} run c }}}} is an unsound meta function that runs 
\colorbox[RGB]{253,246,227}{{{{\color[RGB]{101, 123, 131} c }}}} to completion, possibly
resulting in an infinite loop in the VM.
\paragraph{computation.corec}
\par
\colorbox[RGB]{253,246,227}{{{{\color[RGB]{101, 123, 131} corec f b }}}} is the corecursor for 
\colorbox[RGB]{253,246,227}{{{{\color[RGB]{101, 123, 131} computation α }}}} as a coinductive type.
If 
\colorbox[RGB]{253,246,227}{{{{\color[RGB]{101, 123, 131} f b  }}}{{{\color[RGB]{181, 137, 0} = }}}{{{\color[RGB]{101, 123, 131}  inl a }}}} then 
\colorbox[RGB]{253,246,227}{{{{\color[RGB]{101, 123, 131} corec f b  }}}{{{\color[RGB]{181, 137, 0} = }}}{{{\color[RGB]{101, 123, 131}  return a }}}}, and if 
\colorbox[RGB]{253,246,227}{{{{\color[RGB]{101, 123, 131} f b  }}}{{{\color[RGB]{181, 137, 0} = }}}{{{\color[RGB]{101, 123, 131}  inl b' }}}} then
\colorbox[RGB]{253,246,227}{{{{\color[RGB]{101, 123, 131} corec f b  }}}{{{\color[RGB]{181, 137, 0} = }}}{{{\color[RGB]{101, 123, 131}  think (corec f b') }}}}.
\paragraph{computation.lmap}
\par
left map of 
\colorbox[RGB]{253,246,227}{{{{\color[RGB]{101, 123, 131} ⊕ }}}}\paragraph{computation.rmap}
\par
right map of 
\colorbox[RGB]{253,246,227}{{{{\color[RGB]{101, 123, 131} ⊕ }}}}\paragraph{computation.terminates}
\par
\colorbox[RGB]{253,246,227}{{{{\color[RGB]{101, 123, 131} terminates s }}}} asserts that the computation 
\colorbox[RGB]{253,246,227}{{{{\color[RGB]{101, 123, 131} s }}}} eventually terminates with some value.
\paragraph{computation.promises}
\par
\colorbox[RGB]{253,246,227}{{{{\color[RGB]{101, 123, 131} promises s a }}}}, or 
\colorbox[RGB]{253,246,227}{{{{\color[RGB]{101, 123, 131} s \textasciitilde{} }}}{{{\color[RGB]{181, 137, 0} > }}}{{{\color[RGB]{101, 123, 131}  a }}}}, asserts that although the computation 
\colorbox[RGB]{253,246,227}{{{{\color[RGB]{101, 123, 131} s }}}}may not terminate, if it does, then the result is 
\colorbox[RGB]{253,246,227}{{{{\color[RGB]{101, 123, 131} a }}}}.
\paragraph{computation.length}
\par
\colorbox[RGB]{253,246,227}{{{{\color[RGB]{101, 123, 131} length s }}}} gets the number of steps of a terminating computation
\paragraph{computation.get}
\par
\colorbox[RGB]{253,246,227}{{{{\color[RGB]{101, 123, 131} get s }}}} returns the result of a terminating computation
\paragraph{computation.results}
\par
\colorbox[RGB]{253,246,227}{{{{\color[RGB]{101, 123, 131} results s a n }}}} completely characterizes a terminating computation:
it asserts that 
\colorbox[RGB]{253,246,227}{{{{\color[RGB]{101, 123, 131} s }}}} terminates after exactly 
\colorbox[RGB]{253,246,227}{{{{\color[RGB]{101, 123, 131} n }}}} steps, with result 
\colorbox[RGB]{253,246,227}{{{{\color[RGB]{101, 123, 131} a }}}}.
\paragraph{computation.map}
\par
Map a function on the result of a computation.
\paragraph{computation.bind}
\par
Compose two computations into a monadic 
\colorbox[RGB]{253,246,227}{{{{\color[RGB]{101, 123, 131} bind }}}} operation.
\paragraph{computation.join}
\par
Flatten a computation of computations into a single computation.
\paragraph{computation.orelse}
\par
\colorbox[RGB]{253,246,227}{{{{\color[RGB]{101, 123, 131} c₁  }}}{{{\color[RGB]{181, 137, 0} < }}}{{{\color[RGB]{101, 123, 131} | }}}{{{\color[RGB]{181, 137, 0} > }}}{{{\color[RGB]{101, 123, 131}  c₂ }}}} calculates 
\colorbox[RGB]{253,246,227}{{{{\color[RGB]{101, 123, 131} c₁ }}}} and 
\colorbox[RGB]{253,246,227}{{{{\color[RGB]{101, 123, 131} c₂ }}}} simultaneously, returning
the first one that gives a result.
\paragraph{computation.equiv}
\par
\colorbox[RGB]{253,246,227}{{{{\color[RGB]{101, 123, 131} c₁ \textasciitilde{} c₂ }}}} asserts that 
\colorbox[RGB]{253,246,227}{{{{\color[RGB]{101, 123, 131} c₁ }}}} and 
\colorbox[RGB]{253,246,227}{{{{\color[RGB]{101, 123, 131} c₂ }}}} either both terminate with the same result,
or both loop forever.
\paragraph{computation.lift\_rel}
\par
\colorbox[RGB]{253,246,227}{{{{\color[RGB]{101, 123, 131} lift\_rel R ca cb }}}} is a generalization of 
\colorbox[RGB]{253,246,227}{{{{\color[RGB]{101, 123, 131} equiv }}}} to relations other than
equality. It asserts that if 
\colorbox[RGB]{253,246,227}{{{{\color[RGB]{101, 123, 131} ca }}}} terminates with 
\colorbox[RGB]{253,246,227}{{{{\color[RGB]{101, 123, 131} a }}}}, then 
\colorbox[RGB]{253,246,227}{{{{\color[RGB]{101, 123, 131} cb }}}} terminates with
some 
\colorbox[RGB]{253,246,227}{{{{\color[RGB]{101, 123, 131} b }}}} such that 
\colorbox[RGB]{253,246,227}{{{{\color[RGB]{101, 123, 131} R a b }}}}, and if 
\colorbox[RGB]{253,246,227}{{{{\color[RGB]{101, 123, 131} cb }}}} terminates with 
\colorbox[RGB]{253,246,227}{{{{\color[RGB]{101, 123, 131} b }}}} then 
\colorbox[RGB]{253,246,227}{{{{\color[RGB]{101, 123, 131} ca }}}} terminates
with some 
\colorbox[RGB]{253,246,227}{{{{\color[RGB]{101, 123, 131} a }}}} such that 
\colorbox[RGB]{253,246,227}{{{{\color[RGB]{101, 123, 131} R a b }}}}.
\section{seq/parallel.lean}\paragraph{computation.parallel}
\par
Parallel computation of an infinite stream of computations,
taking the first result
\section{seq/seq.lean}\paragraph{seq}
\par
\colorbox[RGB]{253,246,227}{{{{\color[RGB]{101, 123, 131} seq α }}}} is the type of possibly infinite lists (referred here as sequences).
It is encoded as an infinite stream of options such that if 
\colorbox[RGB]{253,246,227}{{{{\color[RGB]{101, 123, 131} f n  }}}{{{\color[RGB]{181, 137, 0} = }}}{{{\color[RGB]{101, 123, 131}  none }}}}, then
\colorbox[RGB]{253,246,227}{{{{\color[RGB]{101, 123, 131} f m  }}}{{{\color[RGB]{181, 137, 0} = }}}{{{\color[RGB]{101, 123, 131}  none }}}} for all 
\colorbox[RGB]{253,246,227}{{{{\color[RGB]{101, 123, 131} m  }}}{{{\color[RGB]{181, 137, 0} ≥ }}}{{{\color[RGB]{101, 123, 131}  n }}}}.
\paragraph{seq1}
\par
\colorbox[RGB]{253,246,227}{{{{\color[RGB]{101, 123, 131} seq1 α }}}} is the type of nonempty sequences.
\paragraph{seq.nil}
\par
The empty sequence
\paragraph{seq.cons}
\par
Prepend an element to a sequence
\paragraph{seq.nth}
\par
Get the nth element of a sequence (if it exists)
\paragraph{seq.omap}
\par
Functorial action of the functor 
\colorbox[RGB]{253,246,227}{{{{\color[RGB]{101, 123, 131} option (α × \_) }}}}\paragraph{seq.head}
\par
Get the first element of a sequence
\paragraph{seq.tail}
\par
Get the tail of a sequence (or 
\colorbox[RGB]{253,246,227}{{{{\color[RGB]{101, 123, 131} nil }}}} if the sequence is 
\colorbox[RGB]{253,246,227}{{{{\color[RGB]{101, 123, 131} nil }}}})
\paragraph{seq.destruct}
\par
Destructor for a sequence, resulting in either 
\colorbox[RGB]{253,246,227}{{{{\color[RGB]{101, 123, 131} none }}}} (for 
\colorbox[RGB]{253,246,227}{{{{\color[RGB]{101, 123, 131} nil }}}}) or
\colorbox[RGB]{253,246,227}{{{{\color[RGB]{101, 123, 131} some (a, s) }}}} (for 
\colorbox[RGB]{253,246,227}{{{{\color[RGB]{101, 123, 131} cons a s }}}}).
\paragraph{seq.corec}
\par
Corecursor for 
\colorbox[RGB]{253,246,227}{{{{\color[RGB]{101, 123, 131} seq α }}}} as a coinductive type. Iterates 
\colorbox[RGB]{253,246,227}{{{{\color[RGB]{101, 123, 131} f }}}} to produce new elements
of the sequence until 
\colorbox[RGB]{253,246,227}{{{{\color[RGB]{101, 123, 131} none }}}} is obtained.
\paragraph{seq.of\_list}
\par
Embed a list as a sequence
\paragraph{seq.of\_stream}
\par
Embed an infinite stream as a sequence
\paragraph{seq.of\_lazy\_list}
\par
Embed a 
\colorbox[RGB]{253,246,227}{{{{\color[RGB]{101, 123, 131} lazy\_list α }}}} as a sequence. Note that even though this
is non-meta, it will produce infinite sequences if used with
cyclic 
\colorbox[RGB]{253,246,227}{{{{\color[RGB]{101, 123, 131} lazy\_list }}}}s created by meta constructions.
\paragraph{seq.to\_lazy\_list}
\par
Translate a sequence into a 
\colorbox[RGB]{253,246,227}{{{{\color[RGB]{101, 123, 131} lazy\_list }}}}. Since 
\colorbox[RGB]{253,246,227}{{{{\color[RGB]{101, 123, 131} lazy\_list }}}} and 
\colorbox[RGB]{253,246,227}{{{{\color[RGB]{101, 123, 131} list }}}}are isomorphic as non-meta types, this function is necessarily meta.
\paragraph{seq.force\_to\_list}
\par
Translate a sequence to a list. This function will run forever if
run on an infinite sequence.
\paragraph{seq.append}
\par
Append two sequences. If 
\colorbox[RGB]{253,246,227}{{{{\color[RGB]{101, 123, 131} s₁ }}}} is infinite, then 
\colorbox[RGB]{253,246,227}{{{{\color[RGB]{101, 123, 131} s₁  }}}{{{\color[RGB]{181, 137, 0} + }}}{{{\color[RGB]{181, 137, 0} + }}}{{{\color[RGB]{101, 123, 131}  s₂  }}}{{{\color[RGB]{181, 137, 0} = }}}{{{\color[RGB]{101, 123, 131}  s₁ }}}},
otherwise it puts 
\colorbox[RGB]{253,246,227}{{{{\color[RGB]{101, 123, 131} s₂ }}}} at the location of the 
\colorbox[RGB]{253,246,227}{{{{\color[RGB]{101, 123, 131} nil }}}} in 
\colorbox[RGB]{253,246,227}{{{{\color[RGB]{101, 123, 131} s₁ }}}}.
\paragraph{seq.map}
\par
Map a function over a sequence.
\paragraph{seq.join}
\par
Flatten a sequence of sequences. (It is required that the
sequences be nonempty to ensure productivity; in the case
of an infinite sequence of 
\colorbox[RGB]{253,246,227}{{{{\color[RGB]{101, 123, 131} nil }}}}, the first element is never
generated.)
\paragraph{seq.drop}
\par
Remove the first 
\colorbox[RGB]{253,246,227}{{{{\color[RGB]{101, 123, 131} n }}}} elements from the sequence.
\paragraph{seq.take}
\par
Take the first 
\colorbox[RGB]{253,246,227}{{{{\color[RGB]{101, 123, 131} n }}}} elements of the sequence (producing a list)
\paragraph{seq.split\_at}
\par
Split a sequence at 
\colorbox[RGB]{253,246,227}{{{{\color[RGB]{101, 123, 131} n }}}}, producing a finite initial segment
and an infinite tail.
\paragraph{seq.zip\_with}
\par
Combine two sequences with a function
\paragraph{seq.zip}
\par
Pair two sequences into a sequence of pairs
\paragraph{seq.unzip}
\par
Separate a sequence of pairs into two sequences
\paragraph{seq.to\_list}
\par
Convert a sequence which is known to terminate into a list
\paragraph{seq.to\_stream}
\par
Convert a sequence which is known not to terminate into a stream
\paragraph{seq.to\_list\_or\_stream}
\par
Convert a sequence into either a list or a stream depending on whether
it is finite or infinite. (Without decidability of the infiniteness predicate,
this is not constructively possible.)
\paragraph{seq.to\_list'}
\par
Convert a sequence into a list, embedded in a computation to allow for
the possibility of infinite sequences (in which case the computation
never returns anything).
\paragraph{seq1.to\_seq}
\par
Convert a 
\colorbox[RGB]{253,246,227}{{{{\color[RGB]{101, 123, 131} seq1 }}}} to a sequence.
\paragraph{seq1.map}
\par
Map a function on a 
\colorbox[RGB]{253,246,227}{{{{\color[RGB]{101, 123, 131} seq1 }}}}\paragraph{seq1.join}
\par
Flatten a nonempty sequence of nonempty sequences
\paragraph{seq1.ret}
\par
The 
\colorbox[RGB]{253,246,227}{{{{\color[RGB]{101, 123, 131} return }}}} operator for the 
\colorbox[RGB]{253,246,227}{{{{\color[RGB]{101, 123, 131} seq1 }}}} monad,
which produces a singleton sequence.
\paragraph{seq1.bind}
\par
The 
\colorbox[RGB]{253,246,227}{{{{\color[RGB]{101, 123, 131} bind }}}} operator for the 
\colorbox[RGB]{253,246,227}{{{{\color[RGB]{101, 123, 131} seq1 }}}} monad,
which maps 
\colorbox[RGB]{253,246,227}{{{{\color[RGB]{101, 123, 131} f }}}} on each element of 
\colorbox[RGB]{253,246,227}{{{{\color[RGB]{101, 123, 131} s }}}} and appends the results together.
(Not all of 
\colorbox[RGB]{253,246,227}{{{{\color[RGB]{101, 123, 131} s }}}} may be evaluated, because the first few elements of 
\colorbox[RGB]{253,246,227}{{{{\color[RGB]{101, 123, 131} s }}}}may already produce an infinite result.)
\section{seq/wseq.lean}\paragraph{wseq}
\par
Weak sequences.
\par
While the 
\colorbox[RGB]{253,246,227}{{{{\color[RGB]{101, 123, 131} seq }}}} structure allows for lists which may not be finite,
a weak sequence also allows the computation of each element to
involve an indeterminate amount of computation, including possibly
an infinite loop. This is represented as a regular 
\colorbox[RGB]{253,246,227}{{{{\color[RGB]{101, 123, 131} seq }}}} interspersed
with 
\colorbox[RGB]{253,246,227}{{{{\color[RGB]{101, 123, 131} none }}}} elements to indicate that computation is ongoing.
\par
This model is appropriate for Haskell style lazy lists, and is closed
under most interesting computation patterns on infinite lists,
but conversely it is difficult to extract elements from it.
\paragraph{wseq.of\_seq}
\par
Turn a sequence into a weak sequence
\paragraph{wseq.of\_list}
\par
Turn a list into a weak sequence
\paragraph{wseq.of\_stream}
\par
Turn a stream into a weak sequence
\paragraph{wseq.nil}
\par
The empty weak sequence
\paragraph{wseq.cons}
\par
Prepend an element to a weak sequence
\paragraph{wseq.think}
\par
Compute for one tick, without producing any elements
\paragraph{wseq.destruct}
\par
Destruct a weak sequence, to (eventually possibly) produce either
\colorbox[RGB]{253,246,227}{{{{\color[RGB]{101, 123, 131} none }}}} for 
\colorbox[RGB]{253,246,227}{{{{\color[RGB]{101, 123, 131} nil }}}} or 
\colorbox[RGB]{253,246,227}{{{{\color[RGB]{101, 123, 131} some (a, s) }}}} if an element is produced.
\paragraph{wseq.head}
\par
Get the head of a weak sequence. This involves a possibly
infinite computation.
\paragraph{wseq.flatten}
\par
Encode a computation yielding a weak sequence into additional
\colorbox[RGB]{253,246,227}{{{{\color[RGB]{101, 123, 131} think }}}} constructors in a weak sequence
\paragraph{wseq.tail}
\par
Get the tail of a weak sequence. This doesn't need a 
\colorbox[RGB]{253,246,227}{{{{\color[RGB]{101, 123, 131} computation }}}}wrapper, unlike 
\colorbox[RGB]{253,246,227}{{{{\color[RGB]{101, 123, 131} head }}}}, because 
\colorbox[RGB]{253,246,227}{{{{\color[RGB]{101, 123, 131} flatten }}}} allows us to hide this
in the construction of the weak sequence itself.
\paragraph{wseq.drop}
\par
drop the first 
\colorbox[RGB]{253,246,227}{{{{\color[RGB]{101, 123, 131} n }}}} elements from 
\colorbox[RGB]{253,246,227}{{{{\color[RGB]{101, 123, 131} s }}}}.
\paragraph{wseq.nth}
\par
Get the nth element of 
\colorbox[RGB]{253,246,227}{{{{\color[RGB]{101, 123, 131} s }}}}.
\paragraph{wseq.to\_list}
\par
Convert 
\colorbox[RGB]{253,246,227}{{{{\color[RGB]{101, 123, 131} s }}}} to a list (if it is finite and completes in finite time).
\paragraph{wseq.length}
\par
Get the length of 
\colorbox[RGB]{253,246,227}{{{{\color[RGB]{101, 123, 131} s }}}} (if it is finite and completes in finite time).
\paragraph{wseq.is\_finite}
\par
A weak sequence is finite if 
\colorbox[RGB]{253,246,227}{{{{\color[RGB]{101, 123, 131} to\_list s }}}} terminates. Equivalently,
it is a finite number of 
\colorbox[RGB]{253,246,227}{{{{\color[RGB]{101, 123, 131} think }}}} and 
\colorbox[RGB]{253,246,227}{{{{\color[RGB]{101, 123, 131} cons }}}} applied to 
\colorbox[RGB]{253,246,227}{{{{\color[RGB]{101, 123, 131} nil }}}}.
\paragraph{wseq.get}
\par
Get the list corresponding to a finite weak sequence.
\paragraph{wseq.productive}
\par
A weak sequence is 
\emph{productive
} if it never stalls forever - there are
always a finite number of 
\colorbox[RGB]{253,246,227}{{{{\color[RGB]{101, 123, 131} think }}}}s between 
\colorbox[RGB]{253,246,227}{{{{\color[RGB]{101, 123, 131} cons }}}} constructors.
The sequence itself is allowed to be infinite though.
\paragraph{wseq.update\_nth}
\par
Replace the 
\colorbox[RGB]{253,246,227}{{{{\color[RGB]{101, 123, 131} n }}}}th element of 
\colorbox[RGB]{253,246,227}{{{{\color[RGB]{101, 123, 131} s }}}} with 
\colorbox[RGB]{253,246,227}{{{{\color[RGB]{101, 123, 131} a }}}}.
\paragraph{wseq.remove\_nth}
\par
Remove the 
\colorbox[RGB]{253,246,227}{{{{\color[RGB]{101, 123, 131} n }}}}th element of 
\colorbox[RGB]{253,246,227}{{{{\color[RGB]{101, 123, 131} s }}}}.
\paragraph{wseq.filter\_map}
\par
Map the elements of 
\colorbox[RGB]{253,246,227}{{{{\color[RGB]{101, 123, 131} s }}}} over 
\colorbox[RGB]{253,246,227}{{{{\color[RGB]{101, 123, 131} f }}}}, removing any values that yield 
\colorbox[RGB]{253,246,227}{{{{\color[RGB]{101, 123, 131} none }}}}.
\paragraph{wseq.filter}
\par
Select the elements of 
\colorbox[RGB]{253,246,227}{{{{\color[RGB]{101, 123, 131} s }}}} that satisfy 
\colorbox[RGB]{253,246,227}{{{{\color[RGB]{101, 123, 131} p }}}}.
\paragraph{wseq.find}
\par
Get the first element of 
\colorbox[RGB]{253,246,227}{{{{\color[RGB]{101, 123, 131} s }}}} satisfying 
\colorbox[RGB]{253,246,227}{{{{\color[RGB]{101, 123, 131} p }}}}.
\paragraph{wseq.zip\_with}
\par
Zip a function over two weak sequences
\paragraph{wseq.zip}
\par
Zip two weak sequences into a single sequence of pairs
\paragraph{wseq.find\_indexes}
\par
Get the list of indexes of elements of 
\colorbox[RGB]{253,246,227}{{{{\color[RGB]{101, 123, 131} s }}}} satisfying 
\colorbox[RGB]{253,246,227}{{{{\color[RGB]{101, 123, 131} p }}}}\paragraph{wseq.find\_index}
\par
Get the index of the first element of 
\colorbox[RGB]{253,246,227}{{{{\color[RGB]{101, 123, 131} s }}}} satisfying 
\colorbox[RGB]{253,246,227}{{{{\color[RGB]{101, 123, 131} p }}}}\paragraph{wseq.index\_of}
\par
Get the index of the first occurrence of 
\colorbox[RGB]{253,246,227}{{{{\color[RGB]{101, 123, 131} a }}}} in 
\colorbox[RGB]{253,246,227}{{{{\color[RGB]{101, 123, 131} s }}}}\paragraph{wseq.indexes\_of}
\par
Get the indexes of occurrences of 
\colorbox[RGB]{253,246,227}{{{{\color[RGB]{101, 123, 131} a }}}} in 
\colorbox[RGB]{253,246,227}{{{{\color[RGB]{101, 123, 131} s }}}}\paragraph{wseq.union}
\par
\colorbox[RGB]{253,246,227}{{{{\color[RGB]{101, 123, 131} union s1 s2 }}}} is a weak sequence which interleaves 
\colorbox[RGB]{253,246,227}{{{{\color[RGB]{101, 123, 131} s1 }}}} and 
\colorbox[RGB]{253,246,227}{{{{\color[RGB]{101, 123, 131} s2 }}}} in
some order (nondeterministically).
\paragraph{wseq.is\_empty}
\par
Returns 
\colorbox[RGB]{253,246,227}{{{{\color[RGB]{101, 123, 131} tt }}}} if 
\colorbox[RGB]{253,246,227}{{{{\color[RGB]{101, 123, 131} s }}}} is 
\colorbox[RGB]{253,246,227}{{{{\color[RGB]{101, 123, 131} nil }}}} and 
\colorbox[RGB]{253,246,227}{{{{\color[RGB]{101, 123, 131} ff }}}} if 
\colorbox[RGB]{253,246,227}{{{{\color[RGB]{101, 123, 131} s }}}} has an element
\paragraph{wseq.compute}
\par
Calculate one step of computation
\paragraph{wseq.take}
\par
Get the first 
\colorbox[RGB]{253,246,227}{{{{\color[RGB]{101, 123, 131} n }}}} elements of a weak sequence
\paragraph{wseq.split\_at}
\par
Split the sequence at position 
\colorbox[RGB]{253,246,227}{{{{\color[RGB]{101, 123, 131} n }}}} into a finite initial segment
and the weak sequence tail
\paragraph{wseq.any}
\par
Returns 
\colorbox[RGB]{253,246,227}{{{{\color[RGB]{101, 123, 131} tt }}}} if any element of 
\colorbox[RGB]{253,246,227}{{{{\color[RGB]{101, 123, 131} s }}}} satisfies 
\colorbox[RGB]{253,246,227}{{{{\color[RGB]{101, 123, 131} p }}}}\paragraph{wseq.all}
\par
Returns 
\colorbox[RGB]{253,246,227}{{{{\color[RGB]{101, 123, 131} tt }}}} if every element of 
\colorbox[RGB]{253,246,227}{{{{\color[RGB]{101, 123, 131} s }}}} satisfies 
\colorbox[RGB]{253,246,227}{{{{\color[RGB]{101, 123, 131} p }}}}\paragraph{wseq.scanl}
\par
Apply a function to the elements of the sequence to produce a sequence
of partial results. (There is no 
\colorbox[RGB]{253,246,227}{{{{\color[RGB]{101, 123, 131} scanr }}}} because this would require
working from the end of the sequence, which may not exist.)
\paragraph{wseq.inits}
\par
Get the weak sequence of initial segments of the input sequence
\paragraph{wseq.collect}
\par
Like take, but does not wait for a result. Calculates 
\colorbox[RGB]{253,246,227}{{{{\color[RGB]{101, 123, 131} n }}}} steps of
computation and returns the sequence computed so far
\paragraph{wseq.append}
\par
Append two weak sequences. As with 
\colorbox[RGB]{253,246,227}{{{{\color[RGB]{101, 123, 131} seq.append }}}}, this may not use
the second sequence if the first one takes forever to compute
\paragraph{wseq.map}
\par
Map a function over a weak sequence
\paragraph{wseq.join}
\par
Flatten a sequence of weak sequences. (Note that this allows
empty sequences, unlike 
\colorbox[RGB]{253,246,227}{{{{\color[RGB]{101, 123, 131} seq.join }}}}.)
\paragraph{wseq.bind}
\par
Monadic bind operator for weak sequences
\paragraph{wseq.lift\_rel}
\par
Two weak sequences are 
\colorbox[RGB]{253,246,227}{{{{\color[RGB]{101, 123, 131} lift\_rel R }}}} related if they are either both empty,
or they are both nonempty and the heads are 
\colorbox[RGB]{253,246,227}{{{{\color[RGB]{101, 123, 131} R }}}} related and the tails are
\colorbox[RGB]{253,246,227}{{{{\color[RGB]{101, 123, 131} lift\_rel R }}}} related. (This is a coinductive definition.)
\paragraph{wseq.equiv}
\par
If two sequences are equivalent, then they have the same values and
the same computational behavior (i.e. if one loops forever then so does
the other), although they may differ in the number of 
\colorbox[RGB]{253,246,227}{{{{\color[RGB]{101, 123, 131} think }}}}s needed to
arrive at the answer.
\paragraph{wseq.to\_seq}
\par
Given a productive weak sequence, we can collapse all the 
\colorbox[RGB]{253,246,227}{{{{\color[RGB]{101, 123, 131} think }}}}s to
produce a sequence.
\paragraph{wseq.ret}
\par
The monadic 
\colorbox[RGB]{253,246,227}{{{{\color[RGB]{101, 123, 131} return a }}}} is a singleton list containing 
\colorbox[RGB]{253,246,227}{{{{\color[RGB]{101, 123, 131} a }}}}.
\section{set/basic.lean}\paragraph{set.strict\_subset}
\par
\colorbox[RGB]{253,246,227}{{{{\color[RGB]{101, 123, 131} s ⊂ t }}}} means that 
\colorbox[RGB]{253,246,227}{{{{\color[RGB]{101, 123, 131} s }}}} is a strict subset of 
\colorbox[RGB]{253,246,227}{{{{\color[RGB]{101, 123, 131} t }}}}, that is, 
\colorbox[RGB]{253,246,227}{{{{\color[RGB]{101, 123, 131} s ⊆ t }}}} but 
\colorbox[RGB]{253,246,227}{{{{\color[RGB]{101, 123, 131} s  }}}{{{\color[RGB]{181, 137, 0} ≠ }}}{{{\color[RGB]{101, 123, 131}  t }}}}.
\paragraph{set.preimage}
\par
The preimage of 
\colorbox[RGB]{253,246,227}{{{{\color[RGB]{101, 123, 131} s : set β }}}} by 
\colorbox[RGB]{253,246,227}{{{{\color[RGB]{101, 123, 131} f : α  }}}{{{\color[RGB]{133, 153, 0} → }}}{{{\color[RGB]{101, 123, 131}  β }}}}, written 
\colorbox[RGB]{253,246,227}{{{{\color[RGB]{101, 123, 131} f  }}}{{{\color[RGB]{181, 137, 0} ⁻¹ }}}{{{\color[RGB]{101, 123, 131} ' s }}}},
is the set of 
\colorbox[RGB]{253,246,227}{{{{\color[RGB]{101, 123, 131} x : α }}}} such that 
\colorbox[RGB]{253,246,227}{{{{\color[RGB]{101, 123, 131} f x ∈ s }}}}.
\paragraph{set.eq\_on}
\par
Two functions 
\colorbox[RGB]{253,246,227}{{{{\color[RGB]{101, 123, 131} f₁ f₂ : α  }}}{{{\color[RGB]{133, 153, 0} → }}}{{{\color[RGB]{101, 123, 131}  β }}}} are equal on 
\colorbox[RGB]{253,246,227}{{{{\color[RGB]{101, 123, 131} s }}}}if 
\colorbox[RGB]{253,246,227}{{{{\color[RGB]{101, 123, 131} f₁ x  }}}{{{\color[RGB]{181, 137, 0} = }}}{{{\color[RGB]{101, 123, 131}  f₂ x }}}} for all 
\colorbox[RGB]{253,246,227}{{{{\color[RGB]{101, 123, 131} x ∈ a }}}}.
\paragraph{set.image\_image}
\par
A variant of 
\colorbox[RGB]{253,246,227}{{{{\color[RGB]{101, 123, 131} image\_comp }}}}, useful for rewriting
\paragraph{set.image\_id'}
\par
A variant of 
\colorbox[RGB]{253,246,227}{{{{\color[RGB]{101, 123, 131} image\_id }}}}\paragraph{set.range}
\par
Range of a function.
\par
This function is more flexible than 
\colorbox[RGB]{253,246,227}{{{{\color[RGB]{101, 123, 131} f '' univ }}}}, as the image requires that the domain is in Type
and not an arbitrary Sort.
\paragraph{set.pairwise\_on}
\par
The set 
\colorbox[RGB]{253,246,227}{{{{\color[RGB]{101, 123, 131} s }}}} is pairwise 
\colorbox[RGB]{253,246,227}{{{{\color[RGB]{101, 123, 131} r }}}} if 
\colorbox[RGB]{253,246,227}{{{{\color[RGB]{101, 123, 131} r x y }}}} for all 
\emph{distinct
} 
\colorbox[RGB]{253,246,227}{{{{\color[RGB]{101, 123, 131} x y ∈ s }}}}.
\paragraph{set.prod}
\par
The cartesian product 
\colorbox[RGB]{253,246,227}{{{{\color[RGB]{101, 123, 131} prod s t }}}} is the set of 
\colorbox[RGB]{253,246,227}{{{{\color[RGB]{101, 123, 131} (a, b) }}}}such that 
\colorbox[RGB]{253,246,227}{{{{\color[RGB]{101, 123, 131} a ∈ s }}}} and 
\colorbox[RGB]{253,246,227}{{{{\color[RGB]{101, 123, 131} b ∈ t }}}}.
\paragraph{set.inclusion}
\par
\colorbox[RGB]{253,246,227}{{{{\color[RGB]{101, 123, 131} inclusion }}}} is the "identity" function between two subsets 
\colorbox[RGB]{253,246,227}{{{{\color[RGB]{101, 123, 131} s }}}} and 
\colorbox[RGB]{253,246,227}{{{{\color[RGB]{101, 123, 131} t }}}}, where 
\colorbox[RGB]{253,246,227}{{{{\color[RGB]{101, 123, 131} s ⊆ t }}}}\section{set/countable.lean}\paragraph{set.countable}
\par
Countable sets
\par
A set is countable if there exists an encoding of the set into the natural numbers.
An encoding is an injection with a partial inverse, which can be viewed as a
constructive analogue of countability. (For the most part, theorems about
\colorbox[RGB]{253,246,227}{{{{\color[RGB]{101, 123, 131} countable }}}} will be classical and 
\colorbox[RGB]{253,246,227}{{{{\color[RGB]{101, 123, 131} encodable }}}} will be constructive.)
\section{set/default.lean}\section{set/disjointed.lean}\paragraph{pairwise}
\par
A relation 
\colorbox[RGB]{253,246,227}{{{{\color[RGB]{101, 123, 131} p }}}} holds pairwise if 
\colorbox[RGB]{253,246,227}{{{{\color[RGB]{101, 123, 131} p i j }}}} for all 
\colorbox[RGB]{253,246,227}{{{{\color[RGB]{101, 123, 131} i  }}}{{{\color[RGB]{181, 137, 0} ≠ }}}{{{\color[RGB]{101, 123, 131}  j }}}}.
\paragraph{set.disjointed}
\par
If 
\colorbox[RGB]{253,246,227}{{{{\color[RGB]{101, 123, 131} f : ℕ  }}}{{{\color[RGB]{133, 153, 0} → }}}{{{\color[RGB]{101, 123, 131}  set α }}}} is a sequence of sets, then 
\colorbox[RGB]{253,246,227}{{{{\color[RGB]{101, 123, 131} disjointed f }}}} is
the sequence formed with each set subtracted from the later ones
in the sequence, to form a disjoint sequence.
\section{set/enumerate.lean}\section{set/finite.lean}\paragraph{set.finite}
\par
A set is finite if the subtype is a fintype, i.e. there is a
list that enumerates its members.
\paragraph{set.infinite}
\par
A set is infinite if it is not finite.
\paragraph{set.fintype\_of\_finset}
\par
Construct a fintype from a finset with the same elements.
\paragraph{set.to\_finset}
\par
Construct a finset enumerating a set 
\colorbox[RGB]{253,246,227}{{{{\color[RGB]{101, 123, 131} s }}}}, given a 
\colorbox[RGB]{253,246,227}{{{{\color[RGB]{101, 123, 131} fintype }}}} instance.
\paragraph{set.finite.to\_finset}
\par
Get a finset from a finite set
\paragraph{set.finite\_subsets\_of\_finite}
\par
There are finitely many subsets of a given finite set
\section{set/function.lean}\paragraph{set.maps\_to}
\par
\colorbox[RGB]{253,246,227}{{{{\color[RGB]{101, 123, 131} maps\_to f a b }}}} means that the image of 
\colorbox[RGB]{253,246,227}{{{{\color[RGB]{101, 123, 131} a }}}} is contained in 
\colorbox[RGB]{253,246,227}{{{{\color[RGB]{101, 123, 131} b }}}}.
\paragraph{set.inj\_on}
\par
\colorbox[RGB]{253,246,227}{{{{\color[RGB]{101, 123, 131} f }}}} is injective on 
\colorbox[RGB]{253,246,227}{{{{\color[RGB]{101, 123, 131} a }}}} if the restriction of 
\colorbox[RGB]{253,246,227}{{{{\color[RGB]{101, 123, 131} f }}}} to 
\colorbox[RGB]{253,246,227}{{{{\color[RGB]{101, 123, 131} a }}}} is injective.
\paragraph{set.surj\_on}
\par
\colorbox[RGB]{253,246,227}{{{{\color[RGB]{101, 123, 131} f }}}} is surjective from 
\colorbox[RGB]{253,246,227}{{{{\color[RGB]{101, 123, 131} a }}}} to 
\colorbox[RGB]{253,246,227}{{{{\color[RGB]{101, 123, 131} b }}}} if 
\colorbox[RGB]{253,246,227}{{{{\color[RGB]{101, 123, 131} b }}}} is contained in the image of 
\colorbox[RGB]{253,246,227}{{{{\color[RGB]{101, 123, 131} a }}}}.
\paragraph{set.bij\_on}
\par
\colorbox[RGB]{253,246,227}{{{{\color[RGB]{101, 123, 131} f }}}} is bijective from 
\colorbox[RGB]{253,246,227}{{{{\color[RGB]{101, 123, 131} a }}}} to 
\colorbox[RGB]{253,246,227}{{{{\color[RGB]{101, 123, 131} b }}}} if 
\colorbox[RGB]{253,246,227}{{{{\color[RGB]{101, 123, 131} f }}}} is injective on 
\colorbox[RGB]{253,246,227}{{{{\color[RGB]{101, 123, 131} a }}}} and 
\colorbox[RGB]{253,246,227}{{{{\color[RGB]{101, 123, 131} f '' a  }}}{{{\color[RGB]{181, 137, 0} = }}}{{{\color[RGB]{101, 123, 131}  b }}}}.
\paragraph{set.left\_inv\_on}
\par
\colorbox[RGB]{253,246,227}{{{{\color[RGB]{101, 123, 131} g }}}} is a left inverse to 
\colorbox[RGB]{253,246,227}{{{{\color[RGB]{101, 123, 131} f }}}} on 
\colorbox[RGB]{253,246,227}{{{{\color[RGB]{101, 123, 131} a }}}} means that 
\colorbox[RGB]{253,246,227}{{{{\color[RGB]{101, 123, 131} g (f x)  }}}{{{\color[RGB]{181, 137, 0} = }}}{{{\color[RGB]{101, 123, 131}  x }}}} for all 
\colorbox[RGB]{253,246,227}{{{{\color[RGB]{101, 123, 131} x ∈ a }}}}.
\paragraph{set.right\_inv\_on}
\par
\colorbox[RGB]{253,246,227}{{{{\color[RGB]{101, 123, 131} g }}}} is a right inverse to 
\colorbox[RGB]{253,246,227}{{{{\color[RGB]{101, 123, 131} f }}}} on 
\colorbox[RGB]{253,246,227}{{{{\color[RGB]{101, 123, 131} b }}}} if 
\colorbox[RGB]{253,246,227}{{{{\color[RGB]{101, 123, 131} f (g x)  }}}{{{\color[RGB]{181, 137, 0} = }}}{{{\color[RGB]{101, 123, 131}  x }}}} for all 
\colorbox[RGB]{253,246,227}{{{{\color[RGB]{101, 123, 131} x ∈ b }}}}.
\paragraph{set.inv\_on}
\par
\colorbox[RGB]{253,246,227}{{{{\color[RGB]{101, 123, 131} g }}}} is an inverse to 
\colorbox[RGB]{253,246,227}{{{{\color[RGB]{101, 123, 131} f }}}} viewed as a map from 
\colorbox[RGB]{253,246,227}{{{{\color[RGB]{101, 123, 131} a }}}} to 
\colorbox[RGB]{253,246,227}{{{{\color[RGB]{101, 123, 131} b }}}}\section{set/intervals.lean}\paragraph{set.Ioo}
\par
Left-open right-open interval
\paragraph{set.Ico}
\par
Left-closed right-open interval
\paragraph{set.Iio}
\par
Left-infinite right-open interval
\paragraph{set.Icc}
\par
Left-closed right-closed interval
\paragraph{set.Iic}
\par
Left-infinite right-closed interval
\paragraph{set.Ioc}
\par
Left-open right-closed interval
\paragraph{set.Ici}
\par
Left-closed right-infinite interval
\paragraph{set.Ioi}
\par
Left-open right-infinite interval
\section{set/lattice.lean}\paragraph{set.Union}
\par
Indexed union of a family of sets
\paragraph{set.Inter}
\par
Indexed intersection of a family of sets
\paragraph{set.sInter}
\par
Intersection of a set of sets.
\paragraph{disjoint}
\par
Two elements of a lattice are disjoint if their inf is the bottom element.
(This generalizes disjoint sets, viewed as members of the subset lattice.)
\section{sigma/basic.lean}\paragraph{sigma.map}
\par
Map the left and right components of a sigma
\paragraph{psigma.map}
\par
Map the left and right components of a sigma
\section{sigma/default.lean}\section{stream/basic.lean}\section{string/defs.lean}\section{string.lean}\section{subtype.lean}\paragraph{subtype.coind}
\par
Defining a map into a subtype, this can be seen as an "coinduction principle" of 
\colorbox[RGB]{253,246,227}{{{{\color[RGB]{101, 123, 131} subtype }}}}\paragraph{subtype.map}
\par
Restriction of a function to a function on subtypes.
\section{sum.lean}\paragraph{sum.lex}
\par
Lexicographic order for sum. Sort all the 
\colorbox[RGB]{253,246,227}{{{{\color[RGB]{101, 123, 131} inl a }}}} before the 
\colorbox[RGB]{253,246,227}{{{{\color[RGB]{101, 123, 131} inr b }}}},
otherwise use the respective order on 
\colorbox[RGB]{253,246,227}{{{{\color[RGB]{101, 123, 131} α }}}} or 
\colorbox[RGB]{253,246,227}{{{{\color[RGB]{101, 123, 131} β }}}}.
\paragraph{sum.swap}
\par
Swap the factors of a sum type
\section{ulift.lean}\section{vector2.lean}\section{zmod/basic.lean}\section{zmod/quadratic\_reciprocity.lean}\section{zsqrtd/basic.lean}\paragraph{zsqrtd}
\par
The ring of integers adjoined with a square root of 
\colorbox[RGB]{253,246,227}{{{{\color[RGB]{101, 123, 131} d }}}}.
These have the form 
\colorbox[RGB]{253,246,227}{{{{\color[RGB]{101, 123, 131} a  }}}{{{\color[RGB]{181, 137, 0} + }}}{{{\color[RGB]{101, 123, 131}  b √d }}}} where 
\colorbox[RGB]{253,246,227}{{{{\color[RGB]{101, 123, 131} a b : ℤ }}}}. The components
are called 
\colorbox[RGB]{253,246,227}{{{{\color[RGB]{101, 123, 131} re }}}} and 
\colorbox[RGB]{253,246,227}{{{{\color[RGB]{101, 123, 131} im }}}} by analogy to the negative 
\colorbox[RGB]{253,246,227}{{{{\color[RGB]{101, 123, 131} d }}}} case,
but of course both parts are real here since 
\colorbox[RGB]{253,246,227}{{{{\color[RGB]{101, 123, 131} d }}}} is nonnegative.
\paragraph{zsqrtd.of\_int}
\par
Convert an integer to a 
\colorbox[RGB]{253,246,227}{{{{\color[RGB]{101, 123, 131} ℤ√d }}}}\paragraph{zsqrtd.zero}
\par
The zero of the ring
\paragraph{zsqrtd.one}
\par
The one of the ring
\paragraph{zsqrtd.sqrtd}
\par
The representative of 
\colorbox[RGB]{253,246,227}{{{{\color[RGB]{101, 123, 131} √d }}}} in the ring
\paragraph{zsqrtd.add}
\par
Addition of elements of 
\colorbox[RGB]{253,246,227}{{{{\color[RGB]{101, 123, 131} ℤ√d }}}}\paragraph{zsqrtd.neg}
\par
Negation in 
\colorbox[RGB]{253,246,227}{{{{\color[RGB]{101, 123, 131} ℤ√d }}}}\paragraph{zsqrtd.conj}
\par
Conjugation in 
\colorbox[RGB]{253,246,227}{{{{\color[RGB]{101, 123, 131} ℤ√d }}}}. The conjugate of 
\colorbox[RGB]{253,246,227}{{{{\color[RGB]{101, 123, 131} a  }}}{{{\color[RGB]{181, 137, 0} + }}}{{{\color[RGB]{101, 123, 131}  b √d }}}} is 
\colorbox[RGB]{253,246,227}{{{{\color[RGB]{101, 123, 131} a  }}}{{{\color[RGB]{181, 137, 0} - }}}{{{\color[RGB]{101, 123, 131}  b √d }}}}.
\paragraph{zsqrtd.mul}
\par
Multiplication in 
\colorbox[RGB]{253,246,227}{{{{\color[RGB]{101, 123, 131} ℤ√d }}}}\paragraph{zsqrtd.sq\_le}
\par
Read 
\colorbox[RGB]{253,246,227}{{{{\color[RGB]{101, 123, 131} sq\_le a c b d }}}} as 
\colorbox[RGB]{253,246,227}{{{{\color[RGB]{101, 123, 131} a √c  }}}{{{\color[RGB]{181, 137, 0} ≤ }}}{{{\color[RGB]{101, 123, 131}  b √d }}}}\paragraph{zsqrtd.nonnegg}
\par
"Generalized" 
\colorbox[RGB]{253,246,227}{{{{\color[RGB]{101, 123, 131} nonneg }}}}. 
\colorbox[RGB]{253,246,227}{{{{\color[RGB]{101, 123, 131} nonnegg c d x y }}}} means 
\colorbox[RGB]{253,246,227}{{{{\color[RGB]{101, 123, 131} a √c  }}}{{{\color[RGB]{181, 137, 0} + }}}{{{\color[RGB]{101, 123, 131}  b √d  }}}{{{\color[RGB]{181, 137, 0} ≥ }}}{{{\color[RGB]{101, 123, 131}   }}}{{{\color[RGB]{108, 113, 196} 0 }}}};
we are interested in the case 
\colorbox[RGB]{253,246,227}{{{{\color[RGB]{101, 123, 131} c  }}}{{{\color[RGB]{181, 137, 0} = }}}{{{\color[RGB]{101, 123, 131}   }}}{{{\color[RGB]{108, 113, 196} 1 }}}} but this is more symmetric
\paragraph{zsqrtd.nonneg}
\par
Nonnegativity of an element of 
\colorbox[RGB]{253,246,227}{{{{\color[RGB]{101, 123, 131} ℤ√d }}}}.
\paragraph{zsqrtd.nonsquare}
\par
A nonsquare is a natural number that is not equal to the square of an
integer. This is implemented as a typeclass because it's a necessary condition
for much of the Pell equation theory.
\section{zsqrtd/gaussian\_int.lean}\section{algebra/CommRing/adjunctions.lean}\section{algebra/CommRing/basic.lean}\paragraph{Ring}
\par
The category of rings.
\paragraph{CommRing}
\par
The category of commutative rings.
\paragraph{CommRing.to\_Ring}
\par
The functor from commutative rings to rings.
\paragraph{CommRing.forget\_to\_CommMon}
\par
The forgetful functor from commutative rings to (multiplicative) commutative monoids.
\section{algebra/CommRing/colimits.lean}\section{algebra/CommRing/default.lean}\section{algebra/CommRing/limits.lean}\section{algebra/Mon/basic.lean}\paragraph{Mon}
\par
The category of monoids and monoid morphisms.
\paragraph{CommMon}
\par
The category of commutative monoids and monoid morphisms.
\paragraph{CommMon.forget\_to\_Mon}
\par
The forgetful functor from commutative monoids to monoids.
\section{algebra/Mon/colimits.lean}\section{algebra/Mon/default.lean}\section{algebra/archimedean.lean}\paragraph{fract}
\par
The fractional part fract r of r is just r - ⌊r⌋
\paragraph{ceil}
\par
\colorbox[RGB]{253,246,227}{{{{\color[RGB]{101, 123, 131} ceil x }}}} is the smallest integer 
\colorbox[RGB]{253,246,227}{{{{\color[RGB]{101, 123, 131} z }}}} such that 
\colorbox[RGB]{253,246,227}{{{{\color[RGB]{101, 123, 131} x  }}}{{{\color[RGB]{181, 137, 0} ≤ }}}{{{\color[RGB]{101, 123, 131}  z }}}}\paragraph{round}
\par
\colorbox[RGB]{253,246,227}{{{{\color[RGB]{101, 123, 131} round }}}} rounds a number to the nearest integer. 
\colorbox[RGB]{253,246,227}{{{{\color[RGB]{101, 123, 131} round ( }}}{{{\color[RGB]{108, 113, 196} 1 }}}{{{\color[RGB]{101, 123, 131}   }}}{{{\color[RGB]{181, 137, 0} / }}}{{{\color[RGB]{101, 123, 131}   }}}{{{\color[RGB]{108, 113, 196} 2 }}}{{{\color[RGB]{101, 123, 131} )  }}}{{{\color[RGB]{181, 137, 0} = }}}{{{\color[RGB]{101, 123, 131}   }}}{{{\color[RGB]{108, 113, 196} 1 }}}}\section{algebra/associated.lean}\paragraph{is\_unit}
\par
is unit
\paragraph{prime}
\par
prime element of a semiring
\paragraph{irreducible}
\par
\colorbox[RGB]{253,246,227}{{{{\color[RGB]{101, 123, 131} irreducible p }}}} states that 
\colorbox[RGB]{253,246,227}{{{{\color[RGB]{101, 123, 131} p }}}} is non-unit and only factors into units.
\par
We explicitly avoid stating that 
\colorbox[RGB]{253,246,227}{{{{\color[RGB]{101, 123, 131} p }}}} is non-zero, this would require a semiring. Assuming only a
monoid allows us to reuse irreducible for associated elements.
\section{algebra/big\_operators.lean}\paragraph{finset.prod}
\par
\colorbox[RGB]{253,246,227}{{{{\color[RGB]{101, 123, 131} prod s f }}}} is the product of 
\colorbox[RGB]{253,246,227}{{{{\color[RGB]{101, 123, 131} f x }}}} as 
\colorbox[RGB]{253,246,227}{{{{\color[RGB]{101, 123, 131} x }}}} ranges over the elements of the finite set 
\colorbox[RGB]{253,246,227}{{{{\color[RGB]{101, 123, 131} s }}}}.
\paragraph{finset.sum\_range\_id\_mul\_two}
\par
Gauss' summation formula
\paragraph{finset.sum\_range\_id}
\par
Gauss' summation formula
\section{algebra/char\_p.lean}\paragraph{char\_p}
\par
The generator of the kernel of the unique homomorphism ℤ → α for a semiring α
\paragraph{ring\_char}
\par
Noncomuptable function that outputs the unique characteristic of a semiring.
\paragraph{frobenius}
\par
The frobenius map that sends x to x\textasciicircum{}p
\section{algebra/char\_zero.lean}\paragraph{char\_zero}
\par
Typeclass for monoids with characteristic zero.
(This is usually stated on fields but it makes sense for any additive monoid with 1.)
\section{algebra/default.lean}\section{algebra/direct\_sum.lean}\section{algebra/euclidean\_domain.lean}\paragraph{euclidean\_domain.xgcd}
\par
Use the extended GCD algorithm to generate the 
\colorbox[RGB]{253,246,227}{{{{\color[RGB]{101, 123, 131} a }}}} and 
\colorbox[RGB]{253,246,227}{{{{\color[RGB]{101, 123, 131} b }}}} values
satisfying 
\colorbox[RGB]{253,246,227}{{{{\color[RGB]{101, 123, 131} gcd x y  }}}{{{\color[RGB]{181, 137, 0} = }}}{{{\color[RGB]{101, 123, 131}  x  }}}{{{\color[RGB]{181, 137, 0} * }}}{{{\color[RGB]{101, 123, 131}  a  }}}{{{\color[RGB]{181, 137, 0} + }}}{{{\color[RGB]{101, 123, 131}  y  }}}{{{\color[RGB]{181, 137, 0} * }}}{{{\color[RGB]{101, 123, 131}  b }}}}.
\paragraph{euclidean\_domain.gcd\_a}
\par
The extended GCD 
\colorbox[RGB]{253,246,227}{{{{\color[RGB]{101, 123, 131} a }}}} value in the equation 
\colorbox[RGB]{253,246,227}{{{{\color[RGB]{101, 123, 131} gcd x y  }}}{{{\color[RGB]{181, 137, 0} = }}}{{{\color[RGB]{101, 123, 131}  x  }}}{{{\color[RGB]{181, 137, 0} * }}}{{{\color[RGB]{101, 123, 131}  a  }}}{{{\color[RGB]{181, 137, 0} + }}}{{{\color[RGB]{101, 123, 131}  y  }}}{{{\color[RGB]{181, 137, 0} * }}}{{{\color[RGB]{101, 123, 131}  b }}}}.
\paragraph{euclidean\_domain.gcd\_b}
\par
The extended GCD 
\colorbox[RGB]{253,246,227}{{{{\color[RGB]{101, 123, 131} b }}}} value in the equation 
\colorbox[RGB]{253,246,227}{{{{\color[RGB]{101, 123, 131} gcd x y  }}}{{{\color[RGB]{181, 137, 0} = }}}{{{\color[RGB]{101, 123, 131}  x  }}}{{{\color[RGB]{181, 137, 0} * }}}{{{\color[RGB]{101, 123, 131}  a  }}}{{{\color[RGB]{181, 137, 0} + }}}{{{\color[RGB]{101, 123, 131}  y  }}}{{{\color[RGB]{181, 137, 0} * }}}{{{\color[RGB]{101, 123, 131}  b }}}}.
\section{algebra/field.lean}\paragraph{division\_ring\_has\_div'}
\par
Core version 
\colorbox[RGB]{253,246,227}{{{{\color[RGB]{101, 123, 131} division\_ring\_has\_div }}}} erratically requires two instances of 
\colorbox[RGB]{253,246,227}{{{{\color[RGB]{101, 123, 131} division\_ring }}}}\paragraph{units.mk0}
\par
Embed an element of a division ring into the unit group.
By combining this function with the operations on units,
or the 
\colorbox[RGB]{253,246,227}{{{{\color[RGB]{181, 137, 0} / }}}{{{\color[RGB]{101, 123, 131} ₚ }}}} operation, it is possible to write a division
as a partial function with three arguments.
\section{algebra/field\_power.lean}\section{algebra/free.lean}\section{algebra/gcd\_domain.lean}\paragraph{normalization\_domain}
\par
Normalization domain: multiplying with 
\colorbox[RGB]{253,246,227}{{{{\color[RGB]{101, 123, 131} norm\_unit }}}} gives a normal form for associated elements.
\paragraph{gcd\_domain}
\par
GCD domain: an integral domain with normalization and 
\colorbox[RGB]{253,246,227}{{{{\color[RGB]{101, 123, 131} gcd }}}} (greatest common divisor) and
\colorbox[RGB]{253,246,227}{{{{\color[RGB]{101, 123, 131} lcm }}}} (least common multiple) operations. In this setting 
\colorbox[RGB]{253,246,227}{{{{\color[RGB]{101, 123, 131} gcd }}}} and 
\colorbox[RGB]{253,246,227}{{{{\color[RGB]{101, 123, 131} lcm }}}} form a bounded lattice on
the associated elements where 
\colorbox[RGB]{253,246,227}{{{{\color[RGB]{101, 123, 131} gcd }}}} is the infimum, 
\colorbox[RGB]{253,246,227}{{{{\color[RGB]{101, 123, 131} lcm }}}} is the supremum, 
\colorbox[RGB]{253,246,227}{{{{\color[RGB]{108, 113, 196} 1 }}}} is bottom, and
\colorbox[RGB]{253,246,227}{{{{\color[RGB]{108, 113, 196} 0 }}}} is top. The type class focuses on 
\colorbox[RGB]{253,246,227}{{{{\color[RGB]{101, 123, 131} gcd }}}} and we derive the correpsonding 
\colorbox[RGB]{253,246,227}{{{{\color[RGB]{101, 123, 131} lcm }}}} facts from 
\colorbox[RGB]{253,246,227}{{{{\color[RGB]{101, 123, 131} gcd }}}}.
\paragraph{int.lcm}
\par
ℤ specific version of least common multiple.
\section{algebra/group/anti\_hom.lean}\paragraph{is\_group\_anti\_hom}
\par
Predicate for group anti-homomorphism, or a homomorphism
into the opposite group.
\section{algebra/group/basic.lean}\section{algebra/group/conj.lean}\section{algebra/group/default.lean}\section{algebra/group/free\_monoid.lean}\section{algebra/group/hom.lean}\paragraph{is\_group\_hom}
\par
Predicate for group homomorphism.
\section{algebra/group/to\_additive.lean}\section{algebra/group/type\_tags.lean}\section{algebra/group/units.lean}\paragraph{divp}
\par
Partial division. It is defined when the
second argument is invertible, and unlike the division operator
in 
\colorbox[RGB]{253,246,227}{{{{\color[RGB]{101, 123, 131} division\_ring }}}} it is not totalized at zero.
\section{algebra/group/units\_hom.lean}\section{algebra/group/with\_one.lean}\section{algebra/group\_power.lean}\paragraph{monoid.pow}
\par
The power operation in a monoid. 
\colorbox[RGB]{253,246,227}{{{{\color[RGB]{101, 123, 131} a\textasciicircum{}n  }}}{{{\color[RGB]{181, 137, 0} = }}}{{{\color[RGB]{101, 123, 131}  a }}}{{{\color[RGB]{181, 137, 0} * }}}{{{\color[RGB]{101, 123, 131} a }}}{{{\color[RGB]{181, 137, 0} * }}}{{{\color[RGB]{101, 123, 131} ... }}}{{{\color[RGB]{181, 137, 0} * }}}{{{\color[RGB]{101, 123, 131} a }}}} n times.
\paragraph{gpow}
\par
The power operation in a group. This extends 
\colorbox[RGB]{253,246,227}{{{{\color[RGB]{101, 123, 131} monoid.pow }}}} to negative integers
with the definition 
\colorbox[RGB]{253,246,227}{{{{\color[RGB]{101, 123, 131} a\textasciicircum{}( }}}{{{\color[RGB]{181, 137, 0} - }}}{{{\color[RGB]{101, 123, 131} n)  }}}{{{\color[RGB]{181, 137, 0} = }}}{{{\color[RGB]{101, 123, 131}  (a\textasciicircum{}n) }}}{{{\color[RGB]{181, 137, 0} ⁻¹ }}}}.
\section{algebra/module.lean}\paragraph{semimodule}
\par
A semimodule is a generalization of vector spaces to a scalar semiring.
It consists of a scalar semiring 
\colorbox[RGB]{253,246,227}{{{{\color[RGB]{101, 123, 131} α }}}} and an additive monoid of "vectors" 
\colorbox[RGB]{253,246,227}{{{{\color[RGB]{101, 123, 131} β }}}},
connected by a "scalar multiplication" operation 
\colorbox[RGB]{253,246,227}{{{{\color[RGB]{101, 123, 131} r • x : β }}}}(where 
\colorbox[RGB]{253,246,227}{{{{\color[RGB]{101, 123, 131} r : α }}}} and 
\colorbox[RGB]{253,246,227}{{{{\color[RGB]{101, 123, 131} x : β }}}}) with some natural associativity and
distributivity axioms similar to those on a ring.
\paragraph{module}
\par
A module is a generalization of vector spaces to a scalar ring.
It consists of a scalar ring 
\colorbox[RGB]{253,246,227}{{{{\color[RGB]{101, 123, 131} α }}}} and an additive group of "vectors" 
\colorbox[RGB]{253,246,227}{{{{\color[RGB]{101, 123, 131} β }}}},
connected by a "scalar multiplication" operation 
\colorbox[RGB]{253,246,227}{{{{\color[RGB]{101, 123, 131} r • x : β }}}}(where 
\colorbox[RGB]{253,246,227}{{{{\color[RGB]{101, 123, 131} r : α }}}} and 
\colorbox[RGB]{253,246,227}{{{{\color[RGB]{101, 123, 131} x : β }}}}) with some natural associativity and
distributivity axioms similar to those on a ring.
\paragraph{submodule}
\par
A submodule of a module is one which is closed under vector operations.
This is a sufficient condition for the subset of vectors in the submodule
to themselves form a module.
\paragraph{vector\_space}
\par
A vector space is the same as a module, except the scalar ring is actually
a field. (This adds commutativity of the multiplication and existence of inverses.)
This is the traditional generalization of spaces like 
\colorbox[RGB]{253,246,227}{{{{\color[RGB]{101, 123, 131} ℝ\textasciicircum{}n }}}}, which have a natural
addition operation and a way to multiply them by real numbers, but no multiplication
operation between vectors.
\paragraph{subspace}
\par
Subspace of a vector space. Defined to equal 
\colorbox[RGB]{253,246,227}{{{{\color[RGB]{101, 123, 131} submodule }}}}.
\section{algebra/opposites.lean}\section{algebra/order.lean}\paragraph{le\_implies\_le\_of\_le\_of\_le}
\par
monotonicity of 
\colorbox[RGB]{253,246,227}{{{{\color[RGB]{181, 137, 0} ≤ }}}} with respect to 
\colorbox[RGB]{253,246,227}{{{{\color[RGB]{101, 123, 131} → }}}}\paragraph{ordering.compares}
\par
\colorbox[RGB]{253,246,227}{{{{\color[RGB]{101, 123, 131} compares o a b }}}} means that 
\colorbox[RGB]{253,246,227}{{{{\color[RGB]{101, 123, 131} a }}}} and 
\colorbox[RGB]{253,246,227}{{{{\color[RGB]{101, 123, 131} b }}}} have the ordering relation
\colorbox[RGB]{253,246,227}{{{{\color[RGB]{101, 123, 131} o }}}} between them, assuming that the relation 
\colorbox[RGB]{253,246,227}{{{{\color[RGB]{101, 123, 131} a  }}}{{{\color[RGB]{181, 137, 0} < }}}{{{\color[RGB]{101, 123, 131}  b }}}} is defined
\section{algebra/order\_functions.lean}\paragraph{strict\_mono}
\par
A function 
\colorbox[RGB]{253,246,227}{{{{\color[RGB]{101, 123, 131} f }}}} is strictly monotone if 
\colorbox[RGB]{253,246,227}{{{{\color[RGB]{101, 123, 131} a  }}}{{{\color[RGB]{181, 137, 0} < }}}{{{\color[RGB]{101, 123, 131}  b }}}} implies 
\colorbox[RGB]{253,246,227}{{{{\color[RGB]{101, 123, 131} f a  }}}{{{\color[RGB]{181, 137, 0} < }}}{{{\color[RGB]{101, 123, 131}  f b }}}}.
\section{algebra/ordered\_field.lean}\section{algebra/ordered\_group.lean}\paragraph{ordered\_comm\_monoid}
\par
An ordered (additive) commutative monoid is a commutative monoid
with a partial order such that addition is an order embedding, i.e.
\colorbox[RGB]{253,246,227}{{{{\color[RGB]{101, 123, 131} a  }}}{{{\color[RGB]{181, 137, 0} + }}}{{{\color[RGB]{101, 123, 131}  b  }}}{{{\color[RGB]{181, 137, 0} ≤ }}}{{{\color[RGB]{101, 123, 131}  a  }}}{{{\color[RGB]{181, 137, 0} + }}}{{{\color[RGB]{101, 123, 131}  c  }}}{{{\color[RGB]{181, 137, 0} ↔ }}}{{{\color[RGB]{101, 123, 131}  b  }}}{{{\color[RGB]{181, 137, 0} ≤ }}}{{{\color[RGB]{101, 123, 131}  c }}}}. These monoids are automatically cancellative.
\paragraph{canonically\_ordered\_monoid}
\par
A canonically ordered monoid is an ordered commutative monoid
in which the ordering coincides with the divisibility relation,
which is to say, 
\colorbox[RGB]{253,246,227}{{{{\color[RGB]{101, 123, 131} a  }}}{{{\color[RGB]{181, 137, 0} ≤ }}}{{{\color[RGB]{101, 123, 131}  b }}}} iff there exists 
\colorbox[RGB]{253,246,227}{{{{\color[RGB]{101, 123, 131} c }}}} with 
\colorbox[RGB]{253,246,227}{{{{\color[RGB]{101, 123, 131} b  }}}{{{\color[RGB]{181, 137, 0} = }}}{{{\color[RGB]{101, 123, 131}  a  }}}{{{\color[RGB]{181, 137, 0} + }}}{{{\color[RGB]{101, 123, 131}  c }}}}.
This is satisfied by the natural numbers, for example, but not
the integers or other ordered groups.
\paragraph{nonneg\_comm\_group}
\par
This is not so much a new structure as a construction mechanism
for ordered groups, by specifying only the "positive cone" of the group.
\section{algebra/ordered\_ring.lean}\paragraph{nonneg\_ring}
\par
Extend 
\colorbox[RGB]{253,246,227}{{{{\color[RGB]{101, 123, 131} nonneg\_comm\_group }}}} to support ordered rings
specified by their nonnegative elements
\paragraph{linear\_nonneg\_ring}
\par
Extend 
\colorbox[RGB]{253,246,227}{{{{\color[RGB]{101, 123, 131} nonneg\_comm\_group }}}} to support linearly ordered rings
specified by their nonnegative elements
\section{algebra/pi\_instances.lean}\paragraph{prod.inl}
\par
Left injection function for the inner product
From a vector space (and also group and module) perspective the product is the same as the sum of
two vector spaces. 
\colorbox[RGB]{253,246,227}{{{{\color[RGB]{101, 123, 131} inl }}}} and 
\colorbox[RGB]{253,246,227}{{{{\color[RGB]{101, 123, 131} inr }}}} provide the corresponding injection functions.
\paragraph{prod.inr}
\par
Right injection function for the inner product
\section{algebra/pointwise.lean}\section{algebra/punit\_instances.lean}\section{algebra/ring.lean}\paragraph{has\_div\_of\_division\_ring}
\par
this is needed for compatibility between Lean 3.4.2 and Lean 3.5.0c
\paragraph{domain}
\par
A domain is a ring with no zero divisors, i.e. satisfying
the condition 
\colorbox[RGB]{253,246,227}{{{{\color[RGB]{101, 123, 131} a  }}}{{{\color[RGB]{181, 137, 0} * }}}{{{\color[RGB]{101, 123, 131}  b  }}}{{{\color[RGB]{181, 137, 0} = }}}{{{\color[RGB]{101, 123, 131}   }}}{{{\color[RGB]{108, 113, 196} 0 }}}{{{\color[RGB]{101, 123, 131}   }}}{{{\color[RGB]{181, 137, 0} ↔ }}}{{{\color[RGB]{101, 123, 131}  a  }}}{{{\color[RGB]{181, 137, 0} = }}}{{{\color[RGB]{101, 123, 131}   }}}{{{\color[RGB]{108, 113, 196} 0 }}}{{{\color[RGB]{101, 123, 131}   }}}{{{\color[RGB]{181, 137, 0} ∨ }}}{{{\color[RGB]{101, 123, 131}  b  }}}{{{\color[RGB]{181, 137, 0} = }}}{{{\color[RGB]{101, 123, 131}   }}}{{{\color[RGB]{108, 113, 196} 0 }}}}. Alternatively, a domain
is an integral domain without assuming commutativity of multiplication.
\section{algebraic\_geometry/presheafed\_space.lean}\section{algebraic\_geometry/stalks.lean}\section{analysis/asymptotics.lean}\section{analysis/complex/exponential.lean}\paragraph{real.angle.angle}
\par
The type of angles
\paragraph{real.arcsin}
\par
Inverse of the 
\colorbox[RGB]{253,246,227}{{{{\color[RGB]{101, 123, 131} sin }}}} function, returns values in the range 
\colorbox[RGB]{253,246,227}{{{{\color[RGB]{181, 137, 0} - }}}{{{\color[RGB]{101, 123, 131} π  }}}{{{\color[RGB]{181, 137, 0} / }}}{{{\color[RGB]{101, 123, 131}   }}}{{{\color[RGB]{108, 113, 196} 2 }}}{{{\color[RGB]{101, 123, 131}   }}}{{{\color[RGB]{181, 137, 0} ≤ }}}{{{\color[RGB]{101, 123, 131}  arcsin x }}}} and 
\colorbox[RGB]{253,246,227}{{{{\color[RGB]{101, 123, 131} arcsin x  }}}{{{\color[RGB]{181, 137, 0} ≤ }}}{{{\color[RGB]{101, 123, 131}  π  }}}{{{\color[RGB]{181, 137, 0} / }}}{{{\color[RGB]{101, 123, 131}   }}}{{{\color[RGB]{108, 113, 196} 2 }}}}.
If the argument is not between 
\colorbox[RGB]{253,246,227}{{{{\color[RGB]{181, 137, 0} - }}}{{{\color[RGB]{108, 113, 196} 1 }}}} and 
\colorbox[RGB]{253,246,227}{{{{\color[RGB]{108, 113, 196} 1 }}}} it defaults to 
\colorbox[RGB]{253,246,227}{{{{\color[RGB]{108, 113, 196} 0 }}}}\paragraph{real.arccos}
\par
Inverse of the 
\colorbox[RGB]{253,246,227}{{{{\color[RGB]{101, 123, 131} cos }}}} function, returns values in the range 
\colorbox[RGB]{253,246,227}{{{{\color[RGB]{108, 113, 196} 0 }}}{{{\color[RGB]{101, 123, 131}   }}}{{{\color[RGB]{181, 137, 0} ≤ }}}{{{\color[RGB]{101, 123, 131}  arccos x }}}} and 
\colorbox[RGB]{253,246,227}{{{{\color[RGB]{101, 123, 131} arccos x  }}}{{{\color[RGB]{181, 137, 0} ≤ }}}{{{\color[RGB]{101, 123, 131}  π }}}}.
If the argument is not between 
\colorbox[RGB]{253,246,227}{{{{\color[RGB]{181, 137, 0} - }}}{{{\color[RGB]{108, 113, 196} 1 }}}} and 
\colorbox[RGB]{253,246,227}{{{{\color[RGB]{108, 113, 196} 1 }}}} it defaults to 
\colorbox[RGB]{253,246,227}{{{{\color[RGB]{101, 123, 131} π  }}}{{{\color[RGB]{181, 137, 0} / }}}{{{\color[RGB]{101, 123, 131}   }}}{{{\color[RGB]{108, 113, 196} 2 }}}}\paragraph{real.arctan}
\par
Inverse of the 
\colorbox[RGB]{253,246,227}{{{{\color[RGB]{101, 123, 131} tan }}}} function, returns values in the range 
\colorbox[RGB]{253,246,227}{{{{\color[RGB]{181, 137, 0} - }}}{{{\color[RGB]{101, 123, 131} π  }}}{{{\color[RGB]{181, 137, 0} / }}}{{{\color[RGB]{101, 123, 131}   }}}{{{\color[RGB]{108, 113, 196} 2 }}}{{{\color[RGB]{101, 123, 131}   }}}{{{\color[RGB]{181, 137, 0} < }}}{{{\color[RGB]{101, 123, 131}  arctan x }}}} and 
\colorbox[RGB]{253,246,227}{{{{\color[RGB]{101, 123, 131} arctan x  }}}{{{\color[RGB]{181, 137, 0} < }}}{{{\color[RGB]{101, 123, 131}  π  }}}{{{\color[RGB]{181, 137, 0} / }}}{{{\color[RGB]{101, 123, 131}   }}}{{{\color[RGB]{108, 113, 196} 2 }}}}\paragraph{complex.arg}
\par
\colorbox[RGB]{253,246,227}{{{{\color[RGB]{101, 123, 131} arg }}}} returns values in the range (-π, π
{]}
, such that for 
\colorbox[RGB]{253,246,227}{{{{\color[RGB]{101, 123, 131} x  }}}{{{\color[RGB]{181, 137, 0} ≠ }}}{{{\color[RGB]{101, 123, 131}   }}}{{{\color[RGB]{108, 113, 196} 0 }}}},
\colorbox[RGB]{253,246,227}{{{{\color[RGB]{101, 123, 131} sin (arg x)  }}}{{{\color[RGB]{181, 137, 0} = }}}{{{\color[RGB]{101, 123, 131}  x.im  }}}{{{\color[RGB]{181, 137, 0} / }}}{{{\color[RGB]{101, 123, 131}  x,abs }}}} and 
\colorbox[RGB]{253,246,227}{{{{\color[RGB]{101, 123, 131} cos (arg x)  }}}{{{\color[RGB]{181, 137, 0} = }}}{{{\color[RGB]{101, 123, 131}  x.re  }}}{{{\color[RGB]{181, 137, 0} / }}}{{{\color[RGB]{101, 123, 131}  x.abs }}}},
\colorbox[RGB]{253,246,227}{{{{\color[RGB]{101, 123, 131} arg  }}}{{{\color[RGB]{108, 113, 196} 0 }}}} defaults to 
\colorbox[RGB]{253,246,227}{{{{\color[RGB]{108, 113, 196} 0 }}}}\paragraph{complex.log}
\par
Inverse of the 
\colorbox[RGB]{253,246,227}{{{{\color[RGB]{101, 123, 131} exp }}}} function. Returns values such that 
\colorbox[RGB]{253,246,227}{{{{\color[RGB]{101, 123, 131} (log x).im  }}}{{{\color[RGB]{181, 137, 0} > }}}{{{\color[RGB]{101, 123, 131}   }}}{{{\color[RGB]{181, 137, 0} - }}}{{{\color[RGB]{101, 123, 131}  π }}}} and 
\colorbox[RGB]{253,246,227}{{{{\color[RGB]{101, 123, 131} (log x).im  }}}{{{\color[RGB]{181, 137, 0} ≤ }}}{{{\color[RGB]{101, 123, 131}  π }}}}.
\colorbox[RGB]{253,246,227}{{{{\color[RGB]{101, 123, 131} log  }}}{{{\color[RGB]{108, 113, 196} 0 }}}{{{\color[RGB]{101, 123, 131}   }}}{{{\color[RGB]{181, 137, 0} = }}}{{{\color[RGB]{101, 123, 131}   }}}{{{\color[RGB]{108, 113, 196} 0 }}}}\section{analysis/complex/polynomial.lean}\paragraph{complex.exists\_root}
\par
The fundamental theorem of algebra. Every non constant complex polynomial
has a root
\section{analysis/convex.lean}\paragraph{convex}
\par
Convexity of sets
\paragraph{convex\_iff}
\par
Alternative definition of set convexity
\paragraph{convex\_iff\_div}
\par
Another alternative definition of set convexity
\paragraph{segment}
\par
Segments in a vector space
\paragraph{convex\_segment\_iff}
\par
Alternative defintion of set convexity using segments
\paragraph{convex\_on}
\par
Convexity of functions
\section{analysis/normed\_space/banach.lean}\paragraph{exists\_preimage\_norm\_le}
\par
The Banach open mapping theorem: if a bounded linear map between Banach spaces is onto, then
any point has a preimage with controlled norm.
\paragraph{open\_mapping}
\par
The Banach open mapping theorem: a surjective bounded linear map between Banach spaces is open.
\paragraph{linear\_equiv.is\_bounded\_inv}
\par
If a bounded linear map is a bijection, then its inverse is also a bounded linear map.
\section{analysis/normed\_space/basic.lean}\paragraph{normed\_group.of\_add\_dist}
\par
Construct a normed group from a translation invariant distance
\paragraph{normed\_group.of\_add\_dist'}
\par
Construct a normed group from a translation invariant distance
\paragraph{normed\_group.core}
\par
A normed group can be built from a norm that satisfies algebraic properties. This is
formalised in this structure.
\paragraph{rescale\_to\_shell}
\par
If there is a scalar 
\colorbox[RGB]{253,246,227}{{{{\color[RGB]{101, 123, 131} c }}}} with 
\colorbox[RGB]{253,246,227}{{{{\color[RGB]{101, 123, 131} ∥c∥ }}}{{{\color[RGB]{181, 137, 0} > }}}{{{\color[RGB]{108, 113, 196} 1 }}}}, then any element can be moved by scalar multiplication to
any shell of width 
\colorbox[RGB]{253,246,227}{{{{\color[RGB]{101, 123, 131} ∥c∥ }}}}. Also recap information on the norm of the rescaling element that shows
up in applications.
\section{analysis/normed\_space/bounded\_linear\_maps.lean}\paragraph{continuous\_linear\_map.is\_bounded\_linear\_map}
\par
A continuous linear map satisfies 
\colorbox[RGB]{253,246,227}{{{{\color[RGB]{101, 123, 131} is\_bounded\_linear\_map }}}}\paragraph{is\_bounded\_linear\_map.to\_continuous\_linear\_map}
\par
Construct a continuous linear map from is\_bounded\_linear\_map
\paragraph{is\_bounded\_bilinear\_map.linear\_deriv}
\par
Definition of the derivative of a bilinear map 
\colorbox[RGB]{253,246,227}{{{{\color[RGB]{101, 123, 131} f }}}}, given at a point 
\colorbox[RGB]{253,246,227}{{{{\color[RGB]{101, 123, 131} p }}}} by
\colorbox[RGB]{253,246,227}{{{{\color[RGB]{101, 123, 131} q ↦ f(p. }}}{{{\color[RGB]{108, 113, 196} 1 }}}{{{\color[RGB]{101, 123, 131} , q. }}}{{{\color[RGB]{108, 113, 196} 2 }}}{{{\color[RGB]{101, 123, 131} )  }}}{{{\color[RGB]{181, 137, 0} + }}}{{{\color[RGB]{101, 123, 131}  f(q. }}}{{{\color[RGB]{108, 113, 196} 1 }}}{{{\color[RGB]{101, 123, 131} , p. }}}{{{\color[RGB]{108, 113, 196} 2 }}}{{{\color[RGB]{101, 123, 131} ) }}}} as in the standard formula for the derivative of a product.
We define this function here a bounded linear map from 
\colorbox[RGB]{253,246,227}{{{{\color[RGB]{101, 123, 131} E × F }}}} to 
\colorbox[RGB]{253,246,227}{{{{\color[RGB]{101, 123, 131} G }}}}. The fact that this
is indeed the derivative of 
\colorbox[RGB]{253,246,227}{{{{\color[RGB]{101, 123, 131} f }}}} is proved in 
\colorbox[RGB]{253,246,227}{{{{\color[RGB]{101, 123, 131} is\_bounded\_bilinear\_map.has\_fderiv\_at }}}} in
\colorbox[RGB]{253,246,227}{{{{\color[RGB]{101, 123, 131} deriv.lean }}}}\paragraph{is\_bounded\_bilinear\_map.is\_bounded\_linear\_map\_deriv}
\par
Given a bounded bilinear map 
\colorbox[RGB]{253,246,227}{{{{\color[RGB]{101, 123, 131} f }}}}, the map associating to a point 
\colorbox[RGB]{253,246,227}{{{{\color[RGB]{101, 123, 131} p }}}} the derivative of 
\colorbox[RGB]{253,246,227}{{{{\color[RGB]{101, 123, 131} f }}}} at
\colorbox[RGB]{253,246,227}{{{{\color[RGB]{101, 123, 131} p }}}} is itself a bounded linear map.
\section{analysis/normed\_space/deriv.lean}\paragraph{unique\_diff\_within\_at}
\par
A property ensuring that the tangent cone to 
\colorbox[RGB]{253,246,227}{{{{\color[RGB]{101, 123, 131} s }}}} at 
\colorbox[RGB]{253,246,227}{{{{\color[RGB]{101, 123, 131} x }}}} spans a dense subset of the whole space.
The main role of this property is to ensure that the differential within 
\colorbox[RGB]{253,246,227}{{{{\color[RGB]{101, 123, 131} s }}}} at 
\colorbox[RGB]{253,246,227}{{{{\color[RGB]{101, 123, 131} x }}}} is unique,
hence this name. The uniqueness it asserts is proved in 
\colorbox[RGB]{253,246,227}{{{{\color[RGB]{101, 123, 131} unique\_diff\_within\_at.eq }}}}\paragraph{unique\_diff\_on}
\par
A property ensuring that the tangent cone to 
\colorbox[RGB]{253,246,227}{{{{\color[RGB]{101, 123, 131} s }}}} at any of its points spans a dense subset of
the whole space.  The main role of this property is to ensure that the differential along 
\colorbox[RGB]{253,246,227}{{{{\color[RGB]{101, 123, 131} s }}}} is
unique, hence this name. The uniqueness it asserts is proved in 
\colorbox[RGB]{253,246,227}{{{{\color[RGB]{101, 123, 131} unique\_diff\_on.eq }}}}\paragraph{tangent\_cone\_at.lim\_zero}
\par
Auxiliary lemma ensuring that, under the assumptions defining the tangent cone,
the sequence 
\colorbox[RGB]{253,246,227}{{{{\color[RGB]{101, 123, 131} d }}}} tends to 0 at infinity.
\paragraph{tangent\_cone\_inter\_open}
\par
Intersecting with an open set does not change the tangent cone.
\paragraph{unique\_diff\_within\_at.eq}
\par
\colorbox[RGB]{253,246,227}{{{{\color[RGB]{101, 123, 131} unique\_diff\_within\_at }}}} achieves its goal: it implies the uniqueness of the derivative.
\paragraph{has\_fderiv\_at.comp}
\par
The chain rule.
\section{analysis/normed\_space/operator\_norm.lean}\paragraph{continuous\_linear\_map.bound}
\par
A continuous linear map between normed spaces is bounded when the field is nondiscrete.
The continuity ensures boundedness on a ball of some radius δ. The nondiscreteness is then
used to rescale any element into an element of norm in 
{[}
δ/C, δ
{]}
, whose image has a controlled norm.
The norm control for the original element follows by rescaling.
\paragraph{continuous\_linear\_map.op\_norm}
\par
The operator norm of a continuous linear map is the inf of all its bounds.
\paragraph{continuous\_linear\_map.le\_op\_norm}
\par
The fundamental property of the operator norm: ∥f x∥ ≤ ∥f∥ * ∥x∥.
\paragraph{continuous\_linear\_map.unit\_le\_op\_norm}
\par
The image of the unit ball under a continuous linear map is bounded.
\paragraph{continuous\_linear\_map.op\_norm\_le\_bound}
\par
If one controls the norm of every A x, then one controls the norm of A.
\paragraph{continuous\_linear\_map.op\_norm\_triangle}
\par
The operator norm satisfies the triangle inequality.
\paragraph{continuous\_linear\_map.op\_norm\_zero\_iff}
\par
An operator is zero iff its norm vanishes.
\paragraph{continuous\_linear\_map.norm\_id}
\par
The norm of the identity is at most 1. It is in fact 1, except when the space is trivial where
it is 0. It means that one can not do better than an inequality in general.
\paragraph{continuous\_linear\_map.op\_norm\_smul}
\par
The operator norm is homogeneous.
\paragraph{continuous\_linear\_map.to\_normed\_group}
\par
Continuous linear maps themselves form a normed space with respect to
the operator norm.
\paragraph{continuous\_linear\_map.op\_norm\_comp\_le}
\par
The operator norm is submultiplicative.
\paragraph{continuous\_linear\_map.lipschitz}
\par
continuous linear maps are Lipschitz continuous.
\paragraph{continuous\_linear\_map.scalar\_prod\_space\_iso\_norm}
\par
The norm of the tensor product of a scalar linear map and of an element of a normed space
is the product of the norms.
\section{analysis/specific\_limits.lean}\section{category/applicative.lean}\section{category/basic.lean}\paragraph{fish}
\par
This is the Kleisli composition
\section{category/bifunctor.lean}\section{category/bitraversable/basic.lean}\section{category/bitraversable/instances.lean}\section{category/bitraversable/lemmas.lean}\section{category/fold.lean}\paragraph{monoid.foldl}
\par
For a list, foldl f x 
{[}
y₀,y₁
{]}
 reduces as follows
calc  foldl f x 
{[}
y₀,y₁
{]}
= foldl f (f x y₀) 
{[}
y₁
{]}
      : rfl
... = foldl f (f (f x y₀) y₁) 
{[}
{]}
 : rfl
... = f (f x y₀) y₁              : rfl
\par
with f : α → β → α
x : α
{[}
y₀,y₁
{]}
 : list β
\par
We can view the above as a composition of functions:
\par
... = f (f x y₀) y₁              : rfl
... = flip f y₁ (flip f y₀ x)    : rfl
... = (flip f y₁ ∘ flip f y₀) x  : rfl
\par
We can use traverse and const to construct this composition:
\par
calc   const.run (traverse (λ y, const.mk' (flip f y)) 
{[}
y₀,y₁
{]}
) x
= const.run ((::) 
<
\$> const.mk' (flip f y₀) 
<
\emph{> traverse (λ y, const.mk' (flip f y)) 
{[}
y₁
{]}
) x
...  = const.run ((::) 
<
\$> const.mk' (flip f y₀) 
<
}> ( (::) 
<
\$> const.mk' (flip f y₁) 
<
\emph{> traverse (λ y, const.mk' (flip f y)) 
{[}
{]}
 )) x
...  = const.run ((::) 
<
\$> const.mk' (flip f y₀) 
<
}> ( (::) 
<
\$> const.mk' (flip f y₁) 
<
\emph{> pure 
{[}
{]}
 )) x
...  = const.run ( ((::) 
<
\$> const.mk' (flip f y₁) 
<
}> pure 
{[}
{]}
) ∘ ((::) 
<
\$> const.mk' (flip f y₀)) ) x
...  = const.run ( const.mk' (flip f y₁) ∘ const.mk' (flip f y₀) ) x
...  = const.run ( flip f y₁ ∘ flip f y₀ ) x
...  = f (f x y₀) y₁
\par
And this is how 
\colorbox[RGB]{253,246,227}{{{{\color[RGB]{101, 123, 131} const }}}} turns a monoid into an applicative functor and
how the monoid of endofunctions define 
\colorbox[RGB]{253,246,227}{{{{\color[RGB]{101, 123, 131} foldl }}}}.
\paragraph{traversable.to\_list}
\par
Conceptually, 
\colorbox[RGB]{253,246,227}{{{{\color[RGB]{101, 123, 131} to\_list }}}} collects all the elements of a collection
in a list. This idea is formalized by
\par
\colorbox[RGB]{253,246,227}{{{{\color[RGB]{133, 153, 0} lemma }}}{{{\color[RGB]{101, 123, 131}   }}}{{{\color[RGB]{211, 54, 130} to\_list\_spec }}}{{{\color[RGB]{101, 123, 131}   }}}{{{\color[RGB]{101, 123, 131} (x : t α) : to\_list x  }}}{{{\color[RGB]{181, 137, 0} = }}}{{{\color[RGB]{101, 123, 131}  fold\_map free\_monoid.mk x }}}}.
\par
The definition of 
\colorbox[RGB]{253,246,227}{{{{\color[RGB]{101, 123, 131} to\_list }}}} is based on 
\colorbox[RGB]{253,246,227}{{{{\color[RGB]{101, 123, 131} foldl }}}} and 
\colorbox[RGB]{253,246,227}{{{{\color[RGB]{101, 123, 131} list.cons }}}} for
speed. It is faster than using 
\colorbox[RGB]{253,246,227}{{{{\color[RGB]{101, 123, 131} fold\_map free\_monoid.mk }}}} because, by
using 
\colorbox[RGB]{253,246,227}{{{{\color[RGB]{101, 123, 131} foldl }}}} and 
\colorbox[RGB]{253,246,227}{{{{\color[RGB]{101, 123, 131} list.cons }}}}, each insertion is done in constant
time. As a consequence, 
\colorbox[RGB]{253,246,227}{{{{\color[RGB]{101, 123, 131} to\_list }}}} performs in linear.
\par
On the other hand, 
\colorbox[RGB]{253,246,227}{{{{\color[RGB]{101, 123, 131} fold\_map free\_monoid.mk }}}} creates a singleton list
around each element and concatenates all the resulting lists. In
\colorbox[RGB]{253,246,227}{{{{\color[RGB]{101, 123, 131} xs  }}}{{{\color[RGB]{181, 137, 0} + }}}{{{\color[RGB]{181, 137, 0} + }}}{{{\color[RGB]{101, 123, 131}  ys }}}}, concatenation takes a time proportional to 
\colorbox[RGB]{253,246,227}{{{{\color[RGB]{101, 123, 131} length xs }}}}. Since
the order in which concatenation is evaluated is unspecified, nothing
prevents each element of the traversable to be appended at the end
\colorbox[RGB]{253,246,227}{{{{\color[RGB]{101, 123, 131} xs  }}}{{{\color[RGB]{181, 137, 0} + }}}{{{\color[RGB]{181, 137, 0} + }}}{{{\color[RGB]{101, 123, 131}  {[}x{]} }}}} which would yield a 
\colorbox[RGB]{253,246,227}{{{{\color[RGB]{101, 123, 131} O(n²) }}}} run time.
\section{category/functor.lean}\paragraph{functor.comp}
\par
\colorbox[RGB]{253,246,227}{{{{\color[RGB]{101, 123, 131} functor.comp }}}} is a wrapper around 
\colorbox[RGB]{253,246,227}{{{{\color[RGB]{101, 123, 131} function.comp }}}} for types.
It prevents Lean's type class resolution mechanism from trying
a 
\colorbox[RGB]{253,246,227}{{{{\color[RGB]{101, 123, 131} functor (comp F id) }}}} when 
\colorbox[RGB]{253,246,227}{{{{\color[RGB]{101, 123, 131} functor F }}}} would do.
\section{category/monad/basic.lean}\section{category/monad/cont.lean}\section{category/monad/writer.lean}\paragraph{monad\_writer}
\par
An implementation of 
\href{https://hackage.haskell.org/package/mtl-2.2.2/docs/Control-Monad-Reader-Class.html#t:MonadReader}{}MonadReader
.
It does not contain 
\colorbox[RGB]{253,246,227}{{{{\color[RGB]{133, 153, 0} local }}}} because this function cannot be lifted using 
\colorbox[RGB]{253,246,227}{{{{\color[RGB]{101, 123, 131} monad\_lift }}}}.
Instead, the 
\colorbox[RGB]{253,246,227}{{{{\color[RGB]{101, 123, 131} monad\_reader\_adapter }}}} class provides the more general 
\colorbox[RGB]{253,246,227}{{{{\color[RGB]{101, 123, 131} adapt\_reader }}}} function.
\par
Note: This class can be seen as a simplification of the more "principled" definition
\\
\colorbox[RGB]{253,246,227}{\parbox{4.5in}{{{{\color[RGB]{133, 153, 0} class }}}{{{\color[RGB]{101, 123, 131}   }}}{{{\color[RGB]{211, 54, 130} monad\_reader }}}{{{\color[RGB]{101, 123, 131}   }}}{{{\color[RGB]{101, 123, 131} (ρ : out\_param ( }}}{{{\color[RGB]{38, 139, 210} Type }}}{{{\color[RGB]{101, 123, 131}  u)) (n :  }}}{{{\color[RGB]{38, 139, 210} Type }}}{{{\color[RGB]{101, 123, 131}  u  }}}{{{\color[RGB]{133, 153, 0} → }}}{{{\color[RGB]{101, 123, 131}   }}}{{{\color[RGB]{38, 139, 210} Type }}}{{{\color[RGB]{101, 123, 131}  u)  }}}{{{\color[RGB]{181, 137, 0} := }}}{{{\color[RGB]{101, 123, 131} 
 }}}\\

{{{\color[RGB]{101, 123, 131} (lift \{\} \{α :  }}}{{{\color[RGB]{38, 139, 210} Type }}}{{{\color[RGB]{101, 123, 131}  u\} : (∀ \{m :  }}}{{{\color[RGB]{38, 139, 210} Type }}}{{{\color[RGB]{101, 123, 131}  u  }}}{{{\color[RGB]{133, 153, 0} → }}}{{{\color[RGB]{101, 123, 131}   }}}{{{\color[RGB]{38, 139, 210} Type }}}{{{\color[RGB]{101, 123, 131}  u\} {[}monad m{]}, reader\_t ρ m α)  }}}{{{\color[RGB]{133, 153, 0} → }}}{{{\color[RGB]{101, 123, 131}  n α)
 }}}\\

}}\paragraph{monad\_writer\_adapter}
\par
Adapt a monad stack, changing the type of its top-most environment.
\par
This class is comparable to 
\href{https://hackage.haskell.org/package/lens-4.15.4/docs/Control-Lens-Zoom.html#t:Magnify}{}Control.Lens.Magnify
, but does not use lenses (why would it), and is derived automatically for any transformer implementing 
\colorbox[RGB]{253,246,227}{{{{\color[RGB]{101, 123, 131} monad\_functor }}}}.
\par
Note: This class can be seen as a simplification of the more "principled" definition
\\
\colorbox[RGB]{253,246,227}{\parbox{4.5in}{{{{\color[RGB]{133, 153, 0} class }}}{{{\color[RGB]{101, 123, 131}   }}}{{{\color[RGB]{211, 54, 130} monad\_reader\_functor }}}{{{\color[RGB]{101, 123, 131}   }}}{{{\color[RGB]{101, 123, 131} (ρ ρ' : out\_param ( }}}{{{\color[RGB]{38, 139, 210} Type }}}{{{\color[RGB]{101, 123, 131}  u)) (n n' :  }}}{{{\color[RGB]{38, 139, 210} Type }}}{{{\color[RGB]{101, 123, 131}  u  }}}{{{\color[RGB]{133, 153, 0} → }}}{{{\color[RGB]{101, 123, 131}   }}}{{{\color[RGB]{38, 139, 210} Type }}}{{{\color[RGB]{101, 123, 131}  u)  }}}{{{\color[RGB]{181, 137, 0} := }}}{{{\color[RGB]{101, 123, 131} 
 }}}\\

{{{\color[RGB]{101, 123, 131} (map \{\} \{α :  }}}{{{\color[RGB]{38, 139, 210} Type }}}{{{\color[RGB]{101, 123, 131}  u\} : (∀ \{m :  }}}{{{\color[RGB]{38, 139, 210} Type }}}{{{\color[RGB]{101, 123, 131}  u  }}}{{{\color[RGB]{133, 153, 0} → }}}{{{\color[RGB]{101, 123, 131}   }}}{{{\color[RGB]{38, 139, 210} Type }}}{{{\color[RGB]{101, 123, 131}  u\} {[}monad m{]}, reader\_t ρ m α  }}}{{{\color[RGB]{133, 153, 0} → }}}{{{\color[RGB]{101, 123, 131}  reader\_t ρ' m α)  }}}{{{\color[RGB]{133, 153, 0} → }}}{{{\color[RGB]{101, 123, 131}  n α  }}}{{{\color[RGB]{133, 153, 0} → }}}{{{\color[RGB]{101, 123, 131}  n' α)
 }}}\\

}}\section{category/traversable/basic.lean}\section{category/traversable/default.lean}\section{category/traversable/derive.lean}\paragraph{tactic.interactive.nested\_map}
\par
similar to 
\colorbox[RGB]{253,246,227}{{{{\color[RGB]{101, 123, 131} nested\_traverse }}}} but for 
\colorbox[RGB]{253,246,227}{{{{\color[RGB]{101, 123, 131} functor }}}}\paragraph{tactic.interactive.map\_field}
\par
similar to 
\colorbox[RGB]{253,246,227}{{{{\color[RGB]{101, 123, 131} traverse\_field }}}} but for 
\colorbox[RGB]{253,246,227}{{{{\color[RGB]{101, 123, 131} functor }}}}\paragraph{tactic.interactive.map\_constructor}
\par
similar to 
\colorbox[RGB]{253,246,227}{{{{\color[RGB]{101, 123, 131} traverse\_constructor }}}} but for 
\colorbox[RGB]{253,246,227}{{{{\color[RGB]{101, 123, 131} functor }}}}\paragraph{tactic.interactive.mk\_map}
\par
derive the 
\colorbox[RGB]{253,246,227}{{{{\color[RGB]{101, 123, 131} map }}}} definition of a 
\colorbox[RGB]{253,246,227}{{{{\color[RGB]{101, 123, 131} functor }}}}\paragraph{tactic.interactive.derive\_map\_equations}
\par
derive the equations for a specific 
\colorbox[RGB]{253,246,227}{{{{\color[RGB]{101, 123, 131} map }}}} definition
\paragraph{\_private.2444255649.seq\_apply\_constructor}
\par
\colorbox[RGB]{253,246,227}{{{{\color[RGB]{101, 123, 131} seq\_apply\_constructor f {[}x,y,z{]} }}}} synthesizes 
\colorbox[RGB]{253,246,227}{{{{\color[RGB]{101, 123, 131} f  }}}{{{\color[RGB]{181, 137, 0} < }}}{{{\color[RGB]{181, 137, 0} * }}}{{{\color[RGB]{181, 137, 0} > }}}{{{\color[RGB]{101, 123, 131}  x  }}}{{{\color[RGB]{181, 137, 0} < }}}{{{\color[RGB]{181, 137, 0} * }}}{{{\color[RGB]{181, 137, 0} > }}}{{{\color[RGB]{101, 123, 131}  y  }}}{{{\color[RGB]{181, 137, 0} < }}}{{{\color[RGB]{181, 137, 0} * }}}{{{\color[RGB]{181, 137, 0} > }}}{{{\color[RGB]{101, 123, 131}  z }}}}\paragraph{tactic.interactive.nested\_traverse}
\par
\colorbox[RGB]{253,246,227}{{{{\color[RGB]{101, 123, 131} nested\_traverse f α (list (array n (list α))) }}}} synthesizes the expression
\colorbox[RGB]{253,246,227}{{{{\color[RGB]{101, 123, 131} traverse (traverse (traverse f)) }}}}. 
\colorbox[RGB]{253,246,227}{{{{\color[RGB]{101, 123, 131} nested\_traverse }}}} assumes that 
\colorbox[RGB]{253,246,227}{{{{\color[RGB]{101, 123, 131} α }}}} appears in
\colorbox[RGB]{253,246,227}{{{{\color[RGB]{101, 123, 131} (list (array n (list α))) }}}}\paragraph{tactic.interactive.traverse\_field}
\par
For a sum type 
\colorbox[RGB]{253,246,227}{{{{\color[RGB]{133, 153, 0} inductive }}}{{{\color[RGB]{101, 123, 131}   }}}{{{\color[RGB]{211, 54, 130} foo }}}{{{\color[RGB]{101, 123, 131}   }}}{{{\color[RGB]{101, 123, 131} (α :  }}}{{{\color[RGB]{38, 139, 210} Type }}}{{{\color[RGB]{101, 123, 131} ) | foo1 : list α  }}}{{{\color[RGB]{133, 153, 0} → }}}{{{\color[RGB]{101, 123, 131}  ℕ  }}}{{{\color[RGB]{133, 153, 0} → }}}{{{\color[RGB]{101, 123, 131}  foo | ... }}}}\colorbox[RGB]{253,246,227}{{{{\color[RGB]{101, 123, 131} traverse\_field `foo appl\_inst f `α `(x : list α) }}}} synthesizes
\colorbox[RGB]{253,246,227}{{{{\color[RGB]{101, 123, 131} traverse f x }}}} as part of traversing 
\colorbox[RGB]{253,246,227}{{{{\color[RGB]{101, 123, 131} foo1 }}}}.
\paragraph{tactic.interactive.traverse\_constructor}
\par
For a sum type 
\colorbox[RGB]{253,246,227}{{{{\color[RGB]{133, 153, 0} inductive }}}{{{\color[RGB]{101, 123, 131}   }}}{{{\color[RGB]{211, 54, 130} foo }}}{{{\color[RGB]{101, 123, 131}   }}}{{{\color[RGB]{101, 123, 131} (α :  }}}{{{\color[RGB]{38, 139, 210} Type }}}{{{\color[RGB]{101, 123, 131} ) | foo1 : list α  }}}{{{\color[RGB]{133, 153, 0} → }}}{{{\color[RGB]{101, 123, 131}  ℕ  }}}{{{\color[RGB]{133, 153, 0} → }}}{{{\color[RGB]{101, 123, 131}  foo | ... }}}}\colorbox[RGB]{253,246,227}{{{{\color[RGB]{101, 123, 131} traverse\_constructor `foo1 `foo appl\_inst f `α `β {[}`(x : list α), `(y : ℕ){]} }}}}synthesizes 
\colorbox[RGB]{253,246,227}{{{{\color[RGB]{101, 123, 131} foo1  }}}{{{\color[RGB]{181, 137, 0} < }}}{{{\color[RGB]{101, 123, 131} \$ }}}{{{\color[RGB]{181, 137, 0} > }}}{{{\color[RGB]{101, 123, 131}  traverse f x  }}}{{{\color[RGB]{181, 137, 0} < }}}{{{\color[RGB]{181, 137, 0} * }}}{{{\color[RGB]{181, 137, 0} > }}}{{{\color[RGB]{101, 123, 131}  pure y. }}}}\paragraph{tactic.interactive.mk\_traverse}
\par
derive the 
\colorbox[RGB]{253,246,227}{{{{\color[RGB]{101, 123, 131} traverse }}}} definition of a 
\colorbox[RGB]{253,246,227}{{{{\color[RGB]{101, 123, 131} traversable }}}} instance
\paragraph{tactic.interactive.derive\_traverse\_equations}
\par
derive the equations for a specific 
\colorbox[RGB]{253,246,227}{{{{\color[RGB]{101, 123, 131} traverse }}}} definition
\section{category/traversable/equiv.lean}\section{category/traversable/instances.lean}\section{category/traversable/lemmas.lean}\section{category\_theory/adjunction/basic.lean}\paragraph{category\_theory.adjunction}
\par
\colorbox[RGB]{253,246,227}{{{{\color[RGB]{101, 123, 131} F ⊣ G }}}} represents the data of an adjunction between two functors
\colorbox[RGB]{253,246,227}{{{{\color[RGB]{101, 123, 131} F : C ⥤ D }}}} and 
\colorbox[RGB]{253,246,227}{{{{\color[RGB]{101, 123, 131} G : D ⥤ C }}}}. 
\colorbox[RGB]{253,246,227}{{{{\color[RGB]{101, 123, 131} F }}}} is the left adjoint and 
\colorbox[RGB]{253,246,227}{{{{\color[RGB]{101, 123, 131} G }}}} is the right adjoint.
\section{category\_theory/adjunction/default.lean}\section{category\_theory/adjunction/limits.lean}\paragraph{category\_theory.adjunction.left\_adjoint\_preserves\_colimits}
\par
A left adjoint preserves colimits.
\paragraph{category\_theory.adjunction.right\_adjoint\_preserves\_limits}
\par
A right adjoint preserves limits.
\section{category\_theory/category.lean}\paragraph{obviously}
\par
The 
\colorbox[RGB]{253,246,227}{{{{\color[RGB]{101, 123, 131} obviously }}}} tactic is a "replaceable" tactic, which means that its meaning is defined by tactics that are defined later with the 
\colorbox[RGB]{253,246,227}{{{{\color[RGB]{88, 110, 117} @{[}obviously{]} }}}} attribute. It is intended for use with 
\colorbox[RGB]{253,246,227}{{{{\color[RGB]{101, 123, 131} auto\_param }}}}s for structure fields.
\paragraph{category\_theory.category}
\par
The typeclass 
\colorbox[RGB]{253,246,227}{{{{\color[RGB]{101, 123, 131} category C }}}} describes morphisms associated to objects of type 
\colorbox[RGB]{253,246,227}{{{{\color[RGB]{101, 123, 131} C }}}}.
The universe levels of the objects and morphisms are unconstrained, and will often need to be
specified explicitly, as 
\colorbox[RGB]{253,246,227}{{{{\color[RGB]{101, 123, 131} category.\{v\} C }}}}. (See also 
\colorbox[RGB]{253,246,227}{{{{\color[RGB]{101, 123, 131} large\_category }}}} and 
\colorbox[RGB]{253,246,227}{{{{\color[RGB]{101, 123, 131} small\_category }}}}.)
\paragraph{category\_theory.large\_category}
\par
A 
\colorbox[RGB]{253,246,227}{{{{\color[RGB]{101, 123, 131} large\_category }}}} has objects in one universe level higher than the universe level of
the morphisms. It is useful for examples such as the category of types, or the category
of groups, etc.
\paragraph{category\_theory.small\_category}
\par
A 
\colorbox[RGB]{253,246,227}{{{{\color[RGB]{101, 123, 131} small\_category }}}} has objects and morphisms in the same universe level.
\section{category\_theory/comma.lean}\section{category\_theory/concrete\_category.lean}\paragraph{category\_theory.concrete\_category}
\par
\colorbox[RGB]{253,246,227}{{{{\color[RGB]{101, 123, 131} concrete\_category  }}}{{{\color[RGB]{181, 137, 0} @ }}}{{{\color[RGB]{101, 123, 131} hom }}}} collects the evidence that a type constructor 
\colorbox[RGB]{253,246,227}{{{{\color[RGB]{101, 123, 131} c }}}} and a
morphism predicate 
\colorbox[RGB]{253,246,227}{{{{\color[RGB]{101, 123, 131} hom }}}} can be thought of as a concrete category.
\par
In a typical example, 
\colorbox[RGB]{253,246,227}{{{{\color[RGB]{101, 123, 131} c }}}} is the type class 
\colorbox[RGB]{253,246,227}{{{{\color[RGB]{101, 123, 131} topological\_space }}}} and 
\colorbox[RGB]{253,246,227}{{{{\color[RGB]{101, 123, 131} hom }}}} is
\colorbox[RGB]{253,246,227}{{{{\color[RGB]{101, 123, 131} continuous }}}}.
\paragraph{category\_theory.bundled}
\par
\colorbox[RGB]{253,246,227}{{{{\color[RGB]{101, 123, 131} bundled }}}} is a type bundled with a type class instance for that type. Only
the type class is exposed as a parameter.
\paragraph{category\_theory.bundled.map}
\par
Map over the bundled structure
\paragraph{category\_theory.forget}
\par
The forgetful functor from a bundled category to 
\colorbox[RGB]{253,246,227}{{{{\color[RGB]{101, 123, 131} Sort }}}}.
\section{category\_theory/const.lean}\paragraph{category\_theory.functor.const\_comp}
\par
These are actually equal, of course, but not definitionally equal
(the equality requires F.map (𝟙 
\_
) = 𝟙 
\_
). A natural isomorphism is
more convenient than an equality between functors (compare id\_to\_iso).
\section{category\_theory/core.lean}\paragraph{category\_theory.core.functor\_to\_core}
\par
A functor from a groupoid to a category C factors through the core of C.
\section{category\_theory/currying.lean}\section{category\_theory/discrete\_category.lean}\section{category\_theory/elements.lean}\section{category\_theory/epi\_mono.lean}\section{category\_theory/eq\_to\_hom.lean}\paragraph{category\_theory.functor.ext}
\par
Proving equality between functors. This isn't an extensionality lemma,
because usually you don't really want to do this.
\section{category\_theory/equivalence.lean}\paragraph{category\_theory.equivalence}
\par
We define an equivalence as a (half)-adjoint equivalence, a pair of functors with
a unit and counit which are natural isomorphisms and the triangle law 
\colorbox[RGB]{253,246,227}{{{{\color[RGB]{101, 123, 131} Fη ≫ εF  }}}{{{\color[RGB]{181, 137, 0} = }}}{{{\color[RGB]{101, 123, 131}   }}}{{{\color[RGB]{108, 113, 196} 1 }}}}, or in other
words the composite 
\colorbox[RGB]{253,246,227}{{{{\color[RGB]{101, 123, 131} F ⟶ FGF ⟶ F }}}} is the identity.
\par
The triangle equation is written as a family of equalities between morphisms, it is more
complicated if we write it as an equality of natural transformations, because then we would have
to insert natural transformations like 
\colorbox[RGB]{253,246,227}{{{{\color[RGB]{101, 123, 131} F ⟶ F1 }}}}.
\paragraph{category\_theory.equivalence.unit\_inverse\_comp}
\par
The other triangle equality. The proof follows the following proof in Globular:
http://globular.science/1905.001
\paragraph{category\_theory.is\_equivalence}
\par
A functor that is part of a (half) adjoint equivalence
\section{category\_theory/full\_subcategory.lean}\section{category\_theory/fully\_faithful.lean}\section{category\_theory/functor.lean}\paragraph{category\_theory.functor}
\par
\colorbox[RGB]{253,246,227}{{{{\color[RGB]{101, 123, 131} functor C D }}}} represents a functor between categories 
\colorbox[RGB]{253,246,227}{{{{\color[RGB]{101, 123, 131} C }}}} and 
\colorbox[RGB]{253,246,227}{{{{\color[RGB]{101, 123, 131} D }}}}.
\par
To apply a functor 
\colorbox[RGB]{253,246,227}{{{{\color[RGB]{101, 123, 131} F }}}} to an object use 
\colorbox[RGB]{253,246,227}{{{{\color[RGB]{101, 123, 131} F.obj X }}}}, and to a morphism use 
\colorbox[RGB]{253,246,227}{{{{\color[RGB]{101, 123, 131} F.map f }}}}.
\par
The axiom 
\colorbox[RGB]{253,246,227}{{{{\color[RGB]{101, 123, 131} map\_id\_lemma }}}} expresses preservation of identities, and
\colorbox[RGB]{253,246,227}{{{{\color[RGB]{101, 123, 131} map\_comp\_lemma }}}} expresses functoriality.
\paragraph{category\_theory.functor.id}
\par
\colorbox[RGB]{253,246,227}{{{{\color[RGB]{101, 123, 131} functor.id C }}}} is the identity functor on a category 
\colorbox[RGB]{253,246,227}{{{{\color[RGB]{101, 123, 131} C }}}}.
\paragraph{category\_theory.functor.comp}
\par
\colorbox[RGB]{253,246,227}{{{{\color[RGB]{101, 123, 131} F ⋙ G }}}} is the composition of a functor 
\colorbox[RGB]{253,246,227}{{{{\color[RGB]{101, 123, 131} F }}}} and a functor 
\colorbox[RGB]{253,246,227}{{{{\color[RGB]{101, 123, 131} G }}}} (
\colorbox[RGB]{253,246,227}{{{{\color[RGB]{101, 123, 131} F }}}} first, then 
\colorbox[RGB]{253,246,227}{{{{\color[RGB]{101, 123, 131} G }}}}).
\section{category\_theory/functor\_category.lean}\paragraph{category\_theory.functor.category}
\par
\colorbox[RGB]{253,246,227}{{{{\color[RGB]{101, 123, 131} functor.category C D }}}} gives the category structure on functors and natural transformations
between categories 
\colorbox[RGB]{253,246,227}{{{{\color[RGB]{101, 123, 131} C }}}} and 
\colorbox[RGB]{253,246,227}{{{{\color[RGB]{101, 123, 131} D }}}}.
\par
Notice that if 
\colorbox[RGB]{253,246,227}{{{{\color[RGB]{101, 123, 131} C }}}} and 
\colorbox[RGB]{253,246,227}{{{{\color[RGB]{101, 123, 131} D }}}} are both small categories at the same universe level,
this is another small category at that level.
However if 
\colorbox[RGB]{253,246,227}{{{{\color[RGB]{101, 123, 131} C }}}} and 
\colorbox[RGB]{253,246,227}{{{{\color[RGB]{101, 123, 131} D }}}} are both large categories at the same universe level,
this is a small category at the next higher level.
\paragraph{category\_theory.nat\_trans.hcomp}
\par
\colorbox[RGB]{253,246,227}{{{{\color[RGB]{101, 123, 131} hcomp α β }}}} is the horizontal composition of natural transformations.
\section{category\_theory/groupoid.lean}\section{category\_theory/instances/kleisli.lean}\section{category\_theory/instances/rel.lean}\section{category\_theory/isomorphism.lean}\paragraph{category\_theory.is\_iso}
\par
\colorbox[RGB]{253,246,227}{{{{\color[RGB]{101, 123, 131} is\_iso }}}} typeclass expressing that a morphism is invertible.
This contains the data of the inverse, but is a subsingleton type.
\section{category\_theory/limits/cones.lean}\paragraph{category\_theory.functor.cones}
\par
\colorbox[RGB]{253,246,227}{{{{\color[RGB]{101, 123, 131} F.cones }}}} is the functor assigning to an object 
\colorbox[RGB]{253,246,227}{{{{\color[RGB]{101, 123, 131} X }}}} the type of
natural transformations from the constant functor with value 
\colorbox[RGB]{253,246,227}{{{{\color[RGB]{101, 123, 131} X }}}} to 
\colorbox[RGB]{253,246,227}{{{{\color[RGB]{101, 123, 131} F }}}}.
An object representing this functor is a limit of 
\colorbox[RGB]{253,246,227}{{{{\color[RGB]{101, 123, 131} F }}}}.
\paragraph{category\_theory.functor.cocones}
\par
\colorbox[RGB]{253,246,227}{{{{\color[RGB]{101, 123, 131} F.cocones }}}} is the functor assigning to an object 
\colorbox[RGB]{253,246,227}{{{{\color[RGB]{101, 123, 131} X }}}} the type of
natural transformations from 
\colorbox[RGB]{253,246,227}{{{{\color[RGB]{101, 123, 131} F }}}} to the constant functor with value 
\colorbox[RGB]{253,246,227}{{{{\color[RGB]{101, 123, 131} X }}}}.
An object corepresenting this functor is a colimit of 
\colorbox[RGB]{253,246,227}{{{{\color[RGB]{101, 123, 131} F }}}}.
\paragraph{category\_theory.limits.cone}
\par
A 
\colorbox[RGB]{253,246,227}{{{{\color[RGB]{101, 123, 131} c : cone F }}}} is:
\begin{itemize}\item an object 
\colorbox[RGB]{253,246,227}{{{{\color[RGB]{101, 123, 131} c.X }}}} and

\item a natural transformation 
\colorbox[RGB]{253,246,227}{{{{\color[RGB]{101, 123, 131} c.π : c.X ⟶ F }}}} from the constant 
\colorbox[RGB]{253,246,227}{{{{\color[RGB]{101, 123, 131} c.X }}}} functor to 
\colorbox[RGB]{253,246,227}{{{{\color[RGB]{101, 123, 131} F }}}}.

\end{itemize}\par
\colorbox[RGB]{253,246,227}{{{{\color[RGB]{101, 123, 131} cone F }}}} is equivalent, in the obvious way, to 
\colorbox[RGB]{253,246,227}{{{{\color[RGB]{101, 123, 131} Σ X, F.cones.obj X }}}}.
\paragraph{category\_theory.limits.cocone}
\par
A 
\colorbox[RGB]{253,246,227}{{{{\color[RGB]{101, 123, 131} c : cocone F }}}} is
\begin{itemize}\item an object 
\colorbox[RGB]{253,246,227}{{{{\color[RGB]{101, 123, 131} c.X }}}} and

\item a natural transformation 
\colorbox[RGB]{253,246,227}{{{{\color[RGB]{101, 123, 131} c.ι : F ⟶ c.X }}}} from 
\colorbox[RGB]{253,246,227}{{{{\color[RGB]{101, 123, 131} F }}}} to the constant 
\colorbox[RGB]{253,246,227}{{{{\color[RGB]{101, 123, 131} c.X }}}} functor.

\end{itemize}\par
\colorbox[RGB]{253,246,227}{{{{\color[RGB]{101, 123, 131} cocone F }}}} is equivalent, in the obvious way, to 
\colorbox[RGB]{253,246,227}{{{{\color[RGB]{101, 123, 131} Σ X, F.cocones.obj X }}}}.
\paragraph{category\_theory.limits.cone.extend}
\par
A map to the vertex of a cone induces a cone by composition.
\paragraph{category\_theory.limits.cocone.extend}
\par
A map from the vertex of a cocone induces a cocone by composition.
\paragraph{category\_theory.limits.cones.ext}
\par
To give an isomorphism between cones, it suffices to give an
isomorphism between their vertices which commutes with the cone
maps.
\paragraph{category\_theory.limits.cocones.ext}
\par
To give an isomorphism between cocones, it suffices to give an
isomorphism between their vertices which commutes with the cocone
maps.
\paragraph{category\_theory.functor.map\_cone}
\par
The image of a cone in C under a functor G : C ⥤ D is a cone in D.
\paragraph{category\_theory.functor.map\_cocone}
\par
The image of a cocone in C under a functor G : C ⥤ D is a cocone in D.
\section{category\_theory/limits/functor\_category.lean}\section{category\_theory/limits/lattice.lean}\section{category\_theory/limits/limits.lean}\paragraph{category\_theory.limits.is\_limit}
\par
A cone 
\colorbox[RGB]{253,246,227}{{{{\color[RGB]{101, 123, 131} t }}}} on 
\colorbox[RGB]{253,246,227}{{{{\color[RGB]{101, 123, 131} F }}}} is a limit cone if each cone on 
\colorbox[RGB]{253,246,227}{{{{\color[RGB]{101, 123, 131} F }}}} admits a unique
cone morphism to 
\colorbox[RGB]{253,246,227}{{{{\color[RGB]{101, 123, 131} t }}}}.
\paragraph{category\_theory.limits.is\_limit.unique}
\par
Limit cones on 
\colorbox[RGB]{253,246,227}{{{{\color[RGB]{101, 123, 131} F }}}} are unique up to isomorphism.
\paragraph{category\_theory.limits.is\_limit.hom\_ext}
\par
Two morphisms into a limit are equal if their compositions with
each cone morphism are equal.
\paragraph{category\_theory.limits.is\_limit.hom\_iso}
\par
The universal property of a limit cone: a map 
\colorbox[RGB]{253,246,227}{{{{\color[RGB]{101, 123, 131} W ⟶ X }}}} is the same as
a cone on 
\colorbox[RGB]{253,246,227}{{{{\color[RGB]{101, 123, 131} F }}}} with vertex 
\colorbox[RGB]{253,246,227}{{{{\color[RGB]{101, 123, 131} W }}}}.
\paragraph{category\_theory.limits.is\_limit.nat\_iso}
\par
The limit of 
\colorbox[RGB]{253,246,227}{{{{\color[RGB]{101, 123, 131} F }}}} represents the functor taking 
\colorbox[RGB]{253,246,227}{{{{\color[RGB]{101, 123, 131} W }}}} to
the set of cones on 
\colorbox[RGB]{253,246,227}{{{{\color[RGB]{101, 123, 131} F }}}} with vertex 
\colorbox[RGB]{253,246,227}{{{{\color[RGB]{101, 123, 131} W }}}}.
\paragraph{category\_theory.limits.is\_limit.of\_faithful}
\par
If G : C → D is a faithful functor which sends t to a limit cone,
then it suffices to check that the induced maps for the image of t
can be lifted to maps of C.
\paragraph{category\_theory.limits.is\_colimit}
\par
A cocone 
\colorbox[RGB]{253,246,227}{{{{\color[RGB]{101, 123, 131} t }}}} on 
\colorbox[RGB]{253,246,227}{{{{\color[RGB]{101, 123, 131} F }}}} is a colimit cocone if each cocone on 
\colorbox[RGB]{253,246,227}{{{{\color[RGB]{101, 123, 131} F }}}} admits a unique
cocone morphism from 
\colorbox[RGB]{253,246,227}{{{{\color[RGB]{101, 123, 131} t }}}}.
\paragraph{category\_theory.limits.is\_colimit.unique}
\par
Limit cones on 
\colorbox[RGB]{253,246,227}{{{{\color[RGB]{101, 123, 131} F }}}} are unique up to isomorphism.
\paragraph{category\_theory.limits.is\_colimit.hom\_ext}
\par
Two morphisms out of a colimit are equal if their compositions with
each cocone morphism are equal.
\paragraph{category\_theory.limits.is\_colimit.hom\_iso}
\par
The universal property of a colimit cocone: a map 
\colorbox[RGB]{253,246,227}{{{{\color[RGB]{101, 123, 131} X ⟶ W }}}} is the same as
a cocone on 
\colorbox[RGB]{253,246,227}{{{{\color[RGB]{101, 123, 131} F }}}} with vertex 
\colorbox[RGB]{253,246,227}{{{{\color[RGB]{101, 123, 131} W }}}}.
\paragraph{category\_theory.limits.is\_colimit.nat\_iso}
\par
The colimit of 
\colorbox[RGB]{253,246,227}{{{{\color[RGB]{101, 123, 131} F }}}} represents the functor taking 
\colorbox[RGB]{253,246,227}{{{{\color[RGB]{101, 123, 131} W }}}} to
the set of cocones on 
\colorbox[RGB]{253,246,227}{{{{\color[RGB]{101, 123, 131} F }}}} with vertex 
\colorbox[RGB]{253,246,227}{{{{\color[RGB]{101, 123, 131} W }}}}.
\paragraph{category\_theory.limits.is\_colimit.of\_faithful}
\par
If G : C → D is a faithful functor which sends t to a colimit cocone,
then it suffices to check that the induced maps for the image of t
can be lifted to maps of C.
\paragraph{category\_theory.limits.has\_limit}
\par
\colorbox[RGB]{253,246,227}{{{{\color[RGB]{101, 123, 131} has\_limit F }}}} represents a particular chosen limit of the diagram 
\colorbox[RGB]{253,246,227}{{{{\color[RGB]{101, 123, 131} F }}}}.
\paragraph{category\_theory.limits.has\_limits\_of\_shape}
\par
\colorbox[RGB]{253,246,227}{{{{\color[RGB]{101, 123, 131} C }}}} has limits of shape 
\colorbox[RGB]{253,246,227}{{{{\color[RGB]{101, 123, 131} J }}}} if we have chosen a particular limit of
every functor 
\colorbox[RGB]{253,246,227}{{{{\color[RGB]{101, 123, 131} F : J ⥤ C }}}}.
\paragraph{category\_theory.limits.has\_limits}
\par
\colorbox[RGB]{253,246,227}{{{{\color[RGB]{101, 123, 131} C }}}} has all (small) limits if it has limits of every shape.
\paragraph{category\_theory.limits.lim}
\par
\colorbox[RGB]{253,246,227}{{{{\color[RGB]{101, 123, 131} limit F }}}} is functorial in 
\colorbox[RGB]{253,246,227}{{{{\color[RGB]{101, 123, 131} F }}}}, when 
\colorbox[RGB]{253,246,227}{{{{\color[RGB]{101, 123, 131} C }}}} has all limits of shape 
\colorbox[RGB]{253,246,227}{{{{\color[RGB]{101, 123, 131} J }}}}.
\paragraph{category\_theory.limits.has\_colimit}
\par
\colorbox[RGB]{253,246,227}{{{{\color[RGB]{101, 123, 131} has\_colimit F }}}} represents a particular chosen colimit of the diagram 
\colorbox[RGB]{253,246,227}{{{{\color[RGB]{101, 123, 131} F }}}}.
\paragraph{category\_theory.limits.has\_colimits\_of\_shape}
\par
\colorbox[RGB]{253,246,227}{{{{\color[RGB]{101, 123, 131} C }}}} has colimits of shape 
\colorbox[RGB]{253,246,227}{{{{\color[RGB]{101, 123, 131} J }}}} if we have chosen a particular colimit of
every functor 
\colorbox[RGB]{253,246,227}{{{{\color[RGB]{101, 123, 131} F : J ⥤ C }}}}.
\paragraph{category\_theory.limits.has\_colimits}
\par
\colorbox[RGB]{253,246,227}{{{{\color[RGB]{101, 123, 131} C }}}} has all (small) colimits if it has colimits of every shape.
\paragraph{category\_theory.limits.colimit.ι\_desc\_assoc}
\par
We have lots of lemmas describing how to simplify 
\colorbox[RGB]{253,246,227}{{{{\color[RGB]{101, 123, 131} colimit.ι F j ≫ \_ }}}},
and combined with 
\colorbox[RGB]{253,246,227}{{{{\color[RGB]{101, 123, 131} colimit.ext }}}} we rely on these lemmas for many calculations.
\par
However, since 
\colorbox[RGB]{253,246,227}{{{{\color[RGB]{101, 123, 131} category.assoc }}}} is a 
\colorbox[RGB]{253,246,227}{{{{\color[RGB]{88, 110, 117} @{[}simp{]} }}}} lemma, often expressions are
right associated, and it's hard to apply these lemmas about 
\colorbox[RGB]{253,246,227}{{{{\color[RGB]{101, 123, 131} colimit.ι }}}}.
\par
We thus define some additional 
\colorbox[RGB]{253,246,227}{{{{\color[RGB]{88, 110, 117} @{[}simp{]} }}}} lemmas, with an arbitrary extra morphism.
\paragraph{category\_theory.limits.colim}
\par
\colorbox[RGB]{253,246,227}{{{{\color[RGB]{101, 123, 131} colimit F }}}} is functorial in 
\colorbox[RGB]{253,246,227}{{{{\color[RGB]{101, 123, 131} F }}}}, when 
\colorbox[RGB]{253,246,227}{{{{\color[RGB]{101, 123, 131} C }}}} has all colimits of shape 
\colorbox[RGB]{253,246,227}{{{{\color[RGB]{101, 123, 131} J }}}}.
\section{category\_theory/limits/opposites.lean}\section{category\_theory/limits/over.lean}\section{category\_theory/limits/preserves.lean}\paragraph{category\_theory.limits.preserves\_limit\_of\_preserves\_limit\_cone}
\par
If F preserves one limit cone for the diagram K,
then it preserves any limit cone for K.
\paragraph{category\_theory.limits.preserves\_colimit\_of\_preserves\_colimit\_cocone}
\par
If F preserves one colimit cocone for the diagram K,
then it preserves any colimit cocone for K.
\section{category\_theory/limits/shapes/binary\_products.lean}\section{category\_theory/limits/shapes/default.lean}\section{category\_theory/limits/shapes/equalizers.lean}\section{category\_theory/limits/shapes/products.lean}\section{category\_theory/limits/shapes/pullbacks.lean}\section{category\_theory/limits/types.lean}\section{category\_theory/monoidal/category.lean}\section{category\_theory/monoidal/category\_aux.lean}\section{category\_theory/monoidal/functor.lean}\section{category\_theory/monoidal/types.lean}\section{category\_theory/natural\_isomorphism.lean}\paragraph{category\_theory.iso.app}
\par
The application of a natural isomorphism to an object. We put this definition in a different namespace, so that we can use α.app
\section{category\_theory/natural\_transformation.lean}\paragraph{category\_theory.nat\_trans}
\par
\colorbox[RGB]{253,246,227}{{{{\color[RGB]{101, 123, 131} nat\_trans F G }}}} represents a natural transformation between functors 
\colorbox[RGB]{253,246,227}{{{{\color[RGB]{101, 123, 131} F }}}} and 
\colorbox[RGB]{253,246,227}{{{{\color[RGB]{101, 123, 131} G }}}}.
\par
The field 
\colorbox[RGB]{253,246,227}{{{{\color[RGB]{101, 123, 131} app }}}} provides the components of the natural transformation.
\par
Naturality is expressed by 
\colorbox[RGB]{253,246,227}{{{{\color[RGB]{101, 123, 131} α.naturality\_lemma }}}}.
\paragraph{category\_theory.nat\_trans.id}
\par
\colorbox[RGB]{253,246,227}{{{{\color[RGB]{101, 123, 131} nat\_trans.id F }}}} is the identity natural transformation on a functor 
\colorbox[RGB]{253,246,227}{{{{\color[RGB]{101, 123, 131} F }}}}.
\paragraph{category\_theory.nat\_trans.vcomp}
\par
\colorbox[RGB]{253,246,227}{{{{\color[RGB]{101, 123, 131} vcomp α β }}}} is the vertical compositions of natural transformations.
\section{category\_theory/opposites.lean}\paragraph{category\_theory.has\_hom.opposite}
\par
The hom types of the opposite of a category (or graph).
\par
As with the objects, we'll make this irreducible below.
Use 
\colorbox[RGB]{253,246,227}{{{{\color[RGB]{101, 123, 131} f.op }}}} and 
\colorbox[RGB]{253,246,227}{{{{\color[RGB]{101, 123, 131} f.unop }}}} to convert between morphisms of C
and morphisms of Cᵒᵖ.
\paragraph{category\_theory.functor.hom}
\par
\colorbox[RGB]{253,246,227}{{{{\color[RGB]{101, 123, 131} functor.hom }}}} is the hom-pairing, sending (X,Y) to X → Y, contravariant in X and covariant in Y.
\section{category\_theory/pempty.lean}\section{category\_theory/products/associator.lean}\section{category\_theory/products/bifunctor.lean}\section{category\_theory/products/default.lean}\paragraph{category\_theory.prod}
\par
\colorbox[RGB]{253,246,227}{{{{\color[RGB]{101, 123, 131} prod C D }}}} gives the cartesian product of two categories.
\paragraph{category\_theory.uniform\_prod}
\par
\colorbox[RGB]{253,246,227}{{{{\color[RGB]{101, 123, 131} prod.category.uniform C D }}}} is an additional instance specialised so both factors have the same universe levels. This helps typeclass resolution.
\paragraph{category\_theory.prod.inl}
\par
\colorbox[RGB]{253,246,227}{{{{\color[RGB]{101, 123, 131} inl C Z }}}} is the functor 
\colorbox[RGB]{253,246,227}{{{{\color[RGB]{101, 123, 131} X ↦ (X, Z) }}}}.
\paragraph{category\_theory.prod.inr}
\par
\colorbox[RGB]{253,246,227}{{{{\color[RGB]{101, 123, 131} inr D Z }}}} is the functor 
\colorbox[RGB]{253,246,227}{{{{\color[RGB]{101, 123, 131} X ↦ (Z, X) }}}}.
\paragraph{category\_theory.prod.fst}
\par
\colorbox[RGB]{253,246,227}{{{{\color[RGB]{101, 123, 131} fst }}}} is the functor 
\colorbox[RGB]{253,246,227}{{{{\color[RGB]{101, 123, 131} (X, Y) ↦ X }}}}.
\paragraph{category\_theory.prod.snd}
\par
\colorbox[RGB]{253,246,227}{{{{\color[RGB]{101, 123, 131} snd }}}} is the functor 
\colorbox[RGB]{253,246,227}{{{{\color[RGB]{101, 123, 131} (X, Y) ↦ Y }}}}.
\paragraph{category\_theory.functor.prod}
\par
The cartesian product of two functors.
\paragraph{category\_theory.nat\_trans.prod}
\par
The cartesian product of two natural transformations.
\section{category\_theory/punit.lean}\paragraph{category\_theory.functor.of}
\par
The constant functor. For 
\colorbox[RGB]{253,246,227}{{{{\color[RGB]{101, 123, 131} X : C }}}}, 
\colorbox[RGB]{253,246,227}{{{{\color[RGB]{101, 123, 131} of.obj X }}}} is the functor 
\colorbox[RGB]{253,246,227}{{{{\color[RGB]{101, 123, 131} punit ⥤ C }}}}that maps 
\colorbox[RGB]{253,246,227}{{{{\color[RGB]{101, 123, 131} punit.star }}}} to 
\colorbox[RGB]{253,246,227}{{{{\color[RGB]{101, 123, 131} X }}}}.
\section{category\_theory/sparse.lean}\section{category\_theory/types.lean}\section{category\_theory/whiskering.lean}\section{category\_theory/yoneda.lean}\paragraph{category\_theory.yoneda.ext}
\par
Extensionality via Yoneda. The typical usage would be
\\
\colorbox[RGB]{253,246,227}{\parbox{4.5in}{{{{\color[RGB]{147, 161, 161} -{}- }}}{{{\color[RGB]{147, 161, 161}  Goal is `X ≅ Y` }}}{{{\color[RGB]{101, 123, 131} 
 }}}\\
{{{\color[RGB]{101, 123, 131} apply yoneda.ext,
 }}}\\
{{{\color[RGB]{147, 161, 161} -{}- }}}{{{\color[RGB]{147, 161, 161}  Goals are now functions `(Z ⟶ X) → (Z ⟶ Y)`, `(Z ⟶ Y) → (Z ⟶ X)`, and the fact that these }}}{{{\color[RGB]{101, 123, 131} 
 }}}\\
{{{\color[RGB]{101, 123, 131} functions are inverses and natural  }}}{{{\color[RGB]{133, 153, 0} in }}}{{{\color[RGB]{101, 123, 131}  `Z`.
 }}}\\

}}\section{computability/halting.lean}\paragraph{nat.partrec'}
\par
A simplified basis for 
\colorbox[RGB]{253,246,227}{{{{\color[RGB]{101, 123, 131} partrec }}}}.
\section{computability/partrec.lean}\section{computability/partrec\_code.lean}\section{computability/primrec.lean}\paragraph{nat.primrec}
\par
The primitive recursive functions 
\colorbox[RGB]{253,246,227}{{{{\color[RGB]{101, 123, 131} ℕ  }}}{{{\color[RGB]{133, 153, 0} → }}}{{{\color[RGB]{101, 123, 131}  ℕ }}}}.
\paragraph{primcodable}
\par
A 
\colorbox[RGB]{253,246,227}{{{{\color[RGB]{101, 123, 131} primcodable }}}} type is an 
\colorbox[RGB]{253,246,227}{{{{\color[RGB]{101, 123, 131} encodable }}}} type for which
the encode/decode functions are primitive recursive.
\paragraph{primrec}
\par
\colorbox[RGB]{253,246,227}{{{{\color[RGB]{101, 123, 131} primrec f }}}} means 
\colorbox[RGB]{253,246,227}{{{{\color[RGB]{101, 123, 131} f }}}} is primitive recursive (after
encoding its input and output as natural numbers).
\paragraph{primrec₂}
\par
\colorbox[RGB]{253,246,227}{{{{\color[RGB]{101, 123, 131} primrec₂ f }}}} means 
\colorbox[RGB]{253,246,227}{{{{\color[RGB]{101, 123, 131} f }}}} is a binary primitive recursive function.
This is technically unnecessary since we can always curry all
the arguments together, but there are enough natural two-arg
functions that it is convenient to express this directly.
\paragraph{primrec\_pred}
\par
\colorbox[RGB]{253,246,227}{{{{\color[RGB]{101, 123, 131} primrec\_pred p }}}} means 
\colorbox[RGB]{253,246,227}{{{{\color[RGB]{101, 123, 131} p : α  }}}{{{\color[RGB]{133, 153, 0} → }}}{{{\color[RGB]{101, 123, 131}   }}}{{{\color[RGB]{38, 139, 210} Prop }}}} is a (decidable)
primitive recursive predicate, which is to say that
\colorbox[RGB]{253,246,227}{{{{\color[RGB]{101, 123, 131} to\_bool ∘ p : α  }}}{{{\color[RGB]{133, 153, 0} → }}}{{{\color[RGB]{101, 123, 131}  bool }}}} is primitive recursive.
\paragraph{primrec\_rel}
\par
\colorbox[RGB]{253,246,227}{{{{\color[RGB]{101, 123, 131} primrec\_rel p }}}} means 
\colorbox[RGB]{253,246,227}{{{{\color[RGB]{101, 123, 131} p : α  }}}{{{\color[RGB]{133, 153, 0} → }}}{{{\color[RGB]{101, 123, 131}  β  }}}{{{\color[RGB]{133, 153, 0} → }}}{{{\color[RGB]{101, 123, 131}   }}}{{{\color[RGB]{38, 139, 210} Prop }}}} is a (decidable)
primitive recursive relation, which is to say that
\colorbox[RGB]{253,246,227}{{{{\color[RGB]{101, 123, 131} to\_bool ∘ p : α  }}}{{{\color[RGB]{133, 153, 0} → }}}{{{\color[RGB]{101, 123, 131}  β  }}}{{{\color[RGB]{133, 153, 0} → }}}{{{\color[RGB]{101, 123, 131}  bool }}}} is primitive recursive.
\paragraph{nat.primrec'}
\par
An alternative inductive definition of 
\colorbox[RGB]{253,246,227}{{{{\color[RGB]{101, 123, 131} primrec }}}} which
does not use the pairing function on ℕ, and so has to
work with n-ary functions on ℕ instead of unary functions.
We prove that this is equivalent to the regular notion
in 
\colorbox[RGB]{253,246,227}{{{{\color[RGB]{101, 123, 131} to\_prim }}}} and 
\colorbox[RGB]{253,246,227}{{{{\color[RGB]{101, 123, 131} of\_prim }}}}.
\section{computability/turing\_machine.lean}\paragraph{turing.dir}
\par
A direction for the turing machine 
\colorbox[RGB]{253,246,227}{{{{\color[RGB]{101, 123, 131} move }}}} command, either
left or right.
\paragraph{turing.TM0.stmt}
\par
A Turing machine "statement" is just a command to either move
left or right, or write a symbol on the tape.
\paragraph{turing.TM0.machine}
\par
A Post-Turing machine with symbol type 
\colorbox[RGB]{253,246,227}{{{{\color[RGB]{101, 123, 131} Γ }}}} and label type 
\colorbox[RGB]{253,246,227}{{{{\color[RGB]{101, 123, 131} Λ }}}}is a function which, given the current state 
\colorbox[RGB]{253,246,227}{{{{\color[RGB]{101, 123, 131} q : Λ }}}} and
the tape head 
\colorbox[RGB]{253,246,227}{{{{\color[RGB]{101, 123, 131} a : Γ }}}}, either halts (returns 
\colorbox[RGB]{253,246,227}{{{{\color[RGB]{101, 123, 131} none }}}}) or returns
a new state 
\colorbox[RGB]{253,246,227}{{{{\color[RGB]{101, 123, 131} q' : Λ }}}} and a 
\colorbox[RGB]{253,246,227}{{{{\color[RGB]{101, 123, 131} stmt }}}} describing what to do,
either a move left or right, or a write command.
\par
Both 
\colorbox[RGB]{253,246,227}{{{{\color[RGB]{101, 123, 131} Λ }}}} and 
\colorbox[RGB]{253,246,227}{{{{\color[RGB]{101, 123, 131} Γ }}}} are required to be inhabited; the default value
for 
\colorbox[RGB]{253,246,227}{{{{\color[RGB]{101, 123, 131} Γ }}}} is the "blank" tape value, and the default value of 
\colorbox[RGB]{253,246,227}{{{{\color[RGB]{101, 123, 131} Λ }}}} is
the initial state.
\paragraph{turing.TM0.cfg}
\par
The configuration state of a Turing machine during operation
consists of a label (machine state), and a tape, represented in
the form 
\colorbox[RGB]{253,246,227}{{{{\color[RGB]{101, 123, 131} (a, L, R) }}}} meaning the tape looks like 
\colorbox[RGB]{253,246,227}{{{{\color[RGB]{101, 123, 131} L.rev  }}}{{{\color[RGB]{181, 137, 0} + }}}{{{\color[RGB]{181, 137, 0} + }}}{{{\color[RGB]{101, 123, 131}  {[}a{]}  }}}{{{\color[RGB]{181, 137, 0} + }}}{{{\color[RGB]{181, 137, 0} + }}}{{{\color[RGB]{101, 123, 131}  R }}}}with the machine currently reading the 
\colorbox[RGB]{253,246,227}{{{{\color[RGB]{101, 123, 131} a }}}}. The lists are
automatically extended with blanks as the machine moves around.
\paragraph{turing.TM0.step}
\par
Execution semantics of the Turing machine.
\paragraph{turing.TM0.reaches}
\par
The statement 
\colorbox[RGB]{253,246,227}{{{{\color[RGB]{101, 123, 131} reaches M s₁ s₂ }}}} means that 
\colorbox[RGB]{253,246,227}{{{{\color[RGB]{101, 123, 131} s₂ }}}} is obtained
starting from 
\colorbox[RGB]{253,246,227}{{{{\color[RGB]{101, 123, 131} s₁ }}}} after a finite number of steps from 
\colorbox[RGB]{253,246,227}{{{{\color[RGB]{101, 123, 131} s₂ }}}}.
\paragraph{turing.TM0.init}
\par
The initial configuration.
\paragraph{turing.TM0.eval}
\par
Evaluate a Turing machine on initial input to a final state,
if it terminates.
\paragraph{turing.TM0.supports}
\par
The raw definition of a Turing machine does not require that
\colorbox[RGB]{253,246,227}{{{{\color[RGB]{101, 123, 131} Γ }}}} and 
\colorbox[RGB]{253,246,227}{{{{\color[RGB]{101, 123, 131} Λ }}}} are finite, and in practice we will be interested
in the infinite 
\colorbox[RGB]{253,246,227}{{{{\color[RGB]{101, 123, 131} Λ }}}} case. We recover instead a notion of
"effectively finite" Turing machines, which only make use of a
finite subset of their states. We say that a set 
\colorbox[RGB]{253,246,227}{{{{\color[RGB]{101, 123, 131} S ⊆ Λ }}}}supports a Turing machine 
\colorbox[RGB]{253,246,227}{{{{\color[RGB]{101, 123, 131} M }}}} if 
\colorbox[RGB]{253,246,227}{{{{\color[RGB]{101, 123, 131} S }}}} is closed under the
transition function and contains the initial state.
\paragraph{turing.TM1.stmt}
\par
The TM1 model is a simplification and extension of TM0
(Post-Turing model) in the direction of Wang B-machines. The machine's
internal state is extended with a (finite) store 
\colorbox[RGB]{253,246,227}{{{{\color[RGB]{101, 123, 131} σ }}}} of variables
that may be accessed and updated at any time.
A machine is given by a 
\colorbox[RGB]{253,246,227}{{{{\color[RGB]{101, 123, 131} Λ }}}} indexed set of procedures or functions.
Each function has a body which is a 
\colorbox[RGB]{253,246,227}{{{{\color[RGB]{101, 123, 131} stmt }}}}, which can either be a
\colorbox[RGB]{253,246,227}{{{{\color[RGB]{101, 123, 131} move }}}} or 
\colorbox[RGB]{253,246,227}{{{{\color[RGB]{101, 123, 131} write }}}} command, a 
\colorbox[RGB]{253,246,227}{{{{\color[RGB]{101, 123, 131} branch }}}} (if statement based on the
current tape value), a 
\colorbox[RGB]{253,246,227}{{{{\color[RGB]{101, 123, 131} load }}}} (set the variable value),
a 
\colorbox[RGB]{253,246,227}{{{{\color[RGB]{101, 123, 131} goto }}}} (call another function), or 
\colorbox[RGB]{253,246,227}{{{{\color[RGB]{101, 123, 131} halt }}}}. Note that here
most statements do not have labels; 
\colorbox[RGB]{253,246,227}{{{{\color[RGB]{101, 123, 131} goto }}}} commands can only
go to a new function. All commands have access to the variable value
and current tape value.
\paragraph{turing.TM1.cfg}
\par
The configuration of a TM1 machine is given by the currently
evaluating statement, the variable store value, and the tape.
\paragraph{turing.TM1.step\_aux}
\par
The semantics of TM1 evaluation.
\paragraph{turing.TM1.supports}
\par
A set 
\colorbox[RGB]{253,246,227}{{{{\color[RGB]{101, 123, 131} S }}}} of labels supports machine 
\colorbox[RGB]{253,246,227}{{{{\color[RGB]{101, 123, 131} M }}}} if all the 
\colorbox[RGB]{253,246,227}{{{{\color[RGB]{101, 123, 131} goto }}}}statements in the functions in 
\colorbox[RGB]{253,246,227}{{{{\color[RGB]{101, 123, 131} S }}}} refer only to other functions
in 
\colorbox[RGB]{253,246,227}{{{{\color[RGB]{101, 123, 131} S }}}}.
\paragraph{turing.TM2.stmt}
\par
The TM2 model removes the tape entirely from the TM1 model,
replacing it with an arbitrary (finite) collection of stacks.
The operation 
\colorbox[RGB]{253,246,227}{{{{\color[RGB]{101, 123, 131} push }}}} puts an element on one of the stacks,
and 
\colorbox[RGB]{253,246,227}{{{{\color[RGB]{101, 123, 131} pop }}}} removes an element from a stack (and modifying the
internal state based on the result). 
\colorbox[RGB]{253,246,227}{{{{\color[RGB]{101, 123, 131} peek }}}} modifies the
internal state but does not remove an element.
\section{field\_theory/finite.lean}\section{field\_theory/finite\_card.lean}\section{field\_theory/mv\_polynomial.lean}\section{field\_theory/perfect\_closure.lean}\paragraph{perfect\_field}
\par
A perfect field is a field of characteristic p that has p-th root.
\paragraph{perfect\_closure}
\par
The perfect closure is the smallest extension that makes frobenius surjective.
\section{field\_theory/splitting\_field.lean}\paragraph{polynomial.splits}
\par
a polynomial 
\colorbox[RGB]{253,246,227}{{{{\color[RGB]{101, 123, 131} splits }}}} iff it is zero or all of its irreducible factors have 
\colorbox[RGB]{253,246,227}{{{{\color[RGB]{101, 123, 131} degree }}}} 1
\section{field\_theory/subfield.lean}\section{group\_theory/abelianization.lean}\section{group\_theory/category.lean}\paragraph{Group}
\par
The category of groups and group morphisms.
\paragraph{AddCommGroup}
\par
The category of additive commutative groups and group morphisms.
\paragraph{AddCommGroup.forget\_to\_Group}
\par
The forgetful functor from additive commutative groups to groups.
\section{group\_theory/coset.lean}\paragraph{quotient\_group.quotient}
\par
\colorbox[RGB]{253,246,227}{{{{\color[RGB]{101, 123, 131} quotient s }}}} is the quotient type representing the left cosets of 
\colorbox[RGB]{253,246,227}{{{{\color[RGB]{101, 123, 131} s }}}}.
If 
\colorbox[RGB]{253,246,227}{{{{\color[RGB]{101, 123, 131} s }}}} is a normal subgroup, 
\colorbox[RGB]{253,246,227}{{{{\color[RGB]{101, 123, 131} quotient s }}}} is a group
\section{group\_theory/eckmann\_hilton.lean}\section{group\_theory/free\_abelian\_group.lean}\section{group\_theory/free\_group.lean}\paragraph{free\_group.red.step}
\par
Reduction step: 
\colorbox[RGB]{253,246,227}{{{{\color[RGB]{101, 123, 131} w  }}}{{{\color[RGB]{181, 137, 0} * }}}{{{\color[RGB]{101, 123, 131}  x  }}}{{{\color[RGB]{181, 137, 0} * }}}{{{\color[RGB]{101, 123, 131}  x }}}{{{\color[RGB]{181, 137, 0} ⁻¹ }}}{{{\color[RGB]{101, 123, 131}   }}}{{{\color[RGB]{181, 137, 0} * }}}{{{\color[RGB]{101, 123, 131}  v \textasciitilde{} }}}{{{\color[RGB]{181, 137, 0} > }}}{{{\color[RGB]{101, 123, 131}  w  }}}{{{\color[RGB]{181, 137, 0} * }}}{{{\color[RGB]{101, 123, 131}  v }}}}\paragraph{free\_group.red}
\par
Reflexive-transitive closure of red.step
\paragraph{free\_group.red.step.length}
\par
Predicate asserting that word 
\colorbox[RGB]{253,246,227}{{{{\color[RGB]{101, 123, 131} w₁ }}}} can be reduced to 
\colorbox[RGB]{253,246,227}{{{{\color[RGB]{101, 123, 131} w₂ }}}} in one step, i.e. there are words
\colorbox[RGB]{253,246,227}{{{{\color[RGB]{101, 123, 131} w₃ w₄ }}}} and letter 
\colorbox[RGB]{253,246,227}{{{{\color[RGB]{101, 123, 131} x }}}} such that 
\colorbox[RGB]{253,246,227}{{{{\color[RGB]{101, 123, 131} w₁  }}}{{{\color[RGB]{181, 137, 0} = }}}{{{\color[RGB]{101, 123, 131}  w₃xx }}}{{{\color[RGB]{181, 137, 0} ⁻¹ }}}{{{\color[RGB]{101, 123, 131} w₄ }}}} and 
\colorbox[RGB]{253,246,227}{{{{\color[RGB]{101, 123, 131} w₂  }}}{{{\color[RGB]{181, 137, 0} = }}}{{{\color[RGB]{101, 123, 131}  w₃w₄ }}}}\paragraph{free\_group.red.church\_rosser}
\par
Church-Rosser theorem for word reduction: If 
\colorbox[RGB]{253,246,227}{{{{\color[RGB]{101, 123, 131} w1 w2 w3 }}}} are words such that 
\colorbox[RGB]{253,246,227}{{{{\color[RGB]{101, 123, 131} w1 }}}} reduces to 
\colorbox[RGB]{253,246,227}{{{{\color[RGB]{101, 123, 131} w2 }}}}and 
\colorbox[RGB]{253,246,227}{{{{\color[RGB]{101, 123, 131} w3 }}}} respectively, then there is a word 
\colorbox[RGB]{253,246,227}{{{{\color[RGB]{101, 123, 131} w4 }}}} such that 
\colorbox[RGB]{253,246,227}{{{{\color[RGB]{101, 123, 131} w2 }}}} and 
\colorbox[RGB]{253,246,227}{{{{\color[RGB]{101, 123, 131} w3 }}}} reduce to 
\colorbox[RGB]{253,246,227}{{{{\color[RGB]{101, 123, 131} w4 }}}} respectively.
\paragraph{free\_group.red.nil\_iff}
\par
The empty word 
\colorbox[RGB]{253,246,227}{{{{\color[RGB]{101, 123, 131} {[}{]} }}}} only reduces to itself.
\paragraph{free\_group.red.singleton\_iff}
\par
A letter only reduces to itself.
\paragraph{free\_group.red.cons\_nil\_iff\_singleton}
\par
If 
\colorbox[RGB]{253,246,227}{{{{\color[RGB]{101, 123, 131} x }}}} is a letter and 
\colorbox[RGB]{253,246,227}{{{{\color[RGB]{101, 123, 131} w }}}} is a word such that 
\colorbox[RGB]{253,246,227}{{{{\color[RGB]{101, 123, 131} xw }}}} reduces to the empty word, then 
\colorbox[RGB]{253,246,227}{{{{\color[RGB]{101, 123, 131} w }}}} reduces
to 
\colorbox[RGB]{253,246,227}{{{{\color[RGB]{101, 123, 131} x }}}{{{\color[RGB]{181, 137, 0} ⁻¹ }}}}\paragraph{free\_group.red.inv\_of\_red\_of\_ne}
\par
If 
\colorbox[RGB]{253,246,227}{{{{\color[RGB]{101, 123, 131} x }}}} and 
\colorbox[RGB]{253,246,227}{{{{\color[RGB]{101, 123, 131} y }}}} are distinct letters and 
\colorbox[RGB]{253,246,227}{{{{\color[RGB]{101, 123, 131} w₁ w₂ }}}} are words such that 
\colorbox[RGB]{253,246,227}{{{{\color[RGB]{101, 123, 131} xw₁ }}}} reduces to 
\colorbox[RGB]{253,246,227}{{{{\color[RGB]{101, 123, 131} yw₂ }}}}, then
\colorbox[RGB]{253,246,227}{{{{\color[RGB]{101, 123, 131} w₁ }}}} reduces to 
\colorbox[RGB]{253,246,227}{{{{\color[RGB]{101, 123, 131} x }}}{{{\color[RGB]{181, 137, 0} ⁻¹ }}}{{{\color[RGB]{101, 123, 131} yw₂ }}}}.
\paragraph{free\_group.red.sublist}
\par
If 
\colorbox[RGB]{253,246,227}{{{{\color[RGB]{101, 123, 131} w₁ w₂ }}}} are words such that 
\colorbox[RGB]{253,246,227}{{{{\color[RGB]{101, 123, 131} w₁ }}}} reduces to 
\colorbox[RGB]{253,246,227}{{{{\color[RGB]{101, 123, 131} w₂ }}}}, then 
\colorbox[RGB]{253,246,227}{{{{\color[RGB]{101, 123, 131} w₂ }}}} is a sublist of 
\colorbox[RGB]{253,246,227}{{{{\color[RGB]{101, 123, 131} w₁ }}}}.
\paragraph{free\_group}
\par
The free group over a type, i.e. the words formed by the elements of the type and their formal
inverses, quotient by one step reduction.
\paragraph{free\_group.of}
\par
\colorbox[RGB]{253,246,227}{{{{\color[RGB]{101, 123, 131} of x }}}} is the canonical injection from the type to the free group over that type by sending each
element to the equivalence class of the letter that is the element.
\paragraph{free\_group.of.inj}
\par
The canonical injection from the type to the free group is an injection.
\paragraph{free\_group.to\_group}
\par
If 
\colorbox[RGB]{253,246,227}{{{{\color[RGB]{101, 123, 131} β }}}} is a group, then any function from 
\colorbox[RGB]{253,246,227}{{{{\color[RGB]{101, 123, 131} α }}}} to 
\colorbox[RGB]{253,246,227}{{{{\color[RGB]{101, 123, 131} β }}}}extends uniquely to a group homomorphism from
the free group over 
\colorbox[RGB]{253,246,227}{{{{\color[RGB]{101, 123, 131} α }}}} to 
\colorbox[RGB]{253,246,227}{{{{\color[RGB]{101, 123, 131} β }}}}\paragraph{free\_group.map}
\par
Any function from 
\colorbox[RGB]{253,246,227}{{{{\color[RGB]{101, 123, 131} α }}}} to 
\colorbox[RGB]{253,246,227}{{{{\color[RGB]{101, 123, 131} β }}}} extends uniquely
to a group homomorphism from the free group
ver 
\colorbox[RGB]{253,246,227}{{{{\color[RGB]{101, 123, 131} α }}}} to the free group over 
\colorbox[RGB]{253,246,227}{{{{\color[RGB]{101, 123, 131} β }}}}.
\paragraph{free\_group.free\_group\_congr}
\par
Equivalent types give rise to equivalent free groups.
\paragraph{free\_group.prod}
\par
If 
\colorbox[RGB]{253,246,227}{{{{\color[RGB]{101, 123, 131} α }}}} is a group, then any function from 
\colorbox[RGB]{253,246,227}{{{{\color[RGB]{101, 123, 131} α }}}} to 
\colorbox[RGB]{253,246,227}{{{{\color[RGB]{101, 123, 131} α }}}}extends uniquely to a homomorphism from the
free group over 
\colorbox[RGB]{253,246,227}{{{{\color[RGB]{101, 123, 131} α }}}} to 
\colorbox[RGB]{253,246,227}{{{{\color[RGB]{101, 123, 131} α }}}}. This is the multiplicative
version of 
\colorbox[RGB]{253,246,227}{{{{\color[RGB]{101, 123, 131} sum }}}}.
\paragraph{free\_group.sum}
\par
If 
\colorbox[RGB]{253,246,227}{{{{\color[RGB]{101, 123, 131} α }}}} is a group, then any function from 
\colorbox[RGB]{253,246,227}{{{{\color[RGB]{101, 123, 131} α }}}} to 
\colorbox[RGB]{253,246,227}{{{{\color[RGB]{101, 123, 131} α }}}}extends uniquely to a homomorphism from the
free group over 
\colorbox[RGB]{253,246,227}{{{{\color[RGB]{101, 123, 131} α }}}} to 
\colorbox[RGB]{253,246,227}{{{{\color[RGB]{101, 123, 131} α }}}}. This is the additive
version of 
\colorbox[RGB]{253,246,227}{{{{\color[RGB]{101, 123, 131} prod }}}}.
\paragraph{free\_group.reduce}
\par
The maximal reduction of a word. It is computable
iff 
\colorbox[RGB]{253,246,227}{{{{\color[RGB]{101, 123, 131} α }}}} has decidable equality.
\paragraph{free\_group.reduce.red}
\par
The first theorem that characterises the function
\colorbox[RGB]{253,246,227}{{{{\color[RGB]{101, 123, 131} reduce }}}}: a word reduces to its maximal reduction.
\paragraph{free\_group.reduce.min}
\par
The second theorem that characterises the
function 
\colorbox[RGB]{253,246,227}{{{{\color[RGB]{101, 123, 131} reduce }}}}: the maximal reduction of a word
only reduces to itself.
\paragraph{free\_group.reduce.idem}
\par
\colorbox[RGB]{253,246,227}{{{{\color[RGB]{101, 123, 131} reduce }}}} is idempotent, i.e. the maximal reduction
of the maximal reduction of a word is the maximal
reduction of the word.
\paragraph{free\_group.reduce.eq\_of\_red}
\par
If a word reduces to another word, then they have
a common maximal reduction.
\paragraph{free\_group.reduce.sound}
\par
If two words correspond to the same element in
the free group, then they have a common maximal
reduction. This is the proof that the function that
sends an element of the free group to its maximal
reduction is well-defined.
\paragraph{free\_group.reduce.exact}
\par
If two words have a common maximal reduction,
then they correspond to the same element in the free group.
\paragraph{free\_group.reduce.self}
\par
A word and its maximal reduction correspond to
the same element of the free group.
\paragraph{free\_group.reduce.rev}
\par
If words 
\colorbox[RGB]{253,246,227}{{{{\color[RGB]{101, 123, 131} w₁ w₂ }}}} are such that 
\colorbox[RGB]{253,246,227}{{{{\color[RGB]{101, 123, 131} w₁ }}}} reduces to 
\colorbox[RGB]{253,246,227}{{{{\color[RGB]{101, 123, 131} w₂ }}}},
then 
\colorbox[RGB]{253,246,227}{{{{\color[RGB]{101, 123, 131} w₂ }}}} reduces to the maximal reduction of 
\colorbox[RGB]{253,246,227}{{{{\color[RGB]{101, 123, 131} w₁ }}}}.
\paragraph{free\_group.to\_word}
\par
The function that sends an element of the free
group to its maximal reduction.
\paragraph{free\_group.reduce.church\_rosser}
\par
Constructive Church-Rosser theorem (compare 
\colorbox[RGB]{253,246,227}{{{{\color[RGB]{101, 123, 131} church\_rosser }}}}).
\paragraph{free\_group.red.enum}
\par
A list containing every word that 
\colorbox[RGB]{253,246,227}{{{{\color[RGB]{101, 123, 131} w₁ }}}} reduces to.
\section{group\_theory/group\_action.lean}\paragraph{has\_scalar}
\par
Typeclass for types with a scalar multiplication operation, denoted 
\colorbox[RGB]{253,246,227}{{{{\color[RGB]{101, 123, 131} • }}}} (
\colorbox[RGB]{253,246,227}{{{{\color[RGB]{101, 123, 131} \textbackslash{}bu }}}})
\paragraph{mul\_action}
\par
Typeclass for multiplictive actions by monoids. This generalizes group actions.
\paragraph{distrib\_mul\_action}
\par
Typeclass for multiplicative actions on additive structures. This generalizes group modules.
\section{group\_theory/order\_of\_element.lean}\paragraph{order\_of}
\par
\colorbox[RGB]{253,246,227}{{{{\color[RGB]{101, 123, 131} order\_of a }}}} is the order of the element 
\colorbox[RGB]{253,246,227}{{{{\color[RGB]{101, 123, 131} a }}}}, i.e. the 
\colorbox[RGB]{253,246,227}{{{{\color[RGB]{101, 123, 131} n  }}}{{{\color[RGB]{181, 137, 0} ≥ }}}{{{\color[RGB]{101, 123, 131}   }}}{{{\color[RGB]{108, 113, 196} 1 }}}}, s.t. 
\colorbox[RGB]{253,246,227}{{{{\color[RGB]{101, 123, 131} a \textasciicircum{} n  }}}{{{\color[RGB]{181, 137, 0} = }}}{{{\color[RGB]{101, 123, 131}   }}}{{{\color[RGB]{108, 113, 196} 1 }}}}\section{group\_theory/perm/cycles.lean}\section{group\_theory/perm/sign.lean}\paragraph{equiv.perm.swap\_factors}
\par
\colorbox[RGB]{253,246,227}{{{{\color[RGB]{101, 123, 131} swap\_factors }}}} represents a permutation as a product of a list of transpositions.
The representation is non unique and depends on the linear order structure.
For types without linear order 
\colorbox[RGB]{253,246,227}{{{{\color[RGB]{101, 123, 131} trunc\_swap\_factors }}}} can be used
\paragraph{equiv.perm.fin\_pairs\_lt}
\par
set of all pairs (⟨a, b⟩ : Σ a : fin n, fin n) such that b 
<
 a
\paragraph{equiv.perm.sign}
\par
\colorbox[RGB]{253,246,227}{{{{\color[RGB]{101, 123, 131} sign }}}} of a permutation returns the signature or parity of a permutation, 
\colorbox[RGB]{253,246,227}{{{{\color[RGB]{108, 113, 196} 1 }}}} for even
permutations, 
\colorbox[RGB]{253,246,227}{{{{\color[RGB]{181, 137, 0} - }}}{{{\color[RGB]{108, 113, 196} 1 }}}} for odd permutations. It is the unique surjective group homomorphism from
\colorbox[RGB]{253,246,227}{{{{\color[RGB]{101, 123, 131} perm α }}}} to the group with two elements.
\section{group\_theory/presented\_group.lean}\paragraph{presented\_group}
\par
Given a set of relations, rels, over a type α, presented\_group constructs the group with
generators α and relations rels as a quotient of free\_group α.
\paragraph{presented\_group.of}
\par
\colorbox[RGB]{253,246,227}{{{{\color[RGB]{101, 123, 131} of x }}}} is the canonical map from α to a presented group with generators α. The term x is
mapped to the equivalence class of the image of x in free\_group α.
\paragraph{presented\_group.to\_group}
\par
The extension of a map f : α → β that satisfies the given relations to a group homomorphism
from presented\_group rels → β.
\section{group\_theory/quotient\_group.lean}\paragraph{quotient\_group.ker\_lift}
\par
The induced map from the quotient by the kernel to the codomain.
\section{group\_theory/subgroup.lean}\paragraph{is\_subgroup}
\par
\colorbox[RGB]{253,246,227}{{{{\color[RGB]{101, 123, 131} s }}}} is a subgroup: a set containing 1 and closed under multiplication and inverse.
\paragraph{is\_add\_subgroup}
\par
\colorbox[RGB]{253,246,227}{{{{\color[RGB]{101, 123, 131} s }}}} is an additive subgroup: a set containing 0 and closed under addition and negation.
\paragraph{is\_subgroup.trivial}
\par
The trivial subgroup
\paragraph{is\_add\_subgroup.trivial}
\par
The trivial subgroup
\paragraph{group.closure}
\par
\colorbox[RGB]{253,246,227}{{{{\color[RGB]{101, 123, 131} group.closure s }}}} is the subgroup closed over 
\colorbox[RGB]{253,246,227}{{{{\color[RGB]{101, 123, 131} s }}}}, i.e. the smallest subgroup containg s.
\paragraph{add\_group.closure}
\par
\colorbox[RGB]{253,246,227}{{{{\color[RGB]{101, 123, 131} add\_group.closure s }}}} is the additive subgroup closed over 
\colorbox[RGB]{253,246,227}{{{{\color[RGB]{101, 123, 131} s }}}}, i.e. the smallest subgroup containg s.
\paragraph{group.conjugates}
\par
Given an element a, conjugates a is the set of conjugates.
\paragraph{group.conjugates\_of\_set}
\par
Given a set s, conjugates\_of\_set s is the set of all conjugates of
the elements of s.
\paragraph{group.conj\_mem\_conjugates\_of\_set}
\par
The set of conjugates of s is closed under conjugation.
\paragraph{group.normal\_closure}
\par
The normal closure of a set s is the subgroup closure of all the conjugates of
elements of s. It is the smallest normal subgroup containing s.
\paragraph{group.normal\_closure.is\_subgroup}
\par
The normal closure of a set is a subgroup.
\paragraph{group.normal\_closure.is\_normal}
\par
The normal closure of s is a normal subgroup.
\paragraph{group.normal\_closure\_subset}
\par
The normal closure of s is the smallest normal subgroup containing s.
\section{group\_theory/submonoid.lean}\paragraph{is\_submonoid}
\par
\colorbox[RGB]{253,246,227}{{{{\color[RGB]{101, 123, 131} s }}}} is a submonoid: a set containing 1 and closed under multiplication.
\paragraph{is\_add\_submonoid}
\par
\colorbox[RGB]{253,246,227}{{{{\color[RGB]{101, 123, 131} s }}}} is an additive submonoid: a set containing 0 and closed under addition.
\section{group\_theory/sylow.lean}\paragraph{sylow.exists\_prime\_order\_of\_dvd\_card}
\par
Cauchy's theorem
\section{linear\_algebra/basic.lean}\paragraph{submodule.map}
\par
The pushforward
\paragraph{submodule.comap}
\par
The pullback
\paragraph{submodule.map\_subtype.order\_iso}
\par
If N ⊆ M then submodules of N are the same as submodules of M contained in N
\paragraph{submodule.comap\_mkq.order\_iso}
\par
Correspondence Theorem
\paragraph{linear\_map.quot\_ker\_equiv\_range}
\par
First Isomorphism Law
\paragraph{linear\_map.sup\_quotient\_equiv\_quotient\_inf}
\par
Second Isomorphism Law
\paragraph{linear\_map.pi}
\par
\colorbox[RGB]{253,246,227}{{{{\color[RGB]{101, 123, 131} pi }}}} construction for linear functions. From a family of linear functions it produces a linear
function into a family of modules.
\paragraph{linear\_map.proj}
\par
Linear projection
\paragraph{linear\_map.diag}
\par
\colorbox[RGB]{253,246,227}{{{{\color[RGB]{101, 123, 131} diag i j }}}} is the identity map if 
\colorbox[RGB]{253,246,227}{{{{\color[RGB]{101, 123, 131} i  }}}{{{\color[RGB]{181, 137, 0} = }}}{{{\color[RGB]{101, 123, 131}  j }}}} otherwise it is the constant 0 map.
\paragraph{linear\_map.std\_basis}
\par
Standard basis
\paragraph{linear\_map.general\_linear\_group}
\par
The group of invertible linear maps from 
\colorbox[RGB]{253,246,227}{{{{\color[RGB]{101, 123, 131} β }}}} to itself
\section{linear\_algebra/basis.lean}\paragraph{linear\_independent}
\par
Linearly independent set of vectors
\paragraph{is\_basis}
\par
A set of vectors is a basis if it is linearly independent and all vectors are in the span α
\paragraph{is\_basis.constr}
\par
Construct a linear map given the value at the basis.
\section{linear\_algebra/default.lean}\section{linear\_algebra/determinant.lean}\section{linear\_algebra/dimension.lean}\paragraph{mk\_eq\_mk\_of\_basis}
\par
dimension theorem
\paragraph{dim\_range\_add\_dim\_ker}
\par
rank-nullity theorem
\paragraph{dim\_add\_dim\_split}
\par
This is mostly an auxiliary lemma for 
\colorbox[RGB]{253,246,227}{{{{\color[RGB]{101, 123, 131} dim\_sup\_add\_dim\_inf\_eq }}}}\paragraph{linear\_equiv.dim\_eq\_lift}
\par
Version of linear\_equiv.dim\_eq without universe constraints.
\section{linear\_algebra/direct\_sum\_module.lean}\section{linear\_algebra/dual.lean}\section{linear\_algebra/finsupp.lean}\section{linear\_algebra/linear\_combination.lean}\paragraph{lc}
\par
The type of linear coefficients, which are simply the finitely supported functions
from the module 
\colorbox[RGB]{253,246,227}{{{{\color[RGB]{101, 123, 131} β }}}} to the scalar ring 
\colorbox[RGB]{253,246,227}{{{{\color[RGB]{101, 123, 131} α }}}}.
\section{linear\_algebra/matrix.lean}\section{linear\_algebra/tensor\_product.lean}\section{logic/basic.lean}\paragraph{pempty}
\par
\colorbox[RGB]{253,246,227}{{{{\color[RGB]{101, 123, 131} pempty }}}} is the universe-polymorphic analogue of 
\colorbox[RGB]{253,246,227}{{{{\color[RGB]{101, 123, 131} empty }}}}.
\paragraph{forall\_iff\_forall\_surj}
\par
A predicate holds everywhere on the image of a surjective functions iff
it holds everywhere.
\paragraph{classical.subtype\_of\_exists}
\par
A version of classical.indefinite\_description which is definitionally equal to a pair
\section{logic/embedding.lean}\paragraph{function.embedding.some}
\par
Embedding into 
\colorbox[RGB]{253,246,227}{{{{\color[RGB]{101, 123, 131} option }}}}\paragraph{function.embedding.cod\_restrict}
\par
Restrict the codomain of an embedding.
\paragraph{set.embedding\_of\_subset}
\par
The injection map is an embedding between subsets.
\section{logic/function.lean}\paragraph{function.is\_partial\_inv}
\par
\colorbox[RGB]{253,246,227}{{{{\color[RGB]{101, 123, 131} g }}}} is a partial inverse to 
\colorbox[RGB]{253,246,227}{{{{\color[RGB]{101, 123, 131} f }}}} (an injective but not necessarily
surjective function) if 
\colorbox[RGB]{253,246,227}{{{{\color[RGB]{101, 123, 131} g y  }}}{{{\color[RGB]{181, 137, 0} = }}}{{{\color[RGB]{101, 123, 131}  some x }}}} implies 
\colorbox[RGB]{253,246,227}{{{{\color[RGB]{101, 123, 131} f x  }}}{{{\color[RGB]{181, 137, 0} = }}}{{{\color[RGB]{101, 123, 131}  y }}}}, and 
\colorbox[RGB]{253,246,227}{{{{\color[RGB]{101, 123, 131} g y  }}}{{{\color[RGB]{181, 137, 0} = }}}{{{\color[RGB]{101, 123, 131}  none }}}}implies that 
\colorbox[RGB]{253,246,227}{{{{\color[RGB]{101, 123, 131} y }}}} is not in the range of 
\colorbox[RGB]{253,246,227}{{{{\color[RGB]{101, 123, 131} f }}}}.
\paragraph{function.partial\_inv}
\par
We can use choice to construct explicitly a partial inverse for
a given injective function 
\colorbox[RGB]{253,246,227}{{{{\color[RGB]{101, 123, 131} f }}}}.
\paragraph{function.inv\_fun\_on}
\par
Construct the inverse for a function 
\colorbox[RGB]{253,246,227}{{{{\color[RGB]{101, 123, 131} f }}}} on domain 
\colorbox[RGB]{253,246,227}{{{{\color[RGB]{101, 123, 131} s }}}}.
\paragraph{function.inv\_fun}
\par
The inverse of a function (which is a left inverse if 
\colorbox[RGB]{253,246,227}{{{{\color[RGB]{101, 123, 131} f }}}} is injective
and a right inverse if 
\colorbox[RGB]{253,246,227}{{{{\color[RGB]{101, 123, 131} f }}}} is surjective).
\paragraph{function.surj\_inv}
\par
The inverse of a surjective function. (Unlike 
\colorbox[RGB]{253,246,227}{{{{\color[RGB]{101, 123, 131} inv\_fun }}}}, this does not require
\colorbox[RGB]{253,246,227}{{{{\color[RGB]{101, 123, 131} α }}}} to be inhabited.)
\section{logic/relation.lean}\paragraph{relation.refl\_trans\_gen}
\par
\colorbox[RGB]{253,246,227}{{{{\color[RGB]{101, 123, 131} refl\_trans\_gen r }}}}: reflexive transitive closure of 
\colorbox[RGB]{253,246,227}{{{{\color[RGB]{101, 123, 131} r }}}}\paragraph{relation.refl\_gen}
\par
\colorbox[RGB]{253,246,227}{{{{\color[RGB]{101, 123, 131} refl\_gen r }}}}: reflexive closure of 
\colorbox[RGB]{253,246,227}{{{{\color[RGB]{101, 123, 131} r }}}}\paragraph{relation.trans\_gen}
\par
\colorbox[RGB]{253,246,227}{{{{\color[RGB]{101, 123, 131} trans\_gen r }}}}: transitive closure of 
\colorbox[RGB]{253,246,227}{{{{\color[RGB]{101, 123, 131} r }}}}\section{logic/relator.lean}\section{logic/unique.lean}\section{measure\_theory/Meas.lean}\section{measure\_theory/borel\_space.lean}\section{measure\_theory/decomposition.lean}\section{measure\_theory/giry\_monad.lean}\paragraph{measure\_theory.measure.measurable\_space}
\par
Measurability structure on 
\colorbox[RGB]{253,246,227}{{{{\color[RGB]{101, 123, 131} measure }}}}: Measures are measurable w.r.t. all projections
\paragraph{measure\_theory.measure.join}
\par
Monadic join on 
\colorbox[RGB]{253,246,227}{{{{\color[RGB]{101, 123, 131} measure }}}} in the category of measurable spaces and measurable
functions.
\paragraph{measure\_theory.measure.bind}
\par
Monadic bind on 
\colorbox[RGB]{253,246,227}{{{{\color[RGB]{101, 123, 131} measure }}}}, only works in the category of measurable spaces and measurable
functions. When the function 
\colorbox[RGB]{253,246,227}{{{{\color[RGB]{101, 123, 131} f }}}} is not measurable the result is not well defined.
\section{measure\_theory/integration.lean}\paragraph{measure\_theory.lintegral}
\par
The lower Lebesgue integral
\paragraph{measure\_theory.lintegral\_supr}
\par
Monotone convergence theorem -{}- somtimes called Beppo-Levi convergence.
\par
See 
\colorbox[RGB]{253,246,227}{{{{\color[RGB]{101, 123, 131} lintegral\_supr\_directed }}}} for a more general form.
\paragraph{measure\_theory.lintegral\_supr\_directed}
\par
Monotone convergence for a suprema over a directed family and indexed by an encodable type
\section{measure\_theory/lebesgue\_measure.lean}\paragraph{measure\_theory.lebesgue\_length}
\par
Length of an interval. This is the largest monotonic function which correctly
measures all intervals.
\paragraph{measure\_theory.lebesgue\_outer}
\par
The Lebesgue outer measure, as an outer measure of ℝ.
\paragraph{measure\_theory.measure\_theory.measure\_space}
\par
Lebesgue measure on the Borel sets
\par
The outer Lebesgue measure is the completion of this measure. (TODO: proof this)
\section{measure\_theory/measurable\_space.lean}\paragraph{is\_measurable}
\par
\colorbox[RGB]{253,246,227}{{{{\color[RGB]{101, 123, 131} is\_measurable s }}}} means that 
\colorbox[RGB]{253,246,227}{{{{\color[RGB]{101, 123, 131} s }}}} is measurable (in the ambient measure space on 
\colorbox[RGB]{253,246,227}{{{{\color[RGB]{101, 123, 131} α }}}})
\paragraph{measurable\_space.generate\_measurable}
\par
The smallest σ-algebra containing a collection 
\colorbox[RGB]{253,246,227}{{{{\color[RGB]{101, 123, 131} s }}}} of basic sets
\paragraph{measurable\_space.generate\_from}
\par
Construct the smallest measure space containing a collection of basic sets
\paragraph{measurable\_space.map}
\par
The forward image of a measure space under a function. 
\colorbox[RGB]{253,246,227}{{{{\color[RGB]{101, 123, 131} map f m }}}} contains the sets 
\colorbox[RGB]{253,246,227}{{{{\color[RGB]{101, 123, 131} s : set β }}}}whose preimage under 
\colorbox[RGB]{253,246,227}{{{{\color[RGB]{101, 123, 131} f }}}} is measurable.
\paragraph{measurable\_space.comap}
\par
The reverse image of a measure space under a function. 
\colorbox[RGB]{253,246,227}{{{{\color[RGB]{101, 123, 131} comap f m }}}} contains the sets 
\colorbox[RGB]{253,246,227}{{{{\color[RGB]{101, 123, 131} s : set α }}}}such that 
\colorbox[RGB]{253,246,227}{{{{\color[RGB]{101, 123, 131} s }}}} is the 
\colorbox[RGB]{253,246,227}{{{{\color[RGB]{101, 123, 131} f }}}}-preimage of a measurable set in 
\colorbox[RGB]{253,246,227}{{{{\color[RGB]{101, 123, 131} β }}}}.
\paragraph{measurable}
\par
A function 
\colorbox[RGB]{253,246,227}{{{{\color[RGB]{101, 123, 131} f }}}} between measurable spaces is measurable if the preimage of every
measurable set is measurable.
\paragraph{measurable\_equiv}
\par
Equivalences between measurable spaces. Main application is the simplification of measurability
statements along measurable equivalences.
\paragraph{measurable\_space.dynkin\_system}
\par
Dynkin systems
The main purpose of Dynkin systems is to provide a powerful induction rule for σ-algebras generated
by intersection stable set systems.
\paragraph{measurable\_space.dynkin\_system.generate\_has}
\par
The least Dynkin system containing a collection of basic sets.
\section{measure\_theory/measure\_space.lean}\paragraph{measure\_theory.measure'}
\par
Measure projection which is ∞ for non-measurable sets.
\par
\colorbox[RGB]{253,246,227}{{{{\color[RGB]{101, 123, 131} measure' }}}} is mainly used to derive the outer measure, for the main 
\colorbox[RGB]{253,246,227}{{{{\color[RGB]{101, 123, 131} measure }}}} projection.
\paragraph{measure\_theory.outer\_measure'}
\par
outer measure of a measure
\paragraph{measure\_theory.measure.has\_coe\_to\_fun}
\par
Measure projections for a measure space.
\par
For measurable sets this returns the measure assigned by the 
\colorbox[RGB]{253,246,227}{{{{\color[RGB]{101, 123, 131} measure\_of }}}} field in 
\colorbox[RGB]{253,246,227}{{{{\color[RGB]{101, 123, 131} measure }}}}.
But we can extend this to 
\emph{all
} sets, but using the outer measure. This gives us monotonicity and
subadditivity for all sets.
\paragraph{measure\_theory.measure.dirac}
\par
The dirac measure.
\paragraph{measure\_theory.measure.sum}
\par
Sum of an indexed family of measures.
\paragraph{measure\_theory.measure.count}
\par
Counting measure on any measurable space.
\paragraph{measure\_theory.measure.a\_e}
\par
The "almost everywhere" filter of co-null sets.
\paragraph{measure\_theory.measure\_space}
\par
A measure space is a measurable space equipped with a
measure, referred to as 
\colorbox[RGB]{253,246,227}{{{{\color[RGB]{101, 123, 131} volume }}}}.
\paragraph{measure\_theory.all\_ae}
\par
\colorbox[RGB]{253,246,227}{{{{\color[RGB]{101, 123, 131} ∀ₘ a:α, p a }}}} states that the property 
\colorbox[RGB]{253,246,227}{{{{\color[RGB]{101, 123, 131} p }}}} is almost everywhere true in the measure space
associated with 
\colorbox[RGB]{253,246,227}{{{{\color[RGB]{101, 123, 131} α }}}}. This means that the measure of the complementary of 
\colorbox[RGB]{253,246,227}{{{{\color[RGB]{101, 123, 131} p }}}} is 
\colorbox[RGB]{253,246,227}{{{{\color[RGB]{108, 113, 196} 0 }}}}.
\par
In a probability measure, the measure of 
\colorbox[RGB]{253,246,227}{{{{\color[RGB]{101, 123, 131} p }}}} is 
\colorbox[RGB]{253,246,227}{{{{\color[RGB]{108, 113, 196} 1 }}}}, when 
\colorbox[RGB]{253,246,227}{{{{\color[RGB]{101, 123, 131} p }}}} is measurable.
\section{measure\_theory/outer\_measure.lean}\paragraph{measure\_theory.outer\_measure.dirac}
\par
The dirac outer measure.
\paragraph{measure\_theory.outer\_measure.of\_function}
\par
Given any function 
\colorbox[RGB]{253,246,227}{{{{\color[RGB]{101, 123, 131} m }}}} assigning measures to sets satisying 
\colorbox[RGB]{253,246,227}{{{{\color[RGB]{101, 123, 131} m ∅  }}}{{{\color[RGB]{181, 137, 0} = }}}{{{\color[RGB]{101, 123, 131}   }}}{{{\color[RGB]{108, 113, 196} 0 }}}}, there is
a unique maximal outer measure 
\colorbox[RGB]{253,246,227}{{{{\color[RGB]{101, 123, 131} μ }}}} satisfying 
\colorbox[RGB]{253,246,227}{{{{\color[RGB]{101, 123, 131} μ s  }}}{{{\color[RGB]{181, 137, 0} ≤ }}}{{{\color[RGB]{101, 123, 131}  m s }}}} for all 
\colorbox[RGB]{253,246,227}{{{{\color[RGB]{101, 123, 131} s : set α }}}}.
\paragraph{measure\_theory.outer\_measure.caratheodory}
\par
Given an outer measure 
\colorbox[RGB]{253,246,227}{{{{\color[RGB]{101, 123, 131} μ }}}}, the Caratheodory measurable space is
defined such that 
\colorbox[RGB]{253,246,227}{{{{\color[RGB]{101, 123, 131} s }}}} is measurable if 
\colorbox[RGB]{253,246,227}{{{{\color[RGB]{101, 123, 131} ∀t, μ t  }}}{{{\color[RGB]{181, 137, 0} = }}}{{{\color[RGB]{101, 123, 131}  μ (t ∩ s)  }}}{{{\color[RGB]{181, 137, 0} + }}}{{{\color[RGB]{101, 123, 131}  μ (t \textbackslash{} s) }}}}.
\section{measure\_theory/probability\_mass\_function.lean}\paragraph{pmf}
\par
Probability mass functions, i.e. discrete probability measures
\section{meta/coinductive\_predicates.lean}\paragraph{tactic.coinduction}
\par
Prepares coinduction proofs. This tactic constructs the coinduction invariant from
the quantifiers in the current goal.
\par
Current version: do not support mutual inductive rules (i.e. only a since C
\section{meta/expr.lean}\paragraph{expr.is\_num\_eq}
\par
is\_num\_eq n1 n2 returns true if n1 and n2 are both numerals with the same numeral structure,
ignoring differences in type and type class arguments.
\section{meta/rb\_map.lean}\section{number\_theory/dioph.lean}\paragraph{fin2}
\par
An alternate definition of 
\colorbox[RGB]{253,246,227}{{{{\color[RGB]{101, 123, 131} fin n }}}} defined as an inductive type
instead of a subtype of 
\colorbox[RGB]{253,246,227}{{{{\color[RGB]{101, 123, 131} nat }}}}. This is useful for its induction
principle and different definitional equalities.
\paragraph{fin2.to\_nat}
\par
convert a 
\colorbox[RGB]{253,246,227}{{{{\color[RGB]{101, 123, 131} fin2 }}}} into a 
\colorbox[RGB]{253,246,227}{{{{\color[RGB]{101, 123, 131} nat }}}}\paragraph{fin2.opt\_of\_nat}
\par
convert a 
\colorbox[RGB]{253,246,227}{{{{\color[RGB]{101, 123, 131} nat }}}} into a 
\colorbox[RGB]{253,246,227}{{{{\color[RGB]{101, 123, 131} fin2 }}}} if it is in range
\paragraph{fin2.add}
\par
\colorbox[RGB]{253,246,227}{{{{\color[RGB]{101, 123, 131} i  }}}{{{\color[RGB]{181, 137, 0} + }}}{{{\color[RGB]{101, 123, 131}  k : fin2 (n  }}}{{{\color[RGB]{181, 137, 0} + }}}{{{\color[RGB]{101, 123, 131}  k) }}}} when 
\colorbox[RGB]{253,246,227}{{{{\color[RGB]{101, 123, 131} i : fin2 n }}}} and 
\colorbox[RGB]{253,246,227}{{{{\color[RGB]{101, 123, 131} k : ℕ }}}}\paragraph{fin2.left}
\par
\colorbox[RGB]{253,246,227}{{{{\color[RGB]{101, 123, 131} left k }}}} is the embedding 
\colorbox[RGB]{253,246,227}{{{{\color[RGB]{101, 123, 131} fin2 n  }}}{{{\color[RGB]{133, 153, 0} → }}}{{{\color[RGB]{101, 123, 131}  fin2 (k  }}}{{{\color[RGB]{181, 137, 0} + }}}{{{\color[RGB]{101, 123, 131}  n) }}}}\paragraph{fin2.insert\_perm}
\par
\colorbox[RGB]{253,246,227}{{{{\color[RGB]{101, 123, 131} insert\_perm a }}}} is a permutation of 
\colorbox[RGB]{253,246,227}{{{{\color[RGB]{101, 123, 131} fin2 n }}}} with the following properties:
\begin{itemize}\item \colorbox[RGB]{253,246,227}{{{{\color[RGB]{101, 123, 131} insert\_perm a i  }}}{{{\color[RGB]{181, 137, 0} = }}}{{{\color[RGB]{101, 123, 131}  i }}}{{{\color[RGB]{181, 137, 0} + }}}{{{\color[RGB]{108, 113, 196} 1 }}}} if 
\colorbox[RGB]{253,246,227}{{{{\color[RGB]{101, 123, 131} i  }}}{{{\color[RGB]{181, 137, 0} < }}}{{{\color[RGB]{101, 123, 131}  a }}}}
\item \colorbox[RGB]{253,246,227}{{{{\color[RGB]{101, 123, 131} insert\_perm a a  }}}{{{\color[RGB]{181, 137, 0} = }}}{{{\color[RGB]{101, 123, 131}   }}}{{{\color[RGB]{108, 113, 196} 0 }}}}
\item \colorbox[RGB]{253,246,227}{{{{\color[RGB]{101, 123, 131} insert\_perm a i  }}}{{{\color[RGB]{181, 137, 0} = }}}{{{\color[RGB]{101, 123, 131}  i }}}} if 
\colorbox[RGB]{253,246,227}{{{{\color[RGB]{101, 123, 131} i  }}}{{{\color[RGB]{181, 137, 0} > }}}{{{\color[RGB]{101, 123, 131}  a }}}}
\end{itemize}\paragraph{fin2.remap\_left}
\par
\colorbox[RGB]{253,246,227}{{{{\color[RGB]{101, 123, 131} remap\_left f k : fin2 (m  }}}{{{\color[RGB]{181, 137, 0} + }}}{{{\color[RGB]{101, 123, 131}  k)  }}}{{{\color[RGB]{133, 153, 0} → }}}{{{\color[RGB]{101, 123, 131}  fin2 (n  }}}{{{\color[RGB]{181, 137, 0} + }}}{{{\color[RGB]{101, 123, 131}  k) }}}} applies the function
\colorbox[RGB]{253,246,227}{{{{\color[RGB]{101, 123, 131} f : fin2 m  }}}{{{\color[RGB]{133, 153, 0} → }}}{{{\color[RGB]{101, 123, 131}  fin2 n }}}} to inputs less than 
\colorbox[RGB]{253,246,227}{{{{\color[RGB]{101, 123, 131} m }}}}, and leaves the right part
on the right (that is, 
\colorbox[RGB]{253,246,227}{{{{\color[RGB]{101, 123, 131} remap\_left f k (m  }}}{{{\color[RGB]{181, 137, 0} + }}}{{{\color[RGB]{101, 123, 131}  i)  }}}{{{\color[RGB]{181, 137, 0} = }}}{{{\color[RGB]{101, 123, 131}  n  }}}{{{\color[RGB]{181, 137, 0} + }}}{{{\color[RGB]{101, 123, 131}  i }}}}).
\paragraph{fin2.is\_lt}
\par
This is a simple type class inference prover for proof obligations
of the form 
\colorbox[RGB]{253,246,227}{{{{\color[RGB]{101, 123, 131} m  }}}{{{\color[RGB]{181, 137, 0} < }}}{{{\color[RGB]{101, 123, 131}  n }}}} where 
\colorbox[RGB]{253,246,227}{{{{\color[RGB]{101, 123, 131} m n : ℕ }}}}.
\paragraph{fin2.of\_nat'}
\par
Use type class inference to infer the boundedness proof, so that we
can directly convert a 
\colorbox[RGB]{253,246,227}{{{{\color[RGB]{101, 123, 131} nat }}}} into a 
\colorbox[RGB]{253,246,227}{{{{\color[RGB]{101, 123, 131} fin2 n }}}}. This supports
notation like 
\colorbox[RGB]{253,246,227}{{{{\color[RGB]{101, 123, 131} \& }}}{{{\color[RGB]{108, 113, 196} 1 }}}{{{\color[RGB]{101, 123, 131}  : fin  }}}{{{\color[RGB]{108, 113, 196} 3 }}}}.
\paragraph{vector3}
\par
Alternate definition of 
\colorbox[RGB]{253,246,227}{{{{\color[RGB]{101, 123, 131} vector }}}} based on 
\colorbox[RGB]{253,246,227}{{{{\color[RGB]{101, 123, 131} fin2 }}}}.
\paragraph{vector3.nil}
\par
The empty vector
\paragraph{vector3.cons}
\par
The vector cons operation
\paragraph{vector3.nth}
\par
Get the 
\colorbox[RGB]{253,246,227}{{{{\color[RGB]{101, 123, 131} i }}}}th element of a vector
\paragraph{vector3.of\_fn}
\par
Construct a vector from a function on 
\colorbox[RGB]{253,246,227}{{{{\color[RGB]{101, 123, 131} fin2 }}}}.
\paragraph{vector3.head}
\par
Get the head of a nonempty vector.
\paragraph{vector3.tail}
\par
Get the tail of a nonempty vector.
\paragraph{vector3.append}
\par
Append two vectors
\paragraph{vector3.insert}
\par
Insert 
\colorbox[RGB]{253,246,227}{{{{\color[RGB]{101, 123, 131} a }}}} into 
\colorbox[RGB]{253,246,227}{{{{\color[RGB]{101, 123, 131} v }}}} at index 
\colorbox[RGB]{253,246,227}{{{{\color[RGB]{101, 123, 131} i }}}}.
\paragraph{vector\_ex}
\par
"Curried" exists, i.e. ∃ x1 ... xn, f 
{[}
x1, ..., xn
{]}
\paragraph{vector\_all}
\par
"Curried" forall, i.e. ∀ x1 ... xn, f 
{[}
x1, ..., xn
{]}
\paragraph{vector\_allp}
\par
\colorbox[RGB]{253,246,227}{{{{\color[RGB]{101, 123, 131} vector\_allp p v }}}} is equivalent to 
\colorbox[RGB]{253,246,227}{{{{\color[RGB]{101, 123, 131} ∀ i, p (v i) }}}}, but unfolds directly to a conjunction,
i.e. 
\colorbox[RGB]{253,246,227}{{{{\color[RGB]{101, 123, 131} vector\_allp p {[} }}}{{{\color[RGB]{108, 113, 196} 0 }}}{{{\color[RGB]{101, 123, 131} ,  }}}{{{\color[RGB]{108, 113, 196} 1 }}}{{{\color[RGB]{101, 123, 131} ,  }}}{{{\color[RGB]{108, 113, 196} 2 }}}{{{\color[RGB]{101, 123, 131} {]}  }}}{{{\color[RGB]{181, 137, 0} = }}}{{{\color[RGB]{101, 123, 131}  p  }}}{{{\color[RGB]{108, 113, 196} 0 }}}{{{\color[RGB]{101, 123, 131}   }}}{{{\color[RGB]{181, 137, 0} ∧ }}}{{{\color[RGB]{101, 123, 131}  p  }}}{{{\color[RGB]{108, 113, 196} 1 }}}{{{\color[RGB]{101, 123, 131}   }}}{{{\color[RGB]{181, 137, 0} ∧ }}}{{{\color[RGB]{101, 123, 131}  p  }}}{{{\color[RGB]{108, 113, 196} 2 }}}}.
\paragraph{list\_all}
\par
\colorbox[RGB]{253,246,227}{{{{\color[RGB]{101, 123, 131} list\_all p l }}}} is equivalent to 
\colorbox[RGB]{253,246,227}{{{{\color[RGB]{101, 123, 131} ∀ a ∈ l, p a }}}}, but unfolds directly to a conjunction,
i.e. 
\colorbox[RGB]{253,246,227}{{{{\color[RGB]{101, 123, 131} list\_all p {[} }}}{{{\color[RGB]{108, 113, 196} 0 }}}{{{\color[RGB]{101, 123, 131} ,  }}}{{{\color[RGB]{108, 113, 196} 1 }}}{{{\color[RGB]{101, 123, 131} ,  }}}{{{\color[RGB]{108, 113, 196} 2 }}}{{{\color[RGB]{101, 123, 131} {]}  }}}{{{\color[RGB]{181, 137, 0} = }}}{{{\color[RGB]{101, 123, 131}  p  }}}{{{\color[RGB]{108, 113, 196} 0 }}}{{{\color[RGB]{101, 123, 131}   }}}{{{\color[RGB]{181, 137, 0} ∧ }}}{{{\color[RGB]{101, 123, 131}  p  }}}{{{\color[RGB]{108, 113, 196} 1 }}}{{{\color[RGB]{101, 123, 131}   }}}{{{\color[RGB]{181, 137, 0} ∧ }}}{{{\color[RGB]{101, 123, 131}  p  }}}{{{\color[RGB]{108, 113, 196} 2 }}}}.
\paragraph{is\_poly}
\par
A predicate asserting that a function is a multivariate integer polynomial.
(We are being a bit lazy here by allowing many representations for multiplication,
rather than only allowing monomials and addition, but the definition is equivalent
and this is easier to use.)
\paragraph{poly}
\par
The type of multivariate integer polynomials
\paragraph{poly.isp}
\par
The underlying function of a 
\colorbox[RGB]{253,246,227}{{{{\color[RGB]{101, 123, 131} poly }}}} is a polynomial
\paragraph{poly.ext}
\par
Extensionality for 
\colorbox[RGB]{253,246,227}{{{{\color[RGB]{101, 123, 131} poly α }}}}\paragraph{poly.subst}
\par
Construct a 
\colorbox[RGB]{253,246,227}{{{{\color[RGB]{101, 123, 131} poly }}}} given an extensionally equivalent 
\colorbox[RGB]{253,246,227}{{{{\color[RGB]{101, 123, 131} poly }}}}.
\paragraph{poly.proj}
\par
The 
\colorbox[RGB]{253,246,227}{{{{\color[RGB]{101, 123, 131} i }}}}th projection function, 
\colorbox[RGB]{253,246,227}{{{{\color[RGB]{101, 123, 131} x\_i }}}}.
\paragraph{poly.const}
\par
The constant function with value 
\colorbox[RGB]{253,246,227}{{{{\color[RGB]{101, 123, 131} n : ℤ }}}}.
\paragraph{poly.zero}
\par
The zero polynomial
\paragraph{poly.one}
\par
The zero polynomial
\paragraph{poly.sub}
\par
Subtraction of polynomials
\paragraph{poly.neg}
\par
Negation of a polynomial
\paragraph{poly.add}
\par
Addition of polynomials
\paragraph{poly.mul}
\par
Multiplication of polynomials
\paragraph{poly.sumsq}
\par
The sum of squares of a list of polynomials. This is relevant for
Diophantine equations, because it means that a list of equations
can be encoded as a single equation: 
\colorbox[RGB]{253,246,227}{{{{\color[RGB]{101, 123, 131} x  }}}{{{\color[RGB]{181, 137, 0} = }}}{{{\color[RGB]{101, 123, 131}   }}}{{{\color[RGB]{108, 113, 196} 0 }}}{{{\color[RGB]{101, 123, 131}   }}}{{{\color[RGB]{181, 137, 0} ∧ }}}{{{\color[RGB]{101, 123, 131}  y  }}}{{{\color[RGB]{181, 137, 0} = }}}{{{\color[RGB]{101, 123, 131}   }}}{{{\color[RGB]{108, 113, 196} 0 }}}{{{\color[RGB]{101, 123, 131}   }}}{{{\color[RGB]{181, 137, 0} ∧ }}}{{{\color[RGB]{101, 123, 131}  z  }}}{{{\color[RGB]{181, 137, 0} = }}}{{{\color[RGB]{101, 123, 131}   }}}{{{\color[RGB]{108, 113, 196} 0 }}}} is
equivalent to 
\colorbox[RGB]{253,246,227}{{{{\color[RGB]{101, 123, 131} x\textasciicircum{} }}}{{{\color[RGB]{108, 113, 196} 2 }}}{{{\color[RGB]{101, 123, 131}   }}}{{{\color[RGB]{181, 137, 0} + }}}{{{\color[RGB]{101, 123, 131}  y\textasciicircum{} }}}{{{\color[RGB]{108, 113, 196} 2 }}}{{{\color[RGB]{101, 123, 131}   }}}{{{\color[RGB]{181, 137, 0} + }}}{{{\color[RGB]{101, 123, 131}  z\textasciicircum{} }}}{{{\color[RGB]{108, 113, 196} 2 }}}{{{\color[RGB]{101, 123, 131}   }}}{{{\color[RGB]{181, 137, 0} = }}}{{{\color[RGB]{101, 123, 131}   }}}{{{\color[RGB]{108, 113, 196} 0 }}}}.
\paragraph{poly.remap}
\par
Map the index set of variables, replacing 
\colorbox[RGB]{253,246,227}{{{{\color[RGB]{101, 123, 131} x\_i }}}} with 
\colorbox[RGB]{253,246,227}{{{{\color[RGB]{101, 123, 131} x\_(f i) }}}}.
\paragraph{sum.join}
\par
combine two functions into a function on the disjoint union
\paragraph{option.cons}
\par
Functions from 
\colorbox[RGB]{253,246,227}{{{{\color[RGB]{101, 123, 131} option }}}} can be combined similarly to 
\colorbox[RGB]{253,246,227}{{{{\color[RGB]{101, 123, 131} vector.cons }}}}\paragraph{dioph}
\par
A set 
\colorbox[RGB]{253,246,227}{{{{\color[RGB]{101, 123, 131} S ⊆ ℕ\textasciicircum{}α }}}} is diophantine if there exists a polynomial on
\colorbox[RGB]{253,246,227}{{{{\color[RGB]{101, 123, 131} α ⊕ β }}}} such that 
\colorbox[RGB]{253,246,227}{{{{\color[RGB]{101, 123, 131} v ∈ S }}}} iff there exists 
\colorbox[RGB]{253,246,227}{{{{\color[RGB]{101, 123, 131} t : ℕ\textasciicircum{}β }}}} with 
\colorbox[RGB]{253,246,227}{{{{\color[RGB]{101, 123, 131} p (v, t)  }}}{{{\color[RGB]{181, 137, 0} = }}}{{{\color[RGB]{101, 123, 131}   }}}{{{\color[RGB]{108, 113, 196} 0 }}}}.
\paragraph{dioph.dioph\_pfun}
\par
A partial function is Diophantine if its graph is Diophantine.
\paragraph{dioph.dioph\_fn}
\par
A function is Diophantine if its graph is Diophantine.
\section{number\_theory/pell.lean}\paragraph{pell.pell}
\par
The Pell sequences, defined together in mutual recursion.
\paragraph{pell.xn}
\par
The Pell 
\colorbox[RGB]{253,246,227}{{{{\color[RGB]{101, 123, 131} x }}}} sequence.
\paragraph{pell.yn}
\par
The Pell 
\colorbox[RGB]{253,246,227}{{{{\color[RGB]{101, 123, 131} y }}}} sequence.
\paragraph{pell.pell\_zd}
\par
The Pell sequence can also be viewed as an element of 
\colorbox[RGB]{253,246,227}{{{{\color[RGB]{101, 123, 131} ℤ√d }}}}\paragraph{pell.is\_pell}
\par
The property of being a solution to the Pell equation, expressed
as a property of elements of 
\colorbox[RGB]{253,246,227}{{{{\color[RGB]{101, 123, 131} ℤ√d }}}}.
\section{number\_theory/sum\_two\_squares.lean}\section{order/basic.lean}\paragraph{order.preimage}
\par
Given an order 
\colorbox[RGB]{253,246,227}{{{{\color[RGB]{101, 123, 131} R }}}} on 
\colorbox[RGB]{253,246,227}{{{{\color[RGB]{101, 123, 131} β }}}} and a function 
\colorbox[RGB]{253,246,227}{{{{\color[RGB]{101, 123, 131} f : α  }}}{{{\color[RGB]{133, 153, 0} → }}}{{{\color[RGB]{101, 123, 131}  β }}}},
the preimage order on 
\colorbox[RGB]{253,246,227}{{{{\color[RGB]{101, 123, 131} α }}}} is defined by 
\colorbox[RGB]{253,246,227}{{{{\color[RGB]{101, 123, 131} x  }}}{{{\color[RGB]{181, 137, 0} ≤ }}}{{{\color[RGB]{101, 123, 131}  y  }}}{{{\color[RGB]{181, 137, 0} ↔ }}}{{{\color[RGB]{101, 123, 131}  f x  }}}{{{\color[RGB]{181, 137, 0} ≤ }}}{{{\color[RGB]{101, 123, 131}  f y }}}}.
It is the unique order on 
\colorbox[RGB]{253,246,227}{{{{\color[RGB]{101, 123, 131} α }}}} making 
\colorbox[RGB]{253,246,227}{{{{\color[RGB]{101, 123, 131} f }}}} an order embedding
(assuming 
\colorbox[RGB]{253,246,227}{{{{\color[RGB]{101, 123, 131} f }}}} is injective).
\paragraph{monotone}
\par
A function between preorders is monotone if
\colorbox[RGB]{253,246,227}{{{{\color[RGB]{101, 123, 131} a  }}}{{{\color[RGB]{181, 137, 0} ≤ }}}{{{\color[RGB]{101, 123, 131}  b }}}} implies 
\colorbox[RGB]{253,246,227}{{{{\color[RGB]{101, 123, 131} f a  }}}{{{\color[RGB]{181, 137, 0} ≤ }}}{{{\color[RGB]{101, 123, 131}  f b }}}}.
\paragraph{prod.partial\_order}
\par
The pointwise partial order on a product.
(The lexicographic ordering is defined in order/lexicographic.lean, and the instances are
available via the type synonym 
\colorbox[RGB]{253,246,227}{{{{\color[RGB]{101, 123, 131} lex α β  }}}{{{\color[RGB]{181, 137, 0} = }}}{{{\color[RGB]{101, 123, 131}  α × β }}}}.)
\paragraph{no\_top\_order}
\par
order without a top element; somtimes called cofinal
\paragraph{no\_bot\_order}
\par
order without a bottom element; somtimes called coinitial or dense
\paragraph{densely\_ordered}
\par
An order is dense if there is an element between any pair of distinct elements.
\paragraph{partial\_order\_of\_SO}
\par
Construct a partial order from a 
\colorbox[RGB]{253,246,227}{{{{\color[RGB]{101, 123, 131} is\_strict\_order }}}} relation
\paragraph{is\_strict\_total\_order'}
\par
This is basically the same as 
\colorbox[RGB]{253,246,227}{{{{\color[RGB]{101, 123, 131} is\_strict\_total\_order }}}}, but that definition is
in Type (probably by mistake) and also has redundant assumptions.
\paragraph{linear\_order\_of\_STO'}
\par
Construct a linear order from a 
\colorbox[RGB]{253,246,227}{{{{\color[RGB]{101, 123, 131} is\_strict\_total\_order' }}}} relation
\paragraph{decidable\_linear\_order\_of\_STO'}
\par
Construct a decidable linear order from a 
\colorbox[RGB]{253,246,227}{{{{\color[RGB]{101, 123, 131} is\_strict\_total\_order' }}}} relation
\paragraph{is\_order\_connected}
\par
A connected order is one satisfying the condition 
\colorbox[RGB]{253,246,227}{{{{\color[RGB]{101, 123, 131} a  }}}{{{\color[RGB]{181, 137, 0} < }}}{{{\color[RGB]{101, 123, 131}  c  }}}{{{\color[RGB]{133, 153, 0} → }}}{{{\color[RGB]{101, 123, 131}  a  }}}{{{\color[RGB]{181, 137, 0} < }}}{{{\color[RGB]{101, 123, 131}  b  }}}{{{\color[RGB]{181, 137, 0} ∨ }}}{{{\color[RGB]{101, 123, 131}  b  }}}{{{\color[RGB]{181, 137, 0} < }}}{{{\color[RGB]{101, 123, 131}  c }}}}.
This is recognizable as an intuitionistic substitute for 
\colorbox[RGB]{253,246,227}{{{{\color[RGB]{101, 123, 131} a  }}}{{{\color[RGB]{181, 137, 0} ≤ }}}{{{\color[RGB]{101, 123, 131}  b  }}}{{{\color[RGB]{181, 137, 0} ∨ }}}{{{\color[RGB]{101, 123, 131}  b  }}}{{{\color[RGB]{181, 137, 0} ≤ }}}{{{\color[RGB]{101, 123, 131}  a }}}} on
the constructive reals, and is also known as negative transitivity,
since the contrapositive asserts transitivity of the relation 
\colorbox[RGB]{253,246,227}{{{{\color[RGB]{181, 137, 0} ¬ }}}{{{\color[RGB]{101, 123, 131}  a  }}}{{{\color[RGB]{181, 137, 0} < }}}{{{\color[RGB]{101, 123, 131}  b }}}}.
\paragraph{is\_extensional}
\par
An extensional relation is one in which an element is determined by its set
of predecessors. It is named for the 
\colorbox[RGB]{253,246,227}{{{{\color[RGB]{101, 123, 131} x ∈ y }}}} relation in set theory, whose
extensionality is one of the first axioms of ZFC.
\paragraph{is\_well\_order}
\par
A well order is a well-founded linear order.
\paragraph{unbounded}
\par
An unbounded or cofinal set
\paragraph{bounded}
\par
A bounded or final set
\paragraph{well\_founded.min}
\par
The minimum element of a nonempty set in a well-founded order
\paragraph{directed}
\par
A family of elements of α is directed (with respect to a relation 
\colorbox[RGB]{253,246,227}{{{{\color[RGB]{101, 123, 131} ≼ }}}} on α)
if there is a member of the family 
\colorbox[RGB]{253,246,227}{{{{\color[RGB]{101, 123, 131} ≼ }}}}-above any pair in the family.
\paragraph{directed\_on}
\par
A subset of α is directed if there is an element of the set 
\colorbox[RGB]{253,246,227}{{{{\color[RGB]{101, 123, 131} ≼ }}}}-above any
pair of elements in the set.
\section{order/boolean\_algebra.lean}\paragraph{lattice.boolean\_algebra}
\par
A boolean algebra is a bounded distributive lattice with a
complementation operation 
\colorbox[RGB]{253,246,227}{{{{\color[RGB]{181, 137, 0} - }}}} such that 
\colorbox[RGB]{253,246,227}{{{{\color[RGB]{101, 123, 131} x ⊓  }}}{{{\color[RGB]{181, 137, 0} - }}}{{{\color[RGB]{101, 123, 131}  x  }}}{{{\color[RGB]{181, 137, 0} = }}}{{{\color[RGB]{101, 123, 131}  ⊥ }}}} and 
\colorbox[RGB]{253,246,227}{{{{\color[RGB]{101, 123, 131} x ⊔  }}}{{{\color[RGB]{181, 137, 0} - }}}{{{\color[RGB]{101, 123, 131}  x  }}}{{{\color[RGB]{181, 137, 0} = }}}{{{\color[RGB]{101, 123, 131}  ⊤ }}}}.
This is a generalization of (classical) logic of propositions, or
the powerset lattice.
\section{order/bounded\_lattice.lean}\paragraph{lattice.has\_top}
\par
Typeclass for the 
\colorbox[RGB]{253,246,227}{{{{\color[RGB]{101, 123, 131} ⊤ }}}} (
\colorbox[RGB]{253,246,227}{{{{\color[RGB]{101, 123, 131} \textbackslash{}top }}}}) notation
\paragraph{lattice.has\_bot}
\par
Typeclass for the 
\colorbox[RGB]{253,246,227}{{{{\color[RGB]{101, 123, 131} ⊥ }}}} (
\colorbox[RGB]{253,246,227}{{{{\color[RGB]{101, 123, 131} \textbackslash{}bot }}}}) notation
\paragraph{lattice.order\_top}
\par
An 
\colorbox[RGB]{253,246,227}{{{{\color[RGB]{101, 123, 131} order\_top }}}} is a partial order with a maximal element.
(We could state this on preorders, but then it wouldn't be unique
so distinguishing one would seem odd.)
\paragraph{lattice.order\_bot}
\par
An 
\colorbox[RGB]{253,246,227}{{{{\color[RGB]{101, 123, 131} order\_bot }}}} is a partial order with a minimal element.
(We could state this on preorders, but then it wouldn't be unique
so distinguishing one would seem odd.)
\paragraph{lattice.semilattice\_sup\_top}
\par
A 
\colorbox[RGB]{253,246,227}{{{{\color[RGB]{101, 123, 131} semilattice\_sup\_top }}}} is a semilattice with top and join.
\paragraph{lattice.semilattice\_sup\_bot}
\par
A 
\colorbox[RGB]{253,246,227}{{{{\color[RGB]{101, 123, 131} semilattice\_sup\_bot }}}} is a semilattice with bottom and join.
\paragraph{lattice.semilattice\_inf\_top}
\par
A 
\colorbox[RGB]{253,246,227}{{{{\color[RGB]{101, 123, 131} semilattice\_inf\_top }}}} is a semilattice with top and meet.
\paragraph{lattice.semilattice\_inf\_bot}
\par
A 
\colorbox[RGB]{253,246,227}{{{{\color[RGB]{101, 123, 131} semilattice\_inf\_bot }}}} is a semilattice with bottom and meet.
\paragraph{lattice.bounded\_lattice}
\par
A bounded lattice is a lattice with a top and bottom element,
denoted 
\colorbox[RGB]{253,246,227}{{{{\color[RGB]{101, 123, 131} ⊤ }}}} and 
\colorbox[RGB]{253,246,227}{{{{\color[RGB]{101, 123, 131} ⊥ }}}} respectively. This allows for the interpretation
of all finite suprema and infima, taking 
\colorbox[RGB]{253,246,227}{{{{\color[RGB]{101, 123, 131} inf ∅  }}}{{{\color[RGB]{181, 137, 0} = }}}{{{\color[RGB]{101, 123, 131}  ⊤ }}}} and 
\colorbox[RGB]{253,246,227}{{{{\color[RGB]{101, 123, 131} sup ∅  }}}{{{\color[RGB]{181, 137, 0} = }}}{{{\color[RGB]{101, 123, 131}  ⊥ }}}}.
\paragraph{lattice.bounded\_distrib\_lattice}
\par
A bounded distributive lattice is exactly what it sounds like.
\section{order/bounds.lean}\section{order/complete\_boolean\_algebra.lean}\paragraph{lattice.complete\_distrib\_lattice}
\par
A complete distributive lattice is a bit stronger than the name might
suggest; perhaps completely distributive lattice is more descriptive,
as this class includes a requirement that the lattice join
distribute over 
\emph{arbitrary
} infima, and similarly for the dual.
\paragraph{lattice.complete\_boolean\_algebra}
\par
A complete boolean algebra is a completely distributive boolean algebra.
\section{order/complete\_lattice.lean}\paragraph{lattice.has\_Sup}
\par
class for the 
\colorbox[RGB]{253,246,227}{{{{\color[RGB]{101, 123, 131} Sup }}}} operator
\paragraph{lattice.has\_Inf}
\par
class for the 
\colorbox[RGB]{253,246,227}{{{{\color[RGB]{101, 123, 131} Inf }}}} operator
\paragraph{lattice.Sup}
\par
Supremum of a set
\paragraph{lattice.Inf}
\par
Infimum of a set
\paragraph{lattice.supr}
\par
Indexed supremum
\paragraph{lattice.infi}
\par
Indexed infimum
\paragraph{lattice.complete\_lattice}
\par
A complete lattice is a bounded lattice which
has suprema and infima for every subset.
\paragraph{lattice.complete\_linear\_order}
\par
A complete linear order is a linear order whose lattice structure is complete.
\paragraph{lattice.ord\_continuous}
\par
A function 
\colorbox[RGB]{253,246,227}{{{{\color[RGB]{101, 123, 131} f }}}} between complete lattices is order-continuous
if it preserves all suprema.
\section{order/conditionally\_complete\_lattice.lean}\paragraph{bdd\_above\_of\_bdd\_above\_of\_monotone}
\par
The image under a monotone function of a set which is bounded above is bounded above
\paragraph{bdd\_below\_of\_bdd\_below\_of\_monotone}
\par
The image under a monotone function of a set which is bounded below is bounded below
\paragraph{bdd\_above\_top}
\par
When there is a global maximum, every set is bounded above.
\paragraph{bdd\_below\_bot}
\par
When there is a global minimum, every set is bounded below.
\paragraph{bdd\_above\_union}
\par
The union of two sets is bounded above if and only if each of the sets is.
\paragraph{bdd\_above\_insert}
\par
Adding a point to a set preserves its boundedness above.
\paragraph{bdd\_above\_finite}
\par
A finite set is bounded above.
\paragraph{bdd\_above\_finite\_union}
\par
A finite union of sets which are all bounded above is still bounded above.
\paragraph{bdd\_below\_union}
\par
The union of two sets is bounded below if and only if each of the sets is.
\paragraph{bdd\_below\_insert}
\par
Adding a point to a set preserves its boundedness below.
\paragraph{bdd\_below\_finite}
\par
A finite set is bounded below.
\paragraph{bdd\_below\_finite\_union}
\par
A finite union of sets which are all bounded below is still bounded below.
\paragraph{lattice.conditionally\_complete\_lattice}
\par
A conditionally complete lattice is a lattice in which
every nonempty subset which is bounded above has a supremum, and
every nonempty subset which is bounded below has an infimum.
Typical examples are real numbers or natural numbers.
\par
To differentiate the statements from the corresponding statements in (unconditional)
complete lattices, we prefix Inf and Sup by a c everywhere. The same statements should
hold in both worlds, sometimes with additional assumptions of non-emptyness or
boundedness.
\paragraph{lattice.cSup\_intro}
\par
Introduction rule to prove that b is the supremum of s: it suffices to check that b
is larger than all elements of s, and that this is not the case of any w
<
b.
\paragraph{lattice.cInf\_intro}
\par
Introduction rule to prove that b is the infimum of s: it suffices to check that b
is smaller than all elements of s, and that this is not the case of any w>b.
\paragraph{lattice.cSup\_of\_mem\_of\_le}
\par
When an element a of a set s is larger than all elements of the set, it is Sup s
\paragraph{lattice.cInf\_of\_mem\_of\_le}
\par
When an element a of a set s is smaller than all elements of the set, it is Inf s
\paragraph{lattice.lt\_cSup\_of\_lt}
\par
b 
<
 Sup s when there is an element a in s with b 
<
 a, when s is bounded above.
This is essentially an iff, except that the assumptions for the two implications are
slightly different (one needs boundedness above for one direction, nonemptyness and linear
order for the other one), so we formulate separately the two implications, contrary to
the complete\_lattice case.
\paragraph{lattice.cInf\_lt\_of\_lt}
\par
Inf s 
<
 b s when there is an element a in s with a 
<
 b, when s is bounded below.
This is essentially an iff, except that the assumptions for the two implications are
slightly different (one needs boundedness below for one direction, nonemptyness and linear
order for the other one), so we formulate separately the two implications, contrary to
the complete\_lattice case.
\paragraph{lattice.cSup\_singleton}
\par
The supremum of a singleton is the element of the singleton
\paragraph{lattice.cInf\_singleton}
\par
The infimum of a singleton is the element of the singleton
\paragraph{lattice.cInf\_le\_cSup}
\par
If a set is bounded below and above, and nonempty, its infimum is less than or equal to
its supremum.
\paragraph{lattice.cSup\_union}
\par
The sup of a union of sets is the max of the suprema of each subset, under the assumptions
that all sets are bounded above and nonempty.
\paragraph{lattice.cInf\_union}
\par
The inf of a union of sets is the min of the infima of each subset, under the assumptions
that all sets are bounded below and nonempty.
\paragraph{lattice.cSup\_inter\_le}
\par
The supremum of an intersection of sets is bounded by the minimum of the suprema of each
set, if all sets are bounded above and nonempty.
\paragraph{lattice.le\_cInf\_inter}
\par
The infimum of an intersection of sets is bounded below by the maximum of the
infima of each set, if all sets are bounded below and nonempty.
\paragraph{lattice.cSup\_insert}
\par
The supremum of insert a s is the maximum of a and the supremum of s, if s is
nonempty and bounded above.
\paragraph{lattice.cInf\_insert}
\par
The infimum of insert a s is the minimum of a and the infimum of s, if s is
nonempty and bounded below.
\paragraph{lattice.csupr\_le\_csupr}
\par
The indexed supremum of two functions are comparable if the functions are pointwise comparable
\paragraph{lattice.csupr\_le}
\par
The indexed supremum of a function is bounded above by a uniform bound
\paragraph{lattice.le\_csupr}
\par
The indexed supremum of a function is bounded below by the value taken at one point
\paragraph{lattice.cinfi\_le\_cinfi}
\par
The indexed infimum of two functions are comparable if the functions are pointwise comparable
\paragraph{lattice.le\_cinfi}
\par
The indexed minimum of a function is bounded below by a uniform lower bound
\paragraph{lattice.cinfi\_le}
\par
The indexed infimum of a function is bounded above by the value taken at one point
\paragraph{lattice.exists\_lt\_of\_lt\_cSup}
\par
When b 
<
 Sup s, there is an element a in s with b 
<
 a, if s is nonempty and the order is
a linear order.
\paragraph{lattice.exists\_lt\_of\_cInf\_lt}
\par
When Inf s 
<
 b, there is an element a in s with a 
<
 b, if s is nonempty and the order is
a linear order.
\paragraph{lattice.cSup\_intro'}
\par
Introduction rule to prove that b is the supremum of s: it suffices to check that
\begin{enumerate}[1]
\item b is an upper bound

\item every other upper bound b' satisfies b ≤ b'.

\end{enumerate}\paragraph{lattice.lattice.lattice}
\par
This instance is necessary, otherwise the lattice operations would be derived via
conditionally\_complete\_linear\_order\_bot and marked as noncomputable.
\section{order/default.lean}\section{order/filter/basic.lean}\paragraph{filter.has\_mem}
\par
If 
\colorbox[RGB]{253,246,227}{{{{\color[RGB]{101, 123, 131} F }}}} is a filter on 
\colorbox[RGB]{253,246,227}{{{{\color[RGB]{101, 123, 131} α }}}}, and 
\colorbox[RGB]{253,246,227}{{{{\color[RGB]{101, 123, 131} U }}}} a subset of 
\colorbox[RGB]{253,246,227}{{{{\color[RGB]{101, 123, 131} α }}}} then we can write 
\colorbox[RGB]{253,246,227}{{{{\color[RGB]{101, 123, 131} U ∈ F }}}} as on paper.
\paragraph{tactic.interactive.filter\_upwards}
\par
\colorbox[RGB]{253,246,227}{{{{\color[RGB]{101, 123, 131} filter\_upwards {[}h1, ⋯, hn{]} }}}} replaces a goal of the form 
\colorbox[RGB]{253,246,227}{{{{\color[RGB]{101, 123, 131} s ∈ f }}}}and terms 
\colorbox[RGB]{253,246,227}{{{{\color[RGB]{101, 123, 131} h1 : t1 ∈ f, ⋯, hn : tn ∈ f }}}} with 
\colorbox[RGB]{253,246,227}{{{{\color[RGB]{101, 123, 131} ∀x, x ∈ t1  }}}{{{\color[RGB]{133, 153, 0} → }}}{{{\color[RGB]{101, 123, 131}  ⋯  }}}{{{\color[RGB]{133, 153, 0} → }}}{{{\color[RGB]{101, 123, 131}  x ∈ tn  }}}{{{\color[RGB]{133, 153, 0} → }}}{{{\color[RGB]{101, 123, 131}  x ∈ s }}}}.
\par
\colorbox[RGB]{253,246,227}{{{{\color[RGB]{101, 123, 131} filter\_upwards {[}h1, ⋯, hn{]} e }}}} is a short form for 
\colorbox[RGB]{253,246,227}{{{{\color[RGB]{101, 123, 131} \{ filter\_upwards {[}h1, ⋯, hn{]}, exact e \} }}}}.
\paragraph{filter.principal}
\par
The principal filter of 
\colorbox[RGB]{253,246,227}{{{{\color[RGB]{101, 123, 131} s }}}} is the collection of all supersets of 
\colorbox[RGB]{253,246,227}{{{{\color[RGB]{101, 123, 131} s }}}}.
\paragraph{filter.join}
\par
The join of a filter of filters is defined by the relation 
\colorbox[RGB]{253,246,227}{{{{\color[RGB]{101, 123, 131} s ∈ join f  }}}{{{\color[RGB]{181, 137, 0} ↔ }}}{{{\color[RGB]{101, 123, 131}  \{t | s ∈ t\} ∈ f }}}}.
\paragraph{filter.generate\_sets}
\par
\colorbox[RGB]{253,246,227}{{{{\color[RGB]{101, 123, 131} generate\_sets g s }}}}: 
\colorbox[RGB]{253,246,227}{{{{\color[RGB]{101, 123, 131} s }}}} is in the filter closure of 
\colorbox[RGB]{253,246,227}{{{{\color[RGB]{101, 123, 131} g }}}}.
\paragraph{filter.generate}
\par
\colorbox[RGB]{253,246,227}{{{{\color[RGB]{101, 123, 131} generate g }}}} is the smallest filter containing the sets 
\colorbox[RGB]{253,246,227}{{{{\color[RGB]{101, 123, 131} g }}}}.
\paragraph{filter.lattice.has\_inf}
\par
The infimum of filters is the filter generated by intersections
of elements of the two filters.
\paragraph{filter.map}
\par
The forward map of a filter
\paragraph{filter.comap}
\par
The inverse map of a filter
\paragraph{filter.cofinite}
\par
The cofinite filter is the filter of subsets whose complements are finite.
\paragraph{filter.bind}
\par
The monadic bind operation on filter is defined the usual way in terms of 
\colorbox[RGB]{253,246,227}{{{{\color[RGB]{101, 123, 131} map }}}} and 
\colorbox[RGB]{253,246,227}{{{{\color[RGB]{101, 123, 131} join }}}}.
\par
Unfortunately, this 
\colorbox[RGB]{253,246,227}{{{{\color[RGB]{101, 123, 131} bind }}}} does not result in the expected applicative. See 
\colorbox[RGB]{253,246,227}{{{{\color[RGB]{101, 123, 131} filter.seq }}}} for the
applicative instance.
\paragraph{filter.seq}
\par
The applicative sequentiation operation. This is not induced by the bind operation.
\paragraph{filter.tendsto}
\par
\colorbox[RGB]{253,246,227}{{{{\color[RGB]{101, 123, 131} tendsto }}}} is the generic "limit of a function" predicate.
\colorbox[RGB]{253,246,227}{{{{\color[RGB]{101, 123, 131} tendsto f l₁ l₂ }}}} asserts that for every 
\colorbox[RGB]{253,246,227}{{{{\color[RGB]{101, 123, 131} l₂ }}}} neighborhood 
\colorbox[RGB]{253,246,227}{{{{\color[RGB]{101, 123, 131} a }}}},
the 
\colorbox[RGB]{253,246,227}{{{{\color[RGB]{101, 123, 131} f }}}}-preimage of 
\colorbox[RGB]{253,246,227}{{{{\color[RGB]{101, 123, 131} a }}}} is an 
\colorbox[RGB]{253,246,227}{{{{\color[RGB]{101, 123, 131} l₁ }}}} neighborhood.
\paragraph{filter.prod}
\par
Product of filters. This is the filter generated by cartesian products
of elements of the component filters.
\paragraph{filter.at\_top}
\par
\colorbox[RGB]{253,246,227}{{{{\color[RGB]{101, 123, 131} at\_top }}}} is the filter representing the limit 
\colorbox[RGB]{253,246,227}{{{{\color[RGB]{101, 123, 131} → ∞ }}}} on an ordered set.
It is generated by the collection of up-sets 
\colorbox[RGB]{253,246,227}{{{{\color[RGB]{101, 123, 131} \{b | a  }}}{{{\color[RGB]{181, 137, 0} ≤ }}}{{{\color[RGB]{101, 123, 131}  b\} }}}}.
(The preorder need not have a top element for this to be well defined,
and indeed is trivial when a top element exists.)
\paragraph{filter.at\_bot}
\par
\colorbox[RGB]{253,246,227}{{{{\color[RGB]{101, 123, 131} at\_bot }}}} is the filter representing the limit 
\colorbox[RGB]{253,246,227}{{{{\color[RGB]{101, 123, 131} →  }}}{{{\color[RGB]{181, 137, 0} - }}}{{{\color[RGB]{101, 123, 131} ∞ }}}} on an ordered set.
It is generated by the collection of down-sets 
\colorbox[RGB]{253,246,227}{{{{\color[RGB]{101, 123, 131} \{b | b  }}}{{{\color[RGB]{181, 137, 0} ≤ }}}{{{\color[RGB]{101, 123, 131}  a\} }}}}.
(The preorder need not have a bottom element for this to be well defined,
and indeed is trivial when a bottom element exists.)
\paragraph{filter.tendsto\_at\_top\_embedding}
\par
A function 
\colorbox[RGB]{253,246,227}{{{{\color[RGB]{101, 123, 131} f }}}} grows to infinity independent of an order-preserving embedding 
\colorbox[RGB]{253,246,227}{{{{\color[RGB]{101, 123, 131} e }}}}.
\paragraph{filter.map\_at\_top\_eq\_of\_gc}
\par
A function 
\colorbox[RGB]{253,246,227}{{{{\color[RGB]{101, 123, 131} f }}}} maps upwards closed sets (at\_top sets) to upwards closed sets when it is a
Galois insertion. The Galois "insertion" and "connection" is weakened to only require it to be an
insertion and a connetion above 
\colorbox[RGB]{253,246,227}{{{{\color[RGB]{101, 123, 131} b' }}}}.
\paragraph{filter.is\_ultrafilter}
\par
An ultrafilter is a minimal (maximal in the set order) proper filter.
\paragraph{filter.ultrafilter\_iff\_compl\_mem\_iff\_not\_mem}
\par
Equivalent characterization of ultrafilters:
A filter f is an ultrafilter if and only if for each set s,
-s belongs to f if and only if s does not belong to f.
\paragraph{filter.exists\_ultrafilter}
\par
The ultrafilter lemma: Any proper filter is contained in an ultrafilter.
\paragraph{filter.ultrafilter\_of}
\par
Construct an ultrafilter extending a given filter.
The ultrafilter lemma is the assertion that such a filter exists;
we use the axiom of choice to pick one.
\paragraph{filter.sup\_of\_ultrafilters}
\par
A filter equals the intersection of all the ultrafilters which contain it.
\paragraph{filter.tendsto\_iff\_ultrafilter}
\par
The 
\colorbox[RGB]{253,246,227}{{{{\color[RGB]{101, 123, 131} tendsto }}}} relation can be checked on ultrafilters.
\section{order/filter/default.lean}\section{order/filter/filter\_product.lean}\paragraph{filter.bigly\_equal}
\par
Two sequences are bigly equal iff the kernel of their difference is in φ
\paragraph{filter.filterprod}
\par
Ultraproduct, but on a general filter
\paragraph{filter.filter\_product.of}
\par
Equivalence class containing the constant sequence of a term in β
\paragraph{filter.filter\_product.lift}
\par
Lift function to filter product
\paragraph{filter.filter\_product.lift₂}
\par
Lift binary operation to filter product
\paragraph{filter.filter\_product.lift\_rel}
\par
Lift properties to filter product
\paragraph{filter.filter\_product.lift\_rel₂}
\par
Lift binary relations to filter product
\section{order/filter/lift.lean}\paragraph{filter.lift}
\par
A variant on 
\colorbox[RGB]{253,246,227}{{{{\color[RGB]{101, 123, 131} bind }}}} using a function 
\colorbox[RGB]{253,246,227}{{{{\color[RGB]{101, 123, 131} g }}}} taking a set instead of a member of 
\colorbox[RGB]{253,246,227}{{{{\color[RGB]{101, 123, 131} α }}}}.
This is essentially a push-forward along a function mapping each set to a filter.
\paragraph{filter.lift'}
\par
Specialize 
\colorbox[RGB]{253,246,227}{{{{\color[RGB]{101, 123, 131} lift }}}} to functions 
\colorbox[RGB]{253,246,227}{{{{\color[RGB]{101, 123, 131} set α  }}}{{{\color[RGB]{133, 153, 0} → }}}{{{\color[RGB]{101, 123, 131}  set β }}}}. This can be viewed as a generalization of 
\colorbox[RGB]{253,246,227}{{{{\color[RGB]{101, 123, 131} map }}}}.
This is essentially a push-forward along a function mapping each set to a set.
\section{order/filter/partial.lean}\section{order/fixed\_points.lean}\paragraph{lattice.lfp}
\par
Least fixed point of a monotone function
\paragraph{lattice.gfp}
\par
Greatest fixed point of a monotone function
\section{order/galois\_connection.lean}\paragraph{galois\_connection}
\par
A Galois connection is a pair of functions 
\colorbox[RGB]{253,246,227}{{{{\color[RGB]{101, 123, 131} l }}}} and 
\colorbox[RGB]{253,246,227}{{{{\color[RGB]{101, 123, 131} u }}}} satisfying
\colorbox[RGB]{253,246,227}{{{{\color[RGB]{101, 123, 131} l a  }}}{{{\color[RGB]{181, 137, 0} ≤ }}}{{{\color[RGB]{101, 123, 131}  b  }}}{{{\color[RGB]{181, 137, 0} ↔ }}}{{{\color[RGB]{101, 123, 131}  a  }}}{{{\color[RGB]{181, 137, 0} ≤ }}}{{{\color[RGB]{101, 123, 131}  u b }}}}. They are closely connected to adjoint functors
in category theory.
\paragraph{galois\_insertion}
\par
A Galois insertion is a Galois connection where 
\colorbox[RGB]{253,246,227}{{{{\color[RGB]{101, 123, 131} l ∘ u  }}}{{{\color[RGB]{181, 137, 0} = }}}{{{\color[RGB]{101, 123, 131}  id }}}}. It also contains a constructive
choice function, to give better definitional equalities when lifting order structures.
\paragraph{galois\_connection.lift\_order\_bot}
\par
Lift the bottom along a Galois connection
\paragraph{galois\_insertion.lift\_semilattice\_sup}
\par
Lift the suprema along a Galois insertion
\paragraph{galois\_insertion.lift\_semilattice\_inf}
\par
Lift the infima along a Galois insertion
\paragraph{galois\_insertion.lift\_lattice}
\par
Lift the suprema and infima along a Galois insertion
\paragraph{galois\_insertion.lift\_order\_top}
\par
Lift the top along a Galois insertion
\paragraph{galois\_insertion.lift\_bounded\_lattice}
\par
Lift the top, bottom, suprema, and infima along a Galois insertion
\paragraph{galois\_insertion.lift\_complete\_lattice}
\par
Lift all suprema and infima along a Galois insertion
\section{order/lattice.lean}\paragraph{lattice.has\_sup}
\par
Typeclass for the 
\colorbox[RGB]{253,246,227}{{{{\color[RGB]{101, 123, 131} ⊔ }}}} (
\colorbox[RGB]{253,246,227}{{{{\color[RGB]{101, 123, 131} \textbackslash{}lub }}}}) notation
\paragraph{lattice.has\_inf}
\par
Typeclass for the 
\colorbox[RGB]{253,246,227}{{{{\color[RGB]{101, 123, 131} ⊓ }}}} (
\colorbox[RGB]{253,246,227}{{{{\color[RGB]{101, 123, 131} \textbackslash{}glb }}}}) notation
\paragraph{lattice.semilattice\_sup}
\par
A 
\colorbox[RGB]{253,246,227}{{{{\color[RGB]{101, 123, 131} semilattice\_sup }}}} is a join-semilattice, that is, a partial order
with a join (a.k.a. lub / least upper bound, sup / supremum) operation
\colorbox[RGB]{253,246,227}{{{{\color[RGB]{101, 123, 131} ⊔ }}}} which is the least element larger than both factors.
\paragraph{lattice.semilattice\_inf}
\par
A 
\colorbox[RGB]{253,246,227}{{{{\color[RGB]{101, 123, 131} semilattice\_inf }}}} is a meet-semilattice, that is, a partial order
with a meet (a.k.a. glb / greatest lower bound, inf / infimum) operation
\colorbox[RGB]{253,246,227}{{{{\color[RGB]{101, 123, 131} ⊓ }}}} which is the greatest element smaller than both factors.
\paragraph{lattice.lattice}
\par
A lattice is a join-semilattice which is also a meet-semilattice.
\paragraph{lattice.distrib\_lattice}
\par
A distributive lattice is a lattice that satisfies any of four
equivalent distribution properties (of sup over inf or inf over sup,
on the left or right). A classic example of a distributive lattice
is the lattice of subsets of a set, and in fact this example is
generic in the sense that every distributive lattice is realizable
as a sublattice of a powerset lattice.
\section{order/lexicographic.lean}\paragraph{lex\_has\_le}
\par
Dictionary / lexicographic ordering on pairs.
\paragraph{lex\_preorder}
\par
Dictionary / lexicographic preorder for pairs.
\paragraph{lex\_partial\_order}
\par
Dictionary / lexicographic partial\_order for pairs.
\paragraph{lex\_linear\_order}
\par
Dictionary / lexicographic linear\_order for pairs.
\paragraph{lex\_decidable\_linear\_order}
\par
Dictionary / lexicographic decidable\_linear\_order for pairs.
\paragraph{dlex\_has\_le}
\par
Dictionary / lexicographic ordering on dependent pairs.
\par
The 'pointwise' partial order 
\colorbox[RGB]{253,246,227}{{{{\color[RGB]{101, 123, 131} prod.has\_le }}}} doesn't make
sense for dependent pairs, so it's safe to mark these as
instances here.
\paragraph{dlex\_preorder}
\par
Dictionary / lexicographic preorder on dependent pairs.
\paragraph{dlex\_partial\_order}
\par
Dictionary / lexicographic partial\_order for dependent pairs.
\paragraph{dlex\_linear\_order}
\par
Dictionary / lexicographic linear\_order for pairs.
\paragraph{dlex\_decidable\_linear\_order}
\par
Dictionary / lexicographic decidable\_linear\_order for dependent pairs.
\section{order/liminf\_limsup.lean}\paragraph{filter.is\_bounded}
\par
\colorbox[RGB]{253,246,227}{{{{\color[RGB]{101, 123, 131} f.is\_bounded (≺) }}}}: the filter 
\colorbox[RGB]{253,246,227}{{{{\color[RGB]{101, 123, 131} f }}}} is eventually bounded w.r.t. the relation 
\colorbox[RGB]{253,246,227}{{{{\color[RGB]{101, 123, 131} ≺ }}}}, i.e.
eventually, it is bounded by some uniform bound.
\colorbox[RGB]{253,246,227}{{{{\color[RGB]{101, 123, 131} r }}}} will be usually instantiated with 
\colorbox[RGB]{253,246,227}{{{{\color[RGB]{181, 137, 0} ≤ }}}} or 
\colorbox[RGB]{253,246,227}{{{{\color[RGB]{181, 137, 0} ≥ }}}}.
\paragraph{filter.is\_bounded\_iff}
\par
\colorbox[RGB]{253,246,227}{{{{\color[RGB]{101, 123, 131} f }}}} is eventually bounded if and only if, there exists an admissible set on which it is
bounded.
\paragraph{filter.is\_bounded\_under\_of}
\par
A bounded function 
\colorbox[RGB]{253,246,227}{{{{\color[RGB]{101, 123, 131} u }}}} is in particular eventually bounded.
\paragraph{filter.is\_cobounded}
\par
\colorbox[RGB]{253,246,227}{{{{\color[RGB]{101, 123, 131} is\_cobounded (≺) f }}}} states that filter 
\colorbox[RGB]{253,246,227}{{{{\color[RGB]{101, 123, 131} f }}}} is not tend to infinite w.r.t. 
\colorbox[RGB]{253,246,227}{{{{\color[RGB]{101, 123, 131} ≺ }}}}. This is also
called frequently bounded. Will be usually instantiated with 
\colorbox[RGB]{253,246,227}{{{{\color[RGB]{181, 137, 0} ≤ }}}} or 
\colorbox[RGB]{253,246,227}{{{{\color[RGB]{181, 137, 0} ≥ }}}}.
\par
There is a subtlety in this definition: we want 
\colorbox[RGB]{253,246,227}{{{{\color[RGB]{101, 123, 131} f.is\_cobounded }}}} to hold for any 
\colorbox[RGB]{253,246,227}{{{{\color[RGB]{101, 123, 131} f }}}} in the case of
complete lattices. This will be relevant to deduce theorems on complete lattices from their
versions on conditionally complete lattices with additional assumptions. We have to be careful in
the edge case of the trivial filter containing the empty set: the other natural definition
\colorbox[RGB]{253,246,227}{{{{\color[RGB]{181, 137, 0} ¬ }}}{{{\color[RGB]{101, 123, 131}   }}}{{{\color[RGB]{133, 153, 0} ∀ }}}{{{\color[RGB]{101, 123, 131} a, \{n | a  }}}{{{\color[RGB]{181, 137, 0} ≤ }}}{{{\color[RGB]{101, 123, 131}  n\} ∈ f.sets }}}}would not work as well in this case.
\paragraph{filter.is\_cobounded.mk}
\par
To check that a filter is frequently bounded, it suffices to have a witness
which bounds 
\colorbox[RGB]{253,246,227}{{{{\color[RGB]{101, 123, 131} f }}}} at some point for every admissible set.
\par
This is only an implication, as the other direction is wrong for the trivial filter.
\paragraph{filter.is\_cobounded\_of\_is\_bounded}
\par
A filter which is eventually bounded is in particular frequently bounded (in the opposite
direction). At least if the filter is not trivial.
\paragraph{filter.limsup\_eq\_infi\_supr}
\par
In a complete lattice, the limsup of a function is the infimum over sets 
\colorbox[RGB]{253,246,227}{{{{\color[RGB]{101, 123, 131} s }}}} in the filter
of the supremum of the function over 
\colorbox[RGB]{253,246,227}{{{{\color[RGB]{101, 123, 131} s }}}}\paragraph{filter.liminf\_eq\_supr\_infi}
\par
In a complete lattice, the liminf of a function is the infimum over sets 
\colorbox[RGB]{253,246,227}{{{{\color[RGB]{101, 123, 131} s }}}} in the filter
of the supremum of the function over 
\colorbox[RGB]{253,246,227}{{{{\color[RGB]{101, 123, 131} s }}}}\section{order/order\_iso.lean}\paragraph{injective\_of\_increasing}
\par
An increasing function is injective
\paragraph{subtype.order\_embedding}
\par
the induced order on a subtype is an embedding under the natural inclusion.
\paragraph{order\_embedding.rsymm}
\par
An order embedding is also an order embedding between dual orders.
\paragraph{order\_embedding.preimage}
\par
If 
\colorbox[RGB]{253,246,227}{{{{\color[RGB]{101, 123, 131} f }}}} is injective, then it is an order embedding from the
preimage order of 
\colorbox[RGB]{253,246,227}{{{{\color[RGB]{101, 123, 131} s }}}} to 
\colorbox[RGB]{253,246,227}{{{{\color[RGB]{101, 123, 131} s }}}}.
\paragraph{order\_embedding.of\_monotone}
\par
It suffices to prove 
\colorbox[RGB]{253,246,227}{{{{\color[RGB]{101, 123, 131} f }}}} is monotone between strict orders
to show it is an order embedding.
\paragraph{fin.val.order\_embedding}
\par
The inclusion map 
\colorbox[RGB]{253,246,227}{{{{\color[RGB]{101, 123, 131} fin n  }}}{{{\color[RGB]{133, 153, 0} → }}}{{{\color[RGB]{101, 123, 131}  ℕ }}}} is an order embedding.
\paragraph{fin\_fin.order\_embedding}
\par
The inclusion map 
\colorbox[RGB]{253,246,227}{{{{\color[RGB]{101, 123, 131} fin m  }}}{{{\color[RGB]{133, 153, 0} → }}}{{{\color[RGB]{101, 123, 131}  fin n }}}} is an order embedding.
\paragraph{order\_iso}
\par
An order isomorphism is an equivalence that is also an order embedding.
\paragraph{order\_iso.preimage}
\par
Any equivalence lifts to an order isomorphism between 
\colorbox[RGB]{253,246,227}{{{{\color[RGB]{101, 123, 131} s }}}} and its preimage.
\paragraph{set\_coe\_embedding}
\par
A subset 
\colorbox[RGB]{253,246,227}{{{{\color[RGB]{101, 123, 131} p : set α }}}} embeds into 
\colorbox[RGB]{253,246,227}{{{{\color[RGB]{101, 123, 131} α }}}}\paragraph{subrel}
\par
\colorbox[RGB]{253,246,227}{{{{\color[RGB]{101, 123, 131} subrel r p }}}} is the inherited relation on a subset.
\paragraph{order\_embedding.cod\_restrict}
\par
Restrict the codomain of an order embedding
\section{order/zorn.lean}\paragraph{zorn.chain}
\par
A chain is a subset 
\colorbox[RGB]{253,246,227}{{{{\color[RGB]{101, 123, 131} c }}}} satisfying
\colorbox[RGB]{253,246,227}{{{{\color[RGB]{101, 123, 131} x ≺ y  }}}{{{\color[RGB]{181, 137, 0} ∨ }}}{{{\color[RGB]{101, 123, 131}  x  }}}{{{\color[RGB]{181, 137, 0} = }}}{{{\color[RGB]{101, 123, 131}  y  }}}{{{\color[RGB]{181, 137, 0} ∨ }}}{{{\color[RGB]{101, 123, 131}  y ≺ x }}}} for all 
\colorbox[RGB]{253,246,227}{{{{\color[RGB]{101, 123, 131} x y ∈ c }}}}.
\paragraph{zorn.max\_chain\_spec}
\par
Hausdorff's maximality principle
\par
There exists a maximal totally ordered subset of 
\colorbox[RGB]{253,246,227}{{{{\color[RGB]{101, 123, 131} α }}}}.
Note that we do not require 
\colorbox[RGB]{253,246,227}{{{{\color[RGB]{101, 123, 131} α }}}} to be partially ordered by 
\colorbox[RGB]{253,246,227}{{{{\color[RGB]{101, 123, 131} r }}}}.
\paragraph{zorn.zorn}
\par
Zorn's lemma
\par
If every chain has an upper bound, then there is a maximal element
\section{pending/default.lean}\section{ring\_theory/adjoin.lean}\section{ring\_theory/adjoin\_root.lean}\section{ring\_theory/algebra.lean}\paragraph{algebra}
\par
The category of R-algebras where R is a commutative
ring is the under category R ↓ CRing. In the categorical
setting we have a forgetful functor R-Alg ⥤ R-Mod.
However here it extends module in order to preserve
definitional equality in certain cases.
\paragraph{algebra.module}
\par
The codomain of an algebra.
\paragraph{algebra.of\_ring\_hom}
\par
Creating an algebra from a morphism in CRing.
\paragraph{algebra.polynomial}
\par
R
{[}
X
{]}
 is the generator of the category R-Alg.
\paragraph{algebra.mv\_polynomial}
\par
The algebra of multivariate polynomials.
\paragraph{algebra.of\_subring}
\par
Creating an algebra from a subring. This is the dual of ring extension.
\paragraph{algebra.lmul}
\par
The multiplication in an algebra is a bilinear map.
\paragraph{alg\_hom}
\par
Defining the homomorphism in the category R-Alg.
\paragraph{alg\_hom.to\_linear\_map}
\par
R-Alg ⥤ R-Mod
\paragraph{algebra.comap.algebra}
\par
R ⟶ S induces S-Alg ⥤ R-Alg
\paragraph{alg\_hom.comap}
\par
R ⟶ S induces S-Alg ⥤ R-Alg
\paragraph{polynomial.aeval}
\par
A → Hom
\href{R[X],A}{}R-Alg
\paragraph{mv\_polynomial.aeval}
\par
(ι → A) → Hom
\href{R[ι],A}{}R-Alg
\paragraph{alg\_hom\_int}
\par
CRing ⥤ ℤ-Alg
\paragraph{algebra\_int}
\par
CRing ⥤ ℤ-Alg
\paragraph{subalgebra\_of\_subring}
\par
CRing ⥤ ℤ-Alg
\section{ring\_theory/algebra\_operations.lean}\section{ring\_theory/euclidean\_domain.lean}\section{ring\_theory/free\_comm\_ring.lean}\section{ring\_theory/free\_ring.lean}\section{ring\_theory/ideal\_operations.lean}\paragraph{ideal.quotient\_inf\_ring\_equiv\_pi\_quotient}
\par
Chinese Remainder Theorem. Eisenbud Ex.2.6. Similar to Atiyah-Macdonald 1.10 and Stacks 00DT
\paragraph{ideal.order\_iso\_of\_surjective}
\par
Correspondence theorem
\section{ring\_theory/ideals.lean}\paragraph{ideal.quotient.field}
\par
quotient by maximal ideal is a field. def rather than instance, since users will have
computable inverses in some applications
\section{ring\_theory/localization.lean}\paragraph{localization}
\par
The localization of a ring at a submonoid:
the elements of the submonoid become invertible in the localization.
\paragraph{localization.of}
\par
The natural map from the ring to the localization.
\paragraph{localization.to\_units}
\par
The natural map from the submonoid to the unit group of the localization.
\paragraph{localization.funext}
\par
Function extensionality for localisations:
two functions are equal if they agree on elements that are coercions.
\paragraph{localization.fraction\_ring}
\par
The field of fractions of an integral domain.
\section{ring\_theory/multiplicity.lean}\paragraph{multiplicity}
\par
\colorbox[RGB]{253,246,227}{{{{\color[RGB]{101, 123, 131} multiplicity a b }}}} returns the largest natural number 
\colorbox[RGB]{253,246,227}{{{{\color[RGB]{101, 123, 131} n }}}} such that
\colorbox[RGB]{253,246,227}{{{{\color[RGB]{101, 123, 131} a \textasciicircum{} n ∣ b }}}}, as an 
\colorbox[RGB]{253,246,227}{{{{\color[RGB]{101, 123, 131} enat }}}} or natural with infinity. If 
\colorbox[RGB]{253,246,227}{{{{\color[RGB]{101, 123, 131} ∀ n, a \textasciicircum{} n ∣ b }}}},
then it returns 
\colorbox[RGB]{253,246,227}{{{{\color[RGB]{101, 123, 131} ⊤ }}}}\section{ring\_theory/noetherian.lean}\paragraph{submodule.exists\_sub\_one\_mem\_and\_smul\_eq\_zero\_of\_fg\_of\_le\_smul}
\par
Nakayama's Lemma. Atiyah-Macdonald 2.5, Eisenbud 4.7, Matsumura 2.2, Stacks 00DV
\paragraph{submodule.fg\_of\_fg\_map\_of\_fg\_inf\_ker}
\par
If 0 → M' → M → M'' → 0 is exact and M' and M'' are
finitely generated then so is M.
\section{ring\_theory/polynomial.lean}\paragraph{polynomial.degree\_le}
\par
The 
\colorbox[RGB]{253,246,227}{{{{\color[RGB]{101, 123, 131} R }}}}-submodule of 
\colorbox[RGB]{253,246,227}{{{{\color[RGB]{101, 123, 131} R{[}X{]} }}}} consisting of polynomials of degree ≤ 
\colorbox[RGB]{253,246,227}{{{{\color[RGB]{101, 123, 131} n }}}}.
\paragraph{polynomial.restriction}
\par
Given a polynomial, return the polynomial whose coefficients are in
the ring closure of the original coefficients.
\paragraph{polynomial.to\_subring}
\par
Given a polynomial 
\colorbox[RGB]{253,246,227}{{{{\color[RGB]{101, 123, 131} p }}}} and a subring 
\colorbox[RGB]{253,246,227}{{{{\color[RGB]{101, 123, 131} T }}}} that contains the coefficients of 
\colorbox[RGB]{253,246,227}{{{{\color[RGB]{101, 123, 131} p }}}},
return the corresponding polynomial whose coefficients are in 
`
T.
\paragraph{polynomial.of\_subring}
\par
Given a polynomial whose coefficients are in some subring, return
the corresponding polynomial whose coefificents are in the ambient ring.
\paragraph{ideal.of\_polynomial}
\par
Transport an ideal of 
\colorbox[RGB]{253,246,227}{{{{\color[RGB]{101, 123, 131} R{[}X{]} }}}} to an 
\colorbox[RGB]{253,246,227}{{{{\color[RGB]{101, 123, 131} R }}}}-submodule of 
\colorbox[RGB]{253,246,227}{{{{\color[RGB]{101, 123, 131} R{[}X{]} }}}}.
\paragraph{ideal.degree\_le}
\par
Given an ideal 
\colorbox[RGB]{253,246,227}{{{{\color[RGB]{101, 123, 131} I }}}} of 
\colorbox[RGB]{253,246,227}{{{{\color[RGB]{101, 123, 131} R{[}X{]} }}}}, make the 
\colorbox[RGB]{253,246,227}{{{{\color[RGB]{101, 123, 131} R }}}}-submodule of 
\colorbox[RGB]{253,246,227}{{{{\color[RGB]{101, 123, 131} I }}}}consisting of polynomials of degree ≤ 
\colorbox[RGB]{253,246,227}{{{{\color[RGB]{101, 123, 131} n }}}}.
\paragraph{ideal.leading\_coeff\_nth}
\par
Given an ideal 
\colorbox[RGB]{253,246,227}{{{{\color[RGB]{101, 123, 131} I }}}} of 
\colorbox[RGB]{253,246,227}{{{{\color[RGB]{101, 123, 131} R{[}X{]} }}}}, make the ideal in 
\colorbox[RGB]{253,246,227}{{{{\color[RGB]{101, 123, 131} R }}}} of
leading coefficients of polynomials in 
\colorbox[RGB]{253,246,227}{{{{\color[RGB]{101, 123, 131} I }}}} with degree ≤ 
\colorbox[RGB]{253,246,227}{{{{\color[RGB]{101, 123, 131} n }}}}.
\paragraph{ideal.leading\_coeff}
\par
Given an ideal 
\colorbox[RGB]{253,246,227}{{{{\color[RGB]{101, 123, 131} I }}}} in 
\colorbox[RGB]{253,246,227}{{{{\color[RGB]{101, 123, 131} R{[}X{]} }}}}, make the ideal in 
\colorbox[RGB]{253,246,227}{{{{\color[RGB]{101, 123, 131} R }}}} of the
leading coefficients in 
\colorbox[RGB]{253,246,227}{{{{\color[RGB]{101, 123, 131} I }}}}.
\paragraph{is\_noetherian\_ring\_polynomial}
\par
Hilbert basis theorem.
\section{ring\_theory/principal\_ideal\_domain.lean}\paragraph{principal\_ideal\_domain.to\_unique\_factorization\_domain}
\par
The unique factorization domain structure given by the principal ideal domain.
\par
This is not added as type class instance, since the 
\colorbox[RGB]{253,246,227}{{{{\color[RGB]{101, 123, 131} factors }}}} might be computed in a different way.
E.g. factors could return normalized values.
\section{ring\_theory/subring.lean}\paragraph{is\_subring}
\par
\colorbox[RGB]{253,246,227}{{{{\color[RGB]{101, 123, 131} S }}}} is a subring: a set containing 1 and closed under multiplication, addition and and additive inverse.
\section{ring\_theory/unique\_factorization\_domain.lean}\paragraph{unique\_factorization\_domain}
\par
Unique factorization domains.
\par
In a unique factorization domain each element (except zero) is uniquely
represented as a multiset of irreducible factors.
Uniqueness is only up to associated elements.
\par
This is equivalent to defining a unique factorization domain as a domain in
which each element (except zero) is non-uniquely represented as a multiset
of prime factors. This definition is used.
\par
To define a UFD using the traditional definition in terms of multisets
of irreducible factors, use the definition 
\colorbox[RGB]{253,246,227}{{{{\color[RGB]{101, 123, 131} of\_unique\_irreducible\_factorization }}}}\paragraph{associates.factor\_set}
\par
\colorbox[RGB]{253,246,227}{{{{\color[RGB]{101, 123, 131} factor\_set α }}}} representation elements of unique factorization domain as multisets.
\par
\colorbox[RGB]{253,246,227}{{{{\color[RGB]{101, 123, 131} multiset α }}}} produced by 
\colorbox[RGB]{253,246,227}{{{{\color[RGB]{101, 123, 131} factors }}}} are only unique up to associated elements, while the multisets in
\colorbox[RGB]{253,246,227}{{{{\color[RGB]{101, 123, 131} factor\_set α }}}} are unqiue by equality and restricted to irreducible elements. This gives us a
representation of each element as a unique multisets (or the added ⊤ for 0), which has a complete
lattice struture. Infimum is the greatest common divisor and supremum is the least common multiple.
\paragraph{unique\_factorization\_domain.to\_gcd\_domain}
\par
\colorbox[RGB]{253,246,227}{{{{\color[RGB]{101, 123, 131} to\_gcd\_domain }}}} constructs a GCD domain out of a unique factorization domain over a normalization
domain.
\section{set\_theory/cardinal.lean}\paragraph{cardinal}
\par
\colorbox[RGB]{253,246,227}{{{{\color[RGB]{101, 123, 131} cardinal.\{u\} }}}} is the type of cardinal numbers in 
\colorbox[RGB]{253,246,227}{{{{\color[RGB]{38, 139, 210} Type }}}{{{\color[RGB]{101, 123, 131}  u }}}},
defined as the quotient of 
\colorbox[RGB]{253,246,227}{{{{\color[RGB]{38, 139, 210} Type }}}{{{\color[RGB]{101, 123, 131}  u }}}} by existence of an equivalence
(a bijection with explicit inverse).
\paragraph{cardinal.mk}
\par
The cardinal of a type
\paragraph{cardinal.power}
\par
The cardinal exponential. 
\colorbox[RGB]{253,246,227}{{{{\color[RGB]{101, 123, 131} mk α \textasciicircum{} mk β }}}} is the cardinal of 
\colorbox[RGB]{253,246,227}{{{{\color[RGB]{101, 123, 131} β  }}}{{{\color[RGB]{133, 153, 0} → }}}{{{\color[RGB]{101, 123, 131}  α }}}}.
\paragraph{cardinal.min}
\par
The minimum cardinal in a family of cardinals (the existence
of which is provided by 
\colorbox[RGB]{253,246,227}{{{{\color[RGB]{101, 123, 131} injective\_min }}}}).
\paragraph{cardinal.succ}
\par
The successor cardinal - the smallest cardinal greater than
\colorbox[RGB]{253,246,227}{{{{\color[RGB]{101, 123, 131} c }}}}. This is not the same as 
\colorbox[RGB]{253,246,227}{{{{\color[RGB]{101, 123, 131} c  }}}{{{\color[RGB]{181, 137, 0} + }}}{{{\color[RGB]{101, 123, 131}   }}}{{{\color[RGB]{108, 113, 196} 1 }}}} except in the case of finite 
\colorbox[RGB]{253,246,227}{{{{\color[RGB]{101, 123, 131} c }}}}.
\paragraph{cardinal.sum}
\par
The indexed sum of cardinals is the cardinality of the
indexed disjoint union, i.e. sigma type.
\paragraph{cardinal.sup}
\par
The indexed supremum of cardinals is the smallest cardinal above
everything in the family.
\paragraph{cardinal.prod}
\par
The indexed product of cardinals is the cardinality of the Pi type
(dependent product).
\paragraph{cardinal.lift}
\par
The universe lift operation on cardinals
\paragraph{cardinal.omega}
\par
\colorbox[RGB]{253,246,227}{{{{\color[RGB]{101, 123, 131} ω }}}} is the smallest infinite cardinal, also known as ℵ₀.
\paragraph{cardinal.sum\_lt\_prod}
\par
König's theorem
\paragraph{cardinal.powerlt}
\par
The function α\textasciicircum{}\{
<
β\}, defined to be sup
\_
\{γ 
<
 β\} α\textasciicircum{}γ.
We index over \{s : set β.out // mk s 
<
 β \} instead of \{γ // γ 
<
 β\}, because the latter lives in a
higher universe
\section{set\_theory/cofinality.lean}\paragraph{order.cof}
\par
Cofinality of a reflexive order 
\colorbox[RGB]{253,246,227}{{{{\color[RGB]{101, 123, 131} ≼ }}}}. This is the smallest cardinality
of a subset 
\colorbox[RGB]{253,246,227}{{{{\color[RGB]{101, 123, 131} S : set α }}}} such that 
\colorbox[RGB]{253,246,227}{{{{\color[RGB]{101, 123, 131} ∀ a, ∃ b ∈ S, a ≼ b }}}}.
\paragraph{ordinal.cof}
\par
Cofinality of an ordinal. This is the smallest cardinal of a
subset 
\colorbox[RGB]{253,246,227}{{{{\color[RGB]{101, 123, 131} S }}}} of the ordinal which is unbounded, in the sense
\colorbox[RGB]{253,246,227}{{{{\color[RGB]{101, 123, 131} ∀ a, ∃ b ∈ S,  }}}{{{\color[RGB]{181, 137, 0} ¬ }}}{{{\color[RGB]{101, 123, 131} (b  }}}{{{\color[RGB]{181, 137, 0} > }}}{{{\color[RGB]{101, 123, 131}  a) }}}}. It is defined for all ordinals, but
\colorbox[RGB]{253,246,227}{{{{\color[RGB]{101, 123, 131} cof  }}}{{{\color[RGB]{108, 113, 196} 0 }}}{{{\color[RGB]{101, 123, 131}   }}}{{{\color[RGB]{181, 137, 0} = }}}{{{\color[RGB]{101, 123, 131}   }}}{{{\color[RGB]{108, 113, 196} 0 }}}} and 
\colorbox[RGB]{253,246,227}{{{{\color[RGB]{101, 123, 131} cof (succ o)  }}}{{{\color[RGB]{181, 137, 0} = }}}{{{\color[RGB]{101, 123, 131}   }}}{{{\color[RGB]{108, 113, 196} 1 }}}}, so it is only really
interesting on limit ordinals (when it is an infinite cardinal).
\paragraph{ordinal.unbounded\_of\_unbounded\_sUnion}
\par
If the union of s is unbounded and s is smaller than the cofinality, then s has an unbounded member
\paragraph{ordinal.unbounded\_of\_unbounded\_Union}
\par
If the union of s is unbounded and s is smaller than the cofinality, then s has an unbounded member
\paragraph{ordinal.infinite\_pigeonhole}
\par
The infinite pigeonhole principle
\paragraph{ordinal.infinite\_pigeonhole\_card}
\par
pigeonhole principle for a cardinality below the cardinality of the domain
\paragraph{cardinal.is\_limit}
\par
A cardinal is a limit if it is not zero or a successor
cardinal. Note that 
\colorbox[RGB]{253,246,227}{{{{\color[RGB]{101, 123, 131} ω }}}} is a limit cardinal by this definition.
\paragraph{cardinal.is\_strong\_limit}
\par
A cardinal is a strong limit if it is not zero and it is
closed under powersets. Note that 
\colorbox[RGB]{253,246,227}{{{{\color[RGB]{101, 123, 131} ω }}}} is a strong limit by this definition.
\paragraph{cardinal.is\_regular}
\par
A cardinal is regular if it is infinite and it equals its own cofinality.
\paragraph{cardinal.is\_inaccessible}
\par
A cardinal is inaccessible if it is an
uncountable regular strong limit cardinal.
\section{set\_theory/lists.lean}\section{set\_theory/ordinal.lean}\paragraph{initial\_seg.of\_iso}
\par
An order isomorphism is an initial segment
\paragraph{initial\_seg.cod\_restrict}
\par
Restrict the codomain of an initial segment
\paragraph{principal\_seg.of\_element}
\par
Any element of a well order yields a principal segment
\paragraph{principal\_seg.cod\_restrict}
\par
Restrict the codomain of a principal segment
\paragraph{order\_embedding.collapse}
\par
Construct an initial segment from an order embedding.
\paragraph{ordinal}
\par
\colorbox[RGB]{253,246,227}{{{{\color[RGB]{101, 123, 131} ordinal.\{u\} }}}} is the type of well orders in 
\colorbox[RGB]{253,246,227}{{{{\color[RGB]{38, 139, 210} Type }}}{{{\color[RGB]{101, 123, 131}  u }}}},
quotient by order isomorphism.
\paragraph{ordinal.type}
\par
The order type of a well order is an ordinal.
\paragraph{ordinal.typein}
\par
The order type of an element inside a well order.
\paragraph{ordinal.le}
\par
Ordinal less-equal is defined such that
well orders 
\colorbox[RGB]{253,246,227}{{{{\color[RGB]{101, 123, 131} r }}}} and 
\colorbox[RGB]{253,246,227}{{{{\color[RGB]{101, 123, 131} s }}}} satisfy 
\colorbox[RGB]{253,246,227}{{{{\color[RGB]{101, 123, 131} type r  }}}{{{\color[RGB]{181, 137, 0} ≤ }}}{{{\color[RGB]{101, 123, 131}  type s }}}} if there exists
a function embedding 
\colorbox[RGB]{253,246,227}{{{{\color[RGB]{101, 123, 131} r }}}} as an initial segment of 
\colorbox[RGB]{253,246,227}{{{{\color[RGB]{101, 123, 131} s }}}}.
\paragraph{ordinal.lt}
\par
Ordinal less-than is defined such that
well orders 
\colorbox[RGB]{253,246,227}{{{{\color[RGB]{101, 123, 131} r }}}} and 
\colorbox[RGB]{253,246,227}{{{{\color[RGB]{101, 123, 131} s }}}} satisfy 
\colorbox[RGB]{253,246,227}{{{{\color[RGB]{101, 123, 131} type r  }}}{{{\color[RGB]{181, 137, 0} < }}}{{{\color[RGB]{101, 123, 131}  type s }}}} if there exists
a function embedding 
\colorbox[RGB]{253,246,227}{{{{\color[RGB]{101, 123, 131} r }}}} as a principal segment of 
\colorbox[RGB]{253,246,227}{{{{\color[RGB]{101, 123, 131} s }}}}.
\paragraph{ordinal.enum}
\par
\colorbox[RGB]{253,246,227}{{{{\color[RGB]{101, 123, 131} enum r o h }}}} is the 
\colorbox[RGB]{253,246,227}{{{{\color[RGB]{101, 123, 131} o }}}}-th element of 
\colorbox[RGB]{253,246,227}{{{{\color[RGB]{101, 123, 131} α }}}} ordered by 
\colorbox[RGB]{253,246,227}{{{{\color[RGB]{101, 123, 131} r }}}}.
That is, 
\colorbox[RGB]{253,246,227}{{{{\color[RGB]{101, 123, 131} enum }}}} maps an initial segment of the ordinals, those
less than the order type of 
\colorbox[RGB]{253,246,227}{{{{\color[RGB]{101, 123, 131} r }}}}, to the elements of 
\colorbox[RGB]{253,246,227}{{{{\color[RGB]{101, 123, 131} α }}}}.
\paragraph{ordinal.card}
\par
The cardinal of an ordinal is the cardinal of any
set with that order type.
\paragraph{ordinal.succ}
\par
The ordinal successor is the smallest ordinal larger than 
\colorbox[RGB]{253,246,227}{{{{\color[RGB]{101, 123, 131} o }}}}.
It is defined as 
\colorbox[RGB]{253,246,227}{{{{\color[RGB]{101, 123, 131} o  }}}{{{\color[RGB]{181, 137, 0} + }}}{{{\color[RGB]{101, 123, 131}   }}}{{{\color[RGB]{108, 113, 196} 1 }}}}.
\paragraph{ordinal.lift}
\par
The universe lift operation for ordinals, which embeds 
\colorbox[RGB]{253,246,227}{{{{\color[RGB]{101, 123, 131} ordinal.\{u\} }}}} as
a proper initial segment of 
\colorbox[RGB]{253,246,227}{{{{\color[RGB]{101, 123, 131} ordinal.\{v\} }}}} for 
\colorbox[RGB]{253,246,227}{{{{\color[RGB]{101, 123, 131} v  }}}{{{\color[RGB]{181, 137, 0} > }}}{{{\color[RGB]{101, 123, 131}  u }}}}.
\paragraph{ordinal.omega}
\par
\colorbox[RGB]{253,246,227}{{{{\color[RGB]{101, 123, 131} ω }}}} is the first infinite ordinal, defined as the order type of 
\colorbox[RGB]{253,246,227}{{{{\color[RGB]{101, 123, 131} ℕ }}}}.
\paragraph{ordinal.pred}
\par
The ordinal predecessor of 
\colorbox[RGB]{253,246,227}{{{{\color[RGB]{101, 123, 131} o }}}} is 
\colorbox[RGB]{253,246,227}{{{{\color[RGB]{101, 123, 131} o' }}}} if 
\colorbox[RGB]{253,246,227}{{{{\color[RGB]{101, 123, 131} o  }}}{{{\color[RGB]{181, 137, 0} = }}}{{{\color[RGB]{101, 123, 131}  succ o' }}}},
and 
\colorbox[RGB]{253,246,227}{{{{\color[RGB]{101, 123, 131} o }}}} otherwise.
\paragraph{ordinal.is\_limit}
\par
A limit ordinal is an ordinal which is not zero and not a successor.
\paragraph{ordinal.is\_normal}
\par
A normal ordinal function is a strictly increasing function which is
order-continuous.
\paragraph{ordinal.min}
\par
The minimal element of a nonempty family of ordinals
\paragraph{ordinal.omin}
\par
The minimal element of a nonempty set of ordinals
\paragraph{ordinal.univ}
\par
\colorbox[RGB]{253,246,227}{{{{\color[RGB]{101, 123, 131} univ.\{u v\} }}}} is the order type of the ordinals of 
\colorbox[RGB]{253,246,227}{{{{\color[RGB]{38, 139, 210} Type }}}{{{\color[RGB]{101, 123, 131}  u }}}} as a member
of 
\colorbox[RGB]{253,246,227}{{{{\color[RGB]{101, 123, 131} ordinal.\{v\} }}}} (when 
\colorbox[RGB]{253,246,227}{{{{\color[RGB]{101, 123, 131} u  }}}{{{\color[RGB]{181, 137, 0} < }}}{{{\color[RGB]{101, 123, 131}  v }}}}). It is an inaccessible cardinal.
\paragraph{ordinal.sub}
\par
\colorbox[RGB]{253,246,227}{{{{\color[RGB]{101, 123, 131} a  }}}{{{\color[RGB]{181, 137, 0} - }}}{{{\color[RGB]{101, 123, 131}  b }}}} is the unique ordinal satisfying
\colorbox[RGB]{253,246,227}{{{{\color[RGB]{101, 123, 131} b  }}}{{{\color[RGB]{181, 137, 0} + }}}{{{\color[RGB]{101, 123, 131}  (a  }}}{{{\color[RGB]{181, 137, 0} - }}}{{{\color[RGB]{101, 123, 131}  b)  }}}{{{\color[RGB]{181, 137, 0} = }}}{{{\color[RGB]{101, 123, 131}  a }}}} when 
\colorbox[RGB]{253,246,227}{{{{\color[RGB]{101, 123, 131} b  }}}{{{\color[RGB]{181, 137, 0} ≤ }}}{{{\color[RGB]{101, 123, 131}  a }}}}.
\paragraph{ordinal.div}
\par
\colorbox[RGB]{253,246,227}{{{{\color[RGB]{101, 123, 131} a  }}}{{{\color[RGB]{181, 137, 0} / }}}{{{\color[RGB]{101, 123, 131}  b }}}} is the unique ordinal 
\colorbox[RGB]{253,246,227}{{{{\color[RGB]{101, 123, 131} o }}}} satisfying
\colorbox[RGB]{253,246,227}{{{{\color[RGB]{101, 123, 131} a  }}}{{{\color[RGB]{181, 137, 0} = }}}{{{\color[RGB]{101, 123, 131}  b  }}}{{{\color[RGB]{181, 137, 0} * }}}{{{\color[RGB]{101, 123, 131}  o  }}}{{{\color[RGB]{181, 137, 0} + }}}{{{\color[RGB]{101, 123, 131}  o' }}}} with 
\colorbox[RGB]{253,246,227}{{{{\color[RGB]{101, 123, 131} o'  }}}{{{\color[RGB]{181, 137, 0} < }}}{{{\color[RGB]{101, 123, 131}  b }}}}.
\paragraph{ordinal.has\_dvd}
\par
Divisibility is defined by right multiplication:
\colorbox[RGB]{253,246,227}{{{{\color[RGB]{101, 123, 131} a ∣ b }}}} if there exists 
\colorbox[RGB]{253,246,227}{{{{\color[RGB]{101, 123, 131} c }}}} such that 
\colorbox[RGB]{253,246,227}{{{{\color[RGB]{101, 123, 131} b  }}}{{{\color[RGB]{181, 137, 0} = }}}{{{\color[RGB]{101, 123, 131}  a  }}}{{{\color[RGB]{181, 137, 0} * }}}{{{\color[RGB]{101, 123, 131}  c }}}}.
\paragraph{ordinal.has\_mod}
\par
\colorbox[RGB]{253,246,227}{{{{\color[RGB]{101, 123, 131} a \% b }}}} is the unique ordinal 
\colorbox[RGB]{253,246,227}{{{{\color[RGB]{101, 123, 131} o' }}}} satisfying
\colorbox[RGB]{253,246,227}{{{{\color[RGB]{101, 123, 131} a  }}}{{{\color[RGB]{181, 137, 0} = }}}{{{\color[RGB]{101, 123, 131}  b  }}}{{{\color[RGB]{181, 137, 0} * }}}{{{\color[RGB]{101, 123, 131}  o  }}}{{{\color[RGB]{181, 137, 0} + }}}{{{\color[RGB]{101, 123, 131}  o' }}}} with 
\colorbox[RGB]{253,246,227}{{{{\color[RGB]{101, 123, 131} o'  }}}{{{\color[RGB]{181, 137, 0} < }}}{{{\color[RGB]{101, 123, 131}  b }}}}.
\paragraph{cardinal.ord}
\par
The ordinal corresponding to a cardinal 
\colorbox[RGB]{253,246,227}{{{{\color[RGB]{101, 123, 131} c }}}} is the least ordinal
whose cardinal is 
\colorbox[RGB]{253,246,227}{{{{\color[RGB]{101, 123, 131} c }}}}.
\paragraph{cardinal.univ}
\par
The cardinal 
\colorbox[RGB]{253,246,227}{{{{\color[RGB]{101, 123, 131} univ }}}} is the cardinality of ordinal 
\colorbox[RGB]{253,246,227}{{{{\color[RGB]{101, 123, 131} univ }}}}, or
equivalently the cardinal of 
\colorbox[RGB]{253,246,227}{{{{\color[RGB]{101, 123, 131} ordinal.\{u\} }}}}, or 
\colorbox[RGB]{253,246,227}{{{{\color[RGB]{101, 123, 131} cardinal.\{u\} }}}},
as an element of 
\colorbox[RGB]{253,246,227}{{{{\color[RGB]{101, 123, 131} cardinal.\{v\} }}}} (when 
\colorbox[RGB]{253,246,227}{{{{\color[RGB]{101, 123, 131} u  }}}{{{\color[RGB]{181, 137, 0} < }}}{{{\color[RGB]{101, 123, 131}  v }}}}).
\paragraph{ordinal.sup}
\par
The supremum of a family of ordinals
\paragraph{ordinal.bsup}
\par
The supremum of a family of ordinals indexed by the set
of ordinals less than some 
\colorbox[RGB]{253,246,227}{{{{\color[RGB]{101, 123, 131} o : ordinal.\{u\} }}}}.
(This is not a special case of 
\colorbox[RGB]{253,246,227}{{{{\color[RGB]{101, 123, 131} sup }}}} over the subtype,
because 
\colorbox[RGB]{253,246,227}{{{{\color[RGB]{101, 123, 131} \{a  }}}{{{\color[RGB]{181, 137, 0} / }}}{{{\color[RGB]{181, 137, 0} / }}}{{{\color[RGB]{101, 123, 131}  a  }}}{{{\color[RGB]{181, 137, 0} < }}}{{{\color[RGB]{101, 123, 131}  o\} :  }}}{{{\color[RGB]{38, 139, 210} Type }}}{{{\color[RGB]{101, 123, 131}  (u }}}{{{\color[RGB]{181, 137, 0} + }}}{{{\color[RGB]{108, 113, 196} 1 }}}{{{\color[RGB]{101, 123, 131} ) }}}} and 
\colorbox[RGB]{253,246,227}{{{{\color[RGB]{101, 123, 131} sup }}}} only works over
families in 
\colorbox[RGB]{253,246,227}{{{{\color[RGB]{38, 139, 210} Type }}}{{{\color[RGB]{101, 123, 131}  u }}}}.)
\paragraph{ordinal.power}
\par
The ordinal exponential, defined by transfinite recursion.
\paragraph{ordinal.log}
\par
The ordinal logarithm is the solution 
\colorbox[RGB]{253,246,227}{{{{\color[RGB]{101, 123, 131} u }}}} to the equation
\colorbox[RGB]{253,246,227}{{{{\color[RGB]{101, 123, 131} x  }}}{{{\color[RGB]{181, 137, 0} = }}}{{{\color[RGB]{101, 123, 131}  b \textasciicircum{} u  }}}{{{\color[RGB]{181, 137, 0} * }}}{{{\color[RGB]{101, 123, 131}  v  }}}{{{\color[RGB]{181, 137, 0} + }}}{{{\color[RGB]{101, 123, 131}  w }}}} where 
\colorbox[RGB]{253,246,227}{{{{\color[RGB]{101, 123, 131} v  }}}{{{\color[RGB]{181, 137, 0} < }}}{{{\color[RGB]{101, 123, 131}  b }}}} and 
\colorbox[RGB]{253,246,227}{{{{\color[RGB]{101, 123, 131} w  }}}{{{\color[RGB]{181, 137, 0} < }}}{{{\color[RGB]{101, 123, 131}  b }}}}.
\paragraph{ordinal.CNF}
\par
The Cantor normal form of an ordinal is the list of coefficients
in the base-
\colorbox[RGB]{253,246,227}{{{{\color[RGB]{101, 123, 131} b }}}} expansion of 
\colorbox[RGB]{253,246,227}{{{{\color[RGB]{101, 123, 131} o }}}}.
\par
CNF b (b \textasciicircum{} u₁ * v₁ + b \textasciicircum{} u₂ * v₂) = 
{[}
(u₁, v₁), (u₂, v₂)
{]}
\paragraph{ordinal.nfp}
\par
The next fixed point function, the least fixed point of the
normal function 
\colorbox[RGB]{253,246,227}{{{{\color[RGB]{101, 123, 131} f }}}} above 
\colorbox[RGB]{253,246,227}{{{{\color[RGB]{101, 123, 131} a }}}}.
\paragraph{ordinal.deriv}
\par
The derivative of a normal function 
\colorbox[RGB]{253,246,227}{{{{\color[RGB]{101, 123, 131} f }}}} is
the sequence of fixed points of 
\colorbox[RGB]{253,246,227}{{{{\color[RGB]{101, 123, 131} f }}}}.
\paragraph{cardinal.aleph\_idx}
\par
The 
\colorbox[RGB]{253,246,227}{{{{\color[RGB]{101, 123, 131} aleph' }}}} index function, which gives the ordinal index of a cardinal.
(The 
\colorbox[RGB]{253,246,227}{{{{\color[RGB]{101, 123, 131} aleph' }}}} part is because unlike 
\colorbox[RGB]{253,246,227}{{{{\color[RGB]{101, 123, 131} aleph }}}} this counts also the
finite stages. So 
\colorbox[RGB]{253,246,227}{{{{\color[RGB]{101, 123, 131} aleph\_idx n  }}}{{{\color[RGB]{181, 137, 0} = }}}{{{\color[RGB]{101, 123, 131}  n }}}}, 
\colorbox[RGB]{253,246,227}{{{{\color[RGB]{101, 123, 131} aleph\_idx ω  }}}{{{\color[RGB]{181, 137, 0} = }}}{{{\color[RGB]{101, 123, 131}  ω }}}},
\colorbox[RGB]{253,246,227}{{{{\color[RGB]{101, 123, 131} aleph\_idx ℵ₁  }}}{{{\color[RGB]{181, 137, 0} = }}}{{{\color[RGB]{101, 123, 131}  ω  }}}{{{\color[RGB]{181, 137, 0} + }}}{{{\color[RGB]{101, 123, 131}   }}}{{{\color[RGB]{108, 113, 196} 1 }}}} and so on.)
\paragraph{cardinal.aleph'}
\par
The 
\colorbox[RGB]{253,246,227}{{{{\color[RGB]{101, 123, 131} aleph' }}}} function gives the cardinals listed by their ordinal
index, and is the inverse of 
\colorbox[RGB]{253,246,227}{{{{\color[RGB]{101, 123, 131} aleph\_idx }}}}.
\colorbox[RGB]{253,246,227}{{{{\color[RGB]{101, 123, 131} aleph' n  }}}{{{\color[RGB]{181, 137, 0} = }}}{{{\color[RGB]{101, 123, 131}  n }}}}, 
\colorbox[RGB]{253,246,227}{{{{\color[RGB]{101, 123, 131} aleph' ω  }}}{{{\color[RGB]{181, 137, 0} = }}}{{{\color[RGB]{101, 123, 131}  ω }}}}, 
`
aleph' (ω + 1) = ℵ₁, etc.
\paragraph{cardinal.aleph}
\par
The 
\colorbox[RGB]{253,246,227}{{{{\color[RGB]{101, 123, 131} aleph }}}} function gives the infinite cardinals listed by their
ordinal index. 
\colorbox[RGB]{253,246,227}{{{{\color[RGB]{101, 123, 131} aleph  }}}{{{\color[RGB]{108, 113, 196} 0 }}}{{{\color[RGB]{101, 123, 131}   }}}{{{\color[RGB]{181, 137, 0} = }}}{{{\color[RGB]{101, 123, 131}  ω }}}}, 
\colorbox[RGB]{253,246,227}{{{{\color[RGB]{101, 123, 131} aleph  }}}{{{\color[RGB]{108, 113, 196} 1 }}}{{{\color[RGB]{101, 123, 131}   }}}{{{\color[RGB]{181, 137, 0} = }}}{{{\color[RGB]{101, 123, 131}  succ ω }}}} is the first
uncountable cardinal, and so on.
\section{set\_theory/ordinal\_notation.lean}\paragraph{onote}
\par
Recursive definition of an ordinal notation. 
\colorbox[RGB]{253,246,227}{{{{\color[RGB]{101, 123, 131} zero }}}} denotes the
ordinal 0, and 
\colorbox[RGB]{253,246,227}{{{{\color[RGB]{101, 123, 131} oadd e n a }}}} is intended to refer to 
\colorbox[RGB]{253,246,227}{{{{\color[RGB]{101, 123, 131} ω\textasciicircum{}e  }}}{{{\color[RGB]{181, 137, 0} * }}}{{{\color[RGB]{101, 123, 131}  n  }}}{{{\color[RGB]{181, 137, 0} + }}}{{{\color[RGB]{101, 123, 131}  a }}}}.
For this to be valid Cantor normal form, we must have the exponents
decrease to the right, but we can't state this condition until we've
defined 
\colorbox[RGB]{253,246,227}{{{{\color[RGB]{101, 123, 131} repr }}}}, so it is a separate definition 
\colorbox[RGB]{253,246,227}{{{{\color[RGB]{101, 123, 131} NF }}}}.
\paragraph{onote.has\_zero}
\par
Notation for 0
\paragraph{onote.has\_one}
\par
Notation for 1
\paragraph{onote.omega}
\par
Notation for ω
\paragraph{onote.repr}
\par
The ordinal denoted by a notation
\paragraph{onote.to\_string}
\par
Print an ordinal notation
\paragraph{onote.repr'}
\par
Print an ordinal notation
\paragraph{onote.of\_nat}
\par
Convert a 
\colorbox[RGB]{253,246,227}{{{{\color[RGB]{101, 123, 131} nat }}}} into an ordinal
\paragraph{onote.cmp}
\par
Compare ordinal notations
\paragraph{onote.NF\_below}
\par
\colorbox[RGB]{253,246,227}{{{{\color[RGB]{101, 123, 131} NF\_below o b }}}} says that 
\colorbox[RGB]{253,246,227}{{{{\color[RGB]{101, 123, 131} o }}}} is a normal form ordinal notation
satisfying 
\colorbox[RGB]{253,246,227}{{{{\color[RGB]{101, 123, 131} repr o  }}}{{{\color[RGB]{181, 137, 0} < }}}{{{\color[RGB]{101, 123, 131}  ω \textasciicircum{} b }}}}.
\paragraph{onote.NF}
\par
A normal form ordinal notation has the form
\\
\colorbox[RGB]{253,246,227}{\parbox{4.5in}{{{{\color[RGB]{101, 123, 131} ω \textasciicircum{} a₁  }}}{{{\color[RGB]{181, 137, 0} * }}}{{{\color[RGB]{101, 123, 131}  n₁  }}}{{{\color[RGB]{181, 137, 0} + }}}{{{\color[RGB]{101, 123, 131}  ω \textasciicircum{} a₂  }}}{{{\color[RGB]{181, 137, 0} * }}}{{{\color[RGB]{101, 123, 131}  n₂  }}}{{{\color[RGB]{181, 137, 0} + }}}{{{\color[RGB]{101, 123, 131}  ... ω \textasciicircum{} aₖ  }}}{{{\color[RGB]{181, 137, 0} * }}}{{{\color[RGB]{101, 123, 131}  nₖ
 }}}\\

}}\par
where 
\colorbox[RGB]{253,246,227}{{{{\color[RGB]{101, 123, 131} a₁  }}}{{{\color[RGB]{181, 137, 0} > }}}{{{\color[RGB]{101, 123, 131}  a₂  }}}{{{\color[RGB]{181, 137, 0} > }}}{{{\color[RGB]{101, 123, 131}  ...  }}}{{{\color[RGB]{181, 137, 0} > }}}{{{\color[RGB]{101, 123, 131}  aₖ }}}} and all the 
\colorbox[RGB]{253,246,227}{{{{\color[RGB]{101, 123, 131} aᵢ }}}} are
also in normal form.
\par
We will essentially only be interested in normal form
ordinal notations, but to avoid complicating the algorithms
we define everything over general ordinal notations and
only prove correctness with normal form as an invariant.
\paragraph{onote.top\_below}
\par
\colorbox[RGB]{253,246,227}{{{{\color[RGB]{101, 123, 131} top\_below b o }}}} asserts that the largest exponent in 
\colorbox[RGB]{253,246,227}{{{{\color[RGB]{101, 123, 131} o }}}}, if
it exists, is less than 
\colorbox[RGB]{253,246,227}{{{{\color[RGB]{101, 123, 131} b }}}}. This is an auxiliary definition
for decidability of 
\colorbox[RGB]{253,246,227}{{{{\color[RGB]{101, 123, 131} NF }}}}.
\paragraph{onote.add}
\par
Addition of ordinal notations (correct only for normal input)
\paragraph{onote.sub}
\par
Subtraction of ordinal notations (correct only for normal input)
\paragraph{onote.mul}
\par
Multiplication of ordinal notations (correct only for normal input)
\paragraph{onote.split'}
\par
Calculate division and remainder of 
\colorbox[RGB]{253,246,227}{{{{\color[RGB]{101, 123, 131} o }}}} mod ω.
\colorbox[RGB]{253,246,227}{{{{\color[RGB]{101, 123, 131} split' o  }}}{{{\color[RGB]{181, 137, 0} = }}}{{{\color[RGB]{101, 123, 131}  (a, n) }}}} means 
\colorbox[RGB]{253,246,227}{{{{\color[RGB]{101, 123, 131} o  }}}{{{\color[RGB]{181, 137, 0} = }}}{{{\color[RGB]{101, 123, 131}  ω  }}}{{{\color[RGB]{181, 137, 0} * }}}{{{\color[RGB]{101, 123, 131}  a  }}}{{{\color[RGB]{181, 137, 0} + }}}{{{\color[RGB]{101, 123, 131}  n }}}}.
\paragraph{onote.split}
\par
Calculate division and remainder of 
\colorbox[RGB]{253,246,227}{{{{\color[RGB]{101, 123, 131} o }}}} mod ω.
\colorbox[RGB]{253,246,227}{{{{\color[RGB]{101, 123, 131} split o  }}}{{{\color[RGB]{181, 137, 0} = }}}{{{\color[RGB]{101, 123, 131}  (a, n) }}}} means 
\colorbox[RGB]{253,246,227}{{{{\color[RGB]{101, 123, 131} o  }}}{{{\color[RGB]{181, 137, 0} = }}}{{{\color[RGB]{101, 123, 131}  a  }}}{{{\color[RGB]{181, 137, 0} + }}}{{{\color[RGB]{101, 123, 131}  n }}}}, where 
\colorbox[RGB]{253,246,227}{{{{\color[RGB]{101, 123, 131} ω ∣ a }}}}.
\paragraph{onote.scale}
\par
\colorbox[RGB]{253,246,227}{{{{\color[RGB]{101, 123, 131} scale x o }}}} is the ordinal notation for 
\colorbox[RGB]{253,246,227}{{{{\color[RGB]{101, 123, 131} ω \textasciicircum{} x  }}}{{{\color[RGB]{181, 137, 0} * }}}{{{\color[RGB]{101, 123, 131}  o }}}}.
\paragraph{onote.mul\_nat}
\par
\colorbox[RGB]{253,246,227}{{{{\color[RGB]{101, 123, 131} mul\_nat o n }}}} is the ordinal notation for 
\colorbox[RGB]{253,246,227}{{{{\color[RGB]{101, 123, 131} o  }}}{{{\color[RGB]{181, 137, 0} * }}}{{{\color[RGB]{101, 123, 131}  n }}}}.
\paragraph{onote.power}
\par
\colorbox[RGB]{253,246,227}{{{{\color[RGB]{101, 123, 131} power o₁ o₂ }}}} calculates the ordinal notation for
the ordinal exponential 
\colorbox[RGB]{253,246,227}{{{{\color[RGB]{101, 123, 131} o₁ \textasciicircum{} o₂ }}}}.
\paragraph{nonote}
\par
The type of normal ordinal notations. (It would have been
nicer to define this right in the inductive type, but 
\colorbox[RGB]{253,246,227}{{{{\color[RGB]{101, 123, 131} NF o }}}}requires 
\colorbox[RGB]{253,246,227}{{{{\color[RGB]{101, 123, 131} repr }}}} which requires 
\colorbox[RGB]{253,246,227}{{{{\color[RGB]{101, 123, 131} onote }}}}, so all these things
would have to be defined at once, which messes up the VM
representation.)
\paragraph{nonote.mk}
\par
Construct a 
\colorbox[RGB]{253,246,227}{{{{\color[RGB]{101, 123, 131} nonote }}}} from an ordinal notation
(and infer normality)
\paragraph{nonote.repr}
\par
The ordinal represented by an ordinal notation.
(This function is noncomputable because ordinal
arithmetic is noncomputable. In computational applications
\colorbox[RGB]{253,246,227}{{{{\color[RGB]{101, 123, 131} nonote }}}} can be used exclusively without reference
to 
\colorbox[RGB]{253,246,227}{{{{\color[RGB]{101, 123, 131} ordinal }}}}, but this function allows for correctness
results to be stated.)
\paragraph{nonote.of\_nat}
\par
Convert a natural number to an ordinal notation
\paragraph{nonote.cmp}
\par
Compare ordinal notations
\paragraph{nonote.below}
\par
Asserts that 
\colorbox[RGB]{253,246,227}{{{{\color[RGB]{101, 123, 131} repr a  }}}{{{\color[RGB]{181, 137, 0} < }}}{{{\color[RGB]{101, 123, 131}  ω \textasciicircum{} repr b }}}}. Used in 
\colorbox[RGB]{253,246,227}{{{{\color[RGB]{101, 123, 131} nonote.rec\_on }}}}\paragraph{nonote.oadd}
\par
The 
\colorbox[RGB]{253,246,227}{{{{\color[RGB]{101, 123, 131} oadd }}}} pseudo-constructor for 
\colorbox[RGB]{253,246,227}{{{{\color[RGB]{101, 123, 131} nonote }}}}\paragraph{nonote.rec\_on}
\par
This is a recursor-like theorem for 
\colorbox[RGB]{253,246,227}{{{{\color[RGB]{101, 123, 131} nonote }}}} suggesting an
inductive definition, which can't actually be defined this
way due to conflicting dependencies.
\paragraph{nonote.has\_add}
\par
Addition of ordinal notations
\paragraph{nonote.has\_sub}
\par
Subtraction of ordinal notations
\paragraph{nonote.has\_mul}
\par
Multiplication of ordinal notations
\paragraph{nonote.power}
\par
Exponentiation of ordinal notations
\section{set\_theory/schroeder\_bernstein.lean}\section{set\_theory/surreal.lean}\paragraph{pSurreal}
\par
The type of pre-surreal numbers, before we have quotiented
by extensionality. In ZFC, a surreal number is constructed from
two sets of surreal numbers that have been constructed at an earlier
stage. To do this in type theory, we say that a pre-surreal is built
inductively from two families of pre-surreals indexed over any type
in Type u. The resulting type 
\colorbox[RGB]{253,246,227}{{{{\color[RGB]{101, 123, 131} pSurreal.\{u\} }}}} lives in 
\colorbox[RGB]{253,246,227}{{{{\color[RGB]{38, 139, 210} Type }}}{{{\color[RGB]{101, 123, 131}  (u }}}{{{\color[RGB]{181, 137, 0} + }}}{{{\color[RGB]{108, 113, 196} 1 }}}{{{\color[RGB]{101, 123, 131} ) }}}},
reflecting that it is a proper class in ZFC.
\paragraph{pSurreal.le\_lt}
\par
Define simultaneously by mutual induction the 
\colorbox[RGB]{253,246,227}{{{{\color[RGB]{181, 137, 0} <= }}}} and 
\colorbox[RGB]{253,246,227}{{{{\color[RGB]{181, 137, 0} < }}}}relation on surreals. The ZFC definition says that 
\colorbox[RGB]{253,246,227}{{{{\color[RGB]{101, 123, 131} x  }}}{{{\color[RGB]{181, 137, 0} = }}}{{{\color[RGB]{101, 123, 131}  \{xL | xR\} }}}}is less or equal to 
\colorbox[RGB]{253,246,227}{{{{\color[RGB]{101, 123, 131} y  }}}{{{\color[RGB]{181, 137, 0} = }}}{{{\color[RGB]{101, 123, 131}  \{yL | yR\} }}}} if 
\colorbox[RGB]{253,246,227}{{{{\color[RGB]{101, 123, 131} ∀ x₁ ∈ xL, x₁  }}}{{{\color[RGB]{181, 137, 0} < }}}{{{\color[RGB]{101, 123, 131}  y }}}}and 
\colorbox[RGB]{253,246,227}{{{{\color[RGB]{101, 123, 131} ∀ y₂ ∈ yR, x  }}}{{{\color[RGB]{181, 137, 0} < }}}{{{\color[RGB]{101, 123, 131}  y₂ }}}}, where 
\colorbox[RGB]{253,246,227}{{{{\color[RGB]{101, 123, 131} x  }}}{{{\color[RGB]{181, 137, 0} < }}}{{{\color[RGB]{101, 123, 131}  y }}}} is the same as 
\colorbox[RGB]{253,246,227}{{{{\color[RGB]{181, 137, 0} ¬ }}}{{{\color[RGB]{101, 123, 131}  y  }}}{{{\color[RGB]{181, 137, 0} <= }}}{{{\color[RGB]{101, 123, 131}  x }}}}.
This is a tricky induction because it only decreases one side at
a time, and it also swaps the arguments in the definition of 
\colorbox[RGB]{253,246,227}{{{{\color[RGB]{181, 137, 0} < }}}}.
So we define 
\colorbox[RGB]{253,246,227}{{{{\color[RGB]{101, 123, 131} x  }}}{{{\color[RGB]{181, 137, 0} < }}}{{{\color[RGB]{101, 123, 131}  y }}}} and 
\colorbox[RGB]{253,246,227}{{{{\color[RGB]{101, 123, 131} x  }}}{{{\color[RGB]{181, 137, 0} <= }}}{{{\color[RGB]{101, 123, 131}  y }}}} simultaneously.
\paragraph{pSurreal.mk\_le\_mk}
\par
Definition of 
\colorbox[RGB]{253,246,227}{{{{\color[RGB]{101, 123, 131} x  }}}{{{\color[RGB]{181, 137, 0} ≤ }}}{{{\color[RGB]{101, 123, 131}  y }}}} on pre-surreals.
\paragraph{pSurreal.mk\_lt\_mk}
\par
Definition of 
\colorbox[RGB]{253,246,227}{{{{\color[RGB]{101, 123, 131} x  }}}{{{\color[RGB]{181, 137, 0} < }}}{{{\color[RGB]{101, 123, 131}  y }}}} on pre-surreals.
\paragraph{pSurreal.ok}
\par
A pre-surreal is valid (wikipedia calls this "numeric") if
everything in the L set is less than everything in the R set,
and all the elements of L and R are also valid.
\paragraph{pSurreal.equiv}
\par
Define the equivalence relation on pre-surreals. Two pre-surreals
\colorbox[RGB]{253,246,227}{{{{\color[RGB]{101, 123, 131} x }}}}, 
\colorbox[RGB]{253,246,227}{{{{\color[RGB]{101, 123, 131} y }}}} are equivalent if 
\colorbox[RGB]{253,246,227}{{{{\color[RGB]{101, 123, 131} x  }}}{{{\color[RGB]{181, 137, 0} ≤ }}}{{{\color[RGB]{101, 123, 131}  y }}}} and 
\colorbox[RGB]{253,246,227}{{{{\color[RGB]{101, 123, 131} y  }}}{{{\color[RGB]{181, 137, 0} ≤ }}}{{{\color[RGB]{101, 123, 131}  x }}}}.
\paragraph{pSurreal.has\_zero}
\par
The pre-surreal zero is defined by 
\colorbox[RGB]{253,246,227}{{{{\color[RGB]{108, 113, 196} 0 }}}{{{\color[RGB]{101, 123, 131}   }}}{{{\color[RGB]{181, 137, 0} = }}}{{{\color[RGB]{101, 123, 131}  \{ | \} }}}}.
\paragraph{pSurreal.has\_one}
\par
The pre-surreal one is defined by 
\colorbox[RGB]{253,246,227}{{{{\color[RGB]{108, 113, 196} 1 }}}{{{\color[RGB]{101, 123, 131}   }}}{{{\color[RGB]{181, 137, 0} = }}}{{{\color[RGB]{101, 123, 131}  \{  }}}{{{\color[RGB]{108, 113, 196} 0 }}}{{{\color[RGB]{101, 123, 131}  | \} }}}}.
\paragraph{pSurreal.neg}
\par
The negation of 
\colorbox[RGB]{253,246,227}{{{{\color[RGB]{101, 123, 131} \{L | R\} }}}} is 
\colorbox[RGB]{253,246,227}{{{{\color[RGB]{101, 123, 131} \{ }}}{{{\color[RGB]{181, 137, 0} - }}}{{{\color[RGB]{101, 123, 131} R |  }}}{{{\color[RGB]{181, 137, 0} - }}}{{{\color[RGB]{101, 123, 131} L\} }}}}.
\paragraph{pSurreal.add}
\par
The sum of 
\colorbox[RGB]{253,246,227}{{{{\color[RGB]{101, 123, 131} x  }}}{{{\color[RGB]{181, 137, 0} = }}}{{{\color[RGB]{101, 123, 131}  \{xL | xR\} }}}} and 
\colorbox[RGB]{253,246,227}{{{{\color[RGB]{101, 123, 131} y  }}}{{{\color[RGB]{181, 137, 0} = }}}{{{\color[RGB]{101, 123, 131}  \{yL | yR\} }}}} is 
\colorbox[RGB]{253,246,227}{{{{\color[RGB]{101, 123, 131} \{xL  }}}{{{\color[RGB]{181, 137, 0} + }}}{{{\color[RGB]{101, 123, 131}  y, x  }}}{{{\color[RGB]{181, 137, 0} + }}}{{{\color[RGB]{101, 123, 131}  yL | xR  }}}{{{\color[RGB]{181, 137, 0} + }}}{{{\color[RGB]{101, 123, 131}  y, x  }}}{{{\color[RGB]{181, 137, 0} + }}}{{{\color[RGB]{101, 123, 131}  yR\} }}}}.
\paragraph{pSurreal.mul}
\par
The product of 
\colorbox[RGB]{253,246,227}{{{{\color[RGB]{101, 123, 131} x  }}}{{{\color[RGB]{181, 137, 0} = }}}{{{\color[RGB]{101, 123, 131}  \{xL | xR\} }}}} and 
\colorbox[RGB]{253,246,227}{{{{\color[RGB]{101, 123, 131} y  }}}{{{\color[RGB]{181, 137, 0} = }}}{{{\color[RGB]{101, 123, 131}  \{yL | yR\} }}}} is
\colorbox[RGB]{253,246,227}{{{{\color[RGB]{101, 123, 131} \{xL }}}{{{\color[RGB]{181, 137, 0} * }}}{{{\color[RGB]{101, 123, 131} y  }}}{{{\color[RGB]{181, 137, 0} + }}}{{{\color[RGB]{101, 123, 131}  x }}}{{{\color[RGB]{181, 137, 0} * }}}{{{\color[RGB]{101, 123, 131} yL  }}}{{{\color[RGB]{181, 137, 0} - }}}{{{\color[RGB]{101, 123, 131}  xL }}}{{{\color[RGB]{181, 137, 0} * }}}{{{\color[RGB]{101, 123, 131} yL, xR }}}{{{\color[RGB]{181, 137, 0} * }}}{{{\color[RGB]{101, 123, 131} y  }}}{{{\color[RGB]{181, 137, 0} + }}}{{{\color[RGB]{101, 123, 131}  x }}}{{{\color[RGB]{181, 137, 0} * }}}{{{\color[RGB]{101, 123, 131} yR  }}}{{{\color[RGB]{181, 137, 0} - }}}{{{\color[RGB]{101, 123, 131}  xR }}}{{{\color[RGB]{181, 137, 0} * }}}{{{\color[RGB]{101, 123, 131} yR | xL }}}{{{\color[RGB]{181, 137, 0} * }}}{{{\color[RGB]{101, 123, 131} y  }}}{{{\color[RGB]{181, 137, 0} + }}}{{{\color[RGB]{101, 123, 131}  x }}}{{{\color[RGB]{181, 137, 0} * }}}{{{\color[RGB]{101, 123, 131} yR  }}}{{{\color[RGB]{181, 137, 0} - }}}{{{\color[RGB]{101, 123, 131}  xL }}}{{{\color[RGB]{181, 137, 0} * }}}{{{\color[RGB]{101, 123, 131} yR, x }}}{{{\color[RGB]{181, 137, 0} * }}}{{{\color[RGB]{101, 123, 131} yL  }}}{{{\color[RGB]{181, 137, 0} + }}}{{{\color[RGB]{101, 123, 131}  xR }}}{{{\color[RGB]{181, 137, 0} * }}}{{{\color[RGB]{101, 123, 131} y  }}}{{{\color[RGB]{181, 137, 0} - }}}{{{\color[RGB]{101, 123, 131}  xR }}}{{{\color[RGB]{181, 137, 0} * }}}{{{\color[RGB]{101, 123, 131} yL \} }}}}.
\paragraph{pSurreal.inv\_ty}
\par
Because the two halves of the definition of inv produce more elements
of each side, we have to define the two families inductively.
This is the indexing set for the function, and 
\colorbox[RGB]{253,246,227}{{{{\color[RGB]{101, 123, 131} inv\_val }}}} is the function part.
\paragraph{pSurreal.inv\_val}
\par
Because the two halves of the definition of inv produce more elements
of each side, we have to define the two families inductively.
This is the function part, defined by recursion on 
\colorbox[RGB]{253,246,227}{{{{\color[RGB]{101, 123, 131} inv\_ty }}}}.
\paragraph{pSurreal.inv'}
\par
The inverse of a positive surreal number 
\colorbox[RGB]{253,246,227}{{{{\color[RGB]{101, 123, 131} x  }}}{{{\color[RGB]{181, 137, 0} = }}}{{{\color[RGB]{101, 123, 131}  \{L | R\} }}}} is
given by 
\colorbox[RGB]{253,246,227}{{{{\color[RGB]{101, 123, 131} x }}}{{{\color[RGB]{181, 137, 0} ⁻¹ }}}{{{\color[RGB]{101, 123, 131}   }}}{{{\color[RGB]{181, 137, 0} = }}}{{{\color[RGB]{101, 123, 131}  \{ }}}{{{\color[RGB]{108, 113, 196} 0 }}}{{{\color[RGB]{101, 123, 131} , ( }}}{{{\color[RGB]{108, 113, 196} 1 }}}{{{\color[RGB]{101, 123, 131}   }}}{{{\color[RGB]{181, 137, 0} + }}}{{{\color[RGB]{101, 123, 131}  (R  }}}{{{\color[RGB]{181, 137, 0} - }}}{{{\color[RGB]{101, 123, 131}  x)  }}}{{{\color[RGB]{181, 137, 0} * }}}{{{\color[RGB]{101, 123, 131}  x }}}{{{\color[RGB]{181, 137, 0} ⁻¹ }}}{{{\color[RGB]{101, 123, 131} L)  }}}{{{\color[RGB]{181, 137, 0} * }}}{{{\color[RGB]{101, 123, 131}  R, ( }}}{{{\color[RGB]{108, 113, 196} 1 }}}{{{\color[RGB]{101, 123, 131}   }}}{{{\color[RGB]{181, 137, 0} + }}}{{{\color[RGB]{101, 123, 131}  (L  }}}{{{\color[RGB]{181, 137, 0} - }}}{{{\color[RGB]{101, 123, 131}  x)  }}}{{{\color[RGB]{181, 137, 0} * }}}{{{\color[RGB]{101, 123, 131}  x }}}{{{\color[RGB]{181, 137, 0} ⁻¹ }}}{{{\color[RGB]{101, 123, 131} R)  }}}{{{\color[RGB]{181, 137, 0} * }}}{{{\color[RGB]{101, 123, 131}  L | ( }}}{{{\color[RGB]{108, 113, 196} 1 }}}{{{\color[RGB]{101, 123, 131}   }}}{{{\color[RGB]{181, 137, 0} + }}}{{{\color[RGB]{101, 123, 131}  (L  }}}{{{\color[RGB]{181, 137, 0} - }}}{{{\color[RGB]{101, 123, 131}  x)  }}}{{{\color[RGB]{181, 137, 0} * }}}{{{\color[RGB]{101, 123, 131}  x }}}{{{\color[RGB]{181, 137, 0} ⁻¹ }}}{{{\color[RGB]{101, 123, 131} L)  }}}{{{\color[RGB]{181, 137, 0} * }}}{{{\color[RGB]{101, 123, 131}  L, ( }}}{{{\color[RGB]{108, 113, 196} 1 }}}{{{\color[RGB]{101, 123, 131}   }}}{{{\color[RGB]{181, 137, 0} + }}}{{{\color[RGB]{101, 123, 131}  (R  }}}{{{\color[RGB]{181, 137, 0} - }}}{{{\color[RGB]{101, 123, 131}  x)  }}}{{{\color[RGB]{181, 137, 0} * }}}{{{\color[RGB]{101, 123, 131}  x }}}{{{\color[RGB]{181, 137, 0} ⁻¹ }}}{{{\color[RGB]{101, 123, 131} R)  }}}{{{\color[RGB]{181, 137, 0} * }}}{{{\color[RGB]{101, 123, 131}  R\} }}}}.
Because the two halves 
\colorbox[RGB]{253,246,227}{{{{\color[RGB]{101, 123, 131} x }}}{{{\color[RGB]{181, 137, 0} ⁻¹ }}}{{{\color[RGB]{101, 123, 131} L, x }}}{{{\color[RGB]{181, 137, 0} ⁻¹ }}}{{{\color[RGB]{101, 123, 131} R }}}} of 
\colorbox[RGB]{253,246,227}{{{{\color[RGB]{101, 123, 131} x }}}{{{\color[RGB]{181, 137, 0} ⁻¹ }}}} are used in their own
definition, the sets and elements are inductively generated.
\paragraph{pSurreal.inv}
\par
The inverse of a surreal number in terms of the inverse on
positive surreals.
\paragraph{pSurreal.omega}
\par
The pre-surreal number 
\colorbox[RGB]{253,246,227}{{{{\color[RGB]{101, 123, 131} ω }}}}. (In fact all ordinals have surreal
representatives.)
\paragraph{surreal.equiv}
\par
The equivalence on valid pre-surreal numbers.
\paragraph{surreal}
\par
The type of surreal numbers. In ZFC, a surreal number is constructed from
two sets of surreal numbers that have been constructed at an earlier
stage. To do this in type theory, we say that a pre-surreal is built
inductively from two families of pre-surreals indexed over any type
in Type u. The resulting type 
\colorbox[RGB]{253,246,227}{{{{\color[RGB]{101, 123, 131} pSurreal.\{u\} }}}} lives in 
\colorbox[RGB]{253,246,227}{{{{\color[RGB]{38, 139, 210} Type }}}{{{\color[RGB]{101, 123, 131}  (u }}}{{{\color[RGB]{181, 137, 0} + }}}{{{\color[RGB]{108, 113, 196} 1 }}}{{{\color[RGB]{101, 123, 131} ) }}}},
reflecting that it is a proper class in ZFC.
A surreal number is then constructed by discarding the invalid pre-surreals
and quotienting by equivalence so that the ordering becomes a total order.
\paragraph{surreal.mk}
\par
Construct a surreal number from a valid pre-surreal.
\paragraph{surreal.lift}
\par
Lift an equivalence-respecting function on pre-surreals to surreals.
\paragraph{surreal.lift₂}
\par
Lift a binary equivalence-respecting function on pre-surreals to surreals.
\paragraph{surreal.le}
\par
The relation 
\colorbox[RGB]{253,246,227}{{{{\color[RGB]{101, 123, 131} x  }}}{{{\color[RGB]{181, 137, 0} ≤ }}}{{{\color[RGB]{101, 123, 131}  y }}}} on surreals.
\paragraph{surreal.lt}
\par
The relation 
\colorbox[RGB]{253,246,227}{{{{\color[RGB]{101, 123, 131} x  }}}{{{\color[RGB]{181, 137, 0} < }}}{{{\color[RGB]{101, 123, 131}  y }}}} on surreals.
\section{set\_theory/zfc.lean}\paragraph{arity}
\par
The type of 
\colorbox[RGB]{253,246,227}{{{{\color[RGB]{101, 123, 131} n }}}}-ary functions 
\colorbox[RGB]{253,246,227}{{{{\color[RGB]{101, 123, 131} α  }}}{{{\color[RGB]{133, 153, 0} → }}}{{{\color[RGB]{101, 123, 131}  α  }}}{{{\color[RGB]{133, 153, 0} → }}}{{{\color[RGB]{101, 123, 131}  ...  }}}{{{\color[RGB]{133, 153, 0} → }}}{{{\color[RGB]{101, 123, 131}  α }}}}.
\paragraph{pSet}
\par
The type of pre-sets in universe 
\colorbox[RGB]{253,246,227}{{{{\color[RGB]{101, 123, 131} u }}}}. A pre-set
is a family of pre-sets indexed by a type in 
\colorbox[RGB]{253,246,227}{{{{\color[RGB]{38, 139, 210} Type }}}{{{\color[RGB]{101, 123, 131}  u }}}}.
The ZFC universe is defined as a quotient of this
to ensure extensionality.
\paragraph{pSet.type}
\par
The underlying type of a pre-set
\paragraph{pSet.func}
\par
The underlying pre-set family of a pre-set
\paragraph{pSet.equiv}
\par
Two pre-sets are extensionally equivalent if every
element of the first family is extensionally equivalent to
some element of the second family and vice-versa.
\paragraph{pSet.mem}
\par
\colorbox[RGB]{253,246,227}{{{{\color[RGB]{101, 123, 131} x ∈ y }}}} as pre-sets if 
\colorbox[RGB]{253,246,227}{{{{\color[RGB]{101, 123, 131} x }}}} is extensionally equivalent to a member
of the family 
\colorbox[RGB]{253,246,227}{{{{\color[RGB]{101, 123, 131} y }}}}.
\paragraph{pSet.to\_set}
\par
Convert a pre-set to a 
\colorbox[RGB]{253,246,227}{{{{\color[RGB]{101, 123, 131} set }}}} of pre-sets.
\paragraph{pSet.equiv.eq}
\par
Two pre-sets are equivalent iff they have the same members.
\paragraph{pSet.empty}
\par
The empty pre-set
\paragraph{pSet.insert}
\par
Insert an element into a pre-set
\paragraph{pSet.of\_nat}
\par
The n-th von Neumann ordinal
\paragraph{pSet.omega}
\par
The von Neumann ordinal ω
\paragraph{pSet.sep}
\par
The separation operation 
\colorbox[RGB]{253,246,227}{{{{\color[RGB]{101, 123, 131} \{x ∈ a | p x\} }}}}\paragraph{pSet.powerset}
\par
The powerset operator
\paragraph{pSet.Union}
\par
The set union operator
\paragraph{pSet.image}
\par
The image of a function
\paragraph{pSet.lift}
\par
Universe lift operation
\paragraph{pSet.embed}
\par
Embedding of one universe in another
\paragraph{pSet.arity.equiv}
\par
Function equivalence is defined so that 
\colorbox[RGB]{253,246,227}{{{{\color[RGB]{101, 123, 131} f \textasciitilde{} g }}}} iff
\colorbox[RGB]{253,246,227}{{{{\color[RGB]{101, 123, 131} ∀ x y, x \textasciitilde{} y  }}}{{{\color[RGB]{133, 153, 0} → }}}{{{\color[RGB]{101, 123, 131}  f x \textasciitilde{} g y }}}}. This extends to equivalence of n-ary
functions.
\paragraph{pSet.resp}
\par
\colorbox[RGB]{253,246,227}{{{{\color[RGB]{101, 123, 131} resp n }}}} is the collection of n-ary functions on 
\colorbox[RGB]{253,246,227}{{{{\color[RGB]{101, 123, 131} pSet }}}} that respect
equivalence, i.e. when the inputs are equivalent the output is as well.
\paragraph{Set}
\par
The ZFC universe of sets consists of the type of pre-sets,
quotiented by extensional equivalence.
\paragraph{pSet.resp.eval}
\par
An equivalence-respecting function yields an n-ary Set function.
\paragraph{pSet.definable}
\par
A set function is "definable" if it is the image of some n-ary pre-set
function. This isn't exactly definability, but is useful as a sufficient
condition for functions that have a computable image.
\paragraph{Set.to\_set}
\par
Convert a ZFC set into a 
\colorbox[RGB]{253,246,227}{{{{\color[RGB]{101, 123, 131} set }}}} of sets
\paragraph{Set.empty}
\par
The empty set
\paragraph{Set.insert}
\par
\colorbox[RGB]{253,246,227}{{{{\color[RGB]{101, 123, 131} insert x y }}}} is the set 
\colorbox[RGB]{253,246,227}{{{{\color[RGB]{101, 123, 131} \{x\} ∪ y }}}}\paragraph{Set.omega}
\par
\colorbox[RGB]{253,246,227}{{{{\color[RGB]{101, 123, 131} omega }}}} is the first infinite von Neumann ordinal
\paragraph{Set.sep}
\par
\colorbox[RGB]{253,246,227}{{{{\color[RGB]{101, 123, 131} \{x ∈ a | p x\} }}}} is the set of elements in 
\colorbox[RGB]{253,246,227}{{{{\color[RGB]{101, 123, 131} a }}}} satisfying 
\colorbox[RGB]{253,246,227}{{{{\color[RGB]{101, 123, 131} p }}}}\paragraph{Set.powerset}
\par
The powerset operation, the collection of subsets of a set
\paragraph{Set.Union}
\par
The union operator, the collection of elements of elements of a set
\paragraph{Set.union}
\par
The binary union operation
\paragraph{Set.inter}
\par
The binary intersection operation
\paragraph{Set.diff}
\par
The set difference operation
\paragraph{Set.image}
\par
The image of a (definable) set function
\paragraph{Set.pair}
\par
Kuratowski ordered pair
\paragraph{Set.pair\_sep}
\par
A subset of pairs 
\colorbox[RGB]{253,246,227}{{{{\color[RGB]{101, 123, 131} \{(a, b) ∈ x × y | p a b\} }}}}\paragraph{Set.prod}
\par
The cartesian product, 
\colorbox[RGB]{253,246,227}{{{{\color[RGB]{101, 123, 131} \{(a, b) | a ∈ x, b ∈ y\} }}}}\paragraph{Set.is\_func}
\par
\colorbox[RGB]{253,246,227}{{{{\color[RGB]{101, 123, 131} is\_func x y f }}}} is the assertion 
\colorbox[RGB]{253,246,227}{{{{\color[RGB]{101, 123, 131} f : x  }}}{{{\color[RGB]{133, 153, 0} → }}}{{{\color[RGB]{101, 123, 131}  y }}}} where 
\colorbox[RGB]{253,246,227}{{{{\color[RGB]{101, 123, 131} f }}}} is a ZFC function
(a set of ordered pairs)
\paragraph{Set.funs}
\par
\colorbox[RGB]{253,246,227}{{{{\color[RGB]{101, 123, 131} funs x y }}}} is 
\colorbox[RGB]{253,246,227}{{{{\color[RGB]{101, 123, 131} y \textasciicircum{} x }}}}, the set of all set functions 
\colorbox[RGB]{253,246,227}{{{{\color[RGB]{101, 123, 131} x  }}}{{{\color[RGB]{133, 153, 0} → }}}{{{\color[RGB]{101, 123, 131}  y }}}}\paragraph{Set.map}
\par
Graph of a function: 
\colorbox[RGB]{253,246,227}{{{{\color[RGB]{101, 123, 131} map f x }}}} is the ZFC function which maps 
\colorbox[RGB]{253,246,227}{{{{\color[RGB]{101, 123, 131} a ∈ x }}}} to 
\colorbox[RGB]{253,246,227}{{{{\color[RGB]{101, 123, 131} f a }}}}\paragraph{Class.of\_Set}
\par
Coerce a set into a class
\paragraph{Class.univ}
\par
The universal class
\paragraph{Class.to\_Set}
\par
Assert that 
\colorbox[RGB]{253,246,227}{{{{\color[RGB]{101, 123, 131} A }}}} is a set satisfying 
\colorbox[RGB]{253,246,227}{{{{\color[RGB]{101, 123, 131} p }}}}\paragraph{Class.mem}
\par
\colorbox[RGB]{253,246,227}{{{{\color[RGB]{101, 123, 131} A ∈ B }}}} if 
\colorbox[RGB]{253,246,227}{{{{\color[RGB]{101, 123, 131} A }}}} is a set which is a member of 
\colorbox[RGB]{253,246,227}{{{{\color[RGB]{101, 123, 131} B }}}}\paragraph{Class.Cong\_to\_Class}
\par
Convert a conglomerate (a collection of classes) into a class
\paragraph{Class.Class\_to\_Cong}
\par
Convert a class into a conglomerate (a collection of classes)
\paragraph{Class.powerset}
\par
The power class of a class is the class of all subclasses that are sets
\paragraph{Class.Union}
\par
The union of a class is the class of all members of sets in the class
\paragraph{Class.iota}
\par
The definite description operator, which is \{x\} if 
\colorbox[RGB]{253,246,227}{{{{\color[RGB]{101, 123, 131} \{a | p a\}  }}}{{{\color[RGB]{181, 137, 0} = }}}{{{\color[RGB]{101, 123, 131}  \{x\} }}}}and ∅ otherwise
\paragraph{Class.iota\_ex}
\par
Unlike the other set constructors, the 
\colorbox[RGB]{253,246,227}{{{{\color[RGB]{101, 123, 131} iota }}}} definite descriptor
is a set for any set input, but not constructively so, so there is no
associated 
\colorbox[RGB]{253,246,227}{{{{\color[RGB]{101, 123, 131} (Set  }}}{{{\color[RGB]{133, 153, 0} → }}}{{{\color[RGB]{101, 123, 131}   }}}{{{\color[RGB]{38, 139, 210} Prop }}}{{{\color[RGB]{101, 123, 131} )  }}}{{{\color[RGB]{133, 153, 0} → }}}{{{\color[RGB]{101, 123, 131}  Set }}}} function.
\paragraph{Class.fval}
\par
Function value
\paragraph{Set.choice}
\par
A choice function on the set of nonempty sets 
\colorbox[RGB]{253,246,227}{{{{\color[RGB]{101, 123, 131} x }}}}\section{tactic/abel.lean}\paragraph{tactic.interactive.abel1}
\par
Tactic for solving equations in the language of
commutative monoids and groups.
This version of 
\colorbox[RGB]{253,246,227}{{{{\color[RGB]{101, 123, 131} abel }}}} fails if the target is not an equality
that is provable by the axioms of commutative monoids/groups.
\paragraph{tactic.interactive.abel}
\par
Tactic for solving equations in the language of
commutative monoids and groups.
Attempts to prove the goal outright if there is no 
\colorbox[RGB]{253,246,227}{{{{\color[RGB]{133, 153, 0} at }}}}specifier and the target is an equality, but if this
fails it falls back to rewriting all monoid expressions
into a normal form.
\section{tactic/algebra.lean}\section{tactic/alias.lean}\section{tactic/auto\_cases.lean}\paragraph{auto\_cases}
\par
Applies 
\colorbox[RGB]{253,246,227}{{{{\color[RGB]{101, 123, 131} cases }}}} or 
\colorbox[RGB]{253,246,227}{{{{\color[RGB]{101, 123, 131} induction }}}} on certain hypotheses.
\section{tactic/basic.lean}\section{tactic/cache.lean}\paragraph{tactic.reset\_instance\_cache}
\par
Reset the instance cache for the main goal.
\paragraph{tactic.interactive.unfreezeI}
\par
Unfreeze local instances, which allows us to revert
instances in the context.
\paragraph{tactic.interactive.resetI}
\par
Reset the instance cache. This allows any new instances
added to the context to be used in typeclass inference.
\paragraph{tactic.interactive.introI}
\par
Like 
\colorbox[RGB]{253,246,227}{{{{\color[RGB]{101, 123, 131} intro }}}}, but uses the introduced variable
in typeclass inference.
\paragraph{tactic.interactive.introsI}
\par
Like 
\colorbox[RGB]{253,246,227}{{{{\color[RGB]{101, 123, 131} intros }}}}, but uses the introduced variable(s)
in typeclass inference.
\paragraph{tactic.interactive.haveI}
\par
Used to add typeclasses to the context so that they can
be used in typeclass inference. The syntax is the same as 
\colorbox[RGB]{253,246,227}{{{{\color[RGB]{133, 153, 0} have }}}},
but the proof-omitted version is not supported. For
this one must write 
\colorbox[RGB]{253,246,227}{{{{\color[RGB]{133, 153, 0} have }}}{{{\color[RGB]{101, 123, 131}  : t, \{  }}}{{{\color[RGB]{181, 137, 0} < }}}{{{\color[RGB]{101, 123, 131} proof }}}{{{\color[RGB]{181, 137, 0} > }}}{{{\color[RGB]{101, 123, 131}  \}, resetI,  }}}{{{\color[RGB]{181, 137, 0} < }}}{{{\color[RGB]{101, 123, 131} proof }}}{{{\color[RGB]{181, 137, 0} > }}}}.
\paragraph{tactic.interactive.letI}
\par
Used to add typeclasses to the context so that they can
be used in typeclass inference. The syntax is the same as 
\colorbox[RGB]{253,246,227}{{{{\color[RGB]{133, 153, 0} let }}}}.
\paragraph{tactic.interactive.exactI}
\par
Like 
\colorbox[RGB]{253,246,227}{{{{\color[RGB]{101, 123, 131} exact }}}}, but uses all variables in the context
for typeclass inference.
\section{tactic/chain.lean}\paragraph{tactic.chain\_single}
\par
\colorbox[RGB]{253,246,227}{{{{\color[RGB]{101, 123, 131} chain\_many tac }}}} recursively tries 
\colorbox[RGB]{253,246,227}{{{{\color[RGB]{101, 123, 131} tac }}}} on all goals, working depth-first on generated subgoals,
until it no longer succeeds on any goal. 
\colorbox[RGB]{253,246,227}{{{{\color[RGB]{101, 123, 131} chain\_many }}}} automatically makes auxiliary definitions.
\paragraph{tactic.chain\_many}
\par
\colorbox[RGB]{253,246,227}{{{{\color[RGB]{101, 123, 131} chain\_many tac }}}} recursively tries 
\colorbox[RGB]{253,246,227}{{{{\color[RGB]{101, 123, 131} tac }}}} on all goals, working depth-first on generated subgoals,
until it no longer succeeds on any goal. 
\colorbox[RGB]{253,246,227}{{{{\color[RGB]{101, 123, 131} chain\_many }}}} automatically makes auxiliary definitions.
\paragraph{tactic.chain\_iter}
\par
\colorbox[RGB]{253,246,227}{{{{\color[RGB]{101, 123, 131} chain\_many tac }}}} recursively tries 
\colorbox[RGB]{253,246,227}{{{{\color[RGB]{101, 123, 131} tac }}}} on all goals, working depth-first on generated subgoals,
until it no longer succeeds on any goal. 
\colorbox[RGB]{253,246,227}{{{{\color[RGB]{101, 123, 131} chain\_many }}}} automatically makes auxiliary definitions.
\section{tactic/converter/binders.lean}\section{tactic/converter/interactive.lean}\section{tactic/converter/old\_conv.lean}\section{tactic/core.lean}\paragraph{tactic.decl\_mk\_const}
\par
Returns a pair (e, t), where 
\colorbox[RGB]{253,246,227}{{{{\color[RGB]{101, 123, 131} e ← mk\_const d.to\_name }}}}, and 
\colorbox[RGB]{253,246,227}{{{{\color[RGB]{101, 123, 131} t  }}}{{{\color[RGB]{181, 137, 0} = }}}{{{\color[RGB]{101, 123, 131}  d.type }}}}but with universe params updated to match the fresh universe metavariables in 
\colorbox[RGB]{253,246,227}{{{{\color[RGB]{101, 123, 131} e }}}}.
\par
This should have the same effect as just
\\
\colorbox[RGB]{253,246,227}{\parbox{4.5in}{{{{\color[RGB]{133, 153, 0} do }}}{{{\color[RGB]{101, 123, 131}  e ← mk\_const d.to\_name,
 }}}\\
{{{\color[RGB]{101, 123, 131}    t ← infer\_type e,
 }}}\\
{{{\color[RGB]{101, 123, 131}    return (e, t)
 }}}\\

}}\par
but is hopefully faster.
\paragraph{tactic.retrieve}
\par
Runs a tactic for a result, reverting the state after completion
\paragraph{tactic.repeat1}
\par
Repeat a tactic at least once, calling it recursively on all subgoals,
until it fails. This tactic fails if the first invocation fails.
\paragraph{tactic.iterate\_range}
\par
\colorbox[RGB]{253,246,227}{{{{\color[RGB]{101, 123, 131} iterate\_range m n t }}}}: Repeat the given tactic at least 
\colorbox[RGB]{253,246,227}{{{{\color[RGB]{101, 123, 131} m }}}} times and
at most 
\colorbox[RGB]{253,246,227}{{{{\color[RGB]{101, 123, 131} n }}}} times or until 
\colorbox[RGB]{253,246,227}{{{{\color[RGB]{101, 123, 131} t }}}} fails. Fails if 
\colorbox[RGB]{253,246,227}{{{{\color[RGB]{101, 123, 131} t }}}} does not run at least m times.
\paragraph{tactic.get\_expl\_pi\_arity}
\par
Compute the arity of explicit arguments of the given (Pi-)type
\paragraph{tactic.get\_expl\_arity}
\par
Compute the arity of explicit arguments of the given function
\paragraph{tactic.local\_proof}
\par
variation on 
\colorbox[RGB]{253,246,227}{{{{\color[RGB]{133, 153, 0} assert }}}} where a (possibly incomplete)
proof of the assertion is provided as a parameter.
\par
\colorbox[RGB]{253,246,227}{{{{\color[RGB]{101, 123, 131} (h,gs) ← local\_proof `h p tac }}}} creates a local 
\colorbox[RGB]{253,246,227}{{{{\color[RGB]{101, 123, 131} h : p }}}} and
use 
\colorbox[RGB]{253,246,227}{{{{\color[RGB]{101, 123, 131} tac }}}} to (partially) construct a proof for it. 
\colorbox[RGB]{253,246,227}{{{{\color[RGB]{101, 123, 131} gs }}}} is the
list of remaining goals in the proof of 
\colorbox[RGB]{253,246,227}{{{{\color[RGB]{101, 123, 131} h }}}}.
\par
The benefits over assert are:
\begin{itemize}\item unlike with 
\colorbox[RGB]{253,246,227}{{{{\color[RGB]{101, 123, 131} h ←  }}}{{{\color[RGB]{133, 153, 0} assert }}}{{{\color[RGB]{101, 123, 131}  `h p, tac }}}} , 
\colorbox[RGB]{253,246,227}{{{{\color[RGB]{101, 123, 131} h }}}} cannot be used by 
\colorbox[RGB]{253,246,227}{{{{\color[RGB]{101, 123, 131} tac }}}};

\item when 
\colorbox[RGB]{253,246,227}{{{{\color[RGB]{101, 123, 131} tac }}}} does not complete the proof of 
\colorbox[RGB]{253,246,227}{{{{\color[RGB]{101, 123, 131} h }}}}, returning the list
of goals allows one to write a tactic using 
\colorbox[RGB]{253,246,227}{{{{\color[RGB]{101, 123, 131} h }}}} and with the confidence
that a proof will not boil over to goals left over from the proof of 
\colorbox[RGB]{253,246,227}{{{{\color[RGB]{101, 123, 131} h }}}},
unlike what would be the case when using 
\colorbox[RGB]{253,246,227}{{{{\color[RGB]{101, 123, 131} tactic.swap }}}}.

\end{itemize}\paragraph{tactic.get\_goal}
\par
Returns the only goal, or fails if there isn't just one goal.
\paragraph{tactic.iterate\_at\_most\_on\_all\_goals}
\par
\colorbox[RGB]{253,246,227}{{{{\color[RGB]{101, 123, 131} iterate\_at\_most\_on\_all\_goals n t }}}}: repeat the given tactic at most 
\colorbox[RGB]{253,246,227}{{{{\color[RGB]{101, 123, 131} n }}}} times on all goals,
or until it fails. Always succeeds.
\paragraph{tactic.iterate\_at\_most\_on\_subgoals}
\par
\colorbox[RGB]{253,246,227}{{{{\color[RGB]{101, 123, 131} iterate\_at\_most\_on\_subgoals n t }}}}: repeat the tactic 
\colorbox[RGB]{253,246,227}{{{{\color[RGB]{101, 123, 131} t }}}} at most 
\colorbox[RGB]{253,246,227}{{{{\color[RGB]{101, 123, 131} n }}}} times on the first
goal and on all subgoals thus produced, or until it fails. Fails iff 
\colorbox[RGB]{253,246,227}{{{{\color[RGB]{101, 123, 131} t }}}} fails on
current goal.
\paragraph{tactic.apply\_list\_expr}
\par
\colorbox[RGB]{253,246,227}{{{{\color[RGB]{101, 123, 131} apply\_list l }}}}: try to apply the tactics in the list 
\colorbox[RGB]{253,246,227}{{{{\color[RGB]{101, 123, 131} l }}}} on the first goal, and
fail if none succeeds
\paragraph{tactic.build\_list\_expr\_for\_apply}
\par
constructs a list of expressions given a list of p-expressions, as follows:
\begin{itemize}\item if the p-expression is the name of a theorem, use 
\colorbox[RGB]{253,246,227}{{{{\color[RGB]{101, 123, 131} i\_to\_expr\_for\_apply }}}} on it

\item if the p-expression is a user attribute, add all the theorems with this attribute
to the list.

\end{itemize}\paragraph{tactic.apply\_rules}
\par
\colorbox[RGB]{253,246,227}{{{{\color[RGB]{101, 123, 131} apply\_rules hs n }}}}: apply the list of rules 
\colorbox[RGB]{253,246,227}{{{{\color[RGB]{101, 123, 131} hs }}}} (given as pexpr) and 
\colorbox[RGB]{253,246,227}{{{{\color[RGB]{101, 123, 131} assumption }}}} on the
first goal and the resulting subgoals, iteratively, at most 
\colorbox[RGB]{253,246,227}{{{{\color[RGB]{101, 123, 131} n }}}} times
\paragraph{tactic.mk\_iff\_mp\_app}
\par
Auxiliary function for 
\colorbox[RGB]{253,246,227}{{{{\color[RGB]{101, 123, 131} iff\_mp }}}} and 
\colorbox[RGB]{253,246,227}{{{{\color[RGB]{101, 123, 131} iff\_mpr }}}}. Takes a name, which should be either 
\colorbox[RGB]{253,246,227}{{{{\color[RGB]{101, 123, 131}  `iff.mp }}}}or 
\colorbox[RGB]{253,246,227}{{{{\color[RGB]{101, 123, 131}  `iff.mpr }}}}. If the passed expression is an iterated function type eventually producing an
\colorbox[RGB]{253,246,227}{{{{\color[RGB]{101, 123, 131} iff }}}}, returns an expression with the 
\colorbox[RGB]{253,246,227}{{{{\color[RGB]{101, 123, 131} iff }}}} converted to either the forwards or backwards
implication, as requested.
\paragraph{tactic.iff\_mp}
\par
Given an expression whose type is (a possibly iterated function producing) an 
\colorbox[RGB]{253,246,227}{{{{\color[RGB]{101, 123, 131} iff }}}},
create the expression which is the forward implication.
\paragraph{tactic.iff\_mpr}
\par
Given an expression whose type is (a possibly iterated function producing) an 
\colorbox[RGB]{253,246,227}{{{{\color[RGB]{101, 123, 131} iff }}}},
create the expression which is the reverse implication.
\paragraph{tactic.apply\_iff}
\par
Attempts to apply 
\colorbox[RGB]{253,246,227}{{{{\color[RGB]{101, 123, 131} e }}}}, and if that fails, if 
\colorbox[RGB]{253,246,227}{{{{\color[RGB]{101, 123, 131} e }}}} is an 
\colorbox[RGB]{253,246,227}{{{{\color[RGB]{101, 123, 131} iff }}}},
try applying both directions separately.
\paragraph{tactic.change\_with\_at}
\par
assuming olde and newe are defeq when elaborated, replaces occurences of olde with newe at hypothesis h.
\paragraph{tactic.propositional\_goal}
\par
Succeeds only if the current goal is a proposition.
\paragraph{tactic.subsingleton\_goal}
\par
Succeeds only if we can construct an instance showing the
current goal is a subsingleton type.
\paragraph{tactic.terminal\_goal}
\par
Succeeds only if the current goal is "terminal", in the sense
that no other goals depend on it.
\paragraph{tactic.iterate'}
\par
Apply a tactic as many times as possible, collecting the results in a list.
\paragraph{tactic.iterate1}
\par
Like iterate', but fail if the tactic does not succeed at least once.
\paragraph{tactic.successes}
\par
\colorbox[RGB]{253,246,227}{{{{\color[RGB]{101, 123, 131} successes }}}} invokes each tactic in turn, returning the list of successful results.
\paragraph{\_private.2113013827.target'}
\par
Return target after instantiating metavars and whnf
\paragraph{tactic.fsplit}
\par
Just like 
\colorbox[RGB]{253,246,227}{{{{\color[RGB]{101, 123, 131} split }}}}, 
\colorbox[RGB]{253,246,227}{{{{\color[RGB]{101, 123, 131} fsplit }}}} applies the constructor when the type of the target is an inductive data type with one constructor.
However it does not reorder goals or invoke 
\colorbox[RGB]{253,246,227}{{{{\color[RGB]{101, 123, 131} auto\_param }}}} tactics.
\paragraph{tactic.injections\_and\_clear}
\par
Calls 
\colorbox[RGB]{253,246,227}{{{{\color[RGB]{101, 123, 131} injection }}}} on each hypothesis, and then, for each hypothesis on which 
\colorbox[RGB]{253,246,227}{{{{\color[RGB]{101, 123, 131} injection }}}}succeeds, clears the old hypothesis.
\paragraph{tactic.case\_bash}
\par
calls 
\colorbox[RGB]{253,246,227}{{{{\color[RGB]{101, 123, 131} cases }}}} on every local hypothesis, succeeding if
it succeeds on at least one hypothesis.
\paragraph{tactic.find\_local}
\par
\colorbox[RGB]{253,246,227}{{{{\color[RGB]{101, 123, 131} find\_local t }}}} returns a local constant with type t, or fails if none exists.
\paragraph{tactic.dependent\_pose\_core}
\par
\colorbox[RGB]{253,246,227}{{{{\color[RGB]{101, 123, 131} dependent\_pose\_core l }}}}: introduce dependent hypothesis, where the proofs depend on the values
of the previous local constants. 
\colorbox[RGB]{253,246,227}{{{{\color[RGB]{101, 123, 131} l }}}} is a list of local constants and their values.
\paragraph{tactic.mk\_local\_pis\_whnf}
\par
like 
\colorbox[RGB]{253,246,227}{{{{\color[RGB]{101, 123, 131} mk\_local\_pis }}}} but translating into weak head normal form before checking if it is a Π.
\paragraph{tactic.choose1}
\par
Changes 
\colorbox[RGB]{253,246,227}{{{{\color[RGB]{101, 123, 131} (h :  }}}{{{\color[RGB]{133, 153, 0} ∀ }}}{{{\color[RGB]{101, 123, 131} xs, ∃a:α, p a) ⊢ g }}}} to 
\colorbox[RGB]{253,246,227}{{{{\color[RGB]{101, 123, 131} (d :  }}}{{{\color[RGB]{133, 153, 0} ∀ }}}{{{\color[RGB]{101, 123, 131} xs, a) (s :  }}}{{{\color[RGB]{133, 153, 0} ∀ }}}{{{\color[RGB]{101, 123, 131} xs, p (d xs) ⊢ g }}}}\paragraph{tactic.choose}
\par
Changes 
\colorbox[RGB]{253,246,227}{{{{\color[RGB]{101, 123, 131} (h :  }}}{{{\color[RGB]{133, 153, 0} ∀ }}}{{{\color[RGB]{101, 123, 131} xs, ∃ }}}{{{\color[RGB]{133, 153, 0} as }}}{{{\color[RGB]{101, 123, 131} , p  }}}{{{\color[RGB]{133, 153, 0} as }}}{{{\color[RGB]{101, 123, 131} ) ⊢ g }}}} to a list of functions 
\colorbox[RGB]{253,246,227}{{{{\color[RGB]{133, 153, 0} as }}}}, an a final hypothesis on 
\colorbox[RGB]{253,246,227}{{{{\color[RGB]{101, 123, 131} p  }}}{{{\color[RGB]{133, 153, 0} as }}}}\paragraph{tactic.lock\_tactic\_state}
\par
This makes sure that the execution of the tactic does not change the tactic state.
This can be helpful while using rewrite, apply, or expr munging.
Remember to instantiate your metavariables before you're done!
\paragraph{tactic.instance\_stub}
\par
Hole command used to fill in a structure's field when specifying an instance.
\par
In the following:
\\
\colorbox[RGB]{253,246,227}{\parbox{4.5in}{{{{\color[RGB]{133, 153, 0} instance }}}{{{\color[RGB]{101, 123, 131}   }}}{{{\color[RGB]{211, 54, 130} : }}}{{{\color[RGB]{101, 123, 131}   }}}{{{\color[RGB]{101, 123, 131} monad id  }}}{{{\color[RGB]{181, 137, 0} := }}}{{{\color[RGB]{101, 123, 131} 
 }}}\\
{{{\color[RGB]{101, 123, 131} \{! !\}
 }}}\\

}}\par
invoking hole command 
\colorbox[RGB]{253,246,227}{{{{\color[RGB]{101, 123, 131} Instance Stub }}}} produces:
\\
\colorbox[RGB]{253,246,227}{\parbox{4.5in}{{{{\color[RGB]{133, 153, 0} instance }}}{{{\color[RGB]{101, 123, 131}   }}}{{{\color[RGB]{211, 54, 130} : }}}{{{\color[RGB]{101, 123, 131}   }}}{{{\color[RGB]{101, 123, 131} monad id  }}}{{{\color[RGB]{181, 137, 0} := }}}{{{\color[RGB]{101, 123, 131} 
 }}}\\
{{{\color[RGB]{101, 123, 131} \{ map  }}}{{{\color[RGB]{181, 137, 0} := }}}{{{\color[RGB]{101, 123, 131}  \_,
 }}}\\
{{{\color[RGB]{101, 123, 131}   map\_const  }}}{{{\color[RGB]{181, 137, 0} := }}}{{{\color[RGB]{101, 123, 131}  \_,
 }}}\\
{{{\color[RGB]{101, 123, 131}   pure  }}}{{{\color[RGB]{181, 137, 0} := }}}{{{\color[RGB]{101, 123, 131}  \_,
 }}}\\
{{{\color[RGB]{101, 123, 131}   seq  }}}{{{\color[RGB]{181, 137, 0} := }}}{{{\color[RGB]{101, 123, 131}  \_,
 }}}\\
{{{\color[RGB]{101, 123, 131}   seq\_left  }}}{{{\color[RGB]{181, 137, 0} := }}}{{{\color[RGB]{101, 123, 131}  \_,
 }}}\\
{{{\color[RGB]{101, 123, 131}   seq\_right  }}}{{{\color[RGB]{181, 137, 0} := }}}{{{\color[RGB]{101, 123, 131}  \_,
 }}}\\
{{{\color[RGB]{101, 123, 131}   bind  }}}{{{\color[RGB]{181, 137, 0} := }}}{{{\color[RGB]{101, 123, 131}  \_ \}
 }}}\\

}}\paragraph{tactic.match\_stub}
\par
Hole command used to generate a 
\colorbox[RGB]{253,246,227}{{{{\color[RGB]{133, 153, 0} match }}}} expression.
\par
In the following:
\\
\colorbox[RGB]{253,246,227}{\parbox{4.5in}{{{{\color[RGB]{133, 153, 0} meta }}}{{{\color[RGB]{101, 123, 131}   }}}{{{\color[RGB]{133, 153, 0} def }}}{{{\color[RGB]{101, 123, 131}   }}}{{{\color[RGB]{211, 54, 130} foo }}}{{{\color[RGB]{101, 123, 131}   }}}{{{\color[RGB]{101, 123, 131} (e : expr) : tactic unit  }}}{{{\color[RGB]{181, 137, 0} := }}}{{{\color[RGB]{101, 123, 131} 
 }}}\\
{{{\color[RGB]{101, 123, 131} \{! e !\}
 }}}\\

}}\par
invoking hole command 
\colorbox[RGB]{253,246,227}{{{{\color[RGB]{101, 123, 131} Match Stub }}}} produces:
\\
\colorbox[RGB]{253,246,227}{\parbox{4.5in}{{{{\color[RGB]{133, 153, 0} meta }}}{{{\color[RGB]{101, 123, 131}   }}}{{{\color[RGB]{133, 153, 0} def }}}{{{\color[RGB]{101, 123, 131}   }}}{{{\color[RGB]{211, 54, 130} foo }}}{{{\color[RGB]{101, 123, 131}   }}}{{{\color[RGB]{101, 123, 131} (e : expr) : tactic unit  }}}{{{\color[RGB]{181, 137, 0} := }}}{{{\color[RGB]{101, 123, 131} 
 }}}\\
{{{\color[RGB]{133, 153, 0} match }}}{{{\color[RGB]{101, 123, 131}  e  }}}{{{\color[RGB]{133, 153, 0} with }}}{{{\color[RGB]{101, 123, 131} 
 }}}\\
{{{\color[RGB]{101, 123, 131} | (expr.var a)  }}}{{{\color[RGB]{181, 137, 0} := }}}{{{\color[RGB]{101, 123, 131}  \_
 }}}\\
{{{\color[RGB]{101, 123, 131} | (expr.sort a)  }}}{{{\color[RGB]{181, 137, 0} := }}}{{{\color[RGB]{101, 123, 131}  \_
 }}}\\
{{{\color[RGB]{101, 123, 131} | (expr.const a a\_1)  }}}{{{\color[RGB]{181, 137, 0} := }}}{{{\color[RGB]{101, 123, 131}  \_
 }}}\\
{{{\color[RGB]{101, 123, 131} | (expr.mvar a a\_1 a\_2)  }}}{{{\color[RGB]{181, 137, 0} := }}}{{{\color[RGB]{101, 123, 131}  \_
 }}}\\
{{{\color[RGB]{101, 123, 131} | (expr.local\_const a a\_1 a\_2 a\_3)  }}}{{{\color[RGB]{181, 137, 0} := }}}{{{\color[RGB]{101, 123, 131}  \_
 }}}\\
{{{\color[RGB]{101, 123, 131} | (expr.app a a\_1)  }}}{{{\color[RGB]{181, 137, 0} := }}}{{{\color[RGB]{101, 123, 131}  \_
 }}}\\
{{{\color[RGB]{101, 123, 131} | (expr.lam a a\_1 a\_2 a\_3)  }}}{{{\color[RGB]{181, 137, 0} := }}}{{{\color[RGB]{101, 123, 131}  \_
 }}}\\
{{{\color[RGB]{101, 123, 131} | (expr.pi a a\_1 a\_2 a\_3)  }}}{{{\color[RGB]{181, 137, 0} := }}}{{{\color[RGB]{101, 123, 131}  \_
 }}}\\
{{{\color[RGB]{101, 123, 131} | (expr.elet a a\_1 a\_2 a\_3)  }}}{{{\color[RGB]{181, 137, 0} := }}}{{{\color[RGB]{101, 123, 131}  \_
 }}}\\
{{{\color[RGB]{101, 123, 131} | (expr.macro a a\_1)  }}}{{{\color[RGB]{181, 137, 0} := }}}{{{\color[RGB]{101, 123, 131}  \_
 }}}\\
{{{\color[RGB]{133, 153, 0} end }}}{{{\color[RGB]{101, 123, 131} 
 }}}\\

}}\paragraph{tactic.eqn\_stub}
\par
Hole command used to generate a 
\colorbox[RGB]{253,246,227}{{{{\color[RGB]{133, 153, 0} match }}}} expression.
\par
In the following:
\\
\colorbox[RGB]{253,246,227}{\parbox{4.5in}{{{{\color[RGB]{133, 153, 0} meta }}}{{{\color[RGB]{101, 123, 131}   }}}{{{\color[RGB]{133, 153, 0} def }}}{{{\color[RGB]{101, 123, 131}   }}}{{{\color[RGB]{211, 54, 130} foo }}}{{{\color[RGB]{101, 123, 131}   }}}{{{\color[RGB]{101, 123, 131} : \{! expr  }}}{{{\color[RGB]{133, 153, 0} → }}}{{{\color[RGB]{101, 123, 131}  tactic unit !\}  }}}{{{\color[RGB]{147, 161, 161} -{}- }}}{{{\color[RGB]{147, 161, 161}  `:=` is omitted }}}{{{\color[RGB]{101, 123, 131} 
 }}}\\

}}\par
invoking hole command 
\colorbox[RGB]{253,246,227}{{{{\color[RGB]{101, 123, 131} Equations Stub }}}} produces:
\\
\colorbox[RGB]{253,246,227}{\parbox{4.5in}{{{{\color[RGB]{133, 153, 0} meta }}}{{{\color[RGB]{101, 123, 131}   }}}{{{\color[RGB]{133, 153, 0} def }}}{{{\color[RGB]{101, 123, 131}   }}}{{{\color[RGB]{211, 54, 130} foo }}}{{{\color[RGB]{101, 123, 131}   }}}{{{\color[RGB]{101, 123, 131} : expr  }}}{{{\color[RGB]{133, 153, 0} → }}}{{{\color[RGB]{101, 123, 131}  tactic unit
 }}}\\
{{{\color[RGB]{101, 123, 131} | (expr.var a)  }}}{{{\color[RGB]{181, 137, 0} := }}}{{{\color[RGB]{101, 123, 131}  \_
 }}}\\
{{{\color[RGB]{101, 123, 131} | (expr.sort a)  }}}{{{\color[RGB]{181, 137, 0} := }}}{{{\color[RGB]{101, 123, 131}  \_
 }}}\\
{{{\color[RGB]{101, 123, 131} | (expr.const a a\_1)  }}}{{{\color[RGB]{181, 137, 0} := }}}{{{\color[RGB]{101, 123, 131}  \_
 }}}\\
{{{\color[RGB]{101, 123, 131} | (expr.mvar a a\_1 a\_2)  }}}{{{\color[RGB]{181, 137, 0} := }}}{{{\color[RGB]{101, 123, 131}  \_
 }}}\\
{{{\color[RGB]{101, 123, 131} | (expr.local\_const a a\_1 a\_2 a\_3)  }}}{{{\color[RGB]{181, 137, 0} := }}}{{{\color[RGB]{101, 123, 131}  \_
 }}}\\
{{{\color[RGB]{101, 123, 131} | (expr.app a a\_1)  }}}{{{\color[RGB]{181, 137, 0} := }}}{{{\color[RGB]{101, 123, 131}  \_
 }}}\\
{{{\color[RGB]{101, 123, 131} | (expr.lam a a\_1 a\_2 a\_3)  }}}{{{\color[RGB]{181, 137, 0} := }}}{{{\color[RGB]{101, 123, 131}  \_
 }}}\\
{{{\color[RGB]{101, 123, 131} | (expr.pi a a\_1 a\_2 a\_3)  }}}{{{\color[RGB]{181, 137, 0} := }}}{{{\color[RGB]{101, 123, 131}  \_
 }}}\\
{{{\color[RGB]{101, 123, 131} | (expr.elet a a\_1 a\_2 a\_3)  }}}{{{\color[RGB]{181, 137, 0} := }}}{{{\color[RGB]{101, 123, 131}  \_
 }}}\\
{{{\color[RGB]{101, 123, 131} | (expr.macro a a\_1)  }}}{{{\color[RGB]{181, 137, 0} := }}}{{{\color[RGB]{101, 123, 131}  \_
 }}}\\

}}\par
A similar result can be obtained by invoking 
\colorbox[RGB]{253,246,227}{{{{\color[RGB]{101, 123, 131} Equations Stub }}}} on the following:
\\
\colorbox[RGB]{253,246,227}{\parbox{4.5in}{{{{\color[RGB]{133, 153, 0} meta }}}{{{\color[RGB]{101, 123, 131}   }}}{{{\color[RGB]{133, 153, 0} def }}}{{{\color[RGB]{101, 123, 131}   }}}{{{\color[RGB]{211, 54, 130} foo }}}{{{\color[RGB]{101, 123, 131}   }}}{{{\color[RGB]{101, 123, 131} : expr  }}}{{{\color[RGB]{133, 153, 0} → }}}{{{\color[RGB]{101, 123, 131}  tactic unit  }}}{{{\color[RGB]{181, 137, 0} := }}}{{{\color[RGB]{101, 123, 131}   }}}{{{\color[RGB]{147, 161, 161} -{}- }}}{{{\color[RGB]{147, 161, 161}  do not forget to write `:=`!! }}}{{{\color[RGB]{101, 123, 131} 
 }}}\\
{{{\color[RGB]{101, 123, 131} \{! !\}
 }}}\\

}}\\
\colorbox[RGB]{253,246,227}{\parbox{4.5in}{{{{\color[RGB]{133, 153, 0} meta }}}{{{\color[RGB]{101, 123, 131}   }}}{{{\color[RGB]{133, 153, 0} def }}}{{{\color[RGB]{101, 123, 131}   }}}{{{\color[RGB]{211, 54, 130} foo }}}{{{\color[RGB]{101, 123, 131}   }}}{{{\color[RGB]{101, 123, 131} : expr  }}}{{{\color[RGB]{133, 153, 0} → }}}{{{\color[RGB]{101, 123, 131}  tactic unit  }}}{{{\color[RGB]{181, 137, 0} := }}}{{{\color[RGB]{101, 123, 131}   }}}{{{\color[RGB]{147, 161, 161} -{}- }}}{{{\color[RGB]{147, 161, 161}  don't forget to erase `:=`!! }}}{{{\color[RGB]{101, 123, 131} 
 }}}\\
{{{\color[RGB]{101, 123, 131} | (expr.var a)  }}}{{{\color[RGB]{181, 137, 0} := }}}{{{\color[RGB]{101, 123, 131}  \_
 }}}\\
{{{\color[RGB]{101, 123, 131} | (expr.sort a)  }}}{{{\color[RGB]{181, 137, 0} := }}}{{{\color[RGB]{101, 123, 131}  \_
 }}}\\
{{{\color[RGB]{101, 123, 131} | (expr.const a a\_1)  }}}{{{\color[RGB]{181, 137, 0} := }}}{{{\color[RGB]{101, 123, 131}  \_
 }}}\\
{{{\color[RGB]{101, 123, 131} | (expr.mvar a a\_1 a\_2)  }}}{{{\color[RGB]{181, 137, 0} := }}}{{{\color[RGB]{101, 123, 131}  \_
 }}}\\
{{{\color[RGB]{101, 123, 131} | (expr.local\_const a a\_1 a\_2 a\_3)  }}}{{{\color[RGB]{181, 137, 0} := }}}{{{\color[RGB]{101, 123, 131}  \_
 }}}\\
{{{\color[RGB]{101, 123, 131} | (expr.app a a\_1)  }}}{{{\color[RGB]{181, 137, 0} := }}}{{{\color[RGB]{101, 123, 131}  \_
 }}}\\
{{{\color[RGB]{101, 123, 131} | (expr.lam a a\_1 a\_2 a\_3)  }}}{{{\color[RGB]{181, 137, 0} := }}}{{{\color[RGB]{101, 123, 131}  \_
 }}}\\
{{{\color[RGB]{101, 123, 131} | (expr.pi a a\_1 a\_2 a\_3)  }}}{{{\color[RGB]{181, 137, 0} := }}}{{{\color[RGB]{101, 123, 131}  \_
 }}}\\
{{{\color[RGB]{101, 123, 131} | (expr.elet a a\_1 a\_2 a\_3)  }}}{{{\color[RGB]{181, 137, 0} := }}}{{{\color[RGB]{101, 123, 131}  \_
 }}}\\
{{{\color[RGB]{101, 123, 131} | (expr.macro a a\_1)  }}}{{{\color[RGB]{181, 137, 0} := }}}{{{\color[RGB]{101, 123, 131}  \_
 }}}\\

}}\paragraph{tactic.list\_constructors\_hole}
\par
This command lists the constructors that can be used to satisfy the expected type.
\par
When used in the following hole:
\\
\colorbox[RGB]{253,246,227}{\parbox{4.5in}{{{{\color[RGB]{133, 153, 0} def }}}{{{\color[RGB]{101, 123, 131}   }}}{{{\color[RGB]{211, 54, 130} foo }}}{{{\color[RGB]{101, 123, 131}   }}}{{{\color[RGB]{101, 123, 131} : ℤ ⊕ ℕ  }}}{{{\color[RGB]{181, 137, 0} := }}}{{{\color[RGB]{101, 123, 131} 
 }}}\\
{{{\color[RGB]{101, 123, 131} \{! !\}
 }}}\\

}}\par
the command will produce:
\\
\colorbox[RGB]{253,246,227}{\parbox{4.5in}{{{{\color[RGB]{133, 153, 0} def }}}{{{\color[RGB]{101, 123, 131}   }}}{{{\color[RGB]{211, 54, 130} foo }}}{{{\color[RGB]{101, 123, 131}   }}}{{{\color[RGB]{101, 123, 131} : ℤ ⊕ ℕ  }}}{{{\color[RGB]{181, 137, 0} := }}}{{{\color[RGB]{101, 123, 131} 
 }}}\\
{{{\color[RGB]{101, 123, 131} \{! sum.inl, sum.inr !\}
 }}}\\

}}\par
and will display:
\\
\colorbox[RGB]{253,246,227}{\parbox{4.5in}{{{{\color[RGB]{101, 123, 131} sum.inl : ℤ  }}}{{{\color[RGB]{133, 153, 0} → }}}{{{\color[RGB]{101, 123, 131}  ℤ ⊕ ℕ
 }}}\\
{{{\color[RGB]{101, 123, 131} 
 }}}\\
{{{\color[RGB]{101, 123, 131} sum.inr : ℕ  }}}{{{\color[RGB]{133, 153, 0} → }}}{{{\color[RGB]{101, 123, 131}  ℤ ⊕ ℕ
 }}}\\

}}\paragraph{tactic.success\_if\_fail\_with\_msg}
\par
This combinator is for testing purposes. It succeeds if 
\colorbox[RGB]{253,246,227}{{{{\color[RGB]{101, 123, 131} t }}}} fails with message 
\colorbox[RGB]{253,246,227}{{{{\color[RGB]{101, 123, 131} msg }}}},
and fails otherwise.
\section{tactic/default.lean}\section{tactic/elide.lean}\paragraph{tactic.interactive.elide}
\par
The 
\colorbox[RGB]{253,246,227}{{{{\color[RGB]{101, 123, 131} elide n ( }}}{{{\color[RGB]{133, 153, 0} at }}}{{{\color[RGB]{101, 123, 131}  ...) }}}} tactic hides all subterms of the target goal or hypotheses
beyond depth 
\colorbox[RGB]{253,246,227}{{{{\color[RGB]{101, 123, 131} n }}}} by replacing them with 
\colorbox[RGB]{253,246,227}{{{{\color[RGB]{101, 123, 131} hidden }}}}, which is a variant
on the identity function. (Tactics should still mostly be able to see
through the abbreviation, but if you want to unhide the term you can use
\colorbox[RGB]{253,246,227}{{{{\color[RGB]{101, 123, 131} unelide }}}}.)
\paragraph{tactic.interactive.unelide}
\par
The 
\colorbox[RGB]{253,246,227}{{{{\color[RGB]{101, 123, 131} unelide ( }}}{{{\color[RGB]{133, 153, 0} at }}}{{{\color[RGB]{101, 123, 131}  ...) }}}} tactic removes all 
\colorbox[RGB]{253,246,227}{{{{\color[RGB]{101, 123, 131} hidden }}}} subterms in the target
types (usually added by 
\colorbox[RGB]{253,246,227}{{{{\color[RGB]{101, 123, 131} elide }}}}).
\section{tactic/explode.lean}\section{tactic/ext.lean}\paragraph{extensional\_attribute}
\par
Tag lemmas of the form:
\\
\colorbox[RGB]{253,246,227}{\parbox{4.5in}{{{{\color[RGB]{88, 110, 117} @{[}extensionality{]} }}}{{{\color[RGB]{101, 123, 131} 
 }}}\\
{{{\color[RGB]{133, 153, 0} lemma }}}{{{\color[RGB]{101, 123, 131}   }}}{{{\color[RGB]{211, 54, 130} my\_collection.ext }}}{{{\color[RGB]{101, 123, 131}   }}}{{{\color[RGB]{101, 123, 131} (a b : my\_collection)
 }}}\\
{{{\color[RGB]{101, 123, 131}   (h :  }}}{{{\color[RGB]{133, 153, 0} ∀ }}}{{{\color[RGB]{101, 123, 131}  x, a.lookup x  }}}{{{\color[RGB]{181, 137, 0} = }}}{{{\color[RGB]{101, 123, 131}  b.lookup y) :
 }}}\\
{{{\color[RGB]{101, 123, 131}   a  }}}{{{\color[RGB]{181, 137, 0} = }}}{{{\color[RGB]{101, 123, 131}  b  }}}{{{\color[RGB]{181, 137, 0} := }}}{{{\color[RGB]{101, 123, 131}  ...
 }}}\\

}}\par
The attribute indexes extensionality lemma using the type of the
objects (i.e. 
\colorbox[RGB]{253,246,227}{{{{\color[RGB]{101, 123, 131} my\_collection }}}}) which it gets from the statement of
the lemma.  In some cases, the same lemma can be used to state the
extensionality of multiple types that are definitionally equivalent.
\\
\colorbox[RGB]{253,246,227}{\parbox{4.5in}{{{{\color[RGB]{88, 110, 117} attribute {[}extensionality {[}(→),thunk,stream{]} }}}{{{\color[RGB]{101, 123, 131} {]} funext
 }}}\\

}}\par
Those parameters are cumulative. The following are equivalent:
\\
\colorbox[RGB]{253,246,227}{\parbox{4.5in}{{{{\color[RGB]{88, 110, 117} attribute {[}extensionality {[}(→),thunk{]} }}}{{{\color[RGB]{101, 123, 131} {]} funext
 }}}\\
{{{\color[RGB]{88, 110, 117} attribute {[}extensionality {[}stream{]} }}}{{{\color[RGB]{101, 123, 131} {]} funext
 }}}\\

}}\par
and
\\
\colorbox[RGB]{253,246,227}{\parbox{4.5in}{{{{\color[RGB]{88, 110, 117} attribute {[}extensionality {[}(→),thunk,stream{]} }}}{{{\color[RGB]{101, 123, 131} {]} funext
 }}}\\

}}\par
One removes type names from the list for one lemma with:
\\
\colorbox[RGB]{253,246,227}{\parbox{4.5in}{{{{\color[RGB]{88, 110, 117} attribute {[}extensionality {[}-stream,-thunk{]} }}}{{{\color[RGB]{101, 123, 131} {]} funext
 }}}\\

}}\par
Finally, the following:
\\
\colorbox[RGB]{253,246,227}{\parbox{4.5in}{{{{\color[RGB]{88, 110, 117} @{[}extensionality{]} }}}{{{\color[RGB]{101, 123, 131} 
 }}}\\
{{{\color[RGB]{133, 153, 0} lemma }}}{{{\color[RGB]{101, 123, 131}   }}}{{{\color[RGB]{211, 54, 130} my\_collection.ext }}}{{{\color[RGB]{101, 123, 131}   }}}{{{\color[RGB]{101, 123, 131} (a b : my\_collection)
 }}}\\
{{{\color[RGB]{101, 123, 131}   (h :  }}}{{{\color[RGB]{133, 153, 0} ∀ }}}{{{\color[RGB]{101, 123, 131}  x, a.lookup x  }}}{{{\color[RGB]{181, 137, 0} = }}}{{{\color[RGB]{101, 123, 131}  b.lookup y) :
 }}}\\
{{{\color[RGB]{101, 123, 131}   a  }}}{{{\color[RGB]{181, 137, 0} = }}}{{{\color[RGB]{101, 123, 131}  b  }}}{{{\color[RGB]{181, 137, 0} := }}}{{{\color[RGB]{101, 123, 131}  ...
 }}}\\

}}\par
is equivalent to
\\
\colorbox[RGB]{253,246,227}{\parbox{4.5in}{{{{\color[RGB]{88, 110, 117} @{[}extensionality *{]} }}}{{{\color[RGB]{101, 123, 131} 
 }}}\\
{{{\color[RGB]{133, 153, 0} lemma }}}{{{\color[RGB]{101, 123, 131}   }}}{{{\color[RGB]{211, 54, 130} my\_collection.ext }}}{{{\color[RGB]{101, 123, 131}   }}}{{{\color[RGB]{101, 123, 131} (a b : my\_collection)
 }}}\\
{{{\color[RGB]{101, 123, 131}   (h :  }}}{{{\color[RGB]{133, 153, 0} ∀ }}}{{{\color[RGB]{101, 123, 131}  x, a.lookup x  }}}{{{\color[RGB]{181, 137, 0} = }}}{{{\color[RGB]{101, 123, 131}  b.lookup y) :
 }}}\\
{{{\color[RGB]{101, 123, 131}   a  }}}{{{\color[RGB]{181, 137, 0} = }}}{{{\color[RGB]{101, 123, 131}  b  }}}{{{\color[RGB]{181, 137, 0} := }}}{{{\color[RGB]{101, 123, 131}  ...
 }}}\\

}}\par
This allows us specify type synonyms along with the type
that referred to in the lemma statement.
\\
\colorbox[RGB]{253,246,227}{\parbox{4.5in}{{{{\color[RGB]{88, 110, 117} @{[}extensionality {[}*,my\_type\_synonym{]} }}}{{{\color[RGB]{101, 123, 131} {]}
 }}}\\
{{{\color[RGB]{133, 153, 0} lemma }}}{{{\color[RGB]{101, 123, 131}   }}}{{{\color[RGB]{211, 54, 130} my\_collection.ext }}}{{{\color[RGB]{101, 123, 131}   }}}{{{\color[RGB]{101, 123, 131} (a b : my\_collection)
 }}}\\
{{{\color[RGB]{101, 123, 131}   (h :  }}}{{{\color[RGB]{133, 153, 0} ∀ }}}{{{\color[RGB]{101, 123, 131}  x, a.lookup x  }}}{{{\color[RGB]{181, 137, 0} = }}}{{{\color[RGB]{101, 123, 131}  b.lookup y) :
 }}}\\
{{{\color[RGB]{101, 123, 131}   a  }}}{{{\color[RGB]{181, 137, 0} = }}}{{{\color[RGB]{101, 123, 131}  b  }}}{{{\color[RGB]{181, 137, 0} := }}}{{{\color[RGB]{101, 123, 131}  ...
 }}}\\

}}\paragraph{tactic.interactive.ext1}
\par
\colorbox[RGB]{253,246,227}{{{{\color[RGB]{101, 123, 131} ext1 id }}}} selects and apply one extensionality lemma (with attribute
\colorbox[RGB]{253,246,227}{{{{\color[RGB]{101, 123, 131} extensionality }}}}), using 
\colorbox[RGB]{253,246,227}{{{{\color[RGB]{101, 123, 131} id }}}}, if provided, to name a local constant
introduced by the lemma. If 
\colorbox[RGB]{253,246,227}{{{{\color[RGB]{101, 123, 131} id }}}} is omitted, the local constant is
named automatically, as per 
\colorbox[RGB]{253,246,227}{{{{\color[RGB]{101, 123, 131} intro }}}}.
\paragraph{tactic.interactive.ext}
\begin{itemize}\item \colorbox[RGB]{253,246,227}{{{{\color[RGB]{101, 123, 131} ext }}}} applies as many extensionality lemmas as possible;

\item \colorbox[RGB]{253,246,227}{{{{\color[RGB]{101, 123, 131} ext ids }}}}, with 
\colorbox[RGB]{253,246,227}{{{{\color[RGB]{101, 123, 131} ids }}}} a list of identifiers, finds extentionality and applies them
until it runs out of identifiers in 
\colorbox[RGB]{253,246,227}{{{{\color[RGB]{101, 123, 131} ids }}}} to name the local constants.

\end{itemize}\par
When trying to prove:
\\
\colorbox[RGB]{253,246,227}{\parbox{4.5in}{{{{\color[RGB]{101, 123, 131} α β :  }}}{{{\color[RGB]{38, 139, 210} Type }}}{{{\color[RGB]{101, 123, 131} ,
 }}}\\
{{{\color[RGB]{101, 123, 131} f g : α  }}}{{{\color[RGB]{133, 153, 0} → }}}{{{\color[RGB]{101, 123, 131}  set β
 }}}\\
{{{\color[RGB]{101, 123, 131} ⊢ f  }}}{{{\color[RGB]{181, 137, 0} = }}}{{{\color[RGB]{101, 123, 131}  g
 }}}\\

}}\par
applying 
\colorbox[RGB]{253,246,227}{{{{\color[RGB]{101, 123, 131} ext x y }}}} yields:
\\
\colorbox[RGB]{253,246,227}{\parbox{4.5in}{{{{\color[RGB]{101, 123, 131} α β :  }}}{{{\color[RGB]{38, 139, 210} Type }}}{{{\color[RGB]{101, 123, 131} ,
 }}}\\
{{{\color[RGB]{101, 123, 131} f g : α  }}}{{{\color[RGB]{133, 153, 0} → }}}{{{\color[RGB]{101, 123, 131}  set β,
 }}}\\
{{{\color[RGB]{101, 123, 131} x : α,
 }}}\\
{{{\color[RGB]{101, 123, 131} y : β
 }}}\\
{{{\color[RGB]{101, 123, 131} ⊢ y ∈ f x  }}}{{{\color[RGB]{181, 137, 0} ↔ }}}{{{\color[RGB]{101, 123, 131}  y ∈ f x
 }}}\\

}}\par
by applying functional extensionality and set extensionality.
\par
A maximum depth can be provided with 
\colorbox[RGB]{253,246,227}{{{{\color[RGB]{101, 123, 131} ext x y z :  }}}{{{\color[RGB]{108, 113, 196} 3 }}}}.
\section{tactic/fin\_cases.lean}\paragraph{tactic.guard\_mem\_fin}
\par
Checks that the expression looks like 
\colorbox[RGB]{253,246,227}{{{{\color[RGB]{101, 123, 131} x ∈ A }}}} for 
\colorbox[RGB]{253,246,227}{{{{\color[RGB]{101, 123, 131} A : finset α }}}}, 
\colorbox[RGB]{253,246,227}{{{{\color[RGB]{101, 123, 131} multiset α }}}} or 
\colorbox[RGB]{253,246,227}{{{{\color[RGB]{101, 123, 131} A : list α }}}},
and returns the type α.
\paragraph{tactic.interactive.fin\_cases}
\par
\colorbox[RGB]{253,246,227}{{{{\color[RGB]{101, 123, 131} fin\_cases h }}}} performs case analysis on a hypothesis of the form
\colorbox[RGB]{253,246,227}{{{{\color[RGB]{101, 123, 131} h : A }}}}, where 
\colorbox[RGB]{253,246,227}{{{{\color[RGB]{101, 123, 131} {[}fintype A{]} }}}} is available, or
\colorbox[RGB]{253,246,227}{{{{\color[RGB]{101, 123, 131} h ∈ A }}}}, where 
\colorbox[RGB]{253,246,227}{{{{\color[RGB]{101, 123, 131} A : finset X }}}}, 
\colorbox[RGB]{253,246,227}{{{{\color[RGB]{101, 123, 131} A : multiset X }}}} or 
\colorbox[RGB]{253,246,227}{{{{\color[RGB]{101, 123, 131} A : list X }}}}.
\par
\colorbox[RGB]{253,246,227}{{{{\color[RGB]{101, 123, 131} fin\_cases  }}}{{{\color[RGB]{181, 137, 0} * }}}} performs case analysis on all suitable hypotheses.
\par
As an example, in
\\
\colorbox[RGB]{253,246,227}{\parbox{4.5in}{{{{\color[RGB]{133, 153, 0} example }}}{{{\color[RGB]{101, 123, 131}  (f : ℕ  }}}{{{\color[RGB]{133, 153, 0} → }}}{{{\color[RGB]{101, 123, 131}   }}}{{{\color[RGB]{38, 139, 210} Prop }}}{{{\color[RGB]{101, 123, 131} ) (p : fin  }}}{{{\color[RGB]{108, 113, 196} 3 }}}{{{\color[RGB]{101, 123, 131} ) (h0 : f  }}}{{{\color[RGB]{108, 113, 196} 0 }}}{{{\color[RGB]{101, 123, 131} ) (h1 : f  }}}{{{\color[RGB]{108, 113, 196} 1 }}}{{{\color[RGB]{101, 123, 131} ) (h2 : f  }}}{{{\color[RGB]{108, 113, 196} 2 }}}{{{\color[RGB]{101, 123, 131} ) : f p.val  }}}{{{\color[RGB]{181, 137, 0} := }}}{{{\color[RGB]{101, 123, 131} 
 }}}\\
{{{\color[RGB]{133, 153, 0} begin }}}{{{\color[RGB]{101, 123, 131} 
 }}}\\
{{{\color[RGB]{101, 123, 131}   fin\_cases  }}}{{{\color[RGB]{181, 137, 0} * }}}{{{\color[RGB]{101, 123, 131} ; simp,
 }}}\\
{{{\color[RGB]{101, 123, 131}   all\_goals \{ assumption \}
 }}}\\
{{{\color[RGB]{133, 153, 0} end }}}{{{\color[RGB]{101, 123, 131} 
 }}}\\

}}\par
after 
\colorbox[RGB]{253,246,227}{{{{\color[RGB]{101, 123, 131} fin\_cases p; simp }}}}, there are three goals, 
\colorbox[RGB]{253,246,227}{{{{\color[RGB]{101, 123, 131} f  }}}{{{\color[RGB]{108, 113, 196} 0 }}}}, 
\colorbox[RGB]{253,246,227}{{{{\color[RGB]{101, 123, 131} f  }}}{{{\color[RGB]{108, 113, 196} 1 }}}}, and 
\colorbox[RGB]{253,246,227}{{{{\color[RGB]{101, 123, 131} f  }}}{{{\color[RGB]{108, 113, 196} 2 }}}}.
\section{tactic/find.lean}\section{tactic/finish.lean}\section{tactic/generalize\_proofs.lean}\paragraph{tactic.interactive.generalize\_proofs}
\par
Generalize proofs in the goal, naming them with the provided list.
\section{tactic/interactive.lean}\paragraph{tactic.interactive.try\_for}
\par
\colorbox[RGB]{253,246,227}{{{{\color[RGB]{101, 123, 131} try\_for n \{ tac \} }}}} executes 
\colorbox[RGB]{253,246,227}{{{{\color[RGB]{101, 123, 131} tac }}}} for 
\colorbox[RGB]{253,246,227}{{{{\color[RGB]{101, 123, 131} n }}}} ticks, otherwise uses 
\colorbox[RGB]{253,246,227}{{{{\color[RGB]{101, 123, 131} sorry }}}} to close the goal.
Never fails. Useful for debugging.
\paragraph{tactic.interactive.substs}
\par
Multiple subst. 
\colorbox[RGB]{253,246,227}{{{{\color[RGB]{101, 123, 131} substs x y z }}}} is the same as 
\colorbox[RGB]{253,246,227}{{{{\color[RGB]{101, 123, 131} subst x, subst y, subst z }}}}.
\paragraph{tactic.interactive.unfold\_coes}
\par
Unfold coercion-related definitions
\paragraph{tactic.interactive.unfold\_aux}
\par
Unfold auxiliary definitions associated with the current declaration.
\paragraph{tactic.interactive.recover}
\par
For debugging only. This tactic checks the current state for any
missing dropped goals and restores them. Useful when there are no
goals to solve but "result contains meta-variables".
\paragraph{tactic.interactive.continue}
\par
Like 
\colorbox[RGB]{253,246,227}{{{{\color[RGB]{101, 123, 131} try \{ tac \} }}}}, but in the case of failure it continues
from the failure state instead of reverting to the original state.
\paragraph{tactic.interactive.swap}
\par
Move goal 
\colorbox[RGB]{253,246,227}{{{{\color[RGB]{101, 123, 131} n }}}} to the front.
\paragraph{tactic.interactive.clear\_}
\par
Clear all hypotheses starting with 
\colorbox[RGB]{253,246,227}{{{{\color[RGB]{101, 123, 131} \_ }}}}, like 
\colorbox[RGB]{253,246,227}{{{{\color[RGB]{101, 123, 131} \_match }}}} and 
\colorbox[RGB]{253,246,227}{{{{\color[RGB]{101, 123, 131} \_let\_match }}}}.
\paragraph{tactic.interactive.congr'}
\par
Same as the 
\colorbox[RGB]{253,246,227}{{{{\color[RGB]{101, 123, 131} congr }}}} tactic, but takes an optional argument which gives
the depth of recursive applications. This is useful when 
\colorbox[RGB]{253,246,227}{{{{\color[RGB]{101, 123, 131} congr }}}}is too aggressive in breaking down the goal. For example, given
\colorbox[RGB]{253,246,227}{{{{\color[RGB]{101, 123, 131} ⊢ f (g (x  }}}{{{\color[RGB]{181, 137, 0} + }}}{{{\color[RGB]{101, 123, 131}  y))  }}}{{{\color[RGB]{181, 137, 0} = }}}{{{\color[RGB]{101, 123, 131}  f (g (y  }}}{{{\color[RGB]{181, 137, 0} + }}}{{{\color[RGB]{101, 123, 131}  x)) }}}}, 
\colorbox[RGB]{253,246,227}{{{{\color[RGB]{101, 123, 131} congr' }}}} produces the goals 
\colorbox[RGB]{253,246,227}{{{{\color[RGB]{101, 123, 131} ⊢ x  }}}{{{\color[RGB]{181, 137, 0} = }}}{{{\color[RGB]{101, 123, 131}  y }}}}and 
\colorbox[RGB]{253,246,227}{{{{\color[RGB]{101, 123, 131} ⊢ y  }}}{{{\color[RGB]{181, 137, 0} = }}}{{{\color[RGB]{101, 123, 131}  x }}}}, while 
\colorbox[RGB]{253,246,227}{{{{\color[RGB]{101, 123, 131} congr'  }}}{{{\color[RGB]{108, 113, 196} 2 }}}} produces the intended 
\colorbox[RGB]{253,246,227}{{{{\color[RGB]{101, 123, 131} ⊢ x  }}}{{{\color[RGB]{181, 137, 0} + }}}{{{\color[RGB]{101, 123, 131}  y  }}}{{{\color[RGB]{181, 137, 0} = }}}{{{\color[RGB]{101, 123, 131}  y  }}}{{{\color[RGB]{181, 137, 0} + }}}{{{\color[RGB]{101, 123, 131}  x }}}}.
\paragraph{tactic.interactive.replace}
\par
Acts like 
\colorbox[RGB]{253,246,227}{{{{\color[RGB]{133, 153, 0} have }}}}, but removes a hypothesis with the same name as
this one. For example if the state is 
\colorbox[RGB]{253,246,227}{{{{\color[RGB]{101, 123, 131} h : p ⊢ goal }}}} and 
\colorbox[RGB]{253,246,227}{{{{\color[RGB]{101, 123, 131} f : p  }}}{{{\color[RGB]{133, 153, 0} → }}}{{{\color[RGB]{101, 123, 131}  q }}}},
then after 
\colorbox[RGB]{253,246,227}{{{{\color[RGB]{101, 123, 131} replace h  }}}{{{\color[RGB]{181, 137, 0} := }}}{{{\color[RGB]{101, 123, 131}  f h }}}} the goal will be 
\colorbox[RGB]{253,246,227}{{{{\color[RGB]{101, 123, 131} h : q ⊢ goal }}}},
where 
\colorbox[RGB]{253,246,227}{{{{\color[RGB]{133, 153, 0} have }}}{{{\color[RGB]{101, 123, 131}  h  }}}{{{\color[RGB]{181, 137, 0} := }}}{{{\color[RGB]{101, 123, 131}  f h }}}} would result in the state 
\colorbox[RGB]{253,246,227}{{{{\color[RGB]{101, 123, 131} h : p, h : q ⊢ goal }}}}.
This can be used to simulate the 
\colorbox[RGB]{253,246,227}{{{{\color[RGB]{101, 123, 131} specialize }}}} and 
\colorbox[RGB]{253,246,227}{{{{\color[RGB]{101, 123, 131} apply  }}}{{{\color[RGB]{133, 153, 0} at }}}} tactics
of Coq.
\paragraph{tactic.interactive.classical}
\par
Make every propositions in the context decidable
\paragraph{tactic.interactive.generalize\_hyp}
\par
Like 
\colorbox[RGB]{253,246,227}{{{{\color[RGB]{101, 123, 131} generalize }}}} but also considers assumptions
specified by the user. The user can also specify to
omit the goal.
\paragraph{tactic.interactive.convert}
\par
Similar to 
\colorbox[RGB]{253,246,227}{{{{\color[RGB]{101, 123, 131} refine }}}} but generates equality proof obligations
for every discrepancy between the goal and the type of the rule.
\paragraph{tactic.interactive.clean}
\par
Remove identity functions from a term. These are normally
automatically generated with terms like 
\colorbox[RGB]{253,246,227}{{{{\color[RGB]{133, 153, 0} show }}}{{{\color[RGB]{101, 123, 131}  t,  }}}{{{\color[RGB]{133, 153, 0} from }}}{{{\color[RGB]{101, 123, 131}  p }}}} or
\colorbox[RGB]{253,246,227}{{{{\color[RGB]{101, 123, 131} (p : t) }}}} which translate to some variant on 
\colorbox[RGB]{253,246,227}{{{{\color[RGB]{181, 137, 0} @ }}}{{{\color[RGB]{101, 123, 131} id t p }}}} in
order to retain the type.
\paragraph{tactic.interactive.refine\_struct}
\par
\colorbox[RGB]{253,246,227}{{{{\color[RGB]{101, 123, 131} refine\_struct \{ .. \} }}}} acts like 
\colorbox[RGB]{253,246,227}{{{{\color[RGB]{101, 123, 131} refine }}}} but works only with structure instance
literals. It creates a goal for each missing field and tags it with the name of the
field so that 
\colorbox[RGB]{253,246,227}{{{{\color[RGB]{101, 123, 131} have\_field }}}} can be used to generically refer to the field currently
being refined.
\par
As an example, we can use 
\colorbox[RGB]{253,246,227}{{{{\color[RGB]{101, 123, 131} refine\_struct }}}} to automate the construction semigroup
instances:
\\
\colorbox[RGB]{253,246,227}{\parbox{4.5in}{{{{\color[RGB]{101, 123, 131} refine\_struct ( \{ .. \} : semigroup α ),
 }}}\\
{{{\color[RGB]{147, 161, 161} -{}- }}}{{{\color[RGB]{147, 161, 161}  case semigroup, mul }}}{{{\color[RGB]{101, 123, 131} 
 }}}\\
{{{\color[RGB]{147, 161, 161} -{}- }}}{{{\color[RGB]{147, 161, 161}  α : Type u, }}}{{{\color[RGB]{101, 123, 131} 
 }}}\\
{{{\color[RGB]{147, 161, 161} -{}- }}}{{{\color[RGB]{147, 161, 161}  ⊢ α → α → α }}}{{{\color[RGB]{101, 123, 131} 
 }}}\\
{{{\color[RGB]{101, 123, 131} 
 }}}\\
{{{\color[RGB]{147, 161, 161} -{}- }}}{{{\color[RGB]{147, 161, 161}  case semigroup, mul\_assoc }}}{{{\color[RGB]{101, 123, 131} 
 }}}\\
{{{\color[RGB]{147, 161, 161} -{}- }}}{{{\color[RGB]{147, 161, 161}  α : Type u, }}}{{{\color[RGB]{101, 123, 131} 
 }}}\\
{{{\color[RGB]{147, 161, 161} -{}- }}}{{{\color[RGB]{147, 161, 161}  ⊢ ∀ (a b c : α), a * b * c = a * (b * c) }}}{{{\color[RGB]{101, 123, 131} 
 }}}\\

}}\paragraph{tactic.interactive.guard\_hyp'}
\par
\colorbox[RGB]{253,246,227}{{{{\color[RGB]{101, 123, 131} guard\_hyp h  }}}{{{\color[RGB]{181, 137, 0} := }}}{{{\color[RGB]{101, 123, 131}  t }}}} fails if the hypothesis 
\colorbox[RGB]{253,246,227}{{{{\color[RGB]{101, 123, 131} h }}}} does not have type 
\colorbox[RGB]{253,246,227}{{{{\color[RGB]{101, 123, 131} t }}}}.
We use this tactic for writing tests.
Fixes 
\colorbox[RGB]{253,246,227}{{{{\color[RGB]{101, 123, 131} guard\_hyp }}}} by instantiating meta variables
\paragraph{tactic.interactive.guard\_expr\_strict}
\par
\colorbox[RGB]{253,246,227}{{{{\color[RGB]{101, 123, 131} guard\_expr\_strict t  }}}{{{\color[RGB]{181, 137, 0} := }}}{{{\color[RGB]{101, 123, 131}  e }}}} fails if the expr 
\colorbox[RGB]{253,246,227}{{{{\color[RGB]{101, 123, 131} t }}}} is not equal to 
\colorbox[RGB]{253,246,227}{{{{\color[RGB]{101, 123, 131} e }}}}. By contrast
to 
\colorbox[RGB]{253,246,227}{{{{\color[RGB]{101, 123, 131} guard\_expr }}}}, this tests strict (syntactic) equality.
We use this tactic for writing tests.
\paragraph{tactic.interactive.guard\_target\_strict}
\par
\colorbox[RGB]{253,246,227}{{{{\color[RGB]{101, 123, 131} guard\_target\_strict t }}}} fails if the target of the main goal is not syntactically 
\colorbox[RGB]{253,246,227}{{{{\color[RGB]{101, 123, 131} t }}}}.
We use this tactic for writing tests.
\paragraph{tactic.interactive.guard\_hyp\_strict}
\par
\colorbox[RGB]{253,246,227}{{{{\color[RGB]{101, 123, 131} guard\_hyp\_strict h  }}}{{{\color[RGB]{181, 137, 0} := }}}{{{\color[RGB]{101, 123, 131}  t }}}} fails if the hypothesis 
\colorbox[RGB]{253,246,227}{{{{\color[RGB]{101, 123, 131} h }}}} does not have type syntactically equal
to 
\colorbox[RGB]{253,246,227}{{{{\color[RGB]{101, 123, 131} t }}}}.
We use this tactic for writing tests.
\paragraph{tactic.interactive.success\_if\_fail\_with\_msg}
\par
\colorbox[RGB]{253,246,227}{{{{\color[RGB]{101, 123, 131} success\_if\_fail\_with\_msg \{ tac \} msg }}}} succeeds if the interactive tactic 
\colorbox[RGB]{253,246,227}{{{{\color[RGB]{101, 123, 131} tac }}}} fails with
error message 
\colorbox[RGB]{253,246,227}{{{{\color[RGB]{101, 123, 131} msg }}}} (for test writing purposes).
\paragraph{tactic.interactive.have\_field}
\par
\colorbox[RGB]{253,246,227}{{{{\color[RGB]{101, 123, 131} have\_field }}}}, used after 
\colorbox[RGB]{253,246,227}{{{{\color[RGB]{101, 123, 131} refine\_struct \_ }}}} poses 
\colorbox[RGB]{253,246,227}{{{{\color[RGB]{101, 123, 131} field }}}} as a local constant
with the type of the field of the current goal:
\\
\colorbox[RGB]{253,246,227}{\parbox{4.5in}{{{{\color[RGB]{101, 123, 131} refine\_struct (\{ .. \} : semigroup α),
 }}}\\
{{{\color[RGB]{101, 123, 131} \{ have\_field, ... \},
 }}}\\
{{{\color[RGB]{101, 123, 131} \{ have\_field, ... \},
 }}}\\

}}\par
behaves like
\\
\colorbox[RGB]{253,246,227}{\parbox{4.5in}{{{{\color[RGB]{101, 123, 131} refine\_struct (\{ .. \} : semigroup α),
 }}}\\
{{{\color[RGB]{101, 123, 131} \{  }}}{{{\color[RGB]{133, 153, 0} have }}}{{{\color[RGB]{101, 123, 131}  field  }}}{{{\color[RGB]{181, 137, 0} := }}}{{{\color[RGB]{101, 123, 131}   }}}{{{\color[RGB]{181, 137, 0} @ }}}{{{\color[RGB]{101, 123, 131} semigroup.mul, ... \},
 }}}\\
{{{\color[RGB]{101, 123, 131} \{  }}}{{{\color[RGB]{133, 153, 0} have }}}{{{\color[RGB]{101, 123, 131}  field  }}}{{{\color[RGB]{181, 137, 0} := }}}{{{\color[RGB]{101, 123, 131}   }}}{{{\color[RGB]{181, 137, 0} @ }}}{{{\color[RGB]{101, 123, 131} semigroup.mul\_assoc, ... \},
 }}}\\

}}\paragraph{tactic.interactive.apply\_field}
\par
\colorbox[RGB]{253,246,227}{{{{\color[RGB]{101, 123, 131} apply\_field }}}} functions as 
\colorbox[RGB]{253,246,227}{{{{\color[RGB]{101, 123, 131} have\_field, apply field, clear field }}}}\paragraph{tactic.interactive.apply\_rules}
\par
\colorbox[RGB]{253,246,227}{{{{\color[RGB]{101, 123, 131} apply\_rules hs n }}}}: apply the list of rules 
\colorbox[RGB]{253,246,227}{{{{\color[RGB]{101, 123, 131} hs }}}} (given as pexpr) and 
\colorbox[RGB]{253,246,227}{{{{\color[RGB]{101, 123, 131} assumption }}}} on the
first goal and the resulting subgoals, iteratively, at most 
\colorbox[RGB]{253,246,227}{{{{\color[RGB]{101, 123, 131} n }}}} times.
\colorbox[RGB]{253,246,227}{{{{\color[RGB]{101, 123, 131} n }}}} is 50 by default. 
\colorbox[RGB]{253,246,227}{{{{\color[RGB]{101, 123, 131} hs }}}} can contain user attributes: in this case all theorems with this
attribute are added to the list of rules.
\par
example, with or without user attribute:
\\
\colorbox[RGB]{253,246,227}{\parbox{4.5in}{{{{\color[RGB]{88, 110, 117} @{[}user\_attribute{]} }}}{{{\color[RGB]{101, 123, 131} 
 }}}\\
{{{\color[RGB]{133, 153, 0} meta }}}{{{\color[RGB]{101, 123, 131}   }}}{{{\color[RGB]{133, 153, 0} def }}}{{{\color[RGB]{101, 123, 131}   }}}{{{\color[RGB]{211, 54, 130} mono\_rules }}}{{{\color[RGB]{101, 123, 131}   }}}{{{\color[RGB]{101, 123, 131} : user\_attribute  }}}{{{\color[RGB]{181, 137, 0} := }}}{{{\color[RGB]{101, 123, 131} 
 }}}\\
{{{\color[RGB]{101, 123, 131} \{ name  }}}{{{\color[RGB]{181, 137, 0} := }}}{{{\color[RGB]{101, 123, 131}  `mono\_rules,
 }}}\\
{{{\color[RGB]{101, 123, 131}   descr  }}}{{{\color[RGB]{181, 137, 0} := }}}{{{\color[RGB]{101, 123, 131}   }}}{{{\color[RGB]{42, 161, 152} "lemmas usable to prove monotonicity" }}}{{{\color[RGB]{101, 123, 131}  \}
 }}}\\
{{{\color[RGB]{101, 123, 131} 
 }}}\\
{{{\color[RGB]{88, 110, 117} attribute {[}mono\_rules{]} }}}{{{\color[RGB]{101, 123, 131}  add\_le\_add mul\_le\_mul\_of\_nonneg\_right
 }}}\\
{{{\color[RGB]{101, 123, 131} 
 }}}\\
{{{\color[RGB]{133, 153, 0} lemma }}}{{{\color[RGB]{101, 123, 131}   }}}{{{\color[RGB]{211, 54, 130} my\_test }}}{{{\color[RGB]{101, 123, 131}   }}}{{{\color[RGB]{101, 123, 131} \{a b c d e : real\} (h1 : a  }}}{{{\color[RGB]{181, 137, 0} ≤ }}}{{{\color[RGB]{101, 123, 131}  b) (h2 : c  }}}{{{\color[RGB]{181, 137, 0} ≤ }}}{{{\color[RGB]{101, 123, 131}  d) (h3 :  }}}{{{\color[RGB]{108, 113, 196} 0 }}}{{{\color[RGB]{101, 123, 131}   }}}{{{\color[RGB]{181, 137, 0} ≤ }}}{{{\color[RGB]{101, 123, 131}  e) :
 }}}\\
{{{\color[RGB]{101, 123, 131} a  }}}{{{\color[RGB]{181, 137, 0} + }}}{{{\color[RGB]{101, 123, 131}  c  }}}{{{\color[RGB]{181, 137, 0} * }}}{{{\color[RGB]{101, 123, 131}  e  }}}{{{\color[RGB]{181, 137, 0} + }}}{{{\color[RGB]{101, 123, 131}  a  }}}{{{\color[RGB]{181, 137, 0} + }}}{{{\color[RGB]{101, 123, 131}  c  }}}{{{\color[RGB]{181, 137, 0} + }}}{{{\color[RGB]{101, 123, 131}   }}}{{{\color[RGB]{108, 113, 196} 0 }}}{{{\color[RGB]{101, 123, 131}   }}}{{{\color[RGB]{181, 137, 0} ≤ }}}{{{\color[RGB]{101, 123, 131}  b  }}}{{{\color[RGB]{181, 137, 0} + }}}{{{\color[RGB]{101, 123, 131}  d  }}}{{{\color[RGB]{181, 137, 0} * }}}{{{\color[RGB]{101, 123, 131}  e  }}}{{{\color[RGB]{181, 137, 0} + }}}{{{\color[RGB]{101, 123, 131}  b  }}}{{{\color[RGB]{181, 137, 0} + }}}{{{\color[RGB]{101, 123, 131}  d  }}}{{{\color[RGB]{181, 137, 0} + }}}{{{\color[RGB]{101, 123, 131}  e  }}}{{{\color[RGB]{181, 137, 0} := }}}{{{\color[RGB]{101, 123, 131} 
 }}}\\
{{{\color[RGB]{133, 153, 0} by }}}{{{\color[RGB]{101, 123, 131}  apply\_rules mono\_rules
 }}}\\
{{{\color[RGB]{147, 161, 161} -{}- }}}{{{\color[RGB]{147, 161, 161}  any of the following lines would also work: }}}{{{\color[RGB]{101, 123, 131} 
 }}}\\
{{{\color[RGB]{147, 161, 161} -{}- }}}{{{\color[RGB]{147, 161, 161}  add\_le\_add (add\_le\_add (add\_le\_add (add\_le\_add h1 (mul\_le\_mul\_of\_nonneg\_right h2 h3)) h1 ) h2) h3 }}}{{{\color[RGB]{101, 123, 131} 
 }}}\\
{{{\color[RGB]{147, 161, 161} -{}- }}}{{{\color[RGB]{147, 161, 161}  by apply\_rules {[}add\_le\_add, mul\_le\_mul\_of\_nonneg\_right{]} }}}{{{\color[RGB]{101, 123, 131} 
 }}}\\
{{{\color[RGB]{147, 161, 161} -{}- }}}{{{\color[RGB]{147, 161, 161}  by apply\_rules {[}mono\_rules{]} }}}{{{\color[RGB]{101, 123, 131} 
 }}}\\

}}\paragraph{tactic.interactive.h\_generalize}
\par
\colorbox[RGB]{253,246,227}{{{{\color[RGB]{101, 123, 131} h\_generalize Hx : e  }}}{{{\color[RGB]{181, 137, 0} == }}}{{{\color[RGB]{101, 123, 131}  x }}}} matches on 
\colorbox[RGB]{253,246,227}{{{{\color[RGB]{101, 123, 131} cast \_ e }}}} in the goal and replaces it with
\colorbox[RGB]{253,246,227}{{{{\color[RGB]{101, 123, 131} x }}}}. It also adds 
\colorbox[RGB]{253,246,227}{{{{\color[RGB]{101, 123, 131} Hx : e  }}}{{{\color[RGB]{181, 137, 0} == }}}{{{\color[RGB]{101, 123, 131}  x }}}} as an assumption. If 
\colorbox[RGB]{253,246,227}{{{{\color[RGB]{101, 123, 131} cast \_ e }}}} appears multiple
times (not necessarily with the same proof), they are all replaced by 
\colorbox[RGB]{253,246,227}{{{{\color[RGB]{101, 123, 131} x }}}}. 
\colorbox[RGB]{253,246,227}{{{{\color[RGB]{101, 123, 131} cast }}}}\colorbox[RGB]{253,246,227}{{{{\color[RGB]{101, 123, 131} eq.mp }}}}, 
\colorbox[RGB]{253,246,227}{{{{\color[RGB]{101, 123, 131} eq.mpr }}}}, 
\colorbox[RGB]{253,246,227}{{{{\color[RGB]{101, 123, 131} eq.subst }}}}, 
\colorbox[RGB]{253,246,227}{{{{\color[RGB]{101, 123, 131} eq.substr }}}}, 
\colorbox[RGB]{253,246,227}{{{{\color[RGB]{101, 123, 131} eq.rec }}}} and 
\colorbox[RGB]{253,246,227}{{{{\color[RGB]{101, 123, 131} eq.rec\_on }}}} are all treated
as casts.
\par
\colorbox[RGB]{253,246,227}{{{{\color[RGB]{101, 123, 131} h\_generalize Hx : e  }}}{{{\color[RGB]{181, 137, 0} == }}}{{{\color[RGB]{101, 123, 131}  x  }}}{{{\color[RGB]{133, 153, 0} with }}}{{{\color[RGB]{101, 123, 131}  h }}}} adds hypothesis 
\colorbox[RGB]{253,246,227}{{{{\color[RGB]{101, 123, 131} α  }}}{{{\color[RGB]{181, 137, 0} = }}}{{{\color[RGB]{101, 123, 131}  β }}}} with 
\colorbox[RGB]{253,246,227}{{{{\color[RGB]{101, 123, 131} e : α, x : β }}}}.
\par
\colorbox[RGB]{253,246,227}{{{{\color[RGB]{101, 123, 131} h\_generalize Hx : e  }}}{{{\color[RGB]{181, 137, 0} == }}}{{{\color[RGB]{101, 123, 131}  x  }}}{{{\color[RGB]{133, 153, 0} with }}}{{{\color[RGB]{101, 123, 131}  \_ }}}} chooses automatically chooses the name of
assumption 
\colorbox[RGB]{253,246,227}{{{{\color[RGB]{101, 123, 131} α  }}}{{{\color[RGB]{181, 137, 0} = }}}{{{\color[RGB]{101, 123, 131}  β }}}}.
\par
\colorbox[RGB]{253,246,227}{{{{\color[RGB]{101, 123, 131} h\_generalize! Hx : e  }}}{{{\color[RGB]{181, 137, 0} == }}}{{{\color[RGB]{101, 123, 131}  x }}}} reverts 
\colorbox[RGB]{253,246,227}{{{{\color[RGB]{101, 123, 131} Hx }}}}.
\par
when 
\colorbox[RGB]{253,246,227}{{{{\color[RGB]{101, 123, 131} Hx }}}} is omitted, assumption 
\colorbox[RGB]{253,246,227}{{{{\color[RGB]{101, 123, 131} Hx : e  }}}{{{\color[RGB]{181, 137, 0} == }}}{{{\color[RGB]{101, 123, 131}  x }}}} is not added.
\paragraph{tactic.interactive.choose}
\par
\colorbox[RGB]{253,246,227}{{{{\color[RGB]{101, 123, 131} choose a b h  }}}{{{\color[RGB]{133, 153, 0} using }}}{{{\color[RGB]{101, 123, 131}  hyp }}}} takes an hypothesis 
\colorbox[RGB]{253,246,227}{{{{\color[RGB]{101, 123, 131} hyp }}}} of the form
\colorbox[RGB]{253,246,227}{{{{\color[RGB]{101, 123, 131} ∀ (x : X) (y : Y), ∃ (a : A) (b : B), P x y a b }}}} for some 
\colorbox[RGB]{253,246,227}{{{{\color[RGB]{101, 123, 131} P : X  }}}{{{\color[RGB]{133, 153, 0} → }}}{{{\color[RGB]{101, 123, 131}  Y  }}}{{{\color[RGB]{133, 153, 0} → }}}{{{\color[RGB]{101, 123, 131}  A  }}}{{{\color[RGB]{133, 153, 0} → }}}{{{\color[RGB]{101, 123, 131}  B  }}}{{{\color[RGB]{133, 153, 0} → }}}{{{\color[RGB]{101, 123, 131}   }}}{{{\color[RGB]{38, 139, 210} Prop }}}} and outputs
into context a function 
\colorbox[RGB]{253,246,227}{{{{\color[RGB]{101, 123, 131} a : X  }}}{{{\color[RGB]{133, 153, 0} → }}}{{{\color[RGB]{101, 123, 131}  Y  }}}{{{\color[RGB]{133, 153, 0} → }}}{{{\color[RGB]{101, 123, 131}  A }}}}, 
\colorbox[RGB]{253,246,227}{{{{\color[RGB]{101, 123, 131} b : X  }}}{{{\color[RGB]{133, 153, 0} → }}}{{{\color[RGB]{101, 123, 131}  Y  }}}{{{\color[RGB]{133, 153, 0} → }}}{{{\color[RGB]{101, 123, 131}  B }}}} and a proposition 
\colorbox[RGB]{253,246,227}{{{{\color[RGB]{101, 123, 131} h }}}} stating
\colorbox[RGB]{253,246,227}{{{{\color[RGB]{101, 123, 131} ∀ (x : X) (y : Y), P x y (a x y) (b x y) }}}}. It presumably also works with dependent versions.
\par
Example:
\\
\colorbox[RGB]{253,246,227}{\parbox{4.5in}{{{{\color[RGB]{133, 153, 0} example }}}{{{\color[RGB]{101, 123, 131}  (h :  }}}{{{\color[RGB]{133, 153, 0} ∀ }}}{{{\color[RGB]{101, 123, 131} n m : ℕ, ∃i j, m  }}}{{{\color[RGB]{181, 137, 0} = }}}{{{\color[RGB]{101, 123, 131}  n  }}}{{{\color[RGB]{181, 137, 0} + }}}{{{\color[RGB]{101, 123, 131}  i  }}}{{{\color[RGB]{181, 137, 0} ∨ }}}{{{\color[RGB]{101, 123, 131}  m  }}}{{{\color[RGB]{181, 137, 0} + }}}{{{\color[RGB]{101, 123, 131}  j  }}}{{{\color[RGB]{181, 137, 0} = }}}{{{\color[RGB]{101, 123, 131}  n) : true  }}}{{{\color[RGB]{181, 137, 0} := }}}{{{\color[RGB]{101, 123, 131} 
 }}}\\
{{{\color[RGB]{133, 153, 0} begin }}}{{{\color[RGB]{101, 123, 131} 
 }}}\\
{{{\color[RGB]{101, 123, 131}   choose i j h  }}}{{{\color[RGB]{133, 153, 0} using }}}{{{\color[RGB]{101, 123, 131}  h,
 }}}\\
{{{\color[RGB]{101, 123, 131}   guard\_hyp i  }}}{{{\color[RGB]{181, 137, 0} := }}}{{{\color[RGB]{101, 123, 131}  ℕ  }}}{{{\color[RGB]{133, 153, 0} → }}}{{{\color[RGB]{101, 123, 131}  ℕ  }}}{{{\color[RGB]{133, 153, 0} → }}}{{{\color[RGB]{101, 123, 131}  ℕ,
 }}}\\
{{{\color[RGB]{101, 123, 131}   guard\_hyp j  }}}{{{\color[RGB]{181, 137, 0} := }}}{{{\color[RGB]{101, 123, 131}  ℕ  }}}{{{\color[RGB]{133, 153, 0} → }}}{{{\color[RGB]{101, 123, 131}  ℕ  }}}{{{\color[RGB]{133, 153, 0} → }}}{{{\color[RGB]{101, 123, 131}  ℕ,
 }}}\\
{{{\color[RGB]{101, 123, 131}   guard\_hyp h  }}}{{{\color[RGB]{181, 137, 0} := }}}{{{\color[RGB]{101, 123, 131}   }}}{{{\color[RGB]{133, 153, 0} ∀ }}}{{{\color[RGB]{101, 123, 131}  (n m : ℕ), m  }}}{{{\color[RGB]{181, 137, 0} = }}}{{{\color[RGB]{101, 123, 131}  n  }}}{{{\color[RGB]{181, 137, 0} + }}}{{{\color[RGB]{101, 123, 131}  i n m  }}}{{{\color[RGB]{181, 137, 0} ∨ }}}{{{\color[RGB]{101, 123, 131}  m  }}}{{{\color[RGB]{181, 137, 0} + }}}{{{\color[RGB]{101, 123, 131}  j n m  }}}{{{\color[RGB]{181, 137, 0} = }}}{{{\color[RGB]{101, 123, 131}  n,
 }}}\\
{{{\color[RGB]{101, 123, 131}   trivial
 }}}\\
{{{\color[RGB]{133, 153, 0} end }}}{{{\color[RGB]{101, 123, 131} 
 }}}\\

}}\paragraph{tactic.interactive.guard\_target'}
\par
\colorbox[RGB]{253,246,227}{{{{\color[RGB]{101, 123, 131} guard\_target t }}}} fails if the target of the main goal is not 
\colorbox[RGB]{253,246,227}{{{{\color[RGB]{101, 123, 131} t }}}}.
We use this tactic for writing tests.
\paragraph{tactic.interactive.triv}
\par
a weaker version of 
\colorbox[RGB]{253,246,227}{{{{\color[RGB]{101, 123, 131} trivial }}}} that tries to solve the goal by reflexivity or by reducing it to true,
unfolding only 
\colorbox[RGB]{253,246,227}{{{{\color[RGB]{101, 123, 131} reducible }}}} constants.
\paragraph{tactic.interactive.use}
\par
Similar to 
\colorbox[RGB]{253,246,227}{{{{\color[RGB]{101, 123, 131} existsi }}}}. 
\colorbox[RGB]{253,246,227}{{{{\color[RGB]{101, 123, 131} use x }}}} will instantiate the first term of an 
\colorbox[RGB]{253,246,227}{{{{\color[RGB]{101, 123, 131} ∃ }}}} or 
\colorbox[RGB]{253,246,227}{{{{\color[RGB]{101, 123, 131} Σ }}}} goal with 
\colorbox[RGB]{253,246,227}{{{{\color[RGB]{101, 123, 131} x }}}}.
Unlike 
\colorbox[RGB]{253,246,227}{{{{\color[RGB]{101, 123, 131} existsi }}}}, 
\colorbox[RGB]{253,246,227}{{{{\color[RGB]{101, 123, 131} x }}}} is elaborated with respect to the expected type.
\colorbox[RGB]{253,246,227}{{{{\color[RGB]{101, 123, 131} use }}}} will alternatively take a list of terms 
\colorbox[RGB]{253,246,227}{{{{\color[RGB]{101, 123, 131} {[}x0, ..., xn{]} }}}}.
\par
\colorbox[RGB]{253,246,227}{{{{\color[RGB]{101, 123, 131} use }}}} will work with constructors of arbitrary inductive types.
\par
Examples:
\par
example (α : Type) : ∃ S : set α, S = S :=
by use ∅
\par
example : ∃ x : ℤ, x = x :=
by use 42
\par
example : ∃ a b c : ℤ, a + b + c = 6 :=
by use 
{[}
1, 2, 3
{]}
\par
example : ∃ p : ℤ × ℤ, p.1 = 1 :=
by use ⟨1, 42⟩
\par
example : Σ x y : ℤ, (ℤ × ℤ) × ℤ :=
by use 
{[}
1, 2, 3, 4, 5
{]}
\par
inductive foo
| mk : ℕ → bool × ℕ → ℕ → foo
\par
example : foo :=
by use 
{[}
100, tt, 4, 3
{]}
\paragraph{tactic.interactive.clear\_aux\_decl}
\par
\colorbox[RGB]{253,246,227}{{{{\color[RGB]{101, 123, 131} clear\_aux\_decl }}}} clears every 
\colorbox[RGB]{253,246,227}{{{{\color[RGB]{101, 123, 131} aux\_decl }}}} in the local context for the current goal.
This includes the induction hypothesis when using the equation compiler and
\colorbox[RGB]{253,246,227}{{{{\color[RGB]{101, 123, 131} \_let\_match }}}} and 
\colorbox[RGB]{253,246,227}{{{{\color[RGB]{101, 123, 131} \_fun\_match }}}}.
\par
It is useful when using a tactic such as 
\colorbox[RGB]{253,246,227}{{{{\color[RGB]{101, 123, 131} finish }}}}, 
\colorbox[RGB]{253,246,227}{{{{\color[RGB]{101, 123, 131} simp  }}}{{{\color[RGB]{181, 137, 0} * }}}} or 
\colorbox[RGB]{253,246,227}{{{{\color[RGB]{101, 123, 131} subst }}}} that may use these
auxiliary declarations, and produce an error saying the recursion is not well founded.
\paragraph{tactic.interactive.change'}
\par
The logic of 
\colorbox[RGB]{253,246,227}{{{{\color[RGB]{101, 123, 131} change x  }}}{{{\color[RGB]{133, 153, 0} with }}}{{{\color[RGB]{101, 123, 131}  y  }}}{{{\color[RGB]{133, 153, 0} at }}}{{{\color[RGB]{101, 123, 131}  l }}}} fails when there are dependencies.
\colorbox[RGB]{253,246,227}{{{{\color[RGB]{101, 123, 131} change' }}}} mimics the behavior of 
\colorbox[RGB]{253,246,227}{{{{\color[RGB]{101, 123, 131} change }}}}, except in the case of 
\colorbox[RGB]{253,246,227}{{{{\color[RGB]{101, 123, 131} change x  }}}{{{\color[RGB]{133, 153, 0} with }}}{{{\color[RGB]{101, 123, 131}  y  }}}{{{\color[RGB]{133, 153, 0} at }}}{{{\color[RGB]{101, 123, 131}  l }}}}.
In this case, it will correctly replace occurences of 
\colorbox[RGB]{253,246,227}{{{{\color[RGB]{101, 123, 131} x }}}} with 
\colorbox[RGB]{253,246,227}{{{{\color[RGB]{101, 123, 131} y }}}} at all possible hypotheses in 
\colorbox[RGB]{253,246,227}{{{{\color[RGB]{101, 123, 131} l }}}}.
As long as 
\colorbox[RGB]{253,246,227}{{{{\color[RGB]{101, 123, 131} x }}}} and 
\colorbox[RGB]{253,246,227}{{{{\color[RGB]{101, 123, 131} y }}}} are defeq, it should never fail.
\paragraph{tactic.interactive.set}
\par
\colorbox[RGB]{253,246,227}{{{{\color[RGB]{101, 123, 131} set a  }}}{{{\color[RGB]{181, 137, 0} := }}}{{{\color[RGB]{101, 123, 131}  t  }}}{{{\color[RGB]{133, 153, 0} with }}}{{{\color[RGB]{101, 123, 131}  h }}}} is a variant of 
\colorbox[RGB]{253,246,227}{{{{\color[RGB]{133, 153, 0} let }}}{{{\color[RGB]{101, 123, 131}  a  }}}{{{\color[RGB]{181, 137, 0} := }}}{{{\color[RGB]{101, 123, 131}  t }}}}.
It adds the hypothesis 
\colorbox[RGB]{253,246,227}{{{{\color[RGB]{101, 123, 131} h : a  }}}{{{\color[RGB]{181, 137, 0} = }}}{{{\color[RGB]{101, 123, 131}  t }}}} to the local context and replaces 
\colorbox[RGB]{253,246,227}{{{{\color[RGB]{101, 123, 131} t }}}} with 
\colorbox[RGB]{253,246,227}{{{{\color[RGB]{101, 123, 131} a }}}} everywhere it can.
\colorbox[RGB]{253,246,227}{{{{\color[RGB]{101, 123, 131} set a  }}}{{{\color[RGB]{181, 137, 0} := }}}{{{\color[RGB]{101, 123, 131}  t  }}}{{{\color[RGB]{133, 153, 0} with }}}{{{\color[RGB]{101, 123, 131}  ←h }}}} will add 
\colorbox[RGB]{253,246,227}{{{{\color[RGB]{101, 123, 131} h : t  }}}{{{\color[RGB]{181, 137, 0} = }}}{{{\color[RGB]{101, 123, 131}  a }}}} instead.
\colorbox[RGB]{253,246,227}{{{{\color[RGB]{101, 123, 131} set! a  }}}{{{\color[RGB]{181, 137, 0} := }}}{{{\color[RGB]{101, 123, 131}  t  }}}{{{\color[RGB]{133, 153, 0} with }}}{{{\color[RGB]{101, 123, 131}  h }}}} does not do any replacing.
\paragraph{tactic.interactive.clear\_except}
\par
\colorbox[RGB]{253,246,227}{{{{\color[RGB]{101, 123, 131} clear\_except h₀ h₁ }}}} deletes all the assumptions it can except for 
\colorbox[RGB]{253,246,227}{{{{\color[RGB]{101, 123, 131} h₀ }}}} and 
\colorbox[RGB]{253,246,227}{{{{\color[RGB]{101, 123, 131} h₁ }}}}.
\section{tactic/library\_search.lean}\paragraph{tactic.library\_search.library\_defs}
\par
Retrieve all library definitions with a given head symbol.
\paragraph{tactic.interactive.library\_search}
\par
\colorbox[RGB]{253,246,227}{{{{\color[RGB]{101, 123, 131} library\_search }}}} attempts to apply every definition in the library whose head symbol
matches the goal, and then discharge any new goals using 
\colorbox[RGB]{253,246,227}{{{{\color[RGB]{101, 123, 131} solve\_by\_elim }}}}.
\par
If it succeeds, it prints a trace message 
\colorbox[RGB]{253,246,227}{{{{\color[RGB]{101, 123, 131} exact ... }}}} which can replace the invocation
of 
\colorbox[RGB]{253,246,227}{{{{\color[RGB]{101, 123, 131} library\_search }}}}.
\section{tactic/linarith.lean}\paragraph{linarith.comp}
\par
The main datatype for FM elimination.
Variables are represented by natural numbers, each of which has an integer coefficient.
Index 0 is reserved for constants, i.e. 
\colorbox[RGB]{253,246,227}{{{{\color[RGB]{101, 123, 131} coeffs.find  }}}{{{\color[RGB]{108, 113, 196} 0 }}}} is the coefficient of 1.
The represented term is coeffs.keys.sum (λ i, coeffs.find i * Var
{[}
i
{]}
).
str determines the direction of the comparison -{}- is it 
<
 0, ≤ 0, or = 0?
\paragraph{linarith.elim\_var}
\par
If c1 and c2 both contain variable a with opposite coefficients,
produces v1, v2, and c such that a has been cancelled in c := v1
\emph{c1 + v2
}c2
\paragraph{linarith.linarith\_structure}
\par
The state for the elimination monad.
vars: the set of variables present in comps
comps: a set of comparisons
inputs: a set of pairs of exprs (t, pf), where t is a term and pf is a proof that t \{
<
, ≤, =\} 0,
indexed by ℕ.
has\_false: stores a pcomp of 0 
<
 0 if one has been found
TODO: is it more efficient to store comps as a list, to avoid comparisons?
\paragraph{linarith.map\_of\_expr}
\par
Turns an expression into a map from ℕ to ℤ, for use in a comp object.
The expr\_map ℕ argument identifies which expressions have already been assigned numbers.
Returns a new map.
\paragraph{linarith.mk\_linarith\_structure}
\par
Takes a list of proofs of props of the form t \{
<
, ≤, =\} 0, and creates a linarith\_structure.
\paragraph{linarith.mk\_lt\_zero\_pf}
\par
Takes a list of coefficients 
{[}
c
{]}
 and list of expressions, of equal length.
Each expression is a proof of a prop of the form t \{
<
, ≤, =\} 0.
Produces a proof that the sum of (c
*
t) \{
<
, ≤, =\} 0, where the comp is as strong as possible.
\paragraph{linarith.mk\_neg\_eq\_zero\_pf}
\par
Assumes e is a proof that t = 0. Creates a proof that -t = 0.
\paragraph{linarith.prove\_false\_by\_linarith1}
\par
Takes a list of proofs of propositions of the form t \{
<
, ≤, =\} 0,
and tries to prove the goal 
\colorbox[RGB]{253,246,227}{{{{\color[RGB]{101, 123, 131} false }}}}.
\paragraph{linarith.kill\_factors}
\par
e is a term with rational division. produces a natural number n and a proof that n
*
e = e',
where e' has no division.
\paragraph{linarith.prove\_false\_by\_linarith}
\par
Takes a list of proofs of propositions.
Filters out the proofs of linear (in)equalities,
and tries to use them to prove 
\colorbox[RGB]{253,246,227}{{{{\color[RGB]{101, 123, 131} false }}}}.
If pref\_type is given, starts by working over this type
\paragraph{linarith.interactive\_aux}
\par
linarith.interactive\_aux cfg o\_goal restrict\_hyps args:
\begin{itemize}\item cfg is a linarith\_config object

\item o\_goal : option expr is the local constant corresponding to the former goal, if there was one

\item restrict\_hyps : bool is tt if 
\colorbox[RGB]{253,246,227}{{{{\color[RGB]{101, 123, 131} linarith only {[}...{]} }}}} was used

\item args : option (list pexpr) is the optional list of arguments in 
\colorbox[RGB]{253,246,227}{{{{\color[RGB]{101, 123, 131} linarith {[}...{]} }}}}
\end{itemize}\paragraph{tactic.interactive.linarith}
\par
Tries to prove a goal of 
\colorbox[RGB]{253,246,227}{{{{\color[RGB]{101, 123, 131} false }}}} by linear arithmetic on hypotheses.
If the goal is a linear (in)equality, tries to prove it by contradiction.
If the goal is not 
\colorbox[RGB]{253,246,227}{{{{\color[RGB]{101, 123, 131} false }}}} or an inequality, applies 
\colorbox[RGB]{253,246,227}{{{{\color[RGB]{101, 123, 131} exfalso }}}} and tries linarith on the
hypotheses.
\colorbox[RGB]{253,246,227}{{{{\color[RGB]{101, 123, 131} linarith }}}} will use all relevant hypotheses in the local context.
\colorbox[RGB]{253,246,227}{{{{\color[RGB]{101, 123, 131} linarith {[}t1, t2, t3{]} }}}} will add proof terms t1, t2, t3 to the local context.
\colorbox[RGB]{253,246,227}{{{{\color[RGB]{101, 123, 131} linarith only {[}h1, h2, h3, t1, t2, t3{]} }}}} will use only the goal (if relevant), local hypotheses
h1, h2, h3, and proofs t1, t2, t3. It will ignore the rest of the local context.
\colorbox[RGB]{253,246,227}{{{{\color[RGB]{101, 123, 131} linarith! }}}} will use a stronger reducibility setting to identify atoms.
\par
Config options:
\colorbox[RGB]{253,246,227}{{{{\color[RGB]{101, 123, 131} linarith \{exfalso  }}}{{{\color[RGB]{181, 137, 0} := }}}{{{\color[RGB]{101, 123, 131}  ff\} }}}} will fail on a goal that is neither an inequality nor 
\colorbox[RGB]{253,246,227}{{{{\color[RGB]{101, 123, 131} false }}}}\colorbox[RGB]{253,246,227}{{{{\color[RGB]{101, 123, 131} linarith \{restrict\_type  }}}{{{\color[RGB]{181, 137, 0} := }}}{{{\color[RGB]{101, 123, 131}  T\} }}}} will run only on hypotheses that are inequalities over 
\colorbox[RGB]{253,246,227}{{{{\color[RGB]{101, 123, 131} T }}}}\colorbox[RGB]{253,246,227}{{{{\color[RGB]{101, 123, 131} linarith \{discharger  }}}{{{\color[RGB]{181, 137, 0} := }}}{{{\color[RGB]{101, 123, 131}  tac\} }}}} will use 
\colorbox[RGB]{253,246,227}{{{{\color[RGB]{101, 123, 131} tac }}}} instead of 
\colorbox[RGB]{253,246,227}{{{{\color[RGB]{101, 123, 131} ring }}}} for normalization.
Options: 
\colorbox[RGB]{253,246,227}{{{{\color[RGB]{101, 123, 131} ring2 }}}}, 
\colorbox[RGB]{253,246,227}{{{{\color[RGB]{101, 123, 131} ring SOP }}}}, 
\colorbox[RGB]{253,246,227}{{{{\color[RGB]{101, 123, 131} simp }}}}\section{tactic/local\_cache.lean}\paragraph{tactic.local\_cache.cache\_scope.block\_local}
\par
This scope propogates the cache within a 
\colorbox[RGB]{253,246,227}{{{{\color[RGB]{133, 153, 0} begin }}}{{{\color[RGB]{101, 123, 131}  ...  }}}{{{\color[RGB]{133, 153, 0} end }}}} or 
\colorbox[RGB]{253,246,227}{{{{\color[RGB]{133, 153, 0} by }}}} block
and its decendants.
\paragraph{tactic.local\_cache.cache\_scope.def\_local}
\par
This scope propogates the cache within an entire 
\colorbox[RGB]{253,246,227}{{{{\color[RGB]{133, 153, 0} def }}}}/
\colorbox[RGB]{253,246,227}{{{{\color[RGB]{133, 153, 0} lemma }}}}.
\paragraph{tactic.local\_cache.present}
\par
Asks whether the namespace 
\colorbox[RGB]{253,246,227}{{{{\color[RGB]{101, 123, 131} ns }}}} currently has a value-in-cache.
\paragraph{tactic.local\_cache.clear}
\par
Clear cache associated to namespace 
\colorbox[RGB]{253,246,227}{{{{\color[RGB]{101, 123, 131} ns }}}}.
\paragraph{tactic.local\_cache.get}
\par
Gets the (optionally present) value-in-cache for 
\colorbox[RGB]{253,246,227}{{{{\color[RGB]{101, 123, 131} ns }}}}.
\paragraph{tactic.run\_once}
\par
Using the namespace 
\colorbox[RGB]{253,246,227}{{{{\color[RGB]{101, 123, 131} ns }}}} as its key, when called for the first
time 
\colorbox[RGB]{253,246,227}{{{{\color[RGB]{101, 123, 131} run\_once ns t }}}} runs 
\colorbox[RGB]{253,246,227}{{{{\color[RGB]{101, 123, 131} t }}}}, then saves and returns the result.
Upon subsequent invocations in the same tactic block, with the scope
of the caching being inherited by child tactic blocks) we return the
cached result directly.
\par
You can configure the cached scope to be entire 
\colorbox[RGB]{253,246,227}{{{{\color[RGB]{133, 153, 0} def }}}}/
\colorbox[RGB]{253,246,227}{{{{\color[RGB]{133, 153, 0} lemma }}}}s changing
the optional cache\_scope argument to 
\colorbox[RGB]{253,246,227}{{{{\color[RGB]{101, 123, 131} cache\_scope.def\_local }}}}.
Note: the caches backing each scope are different.
\par
If 
\colorbox[RGB]{253,246,227}{{{{\color[RGB]{101, 123, 131} α }}}} is just 
\colorbox[RGB]{253,246,227}{{{{\color[RGB]{101, 123, 131} unit }}}}, this means we just run 
\colorbox[RGB]{253,246,227}{{{{\color[RGB]{101, 123, 131} t }}}} once each tactic
block.
\section{tactic/mk\_iff\_of\_inductive\_prop.lean}\paragraph{\_private.546441557.compact\_relation}
\par
\colorbox[RGB]{253,246,227}{{{{\color[RGB]{101, 123, 131} compact\_relation bs as\_ps }}}}: Produce a relation of the form:
R as := ∃ bs, Λ\_i a\_i = p\_i
{[}
bs
{]}
This relation is user visible, so we compact it by removing each 
\colorbox[RGB]{253,246,227}{{{{\color[RGB]{101, 123, 131} b\_j }}}} where a 
\colorbox[RGB]{253,246,227}{{{{\color[RGB]{101, 123, 131} p\_i  }}}{{{\color[RGB]{181, 137, 0} = }}}{{{\color[RGB]{101, 123, 131}  b\_j }}}}, and
hence 
\colorbox[RGB]{253,246,227}{{{{\color[RGB]{101, 123, 131} a\_i  }}}{{{\color[RGB]{181, 137, 0} = }}}{{{\color[RGB]{101, 123, 131}  b\_j }}}}. We need to take care when there are 
\colorbox[RGB]{253,246,227}{{{{\color[RGB]{101, 123, 131} p\_i }}}} and 
\colorbox[RGB]{253,246,227}{{{{\color[RGB]{101, 123, 131} p\_j }}}} with 
\colorbox[RGB]{253,246,227}{{{{\color[RGB]{101, 123, 131} p\_i  }}}{{{\color[RGB]{181, 137, 0} = }}}{{{\color[RGB]{101, 123, 131}  p\_j  }}}{{{\color[RGB]{181, 137, 0} = }}}{{{\color[RGB]{101, 123, 131}  b\_k }}}}.
\par
TODO: this is copied from Lean's 
\colorbox[RGB]{253,246,227}{{{{\color[RGB]{101, 123, 131} coinductive\_predicates.lean }}}}, export it there.
\paragraph{tactic.mk\_iff\_of\_inductive\_prop}
\par
\colorbox[RGB]{253,246,227}{{{{\color[RGB]{101, 123, 131} mk\_iff\_of\_inductive\_prop i r }}}} makes a iff rule for the inductively defined proposition 
\colorbox[RGB]{253,246,227}{{{{\color[RGB]{101, 123, 131} i }}}}.
The new rule 
\colorbox[RGB]{253,246,227}{{{{\color[RGB]{101, 123, 131} r }}}} has the shape 
\colorbox[RGB]{253,246,227}{{{{\color[RGB]{101, 123, 131} ∀ps is, i  }}}{{{\color[RGB]{133, 153, 0} as }}}{{{\color[RGB]{101, 123, 131}   }}}{{{\color[RGB]{181, 137, 0} ↔ }}}{{{\color[RGB]{101, 123, 131}  ⋁\_j, ∃cs, is  }}}{{{\color[RGB]{181, 137, 0} = }}}{{{\color[RGB]{101, 123, 131}  cs }}}}, where 
\colorbox[RGB]{253,246,227}{{{{\color[RGB]{101, 123, 131} ps }}}} are the type
parameters, 
\colorbox[RGB]{253,246,227}{{{{\color[RGB]{101, 123, 131} is }}}} are the indices, 
\colorbox[RGB]{253,246,227}{{{{\color[RGB]{101, 123, 131} j }}}} ranges over all possible constructors, the 
\colorbox[RGB]{253,246,227}{{{{\color[RGB]{101, 123, 131} cs }}}} are the
parameters for each constructors, the equalities 
\colorbox[RGB]{253,246,227}{{{{\color[RGB]{101, 123, 131} is  }}}{{{\color[RGB]{181, 137, 0} = }}}{{{\color[RGB]{101, 123, 131}  cs }}}} are the instantiations for each
constructor for each of the indices to the inductive type 
\colorbox[RGB]{253,246,227}{{{{\color[RGB]{101, 123, 131} i }}}}.
\par
In each case, we remove constructor parameters (i.e. 
\colorbox[RGB]{253,246,227}{{{{\color[RGB]{101, 123, 131} cs }}}}) when the corresponding equality would
be just 
\colorbox[RGB]{253,246,227}{{{{\color[RGB]{101, 123, 131} c  }}}{{{\color[RGB]{181, 137, 0} = }}}{{{\color[RGB]{101, 123, 131}  i }}}} for some index 
\colorbox[RGB]{253,246,227}{{{{\color[RGB]{101, 123, 131} i }}}}.
\par
For example: 
\colorbox[RGB]{253,246,227}{{{{\color[RGB]{101, 123, 131} mk\_iff\_of\_inductive\_prop }}}} on 
\colorbox[RGB]{253,246,227}{{{{\color[RGB]{101, 123, 131} list.chain }}}} produces:
\par
∀\{α : Type
*
\} (R : α → α → Prop) (a : α) (l : list α),
chain R a l ↔ l = 
{[}
{]}
 ∨ ∃\{b : α\} \{l' : list α\}, R a b ∧ chain R b l ∧ l = b :: l'
\section{tactic/monotonicity/basic.lean}\section{tactic/monotonicity/default.lean}\section{tactic/monotonicity/interactive.lean}\paragraph{tactic.interactive.match\_assoc}
\par
\colorbox[RGB]{253,246,227}{{{{\color[RGB]{101, 123, 131} ( }}}{{{\color[RGB]{133, 153, 0} prefix }}}{{{\color[RGB]{101, 123, 131} ,left,right,suffix) ← match\_assoc unif l r }}}} finds the
longest prefix and suffix common to 
\colorbox[RGB]{253,246,227}{{{{\color[RGB]{101, 123, 131} l }}}} and 
\colorbox[RGB]{253,246,227}{{{{\color[RGB]{101, 123, 131} r }}}} and
returns them along with the differences
\paragraph{tactic.interactive.generalize'}
\par
tactic-facing function, similar to 
\colorbox[RGB]{253,246,227}{{{{\color[RGB]{101, 123, 131} interactive.tactic.generalize }}}} with the
exception that meta variables
\paragraph{tactic.interactive.mono}
\begin{itemize}\item \colorbox[RGB]{253,246,227}{{{{\color[RGB]{101, 123, 131} mono }}}} applies a monotonicity rule.

\item \colorbox[RGB]{253,246,227}{{{{\color[RGB]{101, 123, 131} mono }}}{{{\color[RGB]{181, 137, 0} * }}}} applies monotonicity rules repetitively.

\item \colorbox[RGB]{253,246,227}{{{{\color[RGB]{101, 123, 131} mono  }}}{{{\color[RGB]{133, 153, 0} with }}}{{{\color[RGB]{101, 123, 131}  x  }}}{{{\color[RGB]{181, 137, 0} ≤ }}}{{{\color[RGB]{101, 123, 131}  y }}}} or 
\colorbox[RGB]{253,246,227}{{{{\color[RGB]{101, 123, 131} mono  }}}{{{\color[RGB]{133, 153, 0} with }}}{{{\color[RGB]{101, 123, 131}  {[} }}}{{{\color[RGB]{108, 113, 196} 0 }}}{{{\color[RGB]{101, 123, 131}   }}}{{{\color[RGB]{181, 137, 0} ≤ }}}{{{\color[RGB]{101, 123, 131}  x, }}}{{{\color[RGB]{108, 113, 196} 0 }}}{{{\color[RGB]{101, 123, 131}   }}}{{{\color[RGB]{181, 137, 0} ≤ }}}{{{\color[RGB]{101, 123, 131}  y{]} }}}} creates an assertion for the listed
propositions. Those help to select the right monotonicity rule.

\item \colorbox[RGB]{253,246,227}{{{{\color[RGB]{101, 123, 131} mono left }}}} or 
\colorbox[RGB]{253,246,227}{{{{\color[RGB]{101, 123, 131} mono right }}}} is useful when proving strict orderings:
for 
\colorbox[RGB]{253,246,227}{{{{\color[RGB]{101, 123, 131} x  }}}{{{\color[RGB]{181, 137, 0} + }}}{{{\color[RGB]{101, 123, 131}  y  }}}{{{\color[RGB]{181, 137, 0} < }}}{{{\color[RGB]{101, 123, 131}  w  }}}{{{\color[RGB]{181, 137, 0} + }}}{{{\color[RGB]{101, 123, 131}  z }}}} could be broken down into either
\begin{itemize}\item left:  
\colorbox[RGB]{253,246,227}{{{{\color[RGB]{101, 123, 131} x  }}}{{{\color[RGB]{181, 137, 0} ≤ }}}{{{\color[RGB]{101, 123, 131}  w }}}} and 
\colorbox[RGB]{253,246,227}{{{{\color[RGB]{101, 123, 131} y  }}}{{{\color[RGB]{181, 137, 0} < }}}{{{\color[RGB]{101, 123, 131}  z }}}} or

\item right: 
\colorbox[RGB]{253,246,227}{{{{\color[RGB]{101, 123, 131} x  }}}{{{\color[RGB]{181, 137, 0} < }}}{{{\color[RGB]{101, 123, 131}  w }}}} and 
\colorbox[RGB]{253,246,227}{{{{\color[RGB]{101, 123, 131} y  }}}{{{\color[RGB]{181, 137, 0} ≤ }}}{{{\color[RGB]{101, 123, 131}  z }}}}
\end{itemize}
\end{itemize}\paragraph{tactic.interactive.ac\_mono\_aux}
\par
transforms a goal of the form 
\colorbox[RGB]{253,246,227}{{{{\color[RGB]{101, 123, 131} f x ≼ f y }}}} into 
\colorbox[RGB]{253,246,227}{{{{\color[RGB]{101, 123, 131} x  }}}{{{\color[RGB]{181, 137, 0} ≤ }}}{{{\color[RGB]{101, 123, 131}  y }}}} using lemmas
marked as 
\colorbox[RGB]{253,246,227}{{{{\color[RGB]{101, 123, 131} monotonic }}}}.
\par
Special care is taken when 
\colorbox[RGB]{253,246,227}{{{{\color[RGB]{101, 123, 131} f }}}} is the repeated application of an
associative operator and if the operator is commutative
\paragraph{tactic.interactive.repeat\_until\_or\_at\_most}
\par
(repeat\_until\_or\_at\_most n t u): repeat tactic 
\colorbox[RGB]{253,246,227}{{{{\color[RGB]{101, 123, 131} t }}}} at most n times or until u succeeds
\paragraph{tactic.interactive.ac\_mono}
\par
\colorbox[RGB]{253,246,227}{{{{\color[RGB]{101, 123, 131} ac\_mono }}}} reduces the 
\colorbox[RGB]{253,246,227}{{{{\color[RGB]{101, 123, 131} f x ⊑ f y }}}}, for some relation 
\colorbox[RGB]{253,246,227}{{{{\color[RGB]{101, 123, 131} ⊑ }}}} and a
monotonic function 
\colorbox[RGB]{253,246,227}{{{{\color[RGB]{101, 123, 131} f }}}} to 
\colorbox[RGB]{253,246,227}{{{{\color[RGB]{101, 123, 131} x ≺ y }}}}.
\par
\colorbox[RGB]{253,246,227}{{{{\color[RGB]{101, 123, 131} ac\_mono }}}{{{\color[RGB]{181, 137, 0} * }}}} unwraps monotonic functions until it can't.
\par
\colorbox[RGB]{253,246,227}{{{{\color[RGB]{101, 123, 131} ac\_mono\textasciicircum{}k }}}}, for some literal number 
\colorbox[RGB]{253,246,227}{{{{\color[RGB]{101, 123, 131} k }}}} applies monotonicity 
\colorbox[RGB]{253,246,227}{{{{\color[RGB]{101, 123, 131} k }}}}times.
\par
\colorbox[RGB]{253,246,227}{{{{\color[RGB]{101, 123, 131} ac\_mono h }}}}, with 
\colorbox[RGB]{253,246,227}{{{{\color[RGB]{101, 123, 131} h }}}} a hypothesis, unwraps monotonic functions
and uses 
\colorbox[RGB]{253,246,227}{{{{\color[RGB]{101, 123, 131} h }}}} to solve the remaining goal. Can be combined with * or
\textasciicircum{}k: 
\colorbox[RGB]{253,246,227}{{{{\color[RGB]{101, 123, 131} ac\_mono }}}{{{\color[RGB]{181, 137, 0} * }}}{{{\color[RGB]{101, 123, 131}  h }}}}\par
\colorbox[RGB]{253,246,227}{{{{\color[RGB]{101, 123, 131} ac\_mono : p }}}} asserts 
\colorbox[RGB]{253,246,227}{{{{\color[RGB]{101, 123, 131} p }}}} and uses it to discharge the goal result
unwrapping a series of monotonic functions. Can be combined with * or
\textasciicircum{}k: 
\colorbox[RGB]{253,246,227}{{{{\color[RGB]{101, 123, 131} ac\_mono }}}{{{\color[RGB]{181, 137, 0} * }}}{{{\color[RGB]{101, 123, 131}  : p }}}}\par
In the case where 
\colorbox[RGB]{253,246,227}{{{{\color[RGB]{101, 123, 131} f }}}} is an associative or commutative operator,
\colorbox[RGB]{253,246,227}{{{{\color[RGB]{101, 123, 131} ac\_mono }}}} will consider any possible permutation of its arguments
and use the one the minimizes the difference between the left-hand
side and the right-hand side.
\par
TODO(Simon): with 
\colorbox[RGB]{253,246,227}{{{{\color[RGB]{101, 123, 131} ac\_mono h }}}} and 
\colorbox[RGB]{253,246,227}{{{{\color[RGB]{101, 123, 131} ac\_mono : p }}}} split the remaining
gaol if the provided rule does not solve it completely.
\section{tactic/monotonicity/lemmas.lean}\section{tactic/norm\_cast.lean}\paragraph{norm\_cast.elim\_cast\_attr}
\par
This is an attribute for simplification rules that are
used to normalize casts.
\par
Let r be = or ↔, then elimination lemmas of the shape
Π ..., P ↑a1 ... ↑an r P a1 ... an should be given the
attribute elim\_cast.
\paragraph{norm\_cast.move\_cast\_attr}
\par
This is an attribute for simplification rules that are
used to normalize casts.
\par
Let r be = or ↔, then compositional lemmas of the shape
Π ..., ↑(P a1 ... an) r P ↑a1 ... ↑an should be given the
attribute move\_cast.
\paragraph{norm\_cast.squash\_cast\_attr}
\par
This is an attribute for simplification rules of the shape
Π ..., ↑↑a = ↑a or  Π ..., ↑a = a.
\par
They are used in a heuristic to infer intermediate casts.
\paragraph{tactic.interactive.norm\_cast}
\par
Normalize casts at the given locations by moving them "upwards".
As opposed to simp, norm\_cast can be used without necessarily
closing the goal.
\paragraph{tactic.interactive.rw\_mod\_cast}
\par
Rewrite with the given rule and normalize casts between steps.
\paragraph{tactic.interactive.exact\_mod\_cast}
\par
Normalize the goal and the given expression,
then close the goal with exact.
\paragraph{tactic.interactive.apply\_mod\_cast}
\par
Normalize the goal and the given expression,
then apply the expression to the goal.
\paragraph{tactic.interactive.assumption\_mod\_cast}
\par
Normalize the goal and every expression in the local context,
then close the goal with assumption.
\section{tactic/norm\_num.lean}\paragraph{tactic.interactive.norm\_num1}
\par
Basic version of 
\colorbox[RGB]{253,246,227}{{{{\color[RGB]{101, 123, 131} norm\_num }}}} that does not call 
\colorbox[RGB]{253,246,227}{{{{\color[RGB]{101, 123, 131} simp }}}}.
\paragraph{tactic.interactive.norm\_num}
\par
Normalize numerical expressions. Supports the operations
\colorbox[RGB]{253,246,227}{{{{\color[RGB]{181, 137, 0} + }}}} 
\colorbox[RGB]{253,246,227}{{{{\color[RGB]{181, 137, 0} - }}}} 
\colorbox[RGB]{253,246,227}{{{{\color[RGB]{181, 137, 0} * }}}} 
\colorbox[RGB]{253,246,227}{{{{\color[RGB]{181, 137, 0} / }}}} 
\colorbox[RGB]{253,246,227}{{{{\color[RGB]{101, 123, 131} \textasciicircum{} }}}} 
\colorbox[RGB]{253,246,227}{{{{\color[RGB]{181, 137, 0} < }}}} 
\colorbox[RGB]{253,246,227}{{{{\color[RGB]{181, 137, 0} ≤ }}}} over ordered fields (or other
appropriate classes), as well as 
\colorbox[RGB]{253,246,227}{{{{\color[RGB]{181, 137, 0} - }}}} 
\colorbox[RGB]{253,246,227}{{{{\color[RGB]{181, 137, 0} / }}}} 
\colorbox[RGB]{253,246,227}{{{{\color[RGB]{101, 123, 131} \% }}}} over 
\colorbox[RGB]{253,246,227}{{{{\color[RGB]{101, 123, 131} ℤ }}}} and 
\colorbox[RGB]{253,246,227}{{{{\color[RGB]{101, 123, 131} ℕ }}}}.
\section{tactic/omega/clause.lean}\section{tactic/omega/coeffs.lean}\section{tactic/omega/default.lean}\section{tactic/omega/eq\_elim.lean}\section{tactic/omega/find\_ees.lean}\section{tactic/omega/find\_scalars.lean}\section{tactic/omega/int/dnf.lean}\section{tactic/omega/int/form.lean}\section{tactic/omega/int/main.lean}\section{tactic/omega/int/preterm.lean}\section{tactic/omega/lin\_comb.lean}\section{tactic/omega/main.lean}\section{tactic/omega/misc.lean}\section{tactic/omega/nat/dnf.lean}\section{tactic/omega/nat/form.lean}\section{tactic/omega/nat/main.lean}\section{tactic/omega/nat/neg\_elim.lean}\section{tactic/omega/nat/preterm.lean}\section{tactic/omega/nat/sub\_elim.lean}\section{tactic/omega/prove\_unsats.lean}\section{tactic/omega/term.lean}\section{tactic/pi\_instances.lean}\paragraph{tactic.pi\_instance}
\par
\colorbox[RGB]{253,246,227}{{{{\color[RGB]{101, 123, 131} pi\_instance }}}} constructs an instance of 
\colorbox[RGB]{253,246,227}{{{{\color[RGB]{101, 123, 131} my\_class (Π i : I, f i) }}}}where we know 
\colorbox[RGB]{253,246,227}{{{{\color[RGB]{101, 123, 131} Π i, my\_class (f i) }}}}. If an order relation is required,
it defaults to 
\colorbox[RGB]{253,246,227}{{{{\color[RGB]{101, 123, 131} pi.partial\_order }}}}. Any field of the instance that
\colorbox[RGB]{253,246,227}{{{{\color[RGB]{101, 123, 131} pi\_instance }}}} cannot construct is left untouched and generated as a new goal.
\section{tactic/push\_neg.lean}\paragraph{tactic.interactive.push\_neg}
\par
Push negations in the goal of some assumption.
For instance, given 
\colorbox[RGB]{253,246,227}{{{{\color[RGB]{101, 123, 131} h :  }}}{{{\color[RGB]{181, 137, 0} ¬ }}}{{{\color[RGB]{101, 123, 131}   }}}{{{\color[RGB]{133, 153, 0} ∀ }}}{{{\color[RGB]{101, 123, 131}  x, ∃ y, x  }}}{{{\color[RGB]{181, 137, 0} ≤ }}}{{{\color[RGB]{101, 123, 131}  y }}}}, will be transformed by 
\colorbox[RGB]{253,246,227}{{{{\color[RGB]{101, 123, 131} push\_neg  }}}{{{\color[RGB]{133, 153, 0} at }}}{{{\color[RGB]{101, 123, 131}  h }}}} into
\colorbox[RGB]{253,246,227}{{{{\color[RGB]{101, 123, 131} h : ∃ x,  }}}{{{\color[RGB]{133, 153, 0} ∀ }}}{{{\color[RGB]{101, 123, 131}  y, y  }}}{{{\color[RGB]{181, 137, 0} < }}}{{{\color[RGB]{101, 123, 131}  x }}}}. Variables names are conserved.
\paragraph{tactic.interactive.contrapose}
\par
Transforms the goal into its contrapositive.
\colorbox[RGB]{253,246,227}{{{{\color[RGB]{101, 123, 131} contrapose }}}}     turns a goal 
\colorbox[RGB]{253,246,227}{{{{\color[RGB]{101, 123, 131} P  }}}{{{\color[RGB]{133, 153, 0} → }}}{{{\color[RGB]{101, 123, 131}  Q }}}} into 
\colorbox[RGB]{253,246,227}{{{{\color[RGB]{181, 137, 0} ¬ }}}{{{\color[RGB]{101, 123, 131}  Q  }}}{{{\color[RGB]{133, 153, 0} → }}}{{{\color[RGB]{101, 123, 131}   }}}{{{\color[RGB]{181, 137, 0} ¬ }}}{{{\color[RGB]{101, 123, 131}  P }}}}\colorbox[RGB]{253,246,227}{{{{\color[RGB]{101, 123, 131} contrapose! }}}}    turns a goal 
\colorbox[RGB]{253,246,227}{{{{\color[RGB]{101, 123, 131} P  }}}{{{\color[RGB]{133, 153, 0} → }}}{{{\color[RGB]{101, 123, 131}  Q }}}} into 
\colorbox[RGB]{253,246,227}{{{{\color[RGB]{181, 137, 0} ¬ }}}{{{\color[RGB]{101, 123, 131}  Q  }}}{{{\color[RGB]{133, 153, 0} → }}}{{{\color[RGB]{101, 123, 131}   }}}{{{\color[RGB]{181, 137, 0} ¬ }}}{{{\color[RGB]{101, 123, 131}  P }}}} and pushes negations inside 
\colorbox[RGB]{253,246,227}{{{{\color[RGB]{101, 123, 131} P }}}} and 
\colorbox[RGB]{253,246,227}{{{{\color[RGB]{101, 123, 131} Q }}}}using 
\colorbox[RGB]{253,246,227}{{{{\color[RGB]{101, 123, 131} push\_neg }}}}\colorbox[RGB]{253,246,227}{{{{\color[RGB]{101, 123, 131} contrapose h }}}}   first reverts the local assumption 
\colorbox[RGB]{253,246,227}{{{{\color[RGB]{101, 123, 131} h }}}}, and then uses 
\colorbox[RGB]{253,246,227}{{{{\color[RGB]{101, 123, 131} contrapose }}}} and 
\colorbox[RGB]{253,246,227}{{{{\color[RGB]{101, 123, 131} intro h }}}}\colorbox[RGB]{253,246,227}{{{{\color[RGB]{101, 123, 131} contrapose! h }}}}  first reverts the local assumption 
\colorbox[RGB]{253,246,227}{{{{\color[RGB]{101, 123, 131} h }}}}, and then uses 
\colorbox[RGB]{253,246,227}{{{{\color[RGB]{101, 123, 131} contrapose! }}}} and 
\colorbox[RGB]{253,246,227}{{{{\color[RGB]{101, 123, 131} intro h }}}}\section{tactic/rcases.lean}\paragraph{tactic.rcases\_patt\_inverted}
\par
The parser/printer uses an "inverted" meaning for the 
\colorbox[RGB]{253,246,227}{{{{\color[RGB]{101, 123, 131} many }}}} constructor:
rather than representing a sum of products, here it represents a
product of sums. We fix this by applying 
\colorbox[RGB]{253,246,227}{{{{\color[RGB]{101, 123, 131} invert }}}}, defined below, to
the result.
\paragraph{tactic.rcases.process\_constructor}
\par
Takes the number of fields of a single constructor and patterns to
match its fields against (not necessarily the same number). The
returned lists each contain one element per field of the
constructor. The 
\colorbox[RGB]{253,246,227}{{{{\color[RGB]{101, 123, 131} name }}}} is the name which will be used in the
top-level 
\colorbox[RGB]{253,246,227}{{{{\color[RGB]{101, 123, 131} cases }}}} tactic, and the 
\colorbox[RGB]{253,246,227}{{{{\color[RGB]{101, 123, 131} rcases\_patt }}}} is the pattern which
the field will be matched against by subsequent 
\colorbox[RGB]{253,246,227}{{{{\color[RGB]{101, 123, 131} cases }}}} tactics.
\paragraph{tactic.interactive.rcases}
\par
The 
\colorbox[RGB]{253,246,227}{{{{\color[RGB]{101, 123, 131} rcases }}}} tactic is the same as 
\colorbox[RGB]{253,246,227}{{{{\color[RGB]{101, 123, 131} cases }}}}, but with more flexibility in the
\colorbox[RGB]{253,246,227}{{{{\color[RGB]{133, 153, 0} with }}}} pattern syntax to allow for recursive case splitting. The pattern syntax
uses the following recursive grammar:
\\
\colorbox[RGB]{253,246,227}{\parbox{4.5in}{{{{\color[RGB]{101, 123, 131} patt : }}}{{{\color[RGB]{181, 137, 0} := }}}{{{\color[RGB]{101, 123, 131}  (patt\_list  }}}{{{\color[RGB]{42, 161, 152} "|" }}}{{{\color[RGB]{101, 123, 131} ) }}}{{{\color[RGB]{181, 137, 0} * }}}{{{\color[RGB]{101, 123, 131}  patt\_list
 }}}\\
{{{\color[RGB]{101, 123, 131} patt\_list : }}}{{{\color[RGB]{181, 137, 0} := }}}{{{\color[RGB]{101, 123, 131}  id |  }}}{{{\color[RGB]{42, 161, 152} "\_" }}}{{{\color[RGB]{101, 123, 131}  |  }}}{{{\color[RGB]{42, 161, 152} "⟨" }}}{{{\color[RGB]{101, 123, 131}  (patt  }}}{{{\color[RGB]{42, 161, 152} "," }}}{{{\color[RGB]{101, 123, 131} ) }}}{{{\color[RGB]{181, 137, 0} * }}}{{{\color[RGB]{101, 123, 131}  patt  }}}{{{\color[RGB]{42, 161, 152} "⟩" }}}{{{\color[RGB]{101, 123, 131} 
 }}}\\

}}\par
A pattern like 
\colorbox[RGB]{253,246,227}{{{{\color[RGB]{101, 123, 131} ⟨a, b, c⟩ | ⟨d, e⟩ }}}} will do a split over the inductive datatype,
naming the first three parameters of the first constructor as 
\colorbox[RGB]{253,246,227}{{{{\color[RGB]{101, 123, 131} a,b,c }}}} and the
first two of the second constructor 
\colorbox[RGB]{253,246,227}{{{{\color[RGB]{101, 123, 131} d,e }}}}. If the list is not as long as the
number of arguments to the constructor or the number of constructors, the
remaining variables will be automatically named. If there are nested brackets
such as 
\colorbox[RGB]{253,246,227}{{{{\color[RGB]{101, 123, 131} ⟨⟨a⟩, b | c⟩ | d }}}} then these will cause more case splits as necessary.
If there are too many arguments, such as 
\colorbox[RGB]{253,246,227}{{{{\color[RGB]{101, 123, 131} ⟨a, b, c⟩ }}}} for splitting on
\colorbox[RGB]{253,246,227}{{{{\color[RGB]{101, 123, 131} ∃ x, ∃ y, p x }}}}, then it will be treated as 
\colorbox[RGB]{253,246,227}{{{{\color[RGB]{101, 123, 131} ⟨a, ⟨b, c⟩⟩ }}}}, splitting the last
parameter as necessary.
\par
\colorbox[RGB]{253,246,227}{{{{\color[RGB]{101, 123, 131} rcases }}}} also has special support for quotient types: quotient induction into Prop works like
matching on the constructor 
\colorbox[RGB]{253,246,227}{{{{\color[RGB]{101, 123, 131} quot.mk }}}}.
\par
\colorbox[RGB]{253,246,227}{{{{\color[RGB]{101, 123, 131} rcases? e }}}} will perform case splits on 
\colorbox[RGB]{253,246,227}{{{{\color[RGB]{101, 123, 131} e }}}} in the same way as 
\colorbox[RGB]{253,246,227}{{{{\color[RGB]{101, 123, 131} rcases e }}}},
but rather than accepting a pattern, it does a maximal cases and prints the
pattern that would produce this case splitting. The default maximum depth is 5,
but this can be modified with 
\colorbox[RGB]{253,246,227}{{{{\color[RGB]{101, 123, 131} rcases? e : n }}}}.
\paragraph{tactic.interactive.rintro}
\par
The 
\colorbox[RGB]{253,246,227}{{{{\color[RGB]{101, 123, 131} rintro }}}} tactic is a combination of the 
\colorbox[RGB]{253,246,227}{{{{\color[RGB]{101, 123, 131} intros }}}} tactic with 
\colorbox[RGB]{253,246,227}{{{{\color[RGB]{101, 123, 131} rcases }}}} to
allow for destructuring patterns while introducing variables. See 
\colorbox[RGB]{253,246,227}{{{{\color[RGB]{101, 123, 131} rcases }}}} for
a description of supported patterns. For example, 
\colorbox[RGB]{253,246,227}{{{{\color[RGB]{101, 123, 131} rintro (a | ⟨b, c⟩) ⟨d, e⟩ }}}}will introduce two variables, and then do case splits on both of them producing
two subgoals, one with variables 
\colorbox[RGB]{253,246,227}{{{{\color[RGB]{101, 123, 131} a d e }}}} and the other with 
\colorbox[RGB]{253,246,227}{{{{\color[RGB]{101, 123, 131} b c d e }}}}.
\par
\colorbox[RGB]{253,246,227}{{{{\color[RGB]{101, 123, 131} rintro? }}}} will introduce and case split on variables in the same way as
\colorbox[RGB]{253,246,227}{{{{\color[RGB]{101, 123, 131} rintro }}}}, but will also print the 
\colorbox[RGB]{253,246,227}{{{{\color[RGB]{101, 123, 131} rintro }}}} invocation that would have the same
result. Like 
\colorbox[RGB]{253,246,227}{{{{\color[RGB]{101, 123, 131} rcases? }}}}, 
\colorbox[RGB]{253,246,227}{{{{\color[RGB]{101, 123, 131} rintro? : n }}}} allows for modifying the
depth of splitting; the default is 5.
\paragraph{tactic.interactive.rintros}
\par
Alias for 
\colorbox[RGB]{253,246,227}{{{{\color[RGB]{101, 123, 131} rintro }}}}.
\section{tactic/replacer.lean}\paragraph{tactic.def\_replacer}
\par
Define a new replaceable tactic.
\paragraph{tactic.def\_replacer\_cmd}
\par
Define a new replaceable tactic.
\section{tactic/restate\_axiom.lean}\paragraph{restate\_axiom}
\par
\colorbox[RGB]{253,246,227}{{{{\color[RGB]{101, 123, 131} restate\_axiom }}}} takes a structure field, and makes a new, definitionally simplified copy of it, appending 
\colorbox[RGB]{253,246,227}{{{{\color[RGB]{101, 123, 131} \_lemma }}}} to the name.
The main application is to provide clean versions of structure fields that have been tagged with an auto\_param.
\section{tactic/rewrite.lean}\paragraph{tactic.interactive.assoc\_rewrite}
\par
\colorbox[RGB]{253,246,227}{{{{\color[RGB]{101, 123, 131} assoc\_rewrite {[}h₀,← h₁{]}  }}}{{{\color[RGB]{133, 153, 0} at }}}{{{\color[RGB]{101, 123, 131}  ⊢ h₂ }}}} behaves like 
\colorbox[RGB]{253,246,227}{{{{\color[RGB]{101, 123, 131} rewrite {[}h₀,← h₁{]}  }}}{{{\color[RGB]{133, 153, 0} at }}}{{{\color[RGB]{101, 123, 131}  ⊢ h₂ }}}}with the exception that associativity is used implicitly to make rewriting
possible.
\paragraph{tactic.interactive.assoc\_rw}
\par
synonym for 
\colorbox[RGB]{253,246,227}{{{{\color[RGB]{101, 123, 131} assoc\_rewrite }}}}\section{tactic/rewrite\_all/basic.lean}\section{tactic/rewrite\_all/congr.lean}\section{tactic/rewrite\_all/default.lean}\paragraph{tactic.all\_rewrites}
\par
return a lazy list of (t, n, k) where
\begin{itemize}\item \colorbox[RGB]{253,246,227}{{{{\color[RGB]{101, 123, 131} t }}}} is a 
\colorbox[RGB]{253,246,227}{{{{\color[RGB]{101, 123, 131} tracked\_rewrite }}}} (i.e. a pair 
\colorbox[RGB]{253,246,227}{{{{\color[RGB]{101, 123, 131} (e' : expr, prf : e  }}}{{{\color[RGB]{181, 137, 0} = }}}{{{\color[RGB]{101, 123, 131}  e') }}}})

\item \colorbox[RGB]{253,246,227}{{{{\color[RGB]{101, 123, 131} n }}}} is the index of the rule 
\colorbox[RGB]{253,246,227}{{{{\color[RGB]{101, 123, 131} r }}}} used from 
\colorbox[RGB]{253,246,227}{{{{\color[RGB]{101, 123, 131} rs }}}}, and

\item \colorbox[RGB]{253,246,227}{{{{\color[RGB]{101, 123, 131} k }}}} is the index of 
\colorbox[RGB]{253,246,227}{{{{\color[RGB]{101, 123, 131} t }}}} in 
\colorbox[RGB]{253,246,227}{{{{\color[RGB]{101, 123, 131} all\_rewrites r e }}}}.

\end{itemize}\paragraph{tactic.interactive.perform\_nth\_rewrite}
\par
\colorbox[RGB]{253,246,227}{{{{\color[RGB]{101, 123, 131} perform\_nth\_write n rules }}}} performs only the 
\colorbox[RGB]{253,246,227}{{{{\color[RGB]{101, 123, 131} n }}}}th possible rewrite using the 
\colorbox[RGB]{253,246,227}{{{{\color[RGB]{101, 123, 131} rules }}}}.
\par
The core 
\colorbox[RGB]{253,246,227}{{{{\color[RGB]{101, 123, 131} rewrite }}}} has a 
\colorbox[RGB]{253,246,227}{{{{\color[RGB]{101, 123, 131} occs }}}} configuration setting intended to achieve a similar
purpose, but this doesn't really work. (If a rule matches twice, but with different
values of arguments, the second match will not be identified.)
\paragraph{tactic.interactive.nth\_rewrite\_lhs}
\par
As for 
\colorbox[RGB]{253,246,227}{{{{\color[RGB]{101, 123, 131} perform\_nth\_rewrite }}}}, but only working on the left hand side.
\paragraph{tactic.interactive.nth\_rewrite\_rhs}
\par
As for 
\colorbox[RGB]{253,246,227}{{{{\color[RGB]{101, 123, 131} perform\_nth\_rewrite }}}}, but only working on the right hand side.
\section{tactic/ring.lean}\paragraph{tactic.interactive.ring1}
\par
Tactic for solving equations in the language of rings.
This version of 
\colorbox[RGB]{253,246,227}{{{{\color[RGB]{101, 123, 131} ring }}}} fails if the target is not an equality
that is provable by the axioms of commutative (semi)rings.
\paragraph{tactic.interactive.ring}
\par
Tactic for solving equations in the language of rings.
Attempts to prove the goal outright if there is no 
\colorbox[RGB]{253,246,227}{{{{\color[RGB]{133, 153, 0} at }}}}specifier and the target is an equality, but if this
fails it falls back to rewriting all ring expressions
into a normal form. When writing a normal form,
\colorbox[RGB]{253,246,227}{{{{\color[RGB]{101, 123, 131} ring SOP }}}} will use sum-of-products form instead of horner form.
\colorbox[RGB]{253,246,227}{{{{\color[RGB]{101, 123, 131} ring! }}}} will use a more aggressive reducibility setting to identify atoms.
\section{tactic/ring2.lean}\paragraph{tactic.interactive.ring2}
\par
Tactic for solving equations in the language of rings.
This variant on the 
\colorbox[RGB]{253,246,227}{{{{\color[RGB]{101, 123, 131} ring }}}} tactic uses kernel computation instead
of proof generation.
\section{tactic/scc.lean}\paragraph{tactic.interactive.scc}
\par
Use the available equivalences and implications to prove
a goal of the form 
\colorbox[RGB]{253,246,227}{{{{\color[RGB]{101, 123, 131} p  }}}{{{\color[RGB]{181, 137, 0} ↔ }}}{{{\color[RGB]{101, 123, 131}  q }}}}.
\paragraph{tactic.interactive.scc'}
\par
Collect all the available equivalences and implications and
add assumptions for every equivalence that can be proven using the
strongly connected components technique. Mostly useful for testing.
\section{tactic/simpa.lean}\paragraph{tactic.interactive.simpa}
\par
This is a "finishing" tactic modification of 
\colorbox[RGB]{253,246,227}{{{{\color[RGB]{101, 123, 131} simp }}}}.
\begin{itemize}\item \par
\colorbox[RGB]{253,246,227}{{{{\color[RGB]{101, 123, 131} simpa {[}rules, ...{]}  }}}{{{\color[RGB]{133, 153, 0} using }}}{{{\color[RGB]{101, 123, 131}  e }}}} will simplify the goal and the type of
\colorbox[RGB]{253,246,227}{{{{\color[RGB]{101, 123, 131} e }}}} using 
\colorbox[RGB]{253,246,227}{{{{\color[RGB]{101, 123, 131} rules }}}}, then try to close the goal using 
\colorbox[RGB]{253,246,227}{{{{\color[RGB]{101, 123, 131} e }}}}.

\item \par
\colorbox[RGB]{253,246,227}{{{{\color[RGB]{101, 123, 131} simpa {[}rules, ...{]} }}}} will simplify the goal using 
\colorbox[RGB]{253,246,227}{{{{\color[RGB]{101, 123, 131} rules }}}}, then try
to close it using 
\colorbox[RGB]{253,246,227}{{{{\color[RGB]{101, 123, 131} assumption }}}}.

\end{itemize}\section{tactic/slice.lean}\paragraph{conv.interactive.slice}
\par
\colorbox[RGB]{253,246,227}{{{{\color[RGB]{101, 123, 131} slice }}}} is a conv tactic; if the current focus is a composition of several morphisms,
\colorbox[RGB]{253,246,227}{{{{\color[RGB]{101, 123, 131} slice a b }}}} reassociates as needed, and zooms in on the 
\colorbox[RGB]{253,246,227}{{{{\color[RGB]{101, 123, 131} a }}}}-th through 
\colorbox[RGB]{253,246,227}{{{{\color[RGB]{101, 123, 131} b }}}}-th morphisms.
\par
Thus if the current focus is 
\colorbox[RGB]{253,246,227}{{{{\color[RGB]{101, 123, 131} (a ≫ b) ≫ ((c ≫ d) ≫ e) }}}}, then 
\colorbox[RGB]{253,246,227}{{{{\color[RGB]{101, 123, 131} slice  }}}{{{\color[RGB]{108, 113, 196} 2 }}}{{{\color[RGB]{101, 123, 131}   }}}{{{\color[RGB]{108, 113, 196} 3 }}}} zooms to 
\colorbox[RGB]{253,246,227}{{{{\color[RGB]{101, 123, 131} b ≫ c }}}}.
\paragraph{tactic.interactive.slice\_lhs}
\par
\colorbox[RGB]{253,246,227}{{{{\color[RGB]{101, 123, 131} slice\_lhs a b \{ tac \} }}}} zooms to the left hand side, uses associativity for categorical
composition as needed, zooms in on the 
\colorbox[RGB]{253,246,227}{{{{\color[RGB]{101, 123, 131} a }}}}-th through 
\colorbox[RGB]{253,246,227}{{{{\color[RGB]{101, 123, 131} b }}}}-th morphisms, and invokes 
\colorbox[RGB]{253,246,227}{{{{\color[RGB]{101, 123, 131} tac }}}}.
\paragraph{tactic.interactive.slice\_rhs}
\par
\colorbox[RGB]{253,246,227}{{{{\color[RGB]{101, 123, 131} slice\_rhs a b \{ tac \} }}}} zooms to the right hand side, uses associativity for categorical
composition as needed, zooms in on the 
\colorbox[RGB]{253,246,227}{{{{\color[RGB]{101, 123, 131} a }}}}-th through 
\colorbox[RGB]{253,246,227}{{{{\color[RGB]{101, 123, 131} b }}}}-th morphisms, and invokes 
\colorbox[RGB]{253,246,227}{{{{\color[RGB]{101, 123, 131} tac }}}}.
\section{tactic/solve\_by\_elim.lean}\paragraph{tactic.interactive.apply\_assumption}
\par
\colorbox[RGB]{253,246,227}{{{{\color[RGB]{101, 123, 131} apply\_assumption }}}} looks for an assumption of the form 
\colorbox[RGB]{253,246,227}{{{{\color[RGB]{101, 123, 131} ...  }}}{{{\color[RGB]{133, 153, 0} → }}}{{{\color[RGB]{101, 123, 131}   }}}{{{\color[RGB]{133, 153, 0} ∀ }}}{{{\color[RGB]{101, 123, 131}  \_, ...  }}}{{{\color[RGB]{133, 153, 0} → }}}{{{\color[RGB]{101, 123, 131}  head }}}}where 
\colorbox[RGB]{253,246,227}{{{{\color[RGB]{101, 123, 131} head }}}} matches the current goal.
\par
alternatively, when encountering an assumption of the form 
\colorbox[RGB]{253,246,227}{{{{\color[RGB]{101, 123, 131} sg₀  }}}{{{\color[RGB]{133, 153, 0} → }}}{{{\color[RGB]{101, 123, 131}   }}}{{{\color[RGB]{181, 137, 0} ¬ }}}{{{\color[RGB]{101, 123, 131}  sg₁ }}}},
after the main approach failed, the goal is dismissed and 
\colorbox[RGB]{253,246,227}{{{{\color[RGB]{101, 123, 131} sg₀ }}}} and 
\colorbox[RGB]{253,246,227}{{{{\color[RGB]{101, 123, 131} sg₁ }}}}are made into the new goal.
\par
optional arguments:
\begin{itemize}\item asms: list of rules to consider instead of the local constants

\item tac:  a tactic to run on each subgoals after applying an assumption; if
this tactic fails, the corresponding assumption will be rejected and
the next one will be attempted.

\end{itemize}\paragraph{tactic.interactive.solve\_by\_elim}
\par
\colorbox[RGB]{253,246,227}{{{{\color[RGB]{101, 123, 131} solve\_by\_elim }}}} calls 
\colorbox[RGB]{253,246,227}{{{{\color[RGB]{101, 123, 131} apply\_assumption }}}} on the main goal to find an assumption whose head matches
and then repeatedly calls 
\colorbox[RGB]{253,246,227}{{{{\color[RGB]{101, 123, 131} apply\_assumption }}}} on the generated subgoals until no subgoals remain,
performing at most 
\colorbox[RGB]{253,246,227}{{{{\color[RGB]{101, 123, 131} max\_rep }}}} recursive steps.
\par
\colorbox[RGB]{253,246,227}{{{{\color[RGB]{101, 123, 131} solve\_by\_elim }}}} discharges the current goal or fails
\par
\colorbox[RGB]{253,246,227}{{{{\color[RGB]{101, 123, 131} solve\_by\_elim }}}} performs back-tracking if 
\colorbox[RGB]{253,246,227}{{{{\color[RGB]{101, 123, 131} apply\_assumption }}}} chooses an unproductive assumption
\par
By default, the assumptions passed to apply\_assumption are the local context, 
\colorbox[RGB]{253,246,227}{{{{\color[RGB]{101, 123, 131} congr\_fun }}}} and
\colorbox[RGB]{253,246,227}{{{{\color[RGB]{101, 123, 131} congr\_arg }}}}.
\par
\colorbox[RGB]{253,246,227}{{{{\color[RGB]{101, 123, 131} solve\_by\_elim {[}h₁, h₂, ..., hᵣ{]} }}}} also applies the named lemmas.
\par
`
solve\_by\_elim with attr₁ ... attrᵣ also applied all lemmas tagged with the specified attributes.
\par
\colorbox[RGB]{253,246,227}{{{{\color[RGB]{101, 123, 131} solve\_by\_elim only {[}h₁, h₂, ..., hᵣ{]} }}}} does not include the local context, 
\colorbox[RGB]{253,246,227}{{{{\color[RGB]{101, 123, 131} congr\_fun }}}}, or 
\colorbox[RGB]{253,246,227}{{{{\color[RGB]{101, 123, 131} congr\_arg }}}}unless they are explicitly included.
\par
\colorbox[RGB]{253,246,227}{{{{\color[RGB]{101, 123, 131} solve\_by\_elim {[} }}}{{{\color[RGB]{181, 137, 0} - }}}{{{\color[RGB]{101, 123, 131} id{]} }}}} removes a specified assumption.
\par
\colorbox[RGB]{253,246,227}{{{{\color[RGB]{101, 123, 131} solve\_by\_elim }}}{{{\color[RGB]{181, 137, 0} * }}}} tries to solve all goals together, using backtracking if a solution for one goal
makes other goals impossible.
\par
optional arguments:
\begin{itemize}\item discharger: a subsidiary tactic to try at each step (e.g. 
\colorbox[RGB]{253,246,227}{{{{\color[RGB]{101, 123, 131} cc }}}} may be helpful)

\item max\_rep: number of attempts at discharging generated sub-goals

\end{itemize}\section{tactic/split\_ifs.lean}\paragraph{tactic.interactive.split\_ifs}
\par
Splits all if-then-else-expressions into multiple goals.
\par
Given a goal of the form 
\colorbox[RGB]{253,246,227}{{{{\color[RGB]{101, 123, 131} g ( }}}{{{\color[RGB]{133, 153, 0} if }}}{{{\color[RGB]{101, 123, 131}  p  }}}{{{\color[RGB]{133, 153, 0} then }}}{{{\color[RGB]{101, 123, 131}  x  }}}{{{\color[RGB]{133, 153, 0} else }}}{{{\color[RGB]{101, 123, 131}  y) }}}}, 
\colorbox[RGB]{253,246,227}{{{{\color[RGB]{101, 123, 131} split\_ifs }}}} will produce
two goals: 
\colorbox[RGB]{253,246,227}{{{{\color[RGB]{101, 123, 131} p ⊢ g x }}}} and 
\colorbox[RGB]{253,246,227}{{{{\color[RGB]{181, 137, 0} ¬ }}}{{{\color[RGB]{101, 123, 131} p ⊢ g y }}}}.
\par
If there are multiple ite-expressions, then 
\colorbox[RGB]{253,246,227}{{{{\color[RGB]{101, 123, 131} split\_ifs }}}} will split them all,
starting with a top-most one whose condition does not contain another
ite-expression.
\par
\colorbox[RGB]{253,246,227}{{{{\color[RGB]{101, 123, 131} split\_ifs  }}}{{{\color[RGB]{133, 153, 0} at }}}{{{\color[RGB]{101, 123, 131}   }}}{{{\color[RGB]{181, 137, 0} * }}}} splits all ite-expressions in all hypotheses as well as the goal.
\par
\colorbox[RGB]{253,246,227}{{{{\color[RGB]{101, 123, 131} split\_ifs  }}}{{{\color[RGB]{133, 153, 0} with }}}{{{\color[RGB]{101, 123, 131}  h₁ h₂ h₃ }}}} overrides the default names for the hypotheses.
\section{tactic/squeeze.lean}\section{tactic/subtype\_instance.lean}\paragraph{tactic.mk\_mem\_name}
\par
makes the substructure axiom name from field name, by postfacing with 
\colorbox[RGB]{253,246,227}{{{{\color[RGB]{101, 123, 131} \_mem }}}}\paragraph{tactic.interactive.subtype\_instance}
\par
builds instances for algebraic substructures
\par
Example:
\\
\colorbox[RGB]{253,246,227}{\parbox{4.5in}{{{{\color[RGB]{133, 153, 0} variables }}}{{{\color[RGB]{101, 123, 131}  \{α :  }}}{{{\color[RGB]{38, 139, 210} Type }}}{{{\color[RGB]{181, 137, 0} * }}}{{{\color[RGB]{101, 123, 131} \} {[}monoid α{]} \{s : set α\}
 }}}\\
{{{\color[RGB]{101, 123, 131} 
 }}}\\
{{{\color[RGB]{133, 153, 0} class }}}{{{\color[RGB]{101, 123, 131}   }}}{{{\color[RGB]{211, 54, 130} is\_submonoid }}}{{{\color[RGB]{101, 123, 131}   }}}{{{\color[RGB]{101, 123, 131} (s : set α) :  }}}{{{\color[RGB]{38, 139, 210} Prop }}}{{{\color[RGB]{101, 123, 131}   }}}{{{\color[RGB]{181, 137, 0} := }}}{{{\color[RGB]{101, 123, 131} 
 }}}\\
{{{\color[RGB]{101, 123, 131} (one\_mem : ( }}}{{{\color[RGB]{108, 113, 196} 1 }}}{{{\color[RGB]{101, 123, 131} :α) ∈ s)
 }}}\\
{{{\color[RGB]{101, 123, 131} (mul\_mem \{a b\} : a ∈ s  }}}{{{\color[RGB]{133, 153, 0} → }}}{{{\color[RGB]{101, 123, 131}  b ∈ s  }}}{{{\color[RGB]{133, 153, 0} → }}}{{{\color[RGB]{101, 123, 131}  a  }}}{{{\color[RGB]{181, 137, 0} * }}}{{{\color[RGB]{101, 123, 131}  b ∈ s)
 }}}\\
{{{\color[RGB]{101, 123, 131} 
 }}}\\
{{{\color[RGB]{133, 153, 0} instance }}}{{{\color[RGB]{101, 123, 131}   }}}{{{\color[RGB]{211, 54, 130} subtype.monoid }}}{{{\color[RGB]{101, 123, 131}   }}}{{{\color[RGB]{101, 123, 131} \{s : set α\} {[}is\_submonoid s{]} : monoid s  }}}{{{\color[RGB]{181, 137, 0} := }}}{{{\color[RGB]{101, 123, 131} 
 }}}\\
{{{\color[RGB]{133, 153, 0} by }}}{{{\color[RGB]{101, 123, 131}  subtype\_instance
 }}}\\

}}\section{tactic/tauto.lean}\paragraph{tactic.distrib\_not}
\par
find all assumptions of the shape 
\colorbox[RGB]{253,246,227}{{{{\color[RGB]{181, 137, 0} ¬ }}}{{{\color[RGB]{101, 123, 131}  (p  }}}{{{\color[RGB]{181, 137, 0} ∧ }}}{{{\color[RGB]{101, 123, 131}  q) }}}} or 
\colorbox[RGB]{253,246,227}{{{{\color[RGB]{181, 137, 0} ¬ }}}{{{\color[RGB]{101, 123, 131}  (p  }}}{{{\color[RGB]{181, 137, 0} ∨ }}}{{{\color[RGB]{101, 123, 131}  q) }}}} and
replace them using de Morgan's law.
\paragraph{tactic.interactive.tautology}
\par
\colorbox[RGB]{253,246,227}{{{{\color[RGB]{101, 123, 131} tautology }}}} breaks down assumptions of the form 
\colorbox[RGB]{253,246,227}{{{{\color[RGB]{101, 123, 131} \_  }}}{{{\color[RGB]{181, 137, 0} ∧ }}}{{{\color[RGB]{101, 123, 131}  \_ }}}}, 
\colorbox[RGB]{253,246,227}{{{{\color[RGB]{101, 123, 131} \_  }}}{{{\color[RGB]{181, 137, 0} ∨ }}}{{{\color[RGB]{101, 123, 131}  \_ }}}}, 
\colorbox[RGB]{253,246,227}{{{{\color[RGB]{101, 123, 131} \_  }}}{{{\color[RGB]{181, 137, 0} ↔ }}}{{{\color[RGB]{101, 123, 131}  \_ }}}} and 
\colorbox[RGB]{253,246,227}{{{{\color[RGB]{101, 123, 131} ∃ \_, \_ }}}}and splits a goal of the form 
\colorbox[RGB]{253,246,227}{{{{\color[RGB]{101, 123, 131} \_  }}}{{{\color[RGB]{181, 137, 0} ∧ }}}{{{\color[RGB]{101, 123, 131}  \_ }}}}, 
\colorbox[RGB]{253,246,227}{{{{\color[RGB]{101, 123, 131} \_  }}}{{{\color[RGB]{181, 137, 0} ↔ }}}{{{\color[RGB]{101, 123, 131}  \_ }}}} or 
\colorbox[RGB]{253,246,227}{{{{\color[RGB]{101, 123, 131} ∃ \_, \_ }}}} until it can be discharged
using 
\colorbox[RGB]{253,246,227}{{{{\color[RGB]{101, 123, 131} reflexivity }}}} or 
\colorbox[RGB]{253,246,227}{{{{\color[RGB]{101, 123, 131} solve\_by\_elim }}}}\paragraph{tactic.interactive.tauto}
\par
Shorter name for the tactic 
\colorbox[RGB]{253,246,227}{{{{\color[RGB]{101, 123, 131} tautology }}}}.
\section{tactic/tfae.lean}\paragraph{tactic.interactive.tfae\_have}
\par
In a goal of the form 
\colorbox[RGB]{253,246,227}{{{{\color[RGB]{101, 123, 131} tfae {[}a₀, a₁, a₂{]} }}}},
\colorbox[RGB]{253,246,227}{{{{\color[RGB]{101, 123, 131} tfae\_have : i  }}}{{{\color[RGB]{133, 153, 0} → }}}{{{\color[RGB]{101, 123, 131}  j }}}} creates the assertion 
\colorbox[RGB]{253,246,227}{{{{\color[RGB]{101, 123, 131} aᵢ  }}}{{{\color[RGB]{133, 153, 0} → }}}{{{\color[RGB]{101, 123, 131}  aⱼ }}}}. The other possible
notations are 
\colorbox[RGB]{253,246,227}{{{{\color[RGB]{101, 123, 131} tfae\_have : i ← j }}}} and 
\colorbox[RGB]{253,246,227}{{{{\color[RGB]{101, 123, 131} tfae\_have : i  }}}{{{\color[RGB]{181, 137, 0} ↔ }}}{{{\color[RGB]{101, 123, 131}  j }}}}. The user can
also provide a label for the assertion, as with 
\colorbox[RGB]{253,246,227}{{{{\color[RGB]{133, 153, 0} have }}}}: 
\colorbox[RGB]{253,246,227}{{{{\color[RGB]{101, 123, 131} tfae\_have h : i  }}}{{{\color[RGB]{181, 137, 0} ↔ }}}{{{\color[RGB]{101, 123, 131}  j }}}}.
\paragraph{tactic.interactive.tfae\_finish}
\par
Finds all implications and equivalences in the context
to prove a goal of the form 
\colorbox[RGB]{253,246,227}{{{{\color[RGB]{101, 123, 131} tfae {[}...{]} }}}}.
\section{tactic/tidy.lean}\section{tactic/transfer.lean}\section{tactic/where.lean}\section{tactic/wlog.lean}\paragraph{tactic.interactive.wlog}
\par
Without loss of generality: reduces to one goal under variables permutations.
\par
Given a goal of the form 
\colorbox[RGB]{253,246,227}{{{{\color[RGB]{101, 123, 131} g xs }}}}, a predicate 
\colorbox[RGB]{253,246,227}{{{{\color[RGB]{101, 123, 131} p }}}} over a set of variables, as well as variable
permutations 
\colorbox[RGB]{253,246,227}{{{{\color[RGB]{101, 123, 131} xs\_i }}}}. Then 
\colorbox[RGB]{253,246,227}{{{{\color[RGB]{101, 123, 131} wlog }}}} produces goals of the form
\par
The case goal, i.e. the permutation 
\colorbox[RGB]{253,246,227}{{{{\color[RGB]{101, 123, 131} xs\_i }}}} covers all possible cases:
\colorbox[RGB]{253,246,227}{{{{\color[RGB]{101, 123, 131} ⊢ p xs\_0  }}}{{{\color[RGB]{181, 137, 0} ∨ }}}{{{\color[RGB]{101, 123, 131}  ⋯  }}}{{{\color[RGB]{181, 137, 0} ∨ }}}{{{\color[RGB]{101, 123, 131}  p xs\_n }}}}The main goal, i.e. the goal reduced to 
\colorbox[RGB]{253,246,227}{{{{\color[RGB]{101, 123, 131} xs\_0 }}}}:
\colorbox[RGB]{253,246,227}{{{{\color[RGB]{101, 123, 131} (h : p xs\_0) ⊢ g xs\_0 }}}}The invariant goals, i.e. 
\colorbox[RGB]{253,246,227}{{{{\color[RGB]{101, 123, 131} g }}}} is invariant under 
\colorbox[RGB]{253,246,227}{{{{\color[RGB]{101, 123, 131} xs\_i }}}}:
\colorbox[RGB]{253,246,227}{{{{\color[RGB]{101, 123, 131} (h : p xs\_i) ( }}}{{{\color[RGB]{133, 153, 0} this }}}{{{\color[RGB]{101, 123, 131}  : g xs\_0) ⊢ gs xs\_i }}}}\par
Either the permutation is provided, or a proof of the disjunction is provided to compute the
permutation. The disjunction need to be in assoc normal form, e.g. 
\colorbox[RGB]{253,246,227}{{{{\color[RGB]{101, 123, 131} p₀  }}}{{{\color[RGB]{181, 137, 0} ∨ }}}{{{\color[RGB]{101, 123, 131}  (p₁  }}}{{{\color[RGB]{181, 137, 0} ∨ }}}{{{\color[RGB]{101, 123, 131}  p₂) }}}}. In many cases
the invariant goals can be solved by AC rewriting using 
\colorbox[RGB]{253,246,227}{{{{\color[RGB]{101, 123, 131} cc }}}} etc.
\par
Example:
On a state 
\colorbox[RGB]{253,246,227}{{{{\color[RGB]{101, 123, 131} (n m : ℕ) ⊢ p n m }}}} the tactic 
\colorbox[RGB]{253,246,227}{{{{\color[RGB]{101, 123, 131} wlog h : n  }}}{{{\color[RGB]{181, 137, 0} ≤ }}}{{{\color[RGB]{101, 123, 131}  m  }}}{{{\color[RGB]{133, 153, 0} using }}}{{{\color[RGB]{101, 123, 131}  {[}n m, m n{]} }}}} produces the following
states:
\colorbox[RGB]{253,246,227}{{{{\color[RGB]{101, 123, 131} (n m : ℕ) ⊢ n  }}}{{{\color[RGB]{181, 137, 0} ≤ }}}{{{\color[RGB]{101, 123, 131}  m  }}}{{{\color[RGB]{181, 137, 0} ∨ }}}{{{\color[RGB]{101, 123, 131}  m  }}}{{{\color[RGB]{181, 137, 0} ≤ }}}{{{\color[RGB]{101, 123, 131}  n }}}}\colorbox[RGB]{253,246,227}{{{{\color[RGB]{101, 123, 131} (n m : ℕ) (h : n  }}}{{{\color[RGB]{181, 137, 0} ≤ }}}{{{\color[RGB]{101, 123, 131}  m) ⊢ p n m }}}}\colorbox[RGB]{253,246,227}{{{{\color[RGB]{101, 123, 131} (n m : ℕ) (h : m  }}}{{{\color[RGB]{181, 137, 0} ≤ }}}{{{\color[RGB]{101, 123, 131}  n) ( }}}{{{\color[RGB]{133, 153, 0} this }}}{{{\color[RGB]{101, 123, 131}  : p n m) ⊢ p m n }}}}\par
\colorbox[RGB]{253,246,227}{{{{\color[RGB]{101, 123, 131} wlog }}}} supports different calling conventions. The name 
\colorbox[RGB]{253,246,227}{{{{\color[RGB]{101, 123, 131} h }}}} is used to give a name to the introduced
case hypothesis. If the name is avoided, the default will be 
\colorbox[RGB]{253,246,227}{{{{\color[RGB]{101, 123, 131} case }}}}.
\par
(1) 
\colorbox[RGB]{253,246,227}{{{{\color[RGB]{101, 123, 131} wlog : p xs0  }}}{{{\color[RGB]{133, 153, 0} using }}}{{{\color[RGB]{101, 123, 131}  {[}xs0, …, xsn{]} }}}}Results in the case goal 
\colorbox[RGB]{253,246,227}{{{{\color[RGB]{101, 123, 131} p xs0  }}}{{{\color[RGB]{181, 137, 0} ∨ }}}{{{\color[RGB]{101, 123, 131}  ⋯  }}}{{{\color[RGB]{181, 137, 0} ∨ }}}{{{\color[RGB]{101, 123, 131}  ps xsn }}}}, the main goal 
\colorbox[RGB]{253,246,227}{{{{\color[RGB]{101, 123, 131} (case : p xs0) ⊢ g xs0 }}}} and the
invariance goals 
\colorbox[RGB]{253,246,227}{{{{\color[RGB]{101, 123, 131} (case : p xsi) ( }}}{{{\color[RGB]{133, 153, 0} this }}}{{{\color[RGB]{101, 123, 131}  : g xs0) ⊢ g xsi }}}}.
\par
(2) 
\colorbox[RGB]{253,246,227}{{{{\color[RGB]{101, 123, 131} wlog : p xs0  }}}{{{\color[RGB]{181, 137, 0} := }}}{{{\color[RGB]{101, 123, 131}  r  }}}{{{\color[RGB]{133, 153, 0} using }}}{{{\color[RGB]{101, 123, 131}  xs0 }}}}The expression 
\colorbox[RGB]{253,246,227}{{{{\color[RGB]{101, 123, 131} r }}}} is a proof of the shape 
\colorbox[RGB]{253,246,227}{{{{\color[RGB]{101, 123, 131} p xs0  }}}{{{\color[RGB]{181, 137, 0} ∨ }}}{{{\color[RGB]{101, 123, 131}  ⋯  }}}{{{\color[RGB]{181, 137, 0} ∨ }}}{{{\color[RGB]{101, 123, 131}  p xsi }}}}, it is also used to compute the
variable permutations.
\par
(3) 
\colorbox[RGB]{253,246,227}{{{{\color[RGB]{101, 123, 131} wlog  }}}{{{\color[RGB]{181, 137, 0} := }}}{{{\color[RGB]{101, 123, 131}  r  }}}{{{\color[RGB]{133, 153, 0} using }}}{{{\color[RGB]{101, 123, 131}  xs0 }}}}The expression 
\colorbox[RGB]{253,246,227}{{{{\color[RGB]{101, 123, 131} r }}}} is a proof of the shape 
\colorbox[RGB]{253,246,227}{{{{\color[RGB]{101, 123, 131} p xs0  }}}{{{\color[RGB]{181, 137, 0} ∨ }}}{{{\color[RGB]{101, 123, 131}  ⋯  }}}{{{\color[RGB]{181, 137, 0} ∨ }}}{{{\color[RGB]{101, 123, 131}  p xsi }}}}, it is also used to compute the
variable permutations. This is not as stable as (2), for example 
\colorbox[RGB]{253,246,227}{{{{\color[RGB]{101, 123, 131} p }}}} cannot be a disjunction.
\par
(4) 
\colorbox[RGB]{253,246,227}{{{{\color[RGB]{101, 123, 131} wlog : R x y  }}}{{{\color[RGB]{133, 153, 0} using }}}{{{\color[RGB]{101, 123, 131}  x y }}}} and 
\colorbox[RGB]{253,246,227}{{{{\color[RGB]{101, 123, 131} wlog : R x y }}}}Produces the case 
\colorbox[RGB]{253,246,227}{{{{\color[RGB]{101, 123, 131} R x y  }}}{{{\color[RGB]{181, 137, 0} ∨ }}}{{{\color[RGB]{101, 123, 131}  R y x }}}}. If 
\colorbox[RGB]{253,246,227}{{{{\color[RGB]{101, 123, 131} R }}}} is ≤, then the disjunction discharged using linearity.
If 
\colorbox[RGB]{253,246,227}{{{{\color[RGB]{133, 153, 0} using }}}{{{\color[RGB]{101, 123, 131}  x y }}}} is avoided then 
\colorbox[RGB]{253,246,227}{{{{\color[RGB]{101, 123, 131} x }}}} and 
\colorbox[RGB]{253,246,227}{{{{\color[RGB]{101, 123, 131} y }}}} are the last two variables appearing in the
expression 
\colorbox[RGB]{253,246,227}{{{{\color[RGB]{101, 123, 131} R x y }}}}.
\section{topology/Top/adjunctions.lean}\section{topology/Top/basic.lean}\paragraph{Top}
\par
The category of topological spaces and continuous maps.
\section{topology/Top/default.lean}\section{topology/Top/epi\_mono.lean}\section{topology/Top/limits.lean}\section{topology/Top/open\_nhds.lean}\section{topology/Top/opens.lean}\paragraph{topological\_space.opens.map}
\par
\colorbox[RGB]{253,246,227}{{{{\color[RGB]{101, 123, 131} opens.map f }}}} gives the functor from open sets in Y to open set in X,
given by taking preimages under f.
\section{topology/Top/presheaf.lean}\section{topology/Top/presheaf\_of\_functions.lean}\section{topology/Top/stalks.lean}\paragraph{Top.presheaf.stalk\_functor}
\par
Stalks are functorial with respect to morphisms of presheaves over a fixed 
\colorbox[RGB]{253,246,227}{{{{\color[RGB]{101, 123, 131} X }}}}.
\paragraph{Top.presheaf.stalk}
\par
The stalk of a presheaf 
\colorbox[RGB]{253,246,227}{{{{\color[RGB]{101, 123, 131} F }}}} at a point 
\colorbox[RGB]{253,246,227}{{{{\color[RGB]{101, 123, 131} x }}}} is calculated as the colimit of the functor
nbhds x ⥤ opens F.X ⥤ C
\section{topology/algebra/TopCommRing/basic.lean}\paragraph{TopCommRing.forget\_to\_CommRing}
\par
The forgetful functor to CommRing.
\paragraph{TopCommRing.forget\_to\_Top}
\par
The forgetful functor to Top.
\section{topology/algebra/TopCommRing/default.lean}\section{topology/algebra/continuous\_functions.lean}\section{topology/algebra/group.lean}\paragraph{topological\_group}
\par
A topological group is a group in which the multiplication and inversion operations are
continuous.
\paragraph{topological\_add\_group}
\par
A topological (additive) group is a group in which the addition and negation operations are
continuous.
\paragraph{add\_group\_with\_zero\_nhd}
\par
additive group with a neighbourhood around 0.
Only used to construct a topology and uniform space.
\par
This is currently only available for commutative groups, but it can be extended to
non-commutative groups too.
\section{topology/algebra/group\_completion.lean}\section{topology/algebra/infinite\_sum.lean}\paragraph{has\_sum}
\par
Infinite sum on a topological monoid
The 
\colorbox[RGB]{253,246,227}{{{{\color[RGB]{101, 123, 131} at\_top }}}} filter on 
\colorbox[RGB]{253,246,227}{{{{\color[RGB]{101, 123, 131} finset α }}}} is the limit of all finite sets towards the entire type. So we sum
up bigger and bigger sets. This sum operation is still invariant under reordering, and a absolute
sum operator.
\par
This is based on Mario Carneiro's infinite sum in Metamath.
\paragraph{summable}
\par
\colorbox[RGB]{253,246,227}{{{{\color[RGB]{101, 123, 131} summable f }}}} means that 
\colorbox[RGB]{253,246,227}{{{{\color[RGB]{101, 123, 131} f }}}} has some (infinite) sum. Use 
\colorbox[RGB]{253,246,227}{{{{\color[RGB]{101, 123, 131} tsum }}}} to get the value.
\paragraph{tsum}
\par
\colorbox[RGB]{253,246,227}{{{{\color[RGB]{101, 123, 131} tsum f }}}} is the sum of 
\colorbox[RGB]{253,246,227}{{{{\color[RGB]{101, 123, 131} f }}}} it exists, or 0 otherwise
\section{topology/algebra/module.lean}\paragraph{topological\_semimodule}
\par
A topological semimodule, over a semiring which is also a topological space, is a
semimodule in which scalar multiplication is continuous. In applications, α will be a topological
semiring and β a topological additive semigroup, but this is not needed for the definition
\paragraph{topological\_module}
\par
A topological module, over a ring which is also a topological space, is a module in which
scalar multiplication is continuous. In applications, α will be a topological ring and β a
topological additive group, but this is not needed for the definition
\paragraph{continuous\_linear\_map.has\_coe}
\par
Coerce continuous linear maps to linear maps.
\paragraph{continuous\_linear\_map.to\_fun}
\par
Coerce continuous linear maps to functions.
\paragraph{continuous\_linear\_map.zero}
\par
The continuous map that is constantly zero.
\paragraph{continuous\_linear\_map.id}
\par
the identity map as a continuous linear map.
\paragraph{continuous\_linear\_map.comp}
\par
Composition of bounded linear maps.
\paragraph{continuous\_linear\_map.prod}
\par
The cartesian product of two bounded linear maps, as a bounded linear map.
\paragraph{continuous\_linear\_map.scalar\_prod\_space\_iso}
\par
Associating to a scalar-valued linear map and an element of 
\colorbox[RGB]{253,246,227}{{{{\color[RGB]{101, 123, 131} γ }}}} the
\colorbox[RGB]{253,246,227}{{{{\color[RGB]{101, 123, 131} γ }}}}-valued linear map obtained by multiplying the two (a.k.a. tensoring by 
\colorbox[RGB]{253,246,227}{{{{\color[RGB]{101, 123, 131} γ }}}})
\section{topology/algebra/monoid.lean}\paragraph{topological\_monoid}
\par
A topological monoid is a monoid in which the multiplication is continuous as a function
\colorbox[RGB]{253,246,227}{{{{\color[RGB]{101, 123, 131} α × α  }}}{{{\color[RGB]{133, 153, 0} → }}}{{{\color[RGB]{101, 123, 131}  α }}}}.
\paragraph{topological\_add\_monoid}
\par
A topological (additive) monoid is a monoid in which the addition is
continuous as a function 
\colorbox[RGB]{253,246,227}{{{{\color[RGB]{101, 123, 131} α × α  }}}{{{\color[RGB]{133, 153, 0} → }}}{{{\color[RGB]{101, 123, 131}  α }}}}.
\section{topology/algebra/open\_subgroup.lean}\paragraph{open\_subgroup}
\par
The type of open subgroups of a topological group.
\section{topology/algebra/ordered.lean}\paragraph{ordered\_topology}
\par
(Partially) ordered topology
Also called: partially ordered spaces (pospaces).
\par
Usually ordered topology is used for a topology on linear ordered spaces, where the open intervals
are open sets. This is a generalization as for each linear order where open interals are open sets,
the order relation is closed.
\paragraph{orderable\_topology}
\par
Topologies generated by the open intervals.
\par
This is restricted to linear orders. Only then it is guaranteed that they are also a ordered
topology.
\paragraph{tendsto\_of\_tendsto\_of\_tendsto\_of\_le\_of\_le}
\par
Also known as squeeze or sandwich theorem.
\paragraph{bdd\_below\_of\_compact}
\par
A compact set is bounded below
\paragraph{bdd\_above\_of\_compact}
\par
A compact set is bounded above
\paragraph{Sup\_of\_continuous'}
\par
A continuous monotone function sends supremum to supremum for nonempty sets.
\paragraph{Sup\_of\_continuous}
\par
A continuous monotone function sending bot to bot sends supremum to supremum.
\paragraph{supr\_of\_continuous}
\par
A continuous monotone function sends indexed supremum to indexed supremum.
\paragraph{Inf\_of\_continuous'}
\par
A continuous monotone function sends infimum to infimum for nonempty sets.
\paragraph{Inf\_of\_continuous}
\par
A continuous monotone function sending top to top sends infimum to infimum.
\paragraph{infi\_of\_continuous}
\par
A continuous monotone function sends indexed infimum to indexed infimum.
\paragraph{cSup\_of\_cSup\_of\_monotone\_of\_continuous}
\par
A continuous monotone function sends supremum to supremum in conditionally complete
lattices, under a boundedness assumption.
\paragraph{csupr\_of\_csupr\_of\_monotone\_of\_continuous}
\par
A continuous monotone function sends indexed supremum to indexed supremum in conditionally complete
lattices, under a boundedness assumption.
\paragraph{cInf\_of\_cInf\_of\_monotone\_of\_continuous}
\par
A continuous monotone function sends infimum to infimum in conditionally complete
lattices, under a boundedness assumption.
\paragraph{cinfi\_of\_cinfi\_of\_monotone\_of\_continuous}
\par
A continuous monotone function sends indexed infimum to indexed infimum in conditionally complete
lattices, under a boundedness assumption.
\paragraph{exists\_forall\_le\_of\_compact\_of\_continuous}
\par
The extreme value theorem: a continuous function realizes its minimum on a compact set
\paragraph{exists\_forall\_ge\_of\_compact\_of\_continuous}
\par
The extreme value theorem: a continuous function realizes its maximum on a compact set
\paragraph{le\_nhds\_of\_Limsup\_eq\_Liminf}
\par
If the liminf and the limsup of a filter coincide, then this filter converges to
their common value, at least if the filter is eventually bounded above and below.
\paragraph{Liminf\_eq\_of\_le\_nhds}
\par
If a filter is converging, its limsup coincides with its limit.
\paragraph{Limsup\_eq\_of\_le\_nhds}
\par
If a filter is converging, its liminf coincides with its limit.
\section{topology/algebra/ring.lean}\paragraph{topological\_semiring}
\par
A topological semiring is a semiring where addition and multiplication are continuous.
\paragraph{topological\_ring}
\par
A topological ring is a ring where the ring operations are continuous.
\section{topology/algebra/uniform\_group.lean}\paragraph{uniform\_add\_group}
\par
A uniform (additive) group is a group in which the addition and negation are
uniformly continuous.
\paragraph{dense\_embedding.extend\_Z\_bilin}
\par
Bourbaki GT III.6.5 Theorem I:
ℤ-bilinear continuous maps from dense sub-groups into a complete Hausdorff group extend by continuity.
Note: Bourbaki assumes that α and β are also complete Hausdorff, but this is not necessary.
\section{topology/algebra/uniform\_ring.lean}\section{topology/bases.lean}\paragraph{topological\_space.is\_topological\_basis}
\par
A topological basis is one that satisfies the necessary conditions so that
it suffices to take unions of the basis sets to get a topology (without taking
finite intersections as well).
\paragraph{topological\_space.separable\_space}
\par
A separable space is one with a countable dense subset.
\paragraph{topological\_space.first\_countable\_topology}
\par
A first-countable space is one in which every point has a
countable neighborhood basis.
\paragraph{topological\_space.second\_countable\_topology}
\par
A second-countable space is one with a countable basis.
\section{topology/basic.lean}\paragraph{is\_open}
\par
\colorbox[RGB]{253,246,227}{{{{\color[RGB]{101, 123, 131} is\_open s }}}} means that 
\colorbox[RGB]{253,246,227}{{{{\color[RGB]{101, 123, 131} s }}}} is open in the ambient topological space on 
\colorbox[RGB]{253,246,227}{{{{\color[RGB]{101, 123, 131} α }}}}\paragraph{is\_closed}
\par
A set is closed if its complement is open
\paragraph{interior}
\par
The interior of a set 
\colorbox[RGB]{253,246,227}{{{{\color[RGB]{101, 123, 131} s }}}} is the largest open subset of 
\colorbox[RGB]{253,246,227}{{{{\color[RGB]{101, 123, 131} s }}}}.
\paragraph{closure}
\par
The closure of 
\colorbox[RGB]{253,246,227}{{{{\color[RGB]{101, 123, 131} s }}}} is the smallest closed set containing 
\colorbox[RGB]{253,246,227}{{{{\color[RGB]{101, 123, 131} s }}}}.
\paragraph{frontier}
\par
The frontier of a set is the set of points between the closure and interior.
\paragraph{nhds}
\par
neighbourhood filter
\paragraph{mem\_closure\_iff\_ultrafilter}
\par
\colorbox[RGB]{253,246,227}{{{{\color[RGB]{101, 123, 131} x }}}} belongs to the closure of 
\colorbox[RGB]{253,246,227}{{{{\color[RGB]{101, 123, 131} s }}}} if and only if some ultrafilter
supported on 
\colorbox[RGB]{253,246,227}{{{{\color[RGB]{101, 123, 131} s }}}} converges to 
\colorbox[RGB]{253,246,227}{{{{\color[RGB]{101, 123, 131} x }}}}.
\paragraph{lim}
\par
If 
\colorbox[RGB]{253,246,227}{{{{\color[RGB]{101, 123, 131} f }}}} is a filter, then 
\colorbox[RGB]{253,246,227}{{{{\color[RGB]{101, 123, 131} lim f }}}} is a limit of the filter, if it exists.
\paragraph{locally\_finite}
\par
A family of sets in 
\colorbox[RGB]{253,246,227}{{{{\color[RGB]{101, 123, 131} set α }}}} is locally finite if at every point 
\colorbox[RGB]{253,246,227}{{{{\color[RGB]{101, 123, 131} x:α }}}},
there is a neighborhood of 
\colorbox[RGB]{253,246,227}{{{{\color[RGB]{101, 123, 131} x }}}} which meets only finitely many sets in the family
\paragraph{continuous}
\par
A function between topological spaces is continuous if the preimage
of every open set is open.
\section{topology/bounded\_continuous\_function.lean}\paragraph{continuous\_of\_locally\_uniform\_limit\_of\_continuous}
\par
A locally uniform limit of continuous functions is continuous
\paragraph{continuous\_of\_uniform\_limit\_of\_continuous}
\par
A uniform limit of continuous functions is continuous
\paragraph{bounded\_continuous\_function}
\par
The type of bounded continuous functions from a topological space to a metric space
\paragraph{bounded\_continuous\_function.mk\_of\_compact}
\par
If a function is continuous on a compact space, it is automatically bounded,
and therefore gives rise to an element of the type of bounded continuous functions
\paragraph{bounded\_continuous\_function.mk\_of\_discrete}
\par
If a function is bounded on a discrete space, it is automatically continuous,
and therefore gives rise to an element of the type of bounded continuous functions
\paragraph{bounded\_continuous\_function.has\_dist}
\par
The uniform distance between two bounded continuous functions
\paragraph{bounded\_continuous\_function.dist\_coe\_le\_dist}
\par
The pointwise distance is controlled by the distance between functions, by definition
\paragraph{bounded\_continuous\_function.dist\_le}
\par
The distance between two functions is controlled by the supremum of the pointwise distances
\paragraph{bounded\_continuous\_function.dist\_zero\_of\_empty}
\par
On an empty space, bounded continuous functions are at distance 0
\paragraph{bounded\_continuous\_function.metric\_space}
\par
The type of bounded continuous functions, with the uniform distance, is a metric space.
\paragraph{bounded\_continuous\_function.inhabited}
\par
If the target space is inhabited, so is the space of bounded continuous functions
\paragraph{bounded\_continuous\_function.continuous\_eval}
\par
The evaluation map is continuous, as a joint function of 
\colorbox[RGB]{253,246,227}{{{{\color[RGB]{101, 123, 131} u }}}} and 
\colorbox[RGB]{253,246,227}{{{{\color[RGB]{101, 123, 131} x }}}}\paragraph{bounded\_continuous\_function.continuous\_evalx}
\par
In particular, when 
\colorbox[RGB]{253,246,227}{{{{\color[RGB]{101, 123, 131} x }}}} is fixed, 
\colorbox[RGB]{253,246,227}{{{{\color[RGB]{101, 123, 131} f  }}}{{{\color[RGB]{133, 153, 0} → }}}{{{\color[RGB]{101, 123, 131}  f x }}}} is continuous
\paragraph{bounded\_continuous\_function.continuous\_evalf}
\par
When 
\colorbox[RGB]{253,246,227}{{{{\color[RGB]{101, 123, 131} f }}}} is fixed, 
\colorbox[RGB]{253,246,227}{{{{\color[RGB]{101, 123, 131} x  }}}{{{\color[RGB]{133, 153, 0} → }}}{{{\color[RGB]{101, 123, 131}  f x }}}} is also continuous, by definition
\paragraph{bounded\_continuous\_function.complete\_space}
\par
Bounded continuous functions taking values in a complete space form a complete space.
\paragraph{bounded\_continuous\_function.comp}
\par
Composition (in the target) of a bounded continuous function with a Lipschitz map again
gives a bounded continuous function
\paragraph{bounded\_continuous\_function.continuous\_comp}
\par
The composition operator (in the target) with a Lipschitz map is continuous
\paragraph{bounded\_continuous\_function.cod\_restrict}
\par
Restriction (in the target) of a bounded continuous function taking values in a subset
\paragraph{bounded\_continuous\_function.arzela\_ascoli₁}
\par
First version, with pointwise equicontinuity and range in a compact space
\paragraph{bounded\_continuous\_function.arzela\_ascoli₂}
\par
Second version, with pointwise equicontinuity and range in a compact subset
\paragraph{bounded\_continuous\_function.arzela\_ascoli}
\par
Third (main) version, with pointwise equicontinuity and range in a compact subset, but
without closedness. The closure is then compact
\paragraph{bounded\_continuous\_function.norm\_le}
\par
The norm of a function is controlled by the supremum of the pointwise norms
\paragraph{bounded\_continuous\_function.has\_add}
\par
The pointwise sum of two bounded continuous functions is again bounded continuous.
\paragraph{bounded\_continuous\_function.has\_neg}
\par
The pointwise opposite of a bounded continuous function is again bounded continuous.
\paragraph{bounded\_continuous\_function.of\_normed\_group}
\par
Constructing a bounded continuous function from a uniformly bounded continuous
function taking values in a normed group.
\paragraph{bounded\_continuous\_function.of\_normed\_group\_discrete}
\par
Constructing a bounded continuous function from a uniformly bounded
function on a discrete space, taking values in a normed group
\section{topology/compact\_open.lean}\paragraph{continuous\_map.continuous\_induced}
\par
C(α, -) is a functor.
\section{topology/constructions.lean}\paragraph{compact\_pi\_infinite}
\par
Tychonoff's theorem
\paragraph{continuous\_sigma}
\par
A map out of a sum type is continuous if its restriction to each summand is.
\paragraph{embedding\_sigma\_map}
\par
The sum of embeddings is an embedding.
\paragraph{dense\_embedding.prod}
\par
The product of two dense embeddings is a dense embedding
\paragraph{homeomorph}
\par
α and β are homeomorph, also called topological isomoph
\section{topology/instances/complex.lean}\section{topology/instances/ennreal.lean}\paragraph{ennreal.topological\_space}
\par
Topology on 
\colorbox[RGB]{253,246,227}{{{{\color[RGB]{101, 123, 131} ennreal }}}}.
\par
Note: this is different from the 
\colorbox[RGB]{253,246,227}{{{{\color[RGB]{101, 123, 131} emetric\_space }}}} topology. The 
\colorbox[RGB]{253,246,227}{{{{\color[RGB]{101, 123, 131} emetric\_space }}}} topology has
\colorbox[RGB]{253,246,227}{{{{\color[RGB]{101, 123, 131} is\_open \{⊤\} }}}}, while this topology doesn't have singleton elements.
\paragraph{edist\_ne\_top\_of\_mem\_ball}
\par
In an emetric ball, the distance between points is everywhere finite
\paragraph{metric\_space\_emetric\_ball}
\par
Each ball in an extended metric space gives us a metric space, as the edist
is everywhere finite.
\paragraph{emetric.cauchy\_seq\_iff\_le\_tendsto\_0}
\par
Yet another metric characterization of Cauchy sequences on integers. This one is often the
most efficient.
\section{topology/instances/nnreal.lean}\section{topology/instances/polynomial.lean}\section{topology/instances/real.lean}\section{topology/maps.lean}\paragraph{embedding}
\par
A function between topological spaces is an embedding if it is injective,
and for all 
\colorbox[RGB]{253,246,227}{{{{\color[RGB]{101, 123, 131} s : set α }}}}, 
\colorbox[RGB]{253,246,227}{{{{\color[RGB]{101, 123, 131} s }}}} is open iff it is the preimage of an open set.
\paragraph{dense\_embedding.tendsto\_comap\_nhds\_nhds}
\par
γ -f→ α
g↓     ↓e
δ -h→ β
\paragraph{dense\_embedding.extend}
\par
If 
\colorbox[RGB]{253,246,227}{{{{\color[RGB]{101, 123, 131} e : α  }}}{{{\color[RGB]{133, 153, 0} → }}}{{{\color[RGB]{101, 123, 131}  β }}}} is a dense embedding, then any function 
\colorbox[RGB]{253,246,227}{{{{\color[RGB]{101, 123, 131} α  }}}{{{\color[RGB]{133, 153, 0} → }}}{{{\color[RGB]{101, 123, 131}  γ }}}} extends to a function 
\colorbox[RGB]{253,246,227}{{{{\color[RGB]{101, 123, 131} β  }}}{{{\color[RGB]{133, 153, 0} → }}}{{{\color[RGB]{101, 123, 131}  γ }}}}.
It only extends the parts of 
\colorbox[RGB]{253,246,227}{{{{\color[RGB]{101, 123, 131} β }}}} which are not mapped by 
\colorbox[RGB]{253,246,227}{{{{\color[RGB]{101, 123, 131} e }}}}, everything else equal to 
\colorbox[RGB]{253,246,227}{{{{\color[RGB]{101, 123, 131} f (e a) }}}}.
This allows us to gain equality even if 
\colorbox[RGB]{253,246,227}{{{{\color[RGB]{101, 123, 131} γ }}}} is not T2.
\paragraph{quotient\_map}
\par
A function between topological spaces is a quotient map if it is surjective,
and for all 
\colorbox[RGB]{253,246,227}{{{{\color[RGB]{101, 123, 131} s : set β }}}}, 
\colorbox[RGB]{253,246,227}{{{{\color[RGB]{101, 123, 131} s }}}} is open iff its preimage is an open set.
\paragraph{closed\_embedding}
\par
A closed embedding is an embedding with closed image.
\section{topology/metric\_space/baire.lean}\paragraph{is\_Gδ}
\par
A Gδ set is a countable intersection of open sets.
\paragraph{is\_open.is\_Gδ}
\par
An open set is a Gδ set.
\paragraph{is\_Gδ\_sInter}
\par
A countable intersection of Gδ sets is a Gδ set.
\paragraph{is\_Gδ.union}
\par
The union of two Gδ sets is a Gδ set.
\paragraph{dense\_Inter\_of\_open\_nat}
\par
Baire theorem: a countable intersection of dense open sets is dense. Formulated here when
the source space is ℕ (and subsumed below by 
\colorbox[RGB]{253,246,227}{{{{\color[RGB]{101, 123, 131} dense\_Inter\_of\_open }}}} working with any
encodable source space).
\paragraph{dense\_sInter\_of\_open}
\par
Baire theorem: a countable intersection of dense open sets is dense. Formulated here with ⋂₀.
\paragraph{dense\_bInter\_of\_open}
\par
Baire theorem: a countable intersection of dense open sets is dense. Formulated here with
an index set which is a countable set in any type.
\paragraph{dense\_Inter\_of\_open}
\par
Baire theorem: a countable intersection of dense open sets is dense. Formulated here with
an index set which is an encodable type.
\paragraph{dense\_sInter\_of\_Gδ}
\par
Baire theorem: a countable intersection of dense Gδ sets is dense. Formulated here with ⋂₀.
\paragraph{dense\_bInter\_of\_Gδ}
\par
Baire theorem: a countable intersection of dense Gδ sets is dense. Formulated here with
an index set which is a countable set in any type.
\paragraph{dense\_Inter\_of\_Gδ}
\par
Baire theorem: a countable intersection of dense Gδ sets is dense. Formulated here with
an index set which is an encodable type.
\paragraph{dense\_bUnion\_interior\_of\_closed}
\par
Baire theorem: if countably many closed sets cover the whole space, then their interiors
are dense. Formulated here with an index set which is a countable set in any type.
\paragraph{dense\_sUnion\_interior\_of\_closed}
\par
Baire theorem: if countably many closed sets cover the whole space, then their interiors
are dense. Formulated here with ⋃₀.
\paragraph{dense\_Union\_interior\_of\_closed}
\par
Baire theorem: if countably many closed sets cover the whole space, then their interiors
are dense. Formulated here with an index set which is an encodable type.
\paragraph{nonempty\_interior\_of\_Union\_of\_closed}
\par
One of the most useful consequences of Baire theorem: if a countable union of closed sets
covers the space, then one of the sets has nonempty interior.
\section{topology/metric\_space/basic.lean}\paragraph{uniform\_space\_of\_dist}
\par
Construct a uniform structure from a distance function and metric space axioms
\paragraph{has\_dist}
\par
The distance function (given an ambient metric space on 
\colorbox[RGB]{253,246,227}{{{{\color[RGB]{101, 123, 131} α }}}}), which returns
a nonnegative real number 
\colorbox[RGB]{253,246,227}{{{{\color[RGB]{101, 123, 131} dist x y }}}} given 
\colorbox[RGB]{253,246,227}{{{{\color[RGB]{101, 123, 131} x y : α }}}}.
\paragraph{metric\_space}
\par
Metric space
\par
Each metric space induces a canonical 
\colorbox[RGB]{253,246,227}{{{{\color[RGB]{101, 123, 131} uniform\_space }}}} and hence a canonical 
\colorbox[RGB]{253,246,227}{{{{\color[RGB]{101, 123, 131} topological\_space }}}}.
This is enforced in the type class definition, by extending the 
\colorbox[RGB]{253,246,227}{{{{\color[RGB]{101, 123, 131} uniform\_space }}}} structure. When
instantiating a 
\colorbox[RGB]{253,246,227}{{{{\color[RGB]{101, 123, 131} metric\_space }}}} structure, the uniformity fields are not necessary, they will be
filled in by default. In the same way, each metric space induces an emetric space structure.
It is included in the structure, but filled in by default.
\par
When one instantiates a metric space structure, for instance a product structure,
this makes it possible to use a uniform structure and an edistance that are exactly
the ones for the uniform spaces product and the emetric spaces products, thereby
ensuring that everything in defeq in diamonds.
\paragraph{nndist\_edist}
\par
Express 
\colorbox[RGB]{253,246,227}{{{{\color[RGB]{101, 123, 131} nndist }}}} in terms of 
\colorbox[RGB]{253,246,227}{{{{\color[RGB]{101, 123, 131} edist }}}}\paragraph{edist\_nndist}
\par
Express 
\colorbox[RGB]{253,246,227}{{{{\color[RGB]{101, 123, 131} edist }}}} in terms of 
\colorbox[RGB]{253,246,227}{{{{\color[RGB]{101, 123, 131} nndist }}}}\paragraph{edist\_ne\_top}
\par
In a metric space, the extended distance is always finite
\paragraph{nndist\_self}
\par
\colorbox[RGB]{253,246,227}{{{{\color[RGB]{101, 123, 131} nndist x x }}}} vanishes
\paragraph{dist\_nndist}
\par
Express 
\colorbox[RGB]{253,246,227}{{{{\color[RGB]{101, 123, 131} dist }}}} in terms of 
\colorbox[RGB]{253,246,227}{{{{\color[RGB]{101, 123, 131} nndist }}}}\paragraph{nndist\_dist}
\par
Express 
\colorbox[RGB]{253,246,227}{{{{\color[RGB]{101, 123, 131} nndist }}}} in terms of 
\colorbox[RGB]{253,246,227}{{{{\color[RGB]{101, 123, 131} dist }}}}\paragraph{eq\_of\_nndist\_eq\_zero}
\par
Deduce the equality of points with the vanishing of the nonnegative distance
\paragraph{nndist\_eq\_zero}
\par
Characterize the equality of points with the vanishing of the nonnegative distance
\paragraph{nndist\_triangle}
\par
Triangle inequality for the nonnegative distance
\paragraph{dist\_edist}
\par
Express 
\colorbox[RGB]{253,246,227}{{{{\color[RGB]{101, 123, 131} dist }}}} in terms of 
\colorbox[RGB]{253,246,227}{{{{\color[RGB]{101, 123, 131} edist }}}}\paragraph{metric.ball}
\par
\colorbox[RGB]{253,246,227}{{{{\color[RGB]{101, 123, 131} ball x ε }}}} is the set of all points 
\colorbox[RGB]{253,246,227}{{{{\color[RGB]{101, 123, 131} y }}}} with 
\colorbox[RGB]{253,246,227}{{{{\color[RGB]{101, 123, 131} dist y x  }}}{{{\color[RGB]{181, 137, 0} < }}}{{{\color[RGB]{101, 123, 131}  ε }}}}\paragraph{metric.closed\_ball}
\par
\colorbox[RGB]{253,246,227}{{{{\color[RGB]{101, 123, 131} closed\_ball x ε }}}} is the set of all points 
\colorbox[RGB]{253,246,227}{{{{\color[RGB]{101, 123, 131} y }}}} with 
\colorbox[RGB]{253,246,227}{{{{\color[RGB]{101, 123, 131} dist y x  }}}{{{\color[RGB]{181, 137, 0} ≤ }}}{{{\color[RGB]{101, 123, 131}  ε }}}}\paragraph{metric.totally\_bounded\_of\_finite\_discretization}
\par
A metric space space is totally bounded if one can reconstruct up to any ε>0 any element of the
space from finitely many data.
\paragraph{metric.mem\_uniformity\_edist}
\par
Expressing the uniformity in terms of 
\colorbox[RGB]{253,246,227}{{{{\color[RGB]{101, 123, 131} edist }}}}\paragraph{metric\_space.to\_emetric\_space}
\par
A metric space induces an emetric space
\paragraph{metric.emetric\_ball}
\par
Balls defined using the distance or the edistance coincide
\paragraph{metric.emetric\_closed\_ball}
\par
Closed balls defined using the distance or the edistance coincide
\paragraph{emetric\_space.to\_metric\_space}
\par
One gets a metric space from an emetric space if the edistance
is everywhere finite. We set it up so that the edist and the uniformity are
defeq in the metric space and the emetric space
\paragraph{real.metric\_space}
\par
Instantiate the reals as a metric space.
\paragraph{metric.cauchy\_seq\_iff}
\par
In a metric space, Cauchy sequences are characterized by the fact that, eventually,
the distance between its elements is arbitrarily small
\paragraph{metric.cauchy\_seq\_iff'}
\par
A variation around the metric characterization of Cauchy sequences
\paragraph{cauchy\_seq\_bdd}
\par
A Cauchy sequence on the natural numbers is bounded.
\paragraph{cauchy\_seq\_iff\_le\_tendsto\_0}
\par
Yet another metric characterization of Cauchy sequences on integers. This one is often the
most efficient.
\paragraph{metric.mem\_closure\_iff'}
\par
ε-characterization of the closure in metric spaces
\paragraph{finite\_cover\_balls\_of\_compact}
\par
Any compact set in a metric space can be covered by finitely many balls of a given positive
radius
\paragraph{proper\_space}
\par
A metric space is proper if all closed balls are compact.
\paragraph{locally\_compact\_of\_proper}
\par
A proper space is locally compact
\paragraph{complete\_of\_proper}
\par
A proper space is complete
\paragraph{second\_countable\_of\_proper}
\par
A proper metric space is separable, and therefore second countable. Indeed, any ball is
compact, and therefore admits a countable dense subset. Taking a countable union over the balls
centered at a fixed point and with integer radius, one obtains a countable set which is
dense in the whole space.
\paragraph{metric.second\_countable\_of\_almost\_dense\_set}
\par
A metric space is second countable if, for every ε > 0, there is a countable set which is ε-dense.
\paragraph{metric.second\_countable\_of\_countable\_discretization}
\par
A metric space space is second countable if one can reconstruct up to any ε>0 any element of the
space from countably many data.
\paragraph{metric.bounded}
\par
Boundedness of a subset of a metric space. We formulate the definition to work
even in the empty space.
\paragraph{metric.bounded.subset}
\par
Subsets of a bounded set are also bounded
\paragraph{metric.bounded\_closed\_ball}
\par
Closed balls are bounded
\paragraph{metric.bounded\_ball}
\par
Open balls are bounded
\paragraph{metric.bounded\_iff\_subset\_ball}
\par
Given a point, a bounded subset is included in some ball around this point
\paragraph{metric.bounded\_union}
\par
The union of two bounded sets is bounded iff each of the sets is bounded
\paragraph{metric.bounded\_bUnion}
\par
A finite union of bounded sets is bounded
\paragraph{metric.bounded\_of\_compact}
\par
A compact set is bounded
\paragraph{metric.bounded\_of\_finite}
\par
A finite set is bounded
\paragraph{metric.bounded\_singleton}
\par
A singleton is bounded
\paragraph{metric.bounded\_range\_iff}
\par
Characterization of the boundedness of the range of a function
\paragraph{metric.bounded\_of\_compact\_space}
\par
In a compact space, all sets are bounded
\paragraph{metric.compact\_iff\_closed\_bounded}
\par
In a proper space, a set is compact if and only if it is closed and bounded
\paragraph{metric.diam}
\par
The diameter of a set in a metric space. To get controllable behavior even when the diameter
should be infinite, we express it in terms of the emetric.diameter
\paragraph{metric.diam\_nonneg}
\par
The diameter of a set is always nonnegative
\paragraph{metric.diam\_empty}
\par
The empty set has zero diameter
\paragraph{metric.diam\_singleton}
\par
A singleton has zero diameter
\paragraph{metric.bounded\_iff\_diam\_ne\_top}
\par
Characterize the boundedness of a set in terms of the finiteness of its emetric.diameter.
\paragraph{metric.diam\_eq\_zero\_of\_unbounded}
\par
An unbounded set has zero diameter. If you would prefer to get the value ∞, use 
\colorbox[RGB]{253,246,227}{{{{\color[RGB]{101, 123, 131} emetric.diam }}}}.
This lemma makes it possible to avoid side conditions in some situations
\paragraph{metric.diam\_mono}
\par
If 
\colorbox[RGB]{253,246,227}{{{{\color[RGB]{101, 123, 131} s ⊆ t }}}}, then the diameter of 
\colorbox[RGB]{253,246,227}{{{{\color[RGB]{101, 123, 131} s }}}} is bounded by that of 
\colorbox[RGB]{253,246,227}{{{{\color[RGB]{101, 123, 131} t }}}}, provided 
\colorbox[RGB]{253,246,227}{{{{\color[RGB]{101, 123, 131} t }}}} is bounded.
\paragraph{metric.dist\_le\_diam\_of\_mem}
\par
The distance between two points in a set is controlled by the diameter of the set.
\paragraph{metric.diam\_le\_of\_forall\_dist\_le}
\par
If the distance between any two points in a set is bounded by some constant, this constant
bounds the diameter.
\paragraph{metric.diam\_union}
\par
The diameter of a union is controlled by the sum of the diameters, and the distance between
any two points in each of the sets. This lemma is true without any side condition, since it is
obviously true if 
\colorbox[RGB]{253,246,227}{{{{\color[RGB]{101, 123, 131} s ∪ t }}}} is unbounded.
\paragraph{metric.diam\_union'}
\par
If two sets intersect, the diameter of the union is bounded by the sum of the diameters.
\paragraph{metric.diam\_closed\_ball}
\par
The diameter of a closed ball of radius 
\colorbox[RGB]{253,246,227}{{{{\color[RGB]{101, 123, 131} r }}}} is at most 
\colorbox[RGB]{253,246,227}{{{{\color[RGB]{108, 113, 196} 2 }}}{{{\color[RGB]{101, 123, 131}  r }}}}.
\paragraph{metric.diam\_ball}
\par
The diameter of a ball of radius 
\colorbox[RGB]{253,246,227}{{{{\color[RGB]{101, 123, 131} r }}}} is at most 
\colorbox[RGB]{253,246,227}{{{{\color[RGB]{108, 113, 196} 2 }}}{{{\color[RGB]{101, 123, 131}  r }}}}.
\section{topology/metric\_space/cau\_seq\_filter.lean}\paragraph{ennreal.half\_pow}
\par
An auxiliary positive sequence that tends to 
\colorbox[RGB]{253,246,227}{{{{\color[RGB]{108, 113, 196} 0 }}}} in 
\colorbox[RGB]{253,246,227}{{{{\color[RGB]{101, 123, 131} ennreal }}}}, with good behavior.
\paragraph{sequentially\_complete.B2}
\par
Auxiliary sequence, which is bounded by 
\colorbox[RGB]{253,246,227}{{{{\color[RGB]{101, 123, 131} B }}}}, positive, and tends to 
\colorbox[RGB]{253,246,227}{{{{\color[RGB]{108, 113, 196} 0 }}}}.
\paragraph{sequentially\_complete.set\_seq\_of\_cau\_filter}
\par
Define a decreasing sequence of sets in the filter 
\colorbox[RGB]{253,246,227}{{{{\color[RGB]{101, 123, 131} f }}}}, of diameter bounded by 
\colorbox[RGB]{253,246,227}{{{{\color[RGB]{101, 123, 131} B2 n }}}}.
\paragraph{sequentially\_complete.set\_seq\_of\_cau\_filter\_mem\_sets}
\par
These sets are in the filter.
\paragraph{sequentially\_complete.set\_seq\_of\_cau\_filter\_inhabited}
\par
These sets are nonempty.
\paragraph{sequentially\_complete.set\_seq\_of\_cau\_filter\_spec}
\par
By construction, their diameter is controlled by 
\colorbox[RGB]{253,246,227}{{{{\color[RGB]{101, 123, 131} B2 n }}}}.
\paragraph{sequentially\_complete.set\_seq\_of\_cau\_filter\_monotone}
\par
These sets are nested.
\paragraph{sequentially\_complete.seq\_of\_cau\_filter}
\par
Define the approximating Cauchy sequence for the Cauchy filter 
\colorbox[RGB]{253,246,227}{{{{\color[RGB]{101, 123, 131} f }}}},
obtained by taking a point in each set.
\paragraph{sequentially\_complete.seq\_of\_cau\_filter\_mem\_set\_seq}
\par
The approximating sequence indeed belong to our good sets.
\paragraph{sequentially\_complete.seq\_of\_cau\_filter\_bound}
\par
The distance between points in the sequence is bounded by 
\colorbox[RGB]{253,246,227}{{{{\color[RGB]{101, 123, 131} B2 N }}}}.
\paragraph{sequentially\_complete.seq\_of\_cau\_filter\_is\_cauchy}
\par
The approximating sequence is indeed Cauchy as 
\colorbox[RGB]{253,246,227}{{{{\color[RGB]{101, 123, 131} B2 n }}}} tends to 
\colorbox[RGB]{253,246,227}{{{{\color[RGB]{108, 113, 196} 0 }}}} with 
\colorbox[RGB]{253,246,227}{{{{\color[RGB]{101, 123, 131} n }}}}.
\paragraph{sequentially\_complete.le\_nhds\_cau\_filter\_lim}
\par
If the approximating Cauchy sequence is converging, to a limit 
\colorbox[RGB]{253,246,227}{{{{\color[RGB]{101, 123, 131} y }}}}, then the
original Cauchy filter 
\colorbox[RGB]{253,246,227}{{{{\color[RGB]{101, 123, 131} f }}}} is also converging, to the same limit.
Given 
\colorbox[RGB]{253,246,227}{{{{\color[RGB]{101, 123, 131} t1 }}}} in the filter 
\colorbox[RGB]{253,246,227}{{{{\color[RGB]{101, 123, 131} f }}}} and 
\colorbox[RGB]{253,246,227}{{{{\color[RGB]{101, 123, 131} t2 }}}} a neighborhood of 
\colorbox[RGB]{253,246,227}{{{{\color[RGB]{101, 123, 131} y }}}}, it suffices to show that 
\colorbox[RGB]{253,246,227}{{{{\color[RGB]{101, 123, 131} t1 ∩ t2 }}}} is
nonempty.
Pick 
\colorbox[RGB]{253,246,227}{{{{\color[RGB]{101, 123, 131} ε }}}} so that the ε-eball around 
\colorbox[RGB]{253,246,227}{{{{\color[RGB]{101, 123, 131} y }}}} is contained in 
\colorbox[RGB]{253,246,227}{{{{\color[RGB]{101, 123, 131} t2 }}}}.
Pick 
\colorbox[RGB]{253,246,227}{{{{\color[RGB]{101, 123, 131} n }}}} with 
\colorbox[RGB]{253,246,227}{{{{\color[RGB]{101, 123, 131} B2 n  }}}{{{\color[RGB]{181, 137, 0} < }}}{{{\color[RGB]{101, 123, 131}  ε }}}{{{\color[RGB]{181, 137, 0} / }}}{{{\color[RGB]{108, 113, 196} 2 }}}}, and 
\colorbox[RGB]{253,246,227}{{{{\color[RGB]{101, 123, 131} n2 }}}} such that 
\colorbox[RGB]{253,246,227}{{{{\color[RGB]{101, 123, 131} dist(seq n2, y)  }}}{{{\color[RGB]{181, 137, 0} < }}}{{{\color[RGB]{101, 123, 131}  ε }}}{{{\color[RGB]{181, 137, 0} / }}}{{{\color[RGB]{108, 113, 196} 2 }}}}. Let 
\colorbox[RGB]{253,246,227}{{{{\color[RGB]{101, 123, 131} N  }}}{{{\color[RGB]{181, 137, 0} = }}}{{{\color[RGB]{101, 123, 131}  max(n, n2) }}}}.
We defined 
\colorbox[RGB]{253,246,227}{{{{\color[RGB]{101, 123, 131} seq }}}} by looking at a decreasing sequence of sets of 
\colorbox[RGB]{253,246,227}{{{{\color[RGB]{101, 123, 131} f }}}} with shrinking radius.
The Nth one has radius 
\colorbox[RGB]{253,246,227}{{{{\color[RGB]{181, 137, 0} < }}}{{{\color[RGB]{101, 123, 131}  B2 N  }}}{{{\color[RGB]{181, 137, 0} < }}}{{{\color[RGB]{101, 123, 131}  ε }}}{{{\color[RGB]{181, 137, 0} / }}}{{{\color[RGB]{108, 113, 196} 2 }}}}. This set is in 
\colorbox[RGB]{253,246,227}{{{{\color[RGB]{101, 123, 131} f }}}}, so we can find an element 
\colorbox[RGB]{253,246,227}{{{{\color[RGB]{101, 123, 131} x }}}} that's
also in 
\colorbox[RGB]{253,246,227}{{{{\color[RGB]{101, 123, 131} t1 }}}}.
\colorbox[RGB]{253,246,227}{{{{\color[RGB]{101, 123, 131} dist(x, seq N)  }}}{{{\color[RGB]{181, 137, 0} < }}}{{{\color[RGB]{101, 123, 131}  ε }}}{{{\color[RGB]{181, 137, 0} / }}}{{{\color[RGB]{108, 113, 196} 2 }}}} since 
\colorbox[RGB]{253,246,227}{{{{\color[RGB]{101, 123, 131} seq N }}}} is in this set, and 
\colorbox[RGB]{253,246,227}{{{{\color[RGB]{101, 123, 131} dist (seq N, y)  }}}{{{\color[RGB]{181, 137, 0} < }}}{{{\color[RGB]{101, 123, 131}  ε }}}{{{\color[RGB]{181, 137, 0} / }}}{{{\color[RGB]{108, 113, 196} 2 }}}},
so 
\colorbox[RGB]{253,246,227}{{{{\color[RGB]{101, 123, 131} x }}}} is in the ε-ball around 
\colorbox[RGB]{253,246,227}{{{{\color[RGB]{101, 123, 131} y }}}}, and thus in 
\colorbox[RGB]{253,246,227}{{{{\color[RGB]{101, 123, 131} t2 }}}}.
\paragraph{complete\_of\_cauchy\_seq\_tendsto}
\par
An emetric space in which every Cauchy sequence converges is complete.
\paragraph{emetric.complete\_of\_convergent\_controlled\_sequences}
\par
A very useful criterion to show that a space is complete is to show that all sequences
which satisfy a bound of the form 
\colorbox[RGB]{253,246,227}{{{{\color[RGB]{101, 123, 131} edist (u n) (u m)  }}}{{{\color[RGB]{181, 137, 0} < }}}{{{\color[RGB]{101, 123, 131}  B N }}}} for all 
\colorbox[RGB]{253,246,227}{{{{\color[RGB]{101, 123, 131} n m  }}}{{{\color[RGB]{181, 137, 0} ≥ }}}{{{\color[RGB]{101, 123, 131}  N }}}} are
converging. This is often applied for 
\colorbox[RGB]{253,246,227}{{{{\color[RGB]{101, 123, 131} B N  }}}{{{\color[RGB]{181, 137, 0} = }}}{{{\color[RGB]{101, 123, 131}   }}}{{{\color[RGB]{108, 113, 196} 2 }}}{{{\color[RGB]{101, 123, 131} \textasciicircum{}\{ }}}{{{\color[RGB]{181, 137, 0} - }}}{{{\color[RGB]{101, 123, 131} N\} }}}}, i.e., with a very fast convergence to
\colorbox[RGB]{253,246,227}{{{{\color[RGB]{108, 113, 196} 0 }}}}, which makes it possible to use arguments of converging series, while this is impossible
to do in general for arbitrary Cauchy sequences.
\paragraph{metric.complete\_of\_convergent\_controlled\_sequences}
\par
A very useful criterion to show that a space is complete is to show that all sequences
which satisfy a bound of the form 
\colorbox[RGB]{253,246,227}{{{{\color[RGB]{101, 123, 131} dist (u n) (u m)  }}}{{{\color[RGB]{181, 137, 0} < }}}{{{\color[RGB]{101, 123, 131}  B N }}}} for all 
\colorbox[RGB]{253,246,227}{{{{\color[RGB]{101, 123, 131} n m  }}}{{{\color[RGB]{181, 137, 0} ≥ }}}{{{\color[RGB]{101, 123, 131}  N }}}} are
converging. This is often applied for 
\colorbox[RGB]{253,246,227}{{{{\color[RGB]{101, 123, 131} B N  }}}{{{\color[RGB]{181, 137, 0} = }}}{{{\color[RGB]{101, 123, 131}   }}}{{{\color[RGB]{108, 113, 196} 2 }}}{{{\color[RGB]{101, 123, 131} \textasciicircum{}\{ }}}{{{\color[RGB]{181, 137, 0} - }}}{{{\color[RGB]{101, 123, 131} N\} }}}}, i.e., with a very fast convergence to
\colorbox[RGB]{253,246,227}{{{{\color[RGB]{108, 113, 196} 0 }}}}, which makes it possible to use arguments of converging series, while this is impossible
to do in general for arbitrary Cauchy sequences.
\paragraph{cau\_seq\_iff\_cauchy\_seq}
\par
In a normed field, 
\colorbox[RGB]{253,246,227}{{{{\color[RGB]{101, 123, 131} cau\_seq }}}} coincides with the usual notion of Cauchy sequences.
\paragraph{complete\_space\_of\_cau\_seq\_complete}
\par
A complete normed field is complete as a metric space, as Cauchy sequences converge by
assumption and this suffices to characterize completeness.
\section{topology/metric\_space/closeds.lean}\paragraph{emetric.closeds.emetric\_space}
\par
In emetric spaces, the Hausdorff edistance defines an emetric space structure
on the type of closed subsets
\paragraph{emetric.continuous\_inf\_edist\_Hausdorff\_edist}
\par
The edistance to a closed set depends continuously on the point and the set
\paragraph{emetric.is\_closed\_subsets\_of\_is\_closed}
\par
Subsets of a given closed subset form a closed set
\paragraph{emetric.closeds.edist\_eq}
\par
By definition, the edistance on 
\colorbox[RGB]{253,246,227}{{{{\color[RGB]{101, 123, 131} closeds α }}}} is given by the Hausdorff edistance
\paragraph{emetric.closeds.complete\_space}
\par
In a complete space, the type of closed subsets is complete for the
Hausdorff edistance.
\paragraph{emetric.closeds.compact\_space}
\par
In a compact space, the type of closed subsets is compact.
\paragraph{emetric.nonempty\_compacts.emetric\_space}
\par
In an emetric space, the type of non-empty compact subsets is an emetric space,
where the edistance is the Hausdorff edistance
\paragraph{emetric.nonempty\_compacts.to\_closeds.uniform\_embedding}
\par
\colorbox[RGB]{253,246,227}{{{{\color[RGB]{101, 123, 131} nonempty\_compacts.to\_closeds }}}} is a uniform embedding (as it is an isometry)
\paragraph{emetric.nonempty\_compacts.is\_closed\_in\_closeds}
\par
The range of 
\colorbox[RGB]{253,246,227}{{{{\color[RGB]{101, 123, 131} nonempty\_compacts.to\_closeds }}}} is closed in a complete space
\paragraph{emetric.nonempty\_compacts.complete\_space}
\par
In a complete space, the type of nonempty compact subsets is complete. This follows
from the same statement for closed subsets
\paragraph{emetric.nonempty\_compacts.compact\_space}
\par
In a compact space, the type of nonempty compact subsets is compact. This follows from
the same statement for closed subsets
\paragraph{emetric.nonempty\_compacts.second\_countable\_topology}
\par
In a second countable space, the type of nonempty compact subsets is second countable
\paragraph{metric.nonempty\_compacts.metric\_space}
\par
\colorbox[RGB]{253,246,227}{{{{\color[RGB]{101, 123, 131} nonempty\_compacts α }}}} inherits a metric space structure, as the Hausdorff
edistance between two such sets is finite.
\paragraph{metric.nonempty\_compacts.dist\_eq}
\par
The distance on 
\colorbox[RGB]{253,246,227}{{{{\color[RGB]{101, 123, 131} nonempty\_compacts α }}}} is the Hausdorff distance, by construction
\section{topology/metric\_space/completion.lean}\paragraph{metric.has\_dist}
\par
The distance on the completion is obtained by extending the distance on the original space,
by uniform continuity.
\paragraph{metric.completion.uniform\_continuous\_dist}
\par
The new distance is uniformly continuous.
\paragraph{metric.completion.dist\_eq}
\par
The new distance is an extension of the original distance.
\paragraph{metric.completion.mem\_uniformity\_dist}
\par
Elements of the uniformity (defined generally for completions) can be characterized in terms
of the distance.
\paragraph{metric.completion.eq\_of\_dist\_eq\_zero}
\par
If two points are at distance 0, then they coincide.
\paragraph{metric.completion.metric\_space}
\par
Metric space structure on the completion of a metric space.
\paragraph{metric.completion.coe\_isometry}
\par
The embedding of a metric space in its completion is an isometry.
\section{topology/metric\_space/emetric\_space.lean}\paragraph{uniformity\_dist\_of\_mem\_uniformity}
\par
Characterizing uniformities associated to a (generalized) distance function 
\colorbox[RGB]{253,246,227}{{{{\color[RGB]{101, 123, 131} D }}}}in terms of the elements of the uniformity.
\paragraph{uniform\_space\_of\_edist}
\par
Creating a uniform space from an extended distance.
\paragraph{emetric\_space}
\par
Extended metric spaces, with an extended distance 
\colorbox[RGB]{253,246,227}{{{{\color[RGB]{101, 123, 131} edist }}}} possibly taking the
value ∞
\par
Each emetric space induces a canonical 
\colorbox[RGB]{253,246,227}{{{{\color[RGB]{101, 123, 131} uniform\_space }}}} and hence a canonical 
\colorbox[RGB]{253,246,227}{{{{\color[RGB]{101, 123, 131} topological\_space }}}}.
This is enforced in the type class definition, by extending the 
\colorbox[RGB]{253,246,227}{{{{\color[RGB]{101, 123, 131} uniform\_space }}}} structure. When
instantiating an 
\colorbox[RGB]{253,246,227}{{{{\color[RGB]{101, 123, 131} emetric\_space }}}} structure, the uniformity fields are not necessary, they will be
filled in by default. There is a default value for the uniformity, that can be substituted
in cases of interest, for instance when instantiating an 
\colorbox[RGB]{253,246,227}{{{{\color[RGB]{101, 123, 131} emetric\_space }}}} structure
on a product.
\par
Continuity of 
\colorbox[RGB]{253,246,227}{{{{\color[RGB]{101, 123, 131} edist }}}} is finally proving in 
\colorbox[RGB]{253,246,227}{{{{\color[RGB]{101, 123, 131} topology. }}}{{{\color[RGB]{133, 153, 0} instances }}}{{{\color[RGB]{101, 123, 131} .ennreal }}}}\paragraph{edist\_eq\_zero}
\par
Characterize the equality of points by the vanishing of their extended distance
\paragraph{edist\_triangle\_left}
\par
Triangle inequality for the extended distance
\paragraph{eq\_of\_forall\_edist\_le}
\par
Two points coincide if their distance is 
\colorbox[RGB]{253,246,227}{{{{\color[RGB]{181, 137, 0} < }}}{{{\color[RGB]{101, 123, 131}  ε }}}} for all positive ε
\paragraph{uniformity\_edist'}
\par
Reformulation of the uniform structure in terms of the extended distance
\paragraph{uniformity\_edist''}
\par
Reformulation of the uniform structure in terms of the extended distance on a subtype
\paragraph{mem\_uniformity\_edist}
\par
Characterization of the elements of the uniformity in terms of the extended distance
\paragraph{edist\_mem\_uniformity}
\par
Fixed size neighborhoods of the diagonal belong to the uniform structure
\paragraph{emetric.uniform\_continuous\_iff}
\par
ε-δ characterization of uniform continuity on emetric spaces
\paragraph{emetric.uniform\_embedding\_iff}
\par
ε-δ characterization of uniform embeddings on emetric spaces
\paragraph{emetric.cauchy\_iff}
\par
ε-δ characterization of Cauchy sequences on emetric spaces
\paragraph{to\_separated}
\par
An emetric space is separated
\paragraph{emetric\_space.replace\_uniformity}
\par
Auxiliary function to replace the uniformity on an emetric space with
a uniformity which is equal to the original one, but maybe not defeq.
This is useful if one wants to construct an emetric space with a
specified uniformity.
\paragraph{emetric\_space.induced}
\par
The extended metric induced by an injective function taking values in an emetric space.
\paragraph{subtype.emetric\_space}
\par
Emetric space instance on subsets of emetric spaces
\paragraph{subtype.edist\_eq}
\par
The extended distance on a subset of an emetric space is the restriction of
the original distance, by definition
\paragraph{prod.emetric\_space\_max}
\par
The product of two emetric spaces, with the max distance, is an extended
metric spaces. We make sure that the uniform structure thus constructed is the one
corresponding to the product of uniform spaces, to avoid diamond problems.
\paragraph{emetric\_space\_pi}
\par
The product of a finite number of emetric spaces, with the max distance, is still
an emetric space.
This construction would also work for infinite products, but it would not give rise
to the product topology. Hence, we only formalize it in the good situation of finitely many
spaces.
\paragraph{emetric.ball}
\par
\colorbox[RGB]{253,246,227}{{{{\color[RGB]{101, 123, 131} emetric.ball x ε }}}} is the set of all points 
\colorbox[RGB]{253,246,227}{{{{\color[RGB]{101, 123, 131} y }}}} with 
\colorbox[RGB]{253,246,227}{{{{\color[RGB]{101, 123, 131} edist y x  }}}{{{\color[RGB]{181, 137, 0} < }}}{{{\color[RGB]{101, 123, 131}  ε }}}}\paragraph{emetric.closed\_ball}
\par
\colorbox[RGB]{253,246,227}{{{{\color[RGB]{101, 123, 131} emetric.closed\_ball x ε }}}} is the set of all points 
\colorbox[RGB]{253,246,227}{{{{\color[RGB]{101, 123, 131} y }}}} with 
\colorbox[RGB]{253,246,227}{{{{\color[RGB]{101, 123, 131} edist y x  }}}{{{\color[RGB]{181, 137, 0} ≤ }}}{{{\color[RGB]{101, 123, 131}  ε }}}}\paragraph{emetric.mem\_closure\_iff'}
\par
ε-characterization of the closure in emetric spaces
\paragraph{emetric.cauchy\_seq\_iff}
\par
In an emetric space, Cauchy sequences are characterized by the fact that, eventually,
the edistance between its elements is arbitrarily small
\paragraph{emetric.cauchy\_seq\_iff'}
\par
A variation around the emetric characterization of Cauchy sequences
\paragraph{emetric.countable\_closure\_of\_compact}
\par
A compact set in an emetric space is separable, i.e., it is the closure of a countable set
\paragraph{emetric.second\_countable\_of\_separable}
\par
A separable emetric space is second countable: one obtains a countable basis by taking
the balls centered at points in a dense subset, and with rational radii. We do not register
this as an instance, as there is already an instance going in the other direction
from second countable spaces to separable spaces, and we want to avoid loops.
\paragraph{emetric.diam}
\par
The diameter of a set in an emetric space, named 
\colorbox[RGB]{253,246,227}{{{{\color[RGB]{101, 123, 131} emetric.diam }}}}\paragraph{emetric.edist\_le\_diam\_of\_mem}
\par
If two points belong to some set, their edistance is bounded by the diameter of the set
\paragraph{emetric.diam\_le\_of\_forall\_edist\_le}
\par
If the distance between any two points in a set is bounded by some constant, this constant
bounds the diameter.
\paragraph{emetric.diam\_empty}
\par
The diameter of the empty set vanishes
\paragraph{emetric.diam\_singleton}
\par
The diameter of a singleton vanishes
\paragraph{emetric.diam\_mono}
\par
The diameter is monotonous with respect to inclusion
\paragraph{emetric.diam\_union}
\par
The diameter of a union is controlled by the diameter of the sets, and the edistance
between two points in the sets.
\section{topology/metric\_space/gluing.lean}\paragraph{metric.glue\_dist}
\par
Define a predistance on α ⊕ β, for which Φ p and Ψ p are at distance ε
\paragraph{metric.glue\_metric\_approx}
\par
Given two maps Φ and Ψ intro metric spaces α and β such that the distances between Φ p and Φ q,
and between Ψ p and Ψ q, coincide up to 2 ε where ε > 0, one can almost glue the two spaces α
and β along the images of Φ and Ψ, so that Φ p and Ψ p are at distance ε.
\paragraph{metric.metric\_space\_sum}
\par
The distance on the disjoint union indeed defines a metric space. All the distance properties follow from our
choice of the distance. The harder work is to show that the uniform structure defined by the distance coincides
with the disjoint union uniform structure.
\paragraph{metric.isometry\_on\_inl}
\par
The left injection of a space in a disjoint union in an isometry
\paragraph{metric.isometry\_on\_inr}
\par
The right injection of a space in a disjoint union in an isometry
\paragraph{metric.inductive\_limit\_dist}
\par
Predistance on the disjoint union Σn, X n.
\paragraph{metric.inductive\_limit\_dist\_eq\_dist}
\par
The predistance on the disjoint union Σn, X n can be computed in any X k for large enough k.
\paragraph{metric.inductive\_premetric}
\par
Premetric space structure on Σn, X n.
\paragraph{metric.inductive\_limit}
\par
The type giving the inductive limit in a metric space context.
\paragraph{metric.metric\_space\_inductive\_limit}
\par
Metric space structure on the inductive limit.
\paragraph{metric.to\_inductive\_limit}
\par
Mapping each 
\colorbox[RGB]{253,246,227}{{{{\color[RGB]{101, 123, 131} X n }}}} to the inductive limit.
\paragraph{metric.to\_inductive\_limit\_isometry}
\par
The map 
\colorbox[RGB]{253,246,227}{{{{\color[RGB]{101, 123, 131} to\_inductive\_limit n }}}} mapping 
\colorbox[RGB]{253,246,227}{{{{\color[RGB]{101, 123, 131} X n }}}} to the inductive limit is an isometry.
\paragraph{metric.to\_inductive\_limit\_commute}
\par
The maps 
\colorbox[RGB]{253,246,227}{{{{\color[RGB]{101, 123, 131} to\_inductive\_limit n }}}} are compatible with the maps 
\colorbox[RGB]{253,246,227}{{{{\color[RGB]{101, 123, 131} f n }}}}.
\section{topology/metric\_space/gromov\_hausdorff.lean}\paragraph{\_private.3775878227.isometry\_rel}
\par
Equivalence relation identifying two nonempty compact sets which are isometric
\paragraph{\_private.2047018147.is\_equivalence\_isometry\_rel}
\par
This is indeed an equivalence relation
\paragraph{Gromov\_Hausdorff.isometry\_rel.setoid}
\par
setoid instance identifying two isometric nonempty compact subspaces of ℓ\textasciicircum{}∞(ℝ)
\paragraph{Gromov\_Hausdorff.GH\_space}
\par
The Gromov-Hausdorff space
\paragraph{Gromov\_Hausdorff.to\_GH\_space}
\par
Map any nonempty compact type to GH\_space
\paragraph{Gromov\_Hausdorff.GH\_space.rep}
\par
A metric space representative of any abstract point in GH\_space
\paragraph{Gromov\_Hausdorff.to\_GH\_space\_eq\_to\_GH\_space\_iff\_isometric}
\par
Two nonempty compact spaces have the same image in GH\_space if and only if they are isometric
\paragraph{Gromov\_Hausdorff.has\_dist}
\par
Distance on GH\_space : the distance between two nonempty compact spaces is the infimum
Hausdorff distance between isometric copies of the two spaces in a metric space. For the definition,
we only consider embeddings in ℓ\textasciicircum{}∞(ℝ), but we will prove below that it works for all spaces.
\paragraph{Gromov\_Hausdorff.GH\_dist\_le\_Hausdorff\_dist}
\par
The Gromov-Hausdorff distance between two spaces is bounded by the Hausdorff distance
of isometric copies of the spaces, in any metric space.
\paragraph{Gromov\_Hausdorff.Hausdorff\_dist\_optimal}
\par
The optimal coupling constructed above realizes exactly the Gromov-Hausdorff distance,
essentially by design.
\paragraph{Gromov\_Hausdorff.GH\_dist\_eq\_Hausdorff\_dist}
\par
The Gromov-Hausdorff distance can also be realized by a coupling in ℓ\textasciicircum{}∞(ℝ), by embedding
the optimal coupling through its Kuratowski embedding.
\paragraph{Gromov\_Hausdorff.GH\_space\_metric\_space}
\par
The Gromov-Hausdorff distance defines a genuine distance on the Gromov-Hausdorff space.
\paragraph{topological\_space.nonempty\_compacts.to\_GH\_space}
\par
In particular, nonempty compacts of a metric space map to GH\_space. We register this
in the topological\_space namespace to take advantage of the notation p.to\_GH\_space
\paragraph{Gromov\_Hausdorff.GH\_dist\_le\_of\_approx\_subsets}
\par
If there are subsets which are ε1-dense and ε3-dense in two spaces, and
isometric up to ε2, then the Gromov-Hausdorff distance between the spaces is bounded by
ε1 + ε2/2 + ε3.
\paragraph{Gromov\_Hausdorff.second\_countable}
\par
The Gromov-Hausdorff space is second countable.
\paragraph{Gromov\_Hausdorff.totally\_bounded}
\par
Compactness criterion : a closed set of compact metric spaces is compact if the spaces have
a uniformly bounded diameter, and for all ε the number of balls of radius ε required
to cover the space is uniformly bounded. This is an equivalence, but we only prove the
interesting direction that these conditions imply compactness.
\paragraph{Gromov\_Hausdorff.complete\_space}
\par
The Gromov-Hausdorff space is complete.
\section{topology/metric\_space/gromov\_hausdorff\_realized.lean}\paragraph{Gromov\_Hausdorff.candidates}
\par
The set of functions on α ⊕ β that are candidates distances to realize the
minimum of the Hausdorff distances between α and β in a coupling
\paragraph{\_private.3491321597.candidates\_b}
\par
Version of the set of candidates in bounded\_continuous\_functions, to apply
Arzela-Ascoli
\paragraph{\_private.1005967255.candidates\_dist\_bound}
\par
candidates are bounded by max\_var α β
\paragraph{\_private.273830989.candidates\_lipschitz\_aux}
\par
Technical lemma to prove that candidates are Lipschitz
\paragraph{\_private.2877194623.candidates\_lipschitz}
\par
Candidates are Lipschitz
\paragraph{Gromov\_Hausdorff.candidates\_b\_of\_candidates}
\par
candidates give rise to elements of bounded\_continuous\_functions
\paragraph{\_private.584775865.dist\_mem\_candidates}
\par
The distance on α ⊕ β is a candidate
\paragraph{\_private.1828343803.closed\_candidates\_b}
\par
To apply Arzela-Ascoli, we need to check that the set of candidates is closed and equicontinuous.
Equicontinuity follows from the Lipschitz control, we check closedness
\paragraph{\_private.3530874365.compact\_candidates\_b}
\par
Compactness of candidates (in bounded\_continuous\_functions) follows
\paragraph{Gromov\_Hausdorff.HD}
\par
We will then choose the candidate minimizing the Hausdorff distance. Except that we are not
in a metric space setting, so we need to define our custom version of Hausdorff distance,
called HD, and prove its basic properties.
\paragraph{Gromov\_Hausdorff.HD\_candidates\_b\_dist\_le}
\par
Explicit bound on HD (dist). This means that when looking for minimizers it will
be sufficient to look for functions with HD(f) bounded by this bound.
\paragraph{\_private.3223828837.HD\_continuous}
\par
Conclude that HD, being Lipschitz, is continuous
\paragraph{Gromov\_Hausdorff.premetric\_optimal\_GH\_dist}
\par
With the optimal candidate, construct a premetric space structure on α ⊕ β, on which the
predistance is given by the candidate. Then, we will identify points at 0 predistance
to obtain a genuine metric space
\paragraph{Gromov\_Hausdorff.optimal\_GH\_coupling}
\par
A metric space which realizes the optimal coupling between α and β
\paragraph{Gromov\_Hausdorff.optimal\_GH\_injl}
\par
Injection of α in the optimal coupling between α and β
\paragraph{Gromov\_Hausdorff.isometry\_optimal\_GH\_injl}
\par
The injection of α in the optimal coupling between α and β is an isometry.
\paragraph{Gromov\_Hausdorff.optimal\_GH\_injr}
\par
Injection of β  in the optimal coupling between α and β
\paragraph{Gromov\_Hausdorff.isometry\_optimal\_GH\_injr}
\par
The injection of β in the optimal coupling between α and β is an isometry.
\paragraph{Gromov\_Hausdorff.compact\_space\_optimal\_GH\_coupling}
\par
The optimal coupling between two compact spaces α and β is still a compact space
\paragraph{Gromov\_Hausdorff.Hausdorff\_dist\_optimal\_le\_HD}
\par
For any candidate f, HD(f) is larger than or equal to the Hausdorff distance in the
optimal coupling. This follows from the fact that HD of the optimal candidate is exactly
the Hausdorff distance in the optimal coupling, although we only prove here the inequality
we need.
\section{topology/metric\_space/hausdorff\_distance.lean}\paragraph{emetric.inf\_edist}
\par
The minimal edistance of a point to a set
\paragraph{emetric.inf\_edist\_union}
\par
The edist to a union is the minimum of the edists
\paragraph{emetric.inf\_edist\_singleton}
\par
The edist to a singleton is the edistance to the single point of this singleton
\paragraph{emetric.inf\_edist\_le\_edist\_of\_mem}
\par
The edist to a set is bounded above by the edist to any of its points
\paragraph{emetric.inf\_edist\_zero\_of\_mem}
\par
If a point 
\colorbox[RGB]{253,246,227}{{{{\color[RGB]{101, 123, 131} x }}}} belongs to 
\colorbox[RGB]{253,246,227}{{{{\color[RGB]{101, 123, 131} s }}}}, then its edist to 
\colorbox[RGB]{253,246,227}{{{{\color[RGB]{101, 123, 131} s }}}} vanishes
\paragraph{emetric.inf\_edist\_le\_inf\_edist\_of\_subset}
\par
The edist is monotonous with respect to inclusion
\paragraph{emetric.exists\_edist\_lt\_of\_inf\_edist\_lt}
\par
If the edist to a set is 
\colorbox[RGB]{253,246,227}{{{{\color[RGB]{181, 137, 0} < }}}{{{\color[RGB]{101, 123, 131}  r }}}}, there exists a point in the set at edistance 
\colorbox[RGB]{253,246,227}{{{{\color[RGB]{181, 137, 0} < }}}{{{\color[RGB]{101, 123, 131}  r }}}}\paragraph{emetric.inf\_edist\_le\_inf\_edist\_add\_edist}
\par
The edist of 
\colorbox[RGB]{253,246,227}{{{{\color[RGB]{101, 123, 131} x }}}} to 
\colorbox[RGB]{253,246,227}{{{{\color[RGB]{101, 123, 131} s }}}} is bounded by the sum of the edist of 
\colorbox[RGB]{253,246,227}{{{{\color[RGB]{101, 123, 131} y }}}} to 
\colorbox[RGB]{253,246,227}{{{{\color[RGB]{101, 123, 131} s }}}} and
the edist from 
\colorbox[RGB]{253,246,227}{{{{\color[RGB]{101, 123, 131} x }}}} to 
\colorbox[RGB]{253,246,227}{{{{\color[RGB]{101, 123, 131} y }}}}\paragraph{emetric.continuous\_inf\_edist}
\par
The edist to a set depends continuously on the point
\paragraph{emetric.inf\_edist\_closure}
\par
The edist to a set and to its closure coincide
\paragraph{emetric.mem\_closure\_iff\_inf\_edist\_zero}
\par
A point belongs to the closure of 
\colorbox[RGB]{253,246,227}{{{{\color[RGB]{101, 123, 131} s }}}} iff its infimum edistance to this set vanishes
\paragraph{emetric.mem\_iff\_ind\_edist\_zero\_of\_closed}
\par
Given a closed set 
\colorbox[RGB]{253,246,227}{{{{\color[RGB]{101, 123, 131} s }}}}, a point belongs to 
\colorbox[RGB]{253,246,227}{{{{\color[RGB]{101, 123, 131} s }}}} iff its infimum edistance to this set vanishes
\paragraph{emetric.inf\_edist\_image}
\par
The infimum edistance is invariant under isometries
\paragraph{emetric.Hausdorff\_edist}
\par
The Hausdorff edistance between two sets is the smallest 
\colorbox[RGB]{253,246,227}{{{{\color[RGB]{101, 123, 131} r }}}} such that each set
is contained in the 
\colorbox[RGB]{253,246,227}{{{{\color[RGB]{101, 123, 131} r }}}}-neighborhood of the other one
\paragraph{emetric.Hausdorff\_edist\_self}
\par
The Hausdorff edistance of a set to itself vanishes
\paragraph{emetric.Hausdorff\_edist\_comm}
\par
The Haudorff edistances of 
\colorbox[RGB]{253,246,227}{{{{\color[RGB]{101, 123, 131} s }}}} to 
\colorbox[RGB]{253,246,227}{{{{\color[RGB]{101, 123, 131} t }}}} and of 
\colorbox[RGB]{253,246,227}{{{{\color[RGB]{101, 123, 131} t }}}} to 
\colorbox[RGB]{253,246,227}{{{{\color[RGB]{101, 123, 131} s }}}} coincide
\paragraph{emetric.Hausdorff\_edist\_le\_of\_inf\_edist}
\par
Bounding the Hausdorff edistance by bounding the edistance of any point
in each set to the other set
\paragraph{emetric.Hausdorff\_edist\_le\_of\_mem\_edist}
\par
Bounding the Hausdorff edistance by exhibiting, for any point in each set,
another point in the other set at controlled distance
\paragraph{emetric.inf\_edist\_le\_Hausdorff\_edist\_of\_mem}
\par
The distance to a set is controlled by the Hausdorff distance
\paragraph{emetric.exists\_edist\_lt\_of\_Hausdorff\_edist\_lt}
\par
If the Hausdorff distance is 
\colorbox[RGB]{253,246,227}{{{{\color[RGB]{181, 137, 0} < }}}{{{\color[RGB]{101, 123, 131} r }}}}, then any point in one of the sets has
a corresponding point at distance 
\colorbox[RGB]{253,246,227}{{{{\color[RGB]{181, 137, 0} < }}}{{{\color[RGB]{101, 123, 131} r }}}} in the other set
\paragraph{emetric.inf\_edist\_le\_inf\_edist\_add\_Hausdorff\_edist}
\par
The distance from 
\colorbox[RGB]{253,246,227}{{{{\color[RGB]{101, 123, 131} x }}}} to 
\colorbox[RGB]{253,246,227}{{{{\color[RGB]{101, 123, 131} s }}}}or 
\colorbox[RGB]{253,246,227}{{{{\color[RGB]{101, 123, 131} t }}}} is controlled in terms of the Hausdorff distance
between 
\colorbox[RGB]{253,246,227}{{{{\color[RGB]{101, 123, 131} s }}}} and 
\colorbox[RGB]{253,246,227}{{{{\color[RGB]{101, 123, 131} t }}}}\paragraph{emetric.Hausdorff\_edist\_image}
\par
The Hausdorff edistance is invariant under eisometries
\paragraph{emetric.Hausdorff\_edist\_le\_ediam}
\par
The Hausdorff distance is controlled by the diameter of the union
\paragraph{emetric.Hausdorff\_edist\_triangle}
\par
The Hausdorff distance satisfies the triangular inequality
\paragraph{emetric.Hausdorff\_edist\_self\_closure}
\par
The Hausdorff edistance between a set and its closure vanishes
\paragraph{emetric.Hausdorff\_edist\_closure₁}
\par
Replacing a set by its closure does not change the Hausdorff edistance.
\paragraph{emetric.Hausdorff\_edist\_closure₂}
\par
Replacing a set by its closure does not change the Hausdorff edistance.
\paragraph{emetric.Hausdorff\_edist\_closure}
\par
The Hausdorff edistance between sets or their closures is the same
\paragraph{emetric.Hausdorff\_edist\_zero\_iff\_closure\_eq\_closure}
\par
Two sets are at zero Hausdorff edistance if and only if they have the same closure
\paragraph{emetric.Hausdorff\_edist\_zero\_iff\_eq\_of\_closed}
\par
Two closed sets are at zero Hausdorff edistance if and only if they coincide
\paragraph{emetric.Hausdorff\_edist\_empty}
\par
The Haudorff edistance to the empty set is infinite
\paragraph{emetric.ne\_empty\_of\_Hausdorff\_edist\_ne\_top}
\par
If a set is at finite Hausdorff edistance of a nonempty set, it is nonempty
\paragraph{metric.inf\_dist}
\par
The minimal distance of a point to a set
\paragraph{metric.inf\_dist\_nonneg}
\par
the minimal distance is always nonnegative
\paragraph{metric.inf\_dist\_empty}
\par
the minimal distance to the empty set is 0 (if you want to have the more reasonable
value ∞ instead, use 
\colorbox[RGB]{253,246,227}{{{{\color[RGB]{101, 123, 131} inf\_edist }}}}, which takes values in ennreal)
\paragraph{metric.inf\_edist\_ne\_top}
\par
In a metric space, the minimal edistance to a nonempty set is finite
\paragraph{metric.inf\_dist\_zero\_of\_mem}
\par
The minimal distance of a point to a set containing it vanishes
\paragraph{metric.inf\_dist\_singleton}
\par
The minimal distance to a singleton is the distance to the unique point in this singleton
\paragraph{metric.inf\_dist\_le\_dist\_of\_mem}
\par
The minimal distance to a set is bounded by the distance to any point in this set
\paragraph{metric.inf\_dist\_le\_inf\_dist\_of\_subset}
\par
The minimal distance is monotonous with respect to inclusion
\paragraph{metric.exists\_dist\_lt\_of\_inf\_dist\_lt}
\par
If the minimal distance to a set is 
\colorbox[RGB]{253,246,227}{{{{\color[RGB]{181, 137, 0} < }}}{{{\color[RGB]{101, 123, 131} r }}}}, there exists a point in this set at distance 
\colorbox[RGB]{253,246,227}{{{{\color[RGB]{181, 137, 0} < }}}{{{\color[RGB]{101, 123, 131} r }}}}\paragraph{metric.inf\_dist\_le\_inf\_dist\_add\_dist}
\par
The minimal distance from 
\colorbox[RGB]{253,246,227}{{{{\color[RGB]{101, 123, 131} x }}}} to 
\colorbox[RGB]{253,246,227}{{{{\color[RGB]{101, 123, 131} s }}}} is bounded by the distance from 
\colorbox[RGB]{253,246,227}{{{{\color[RGB]{101, 123, 131} y }}}} to 
\colorbox[RGB]{253,246,227}{{{{\color[RGB]{101, 123, 131} s }}}}, modulo
the distance between 
\colorbox[RGB]{253,246,227}{{{{\color[RGB]{101, 123, 131} x }}}} and 
\colorbox[RGB]{253,246,227}{{{{\color[RGB]{101, 123, 131} y }}}}\paragraph{metric.uniform\_continuous\_inf\_dist}
\par
The minimal distance to a set is uniformly continuous
\paragraph{metric.continuous\_inf\_dist}
\par
The minimal distance to a set is continuous
\paragraph{metric.inf\_dist\_eq\_closure}
\par
The minimal distance to a set and its closure coincide
\paragraph{metric.mem\_closure\_iff\_inf\_dist\_zero}
\par
A point belongs to the closure of 
\colorbox[RGB]{253,246,227}{{{{\color[RGB]{101, 123, 131} s }}}} iff its infimum distance to this set vanishes
\paragraph{metric.mem\_iff\_ind\_dist\_zero\_of\_closed}
\par
Given a closed set 
\colorbox[RGB]{253,246,227}{{{{\color[RGB]{101, 123, 131} s }}}}, a point belongs to 
\colorbox[RGB]{253,246,227}{{{{\color[RGB]{101, 123, 131} s }}}} iff its infimum distance to this set vanishes
\paragraph{metric.inf\_dist\_image}
\par
The infimum distance is invariant under isometries
\paragraph{metric.Hausdorff\_dist}
\par
The Hausdorff distance between two sets is the smallest nonnegative 
\colorbox[RGB]{253,246,227}{{{{\color[RGB]{101, 123, 131} r }}}} such that each set is
included in the 
\colorbox[RGB]{253,246,227}{{{{\color[RGB]{101, 123, 131} r }}}}-neighborhood of the other. If there is no such 
\colorbox[RGB]{253,246,227}{{{{\color[RGB]{101, 123, 131} r }}}}, it is defined to
be 
\colorbox[RGB]{253,246,227}{{{{\color[RGB]{108, 113, 196} 0 }}}}, arbitrarily
\paragraph{metric.Hausdorff\_dist\_nonneg}
\par
The Hausdorff distance is nonnegative
\paragraph{metric.Hausdorff\_edist\_ne\_top\_of\_ne\_empty\_of\_bounded}
\par
If two sets are nonempty and bounded in a metric space, they are at finite Hausdorff edistance
\paragraph{metric.Hausdorff\_dist\_self\_zero}
\par
The Hausdorff distance between a set and itself is zero
\paragraph{metric.Hausdorff\_dist\_comm}
\par
The Hausdorff distance from 
\colorbox[RGB]{253,246,227}{{{{\color[RGB]{101, 123, 131} s }}}} to 
\colorbox[RGB]{253,246,227}{{{{\color[RGB]{101, 123, 131} t }}}} and from 
\colorbox[RGB]{253,246,227}{{{{\color[RGB]{101, 123, 131} t }}}} to 
\colorbox[RGB]{253,246,227}{{{{\color[RGB]{101, 123, 131} s }}}} coincide
\paragraph{metric.Hausdorff\_dist\_empty}
\par
The Hausdorff distance to the empty set vanishes (if you want to have the more reasonable
value ∞ instead, use 
\colorbox[RGB]{253,246,227}{{{{\color[RGB]{101, 123, 131} Hausdorff\_edist }}}}, which takes values in ennreal)
\paragraph{metric.Hausdorff\_dist\_empty'}
\par
The Hausdorff distance to the empty set vanishes (if you want to have the more reasonable
value ∞ instead, use 
\colorbox[RGB]{253,246,227}{{{{\color[RGB]{101, 123, 131} Hausdorff\_edist }}}}, which takes values in ennreal)
\paragraph{metric.Hausdorff\_dist\_le\_of\_inf\_dist}
\par
Bounding the Hausdorff distance by bounding the distance of any point
in each set to the other set
\paragraph{metric.Hausdorff\_dist\_le\_of\_mem\_dist}
\par
Bounding the Hausdorff distance by exhibiting, for any point in each set,
another point in the other set at controlled distance
\paragraph{metric.Hausdorff\_dist\_le\_diam}
\par
The Hausdorff distance is controlled by the diameter of the union
\paragraph{metric.inf\_dist\_le\_Hausdorff\_dist\_of\_mem}
\par
The distance to a set is controlled by the Hausdorff distance
\paragraph{metric.exists\_dist\_lt\_of\_Hausdorff\_dist\_lt}
\par
If the Hausdorff distance is 
\colorbox[RGB]{253,246,227}{{{{\color[RGB]{181, 137, 0} < }}}{{{\color[RGB]{101, 123, 131} r }}}}, then any point in one of the sets is at distance
\colorbox[RGB]{253,246,227}{{{{\color[RGB]{181, 137, 0} < }}}{{{\color[RGB]{101, 123, 131} r }}}} of a point in the other set
\paragraph{metric.exists\_dist\_lt\_of\_Hausdorff\_dist\_lt'}
\par
If the Hausdorff distance is 
\colorbox[RGB]{253,246,227}{{{{\color[RGB]{181, 137, 0} < }}}{{{\color[RGB]{101, 123, 131} r }}}}, then any point in one of the sets is at distance
\colorbox[RGB]{253,246,227}{{{{\color[RGB]{181, 137, 0} < }}}{{{\color[RGB]{101, 123, 131} r }}}} of a point in the other set
\paragraph{metric.inf\_dist\_le\_inf\_dist\_add\_Hausdorff\_dist}
\par
The infimum distance to 
\colorbox[RGB]{253,246,227}{{{{\color[RGB]{101, 123, 131} s }}}} and 
\colorbox[RGB]{253,246,227}{{{{\color[RGB]{101, 123, 131} t }}}} are the same, up to the Hausdorff distance
between 
\colorbox[RGB]{253,246,227}{{{{\color[RGB]{101, 123, 131} s }}}} and 
\colorbox[RGB]{253,246,227}{{{{\color[RGB]{101, 123, 131} t }}}}\paragraph{metric.Hausdorff\_dist\_image}
\par
The Hausdorff distance is invariant under isometries
\paragraph{metric.Hausdorff\_dist\_triangle}
\par
The Hausdorff distance satisfies the triangular inequality
\paragraph{metric.Hausdorff\_dist\_triangle'}
\par
The Hausdorff distance satisfies the triangular inequality
\paragraph{metric.Hausdorff\_dist\_self\_closure}
\par
The Hausdorff distance between a set and its closure vanish
\paragraph{metric.Hausdorff\_dist\_closure₁}
\par
Replacing a set by its closure does not change the Hausdorff distance.
\paragraph{metric.Hausdorff\_dist\_closure₂}
\par
Replacing a set by its closure does not change the Hausdorff distance.
\paragraph{metric.Hausdorff\_dist\_closure}
\par
The Hausdorff distance between two sets and their closures coincide
\paragraph{metric.Hausdorff\_dist\_zero\_iff\_closure\_eq\_closure}
\par
Two sets are at zero Hausdorff distance if and only if they have the same closures
\paragraph{metric.Hausdorff\_dist\_zero\_iff\_eq\_of\_closed}
\par
Two closed sets are at zero Hausdorff distance if and only if they coincide
\section{topology/metric\_space/isometry.lean}\paragraph{isometry}
\par
An isometry (also known as isometric embedding) is a map preserving the edistance
between emetric spaces, or equivalently the distance between metric space.
\paragraph{isometry\_emetric\_iff\_metric}
\par
On metric spaces, a map is an isometry if and only if it preserves distances.
\paragraph{isometry.edist\_eq}
\par
An isometry preserves edistances.
\paragraph{isometry.dist\_eq}
\par
An isometry preserves distances.
\paragraph{isometry.injective}
\par
An isometry is injective
\paragraph{isometry\_subsingleton}
\par
Any map on a subsingleton is an isometry
\paragraph{isometry\_id}
\par
The identity is an isometry
\paragraph{isometry.comp}
\par
The composition of isometries is an isometry
\paragraph{isometry.uniform\_embedding}
\par
An isometry is an embedding
\paragraph{isometry.continuous}
\par
An isometry is continuous.
\paragraph{isometry.inv}
\par
The inverse of an isometry is an isometry.
\paragraph{emetric.isometry.diam\_image}
\par
Isometries preserve the diameter
\paragraph{isometry\_subtype\_val}
\par
The injection from a subtype is an isometry
\paragraph{metric.isometry.diam\_image}
\par
An isometry preserves the diameter in metric spaces
\paragraph{isometric}
\par
α and β are isometric if there is an isometric bijection between them.
\paragraph{isometry.isometric\_on\_range}
\par
An isometry induces an isometric isomorphism between the source space and the
range of the isometry.
\paragraph{Kuratowski\_embedding.embedding\_of\_subset}
\par
A metric space can be embedded in 
\colorbox[RGB]{253,246,227}{{{{\color[RGB]{101, 123, 131} l\textasciicircum{}∞(ℝ) }}}} via the distances to points in
a fixed countable set, if this set is dense. This map is given in the next definition,
without density assumptions.
\paragraph{Kuratowski\_embedding.embedding\_of\_subset\_dist\_le}
\par
The embedding map is always a semi-contraction.
\paragraph{Kuratowski\_embedding.embedding\_of\_subset\_isometry}
\par
When the reference set is dense, the embedding map is an isometry on its image.
\paragraph{Kuratowski\_embedding.exists\_isometric\_embedding}
\par
Every separable metric space embeds isometrically in ℓ\_infty\_ℝ.
\paragraph{Kuratowski\_embedding.Kuratowski\_embedding}
\par
The Kuratowski embedding is an isometric embedding of a separable metric space in ℓ\textasciicircum{}∞(ℝ)
\paragraph{Kuratowski\_embedding.Kuratowski\_embedding\_isometry}
\par
The Kuratowski embedding is an isometry
\paragraph{Kuratowski\_embedding.nonempty\_compacts.Kuratowski\_embedding}
\par
Version of the Kuratowski embedding for nonempty compacts
\section{topology/metric\_space/lipschitz.lean}\paragraph{uniform\_continuous\_of\_lipschitz}
\par
A Lipschitz function is uniformly continuous
\paragraph{continuous\_of\_lipschitz}
\par
A Lipschitz function is continuous
\paragraph{lipschitz\_with}
\par
\colorbox[RGB]{253,246,227}{{{{\color[RGB]{101, 123, 131} lipschitz\_with K f }}}}: the function 
\colorbox[RGB]{253,246,227}{{{{\color[RGB]{101, 123, 131} f }}}} is Lipschitz continuous w.r.t. the Lipschitz
constant 
\colorbox[RGB]{253,246,227}{{{{\color[RGB]{101, 123, 131} K }}}}.
\paragraph{lipschitz\_with.exists\_fixed\_point\_of\_contraction}
\par
Banach fixed-point theorem, contraction mapping theorem
\section{topology/metric\_space/premetric\_space.lean}\paragraph{premetric.dist\_setoid}
\par
The canonical equivalence relation on a premetric space.
\paragraph{premetric.metric\_quot}
\par
The canonical quotient of a premetric space, identifying points at distance 0.
\section{topology/opens.lean}\paragraph{topological\_space.opens}
\par
The type of open subsets of a topological space.
\paragraph{topological\_space.closeds}
\par
The type of closed subsets of a topological space.
\paragraph{topological\_space.nonempty\_compacts}
\par
The type of non-empty compact subsets of a topological space. The
non-emptiness will be useful in metric spaces, as we will be able to put
a distance (and not merely an edistance) on this space.
\paragraph{topological\_space.nonempty\_compacts.to\_closeds}
\par
Associate to a nonempty compact subset the corresponding closed subset
\section{topology/order.lean}\paragraph{topological\_space.generate\_open}
\par
The least topology containing a collection of basic sets.
\paragraph{topological\_space.generate\_from}
\par
The smallest topological space containing the collection 
\colorbox[RGB]{253,246,227}{{{{\color[RGB]{101, 123, 131} g }}}} of basic sets
\paragraph{topological\_space.induced}
\par
Given 
\colorbox[RGB]{253,246,227}{{{{\color[RGB]{101, 123, 131} f : α  }}}{{{\color[RGB]{133, 153, 0} → }}}{{{\color[RGB]{101, 123, 131}  β }}}} and a topology on 
\colorbox[RGB]{253,246,227}{{{{\color[RGB]{101, 123, 131} β }}}}, the induced topology on 
\colorbox[RGB]{253,246,227}{{{{\color[RGB]{101, 123, 131} α }}}} is the collection of
sets that are preimages of some open set in 
\colorbox[RGB]{253,246,227}{{{{\color[RGB]{101, 123, 131} β }}}}. This is the coarsest topology that
makes 
\colorbox[RGB]{253,246,227}{{{{\color[RGB]{101, 123, 131} f }}}} continuous.
\paragraph{topological\_space.coinduced}
\par
Given 
\colorbox[RGB]{253,246,227}{{{{\color[RGB]{101, 123, 131} f : α  }}}{{{\color[RGB]{133, 153, 0} → }}}{{{\color[RGB]{101, 123, 131}  β }}}} and a topology on 
\colorbox[RGB]{253,246,227}{{{{\color[RGB]{101, 123, 131} α }}}}, the coinduced topology on 
\colorbox[RGB]{253,246,227}{{{{\color[RGB]{101, 123, 131} β }}}} is defined
such that 
\colorbox[RGB]{253,246,227}{{{{\color[RGB]{101, 123, 131} s:set β }}}} is open if the preimage of 
\colorbox[RGB]{253,246,227}{{{{\color[RGB]{101, 123, 131} s }}}} is open. This is the finest topology that
makes 
\colorbox[RGB]{253,246,227}{{{{\color[RGB]{101, 123, 131} f }}}} continuous.
\section{topology/separation.lean}\paragraph{t0\_space}
\par
A T₀ space, also known as a Kolmogorov space, is a topological space
where for every pair 
\colorbox[RGB]{253,246,227}{{{{\color[RGB]{101, 123, 131} x  }}}{{{\color[RGB]{181, 137, 0} ≠ }}}{{{\color[RGB]{101, 123, 131}  y }}}}, there is an open set containing one but not the other.
\paragraph{t1\_space}
\par
A T₁ space, also known as a Fréchet space, is a topological space
where every singleton set is closed. Equivalently, for every pair
\colorbox[RGB]{253,246,227}{{{{\color[RGB]{101, 123, 131} x  }}}{{{\color[RGB]{181, 137, 0} ≠ }}}{{{\color[RGB]{101, 123, 131}  y }}}}, there is an open set containing 
\colorbox[RGB]{253,246,227}{{{{\color[RGB]{101, 123, 131} x }}}} and not 
\colorbox[RGB]{253,246,227}{{{{\color[RGB]{101, 123, 131} y }}}}.
\paragraph{t2\_space}
\par
A T₂ space, also known as a Hausdorff space, is one in which for every
\colorbox[RGB]{253,246,227}{{{{\color[RGB]{101, 123, 131} x  }}}{{{\color[RGB]{181, 137, 0} ≠ }}}{{{\color[RGB]{101, 123, 131}  y }}}} there exists disjoint open sets around 
\colorbox[RGB]{253,246,227}{{{{\color[RGB]{101, 123, 131} x }}}} and 
\colorbox[RGB]{253,246,227}{{{{\color[RGB]{101, 123, 131} y }}}}. This is
the most widely used of the separation axioms.
\paragraph{regular\_space}
\par
A T₃ space, also known as a regular space (although this condition sometimes
omits T₂), is one in which for every closed 
\colorbox[RGB]{253,246,227}{{{{\color[RGB]{101, 123, 131} C }}}} and 
\colorbox[RGB]{253,246,227}{{{{\color[RGB]{101, 123, 131} x ∉ C }}}}, there exist
disjoint open sets containing 
\colorbox[RGB]{253,246,227}{{{{\color[RGB]{101, 123, 131} x }}}} and 
\colorbox[RGB]{253,246,227}{{{{\color[RGB]{101, 123, 131} C }}}} respectively.
\paragraph{normal\_space}
\par
A T₄ space, also known as a normal space (although this condition sometimes
omits T₂), is one in which for every pair of disjoint closed sets 
\colorbox[RGB]{253,246,227}{{{{\color[RGB]{101, 123, 131} C }}}} and 
\colorbox[RGB]{253,246,227}{{{{\color[RGB]{101, 123, 131} D }}}},
there exist disjoint open sets containing 
\colorbox[RGB]{253,246,227}{{{{\color[RGB]{101, 123, 131} C }}}} and 
\colorbox[RGB]{253,246,227}{{{{\color[RGB]{101, 123, 131} D }}}} respectively.
\section{topology/sequences.lean}\paragraph{topological\_space.seq\_tendsto\_iff}
\par
A sequence converges in the sence of topological spaces iff the associated statement for filter
holds.
\paragraph{sequential\_closure}
\par
The sequential closure of a subset M ⊆ α of a topological space α is
the set of all p ∈ α which arise as limit of sequences in M.
\paragraph{is\_seq\_closed\_of\_def}
\par
A convenience lemma for showing that a set is sequentially closed.
\paragraph{sequential\_closure\_subset\_closure}
\par
The sequential closure of a set is contained in the closure of that set.
The converse is not true.
\paragraph{is\_seq\_closed\_of\_is\_closed}
\par
A set is sequentially closed if it is closed.
\paragraph{mem\_of\_is\_seq\_closed}
\par
The limit of a convergent sequence in a sequentially closed set is in that set.
\paragraph{mem\_of\_is\_closed\_sequential}
\par
The limit of a convergent sequence in a closed set is in that set.
\paragraph{sequential\_space}
\par
A sequential space is a space in which 'sequences are enough to probe the topology'. This can be
formalised by demanding that the sequential closure and the closure coincide. The following
statements show that other topological properties can be deduced from sequences in sequential
spaces.
\paragraph{is\_seq\_closed\_iff\_is\_closed}
\par
In a sequential space, a set is closed iff it's sequentially closed.
\paragraph{sequentially\_continuous}
\par
A function between topological spaces is sequentially continuous if it commutes with limit of
convergent sequences.
\paragraph{continuous\_iff\_sequentially\_continuous}
\par
In a sequential space, continuity and sequential continuity coincide.
\paragraph{metric.sequential\_space}
\par
Show that every metric space is sequential.
\section{topology/stone\_cech.lean}\paragraph{ultrafilter\_basis}
\par
Basis for the topology on 
\colorbox[RGB]{253,246,227}{{{{\color[RGB]{101, 123, 131} ultrafilter α }}}}.
\paragraph{ultrafilter\_is\_open\_basic}
\par
The basic open sets for the topology on ultrafilters are open.
\paragraph{ultrafilter\_is\_closed\_basic}
\par
The basic open sets for the topology on ultrafilters are also closed.
\paragraph{ultrafilter\_converges\_iff}
\par
Every ultrafilter 
\colorbox[RGB]{253,246,227}{{{{\color[RGB]{101, 123, 131} u }}}} on 
\colorbox[RGB]{253,246,227}{{{{\color[RGB]{101, 123, 131} ultrafilter α }}}} converges to a unique
point of 
\colorbox[RGB]{253,246,227}{{{{\color[RGB]{101, 123, 131} ultrafilter α }}}}, namely 
\colorbox[RGB]{253,246,227}{{{{\color[RGB]{101, 123, 131} mjoin u }}}}.
\paragraph{dense\_embedding\_pure}
\par
\colorbox[RGB]{253,246,227}{{{{\color[RGB]{101, 123, 131} pure : α  }}}{{{\color[RGB]{133, 153, 0} → }}}{{{\color[RGB]{101, 123, 131}  ultrafilter α }}}} defines a dense embedding of 
\colorbox[RGB]{253,246,227}{{{{\color[RGB]{101, 123, 131} α }}}} in 
\colorbox[RGB]{253,246,227}{{{{\color[RGB]{101, 123, 131} ultrafilter α }}}}.
\paragraph{ultrafilter.extend}
\par
The extension of a function 
\colorbox[RGB]{253,246,227}{{{{\color[RGB]{101, 123, 131} α  }}}{{{\color[RGB]{133, 153, 0} → }}}{{{\color[RGB]{101, 123, 131}  γ }}}} to a function 
\colorbox[RGB]{253,246,227}{{{{\color[RGB]{101, 123, 131} ultrafilter α  }}}{{{\color[RGB]{133, 153, 0} → }}}{{{\color[RGB]{101, 123, 131}  γ }}}}.
When 
\colorbox[RGB]{253,246,227}{{{{\color[RGB]{101, 123, 131} γ }}}} is a compact Hausdorff space it will be continuous.
\paragraph{ultrafilter\_extend\_eq\_iff}
\par
The value of 
\colorbox[RGB]{253,246,227}{{{{\color[RGB]{101, 123, 131} ultrafilter.extend f }}}} on an ultrafilter 
\colorbox[RGB]{253,246,227}{{{{\color[RGB]{101, 123, 131} b }}}} is the
unique limit of the ultrafilter 
\colorbox[RGB]{253,246,227}{{{{\color[RGB]{101, 123, 131} b.map f }}}} in 
\colorbox[RGB]{253,246,227}{{{{\color[RGB]{101, 123, 131} γ }}}}.
\paragraph{stone\_cech}
\par
The Stone-Čech compactification of a topological space.
\paragraph{stone\_cech\_unit}
\par
The natural map from α to its Stone-Čech compactification.
\paragraph{stone\_cech\_unit\_dense}
\par
The image of stone\_cech\_unit is dense. (But stone\_cech\_unit need
not be an embedding, for example if α is not Hausdorff.)
\paragraph{stone\_cech\_extend}
\par
The extension of a continuous function from α to a compact
Hausdorff space γ to the Stone-Čech compactification of α.
\section{topology/subset\_properties.lean}\paragraph{compact}
\par
A set 
\colorbox[RGB]{253,246,227}{{{{\color[RGB]{101, 123, 131} s }}}} is compact if for every filter 
\colorbox[RGB]{253,246,227}{{{{\color[RGB]{101, 123, 131} f }}}} that contains 
\colorbox[RGB]{253,246,227}{{{{\color[RGB]{101, 123, 131} s }}}},
every set of 
\colorbox[RGB]{253,246,227}{{{{\color[RGB]{101, 123, 131} f }}}} also meets every neighborhood of some 
\colorbox[RGB]{253,246,227}{{{{\color[RGB]{101, 123, 131} a ∈ s }}}}.
\paragraph{compact\_space}
\par
Type class for compact spaces. Separation is sometimes included in the definition, especially
in the French literature, but we do not include it here.
\paragraph{locally\_compact\_space}
\par
There are various definitions of "locally compact space" in the literature, which agree for
Hausdorff spaces but not in general. This one is the precise condition on X needed for the
evaluation 
\colorbox[RGB]{253,246,227}{{{{\color[RGB]{101, 123, 131} map C(X, Y) × X  }}}{{{\color[RGB]{133, 153, 0} → }}}{{{\color[RGB]{101, 123, 131}  Y }}}} to be continuous for all 
\colorbox[RGB]{253,246,227}{{{{\color[RGB]{101, 123, 131} Y }}}} when 
\colorbox[RGB]{253,246,227}{{{{\color[RGB]{101, 123, 131} C(X, Y) }}}} is given the
compact-open topology.
\paragraph{is\_irreducible}
\par
A irreducible set is one where there is no non-trivial pair of disjoint opens.
\paragraph{irreducible\_space}
\par
A irreducible space is one where there is no non-trivial pair of disjoint opens.
\paragraph{is\_connected}
\par
A connected set is one where there is no non-trivial open partition.
\paragraph{connected\_space}
\par
A connected space is one where there is no non-trivial open partition.
\section{topology/uniform\_space/basic.lean}\paragraph{id\_rel}
\par
The identity relation, or the graph of the identity function
\paragraph{comp\_rel}
\par
The composition of relations
\paragraph{uniform\_space.core}
\par
This core description of a uniform space is outside of the type class hierarchy. It is useful
for constructions of uniform spaces, when the topology is derived from the uniform space.
\paragraph{uniform\_space.core.to\_topological\_space}
\par
A uniform space generates a topological space
\paragraph{uniform\_space}
\par
A uniform space is a generalization of the "uniform" topological aspects of a
metric space. It consists of a filter on 
\colorbox[RGB]{253,246,227}{{{{\color[RGB]{101, 123, 131} α × α }}}} called the "uniformity", which
satisfies properties analogous to the reflexivity, symmetry, and triangle properties
of a metric.
\par
A metric space has a natural uniformity, and a uniform space has a natural topology.
A topological group also has a natural uniformity, even when it is not metrizable.
\paragraph{uniformity}
\par
The uniformity is a filter on α × α (inferred from an ambient uniform space
structure on α).
\paragraph{uniform\_space.comap}
\par
Given 
\colorbox[RGB]{253,246,227}{{{{\color[RGB]{101, 123, 131} f : α  }}}{{{\color[RGB]{133, 153, 0} → }}}{{{\color[RGB]{101, 123, 131}  β }}}} and a uniformity 
\colorbox[RGB]{253,246,227}{{{{\color[RGB]{101, 123, 131} u }}}} on 
\colorbox[RGB]{253,246,227}{{{{\color[RGB]{101, 123, 131} β }}}}, the inverse image of 
\colorbox[RGB]{253,246,227}{{{{\color[RGB]{101, 123, 131} u }}}} under 
\colorbox[RGB]{253,246,227}{{{{\color[RGB]{101, 123, 131} f }}}}is the inverse image in the filter sense of the induced function 
\colorbox[RGB]{253,246,227}{{{{\color[RGB]{101, 123, 131} α × α  }}}{{{\color[RGB]{133, 153, 0} → }}}{{{\color[RGB]{101, 123, 131}  β × β }}}}.
\paragraph{uniform\_space.core.sum}
\par
Uniformity on a disjoint union. Entourages of the diagonal in the union are obtained
by taking independently an entourage of the diagonal in the first part, and an entourage of
the diagonal in the second part.
\paragraph{union\_mem\_uniformity\_sum}
\par
The union of an entourage of the diagonal in each set of a disjoint union is again an entourage of the diagonal.
\section{topology/uniform\_space/cauchy.lean}\paragraph{cauchy}
\par
A filter 
\colorbox[RGB]{253,246,227}{{{{\color[RGB]{101, 123, 131} f }}}} is Cauchy if for every entourage 
\colorbox[RGB]{253,246,227}{{{{\color[RGB]{101, 123, 131} r }}}}, there exists an
\colorbox[RGB]{253,246,227}{{{{\color[RGB]{101, 123, 131} s ∈ f }}}} such that 
\colorbox[RGB]{253,246,227}{{{{\color[RGB]{101, 123, 131} s × s ⊆ r }}}}. This is a generalization of Cauchy
sequences, because if 
\colorbox[RGB]{253,246,227}{{{{\color[RGB]{101, 123, 131} a : ℕ  }}}{{{\color[RGB]{133, 153, 0} → }}}{{{\color[RGB]{101, 123, 131}  α }}}} then the filter of sets containing
cofinitely many of the 
\colorbox[RGB]{253,246,227}{{{{\color[RGB]{101, 123, 131} a n }}}} is Cauchy iff 
\colorbox[RGB]{253,246,227}{{{{\color[RGB]{101, 123, 131} a }}}} is a Cauchy sequence.
\paragraph{cauchy\_seq}
\par
Cauchy sequences. Usually defined on ℕ, but often it is also useful to say that a function
defined on ℝ is Cauchy at +∞ to deduce convergence. Therefore, we define it in a type class that
is general enough to cover both ℕ and ℝ, which are the main motivating examples.
\paragraph{complete\_space}
\par
A complete space is defined here using uniformities. A uniform space
is complete if every Cauchy filter converges.
\paragraph{complete\_space\_of\_is\_complete\_univ}
\par
If 
\colorbox[RGB]{253,246,227}{{{{\color[RGB]{101, 123, 131} univ }}}} is complete, the space is a complete space
\paragraph{cauchy\_seq\_tendsto\_of\_complete}
\par
A Cauchy sequence in a complete space converges
\paragraph{totally\_bounded}
\par
A set 
\colorbox[RGB]{253,246,227}{{{{\color[RGB]{101, 123, 131} s }}}} is totally bounded if for every entourage 
\colorbox[RGB]{253,246,227}{{{{\color[RGB]{101, 123, 131} d }}}} there is a finite
set of points 
\colorbox[RGB]{253,246,227}{{{{\color[RGB]{101, 123, 131} t }}}} such that every element of 
\colorbox[RGB]{253,246,227}{{{{\color[RGB]{101, 123, 131} s }}}} is 
\colorbox[RGB]{253,246,227}{{{{\color[RGB]{101, 123, 131} d }}}}-near to some element of 
\colorbox[RGB]{253,246,227}{{{{\color[RGB]{101, 123, 131} t }}}}.
\section{topology/uniform\_space/complete\_separated.lean}\section{topology/uniform\_space/completion.lean}\paragraph{Cauchy}
\par
Space of Cauchy filters
\par
This is essentially the completion of a uniform space. The embeddings are the neighbourhood filters.
This space is not minimal, the separated uniform space (i.e. quotiented on the intersection of all
entourages) is necessary for this.
\paragraph{Cauchy.pure\_cauchy}
\par
Embedding of 
\colorbox[RGB]{253,246,227}{{{{\color[RGB]{101, 123, 131} α }}}} into its completion
\paragraph{uniform\_space.completion}
\par
Hausdorff completion of 
\colorbox[RGB]{253,246,227}{{{{\color[RGB]{101, 123, 131} α }}}}\paragraph{uniform\_space.completion.has\_coe}
\par
Automatic coercion from 
\colorbox[RGB]{253,246,227}{{{{\color[RGB]{101, 123, 131} α }}}} to its completion. Not always injective.
\paragraph{uniform\_space.completion.extension}
\par
"Extension" to the completion. Based on 
\colorbox[RGB]{253,246,227}{{{{\color[RGB]{101, 123, 131} Cauchy.extend }}}}, which is defined for any map 
\colorbox[RGB]{253,246,227}{{{{\color[RGB]{101, 123, 131} f }}}} but
returns an arbitrary constant value if 
\colorbox[RGB]{253,246,227}{{{{\color[RGB]{101, 123, 131} f }}}} is not uniformly continuous
\paragraph{uniform\_space.completion.map}
\par
Completion functor acting on morphisms
\section{topology/uniform\_space/pi.lean}\section{topology/uniform\_space/separation.lean}\paragraph{separation\_rel}
\par
The separation relation is the intersection of all entourages.
Two points which are related by the separation relation are "indistinguishable"
according to the uniform structure.
\section{topology/uniform\_space/uniform\_embedding.lean}\paragraph{is\_complete\_image\_iff}
\par
A set is complete iff its image under a uniform embedding is complete.
\end{document}